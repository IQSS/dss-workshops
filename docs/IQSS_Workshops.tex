\documentclass[]{book}
\usepackage{lmodern}
\usepackage{amssymb,amsmath}
\usepackage{ifxetex,ifluatex}
\usepackage{fixltx2e} % provides \textsubscript
\ifnum 0\ifxetex 1\fi\ifluatex 1\fi=0 % if pdftex
  \usepackage[T1]{fontenc}
  \usepackage[utf8]{inputenc}
\else % if luatex or xelatex
  \ifxetex
    \usepackage{mathspec}
  \else
    \usepackage{fontspec}
  \fi
  \defaultfontfeatures{Ligatures=TeX,Scale=MatchLowercase}
\fi
% use upquote if available, for straight quotes in verbatim environments
\IfFileExists{upquote.sty}{\usepackage{upquote}}{}
% use microtype if available
\IfFileExists{microtype.sty}{%
\usepackage{microtype}
\UseMicrotypeSet[protrusion]{basicmath} % disable protrusion for tt fonts
}{}
\usepackage[margin=1.5in]{geometry}
\usepackage{hyperref}
\hypersetup{unicode=true,
            pdftitle={IQSS Workshops},
            pdfborder={0 0 0},
            breaklinks=true}
\urlstyle{same}  % don't use monospace font for urls
\usepackage{natbib}
\bibliographystyle{apalike}
\usepackage{color}
\usepackage{fancyvrb}
\newcommand{\VerbBar}{|}
\newcommand{\VERB}{\Verb[commandchars=\\\{\}]}
\DefineVerbatimEnvironment{Highlighting}{Verbatim}{commandchars=\\\{\}}
% Add ',fontsize=\small' for more characters per line
\usepackage{framed}
\definecolor{shadecolor}{RGB}{248,248,248}
\newenvironment{Shaded}{\begin{snugshade}}{\end{snugshade}}
\newcommand{\KeywordTok}[1]{\textcolor[rgb]{0.13,0.29,0.53}{\textbf{#1}}}
\newcommand{\DataTypeTok}[1]{\textcolor[rgb]{0.13,0.29,0.53}{#1}}
\newcommand{\DecValTok}[1]{\textcolor[rgb]{0.00,0.00,0.81}{#1}}
\newcommand{\BaseNTok}[1]{\textcolor[rgb]{0.00,0.00,0.81}{#1}}
\newcommand{\FloatTok}[1]{\textcolor[rgb]{0.00,0.00,0.81}{#1}}
\newcommand{\ConstantTok}[1]{\textcolor[rgb]{0.00,0.00,0.00}{#1}}
\newcommand{\CharTok}[1]{\textcolor[rgb]{0.31,0.60,0.02}{#1}}
\newcommand{\SpecialCharTok}[1]{\textcolor[rgb]{0.00,0.00,0.00}{#1}}
\newcommand{\StringTok}[1]{\textcolor[rgb]{0.31,0.60,0.02}{#1}}
\newcommand{\VerbatimStringTok}[1]{\textcolor[rgb]{0.31,0.60,0.02}{#1}}
\newcommand{\SpecialStringTok}[1]{\textcolor[rgb]{0.31,0.60,0.02}{#1}}
\newcommand{\ImportTok}[1]{#1}
\newcommand{\CommentTok}[1]{\textcolor[rgb]{0.56,0.35,0.01}{\textit{#1}}}
\newcommand{\DocumentationTok}[1]{\textcolor[rgb]{0.56,0.35,0.01}{\textbf{\textit{#1}}}}
\newcommand{\AnnotationTok}[1]{\textcolor[rgb]{0.56,0.35,0.01}{\textbf{\textit{#1}}}}
\newcommand{\CommentVarTok}[1]{\textcolor[rgb]{0.56,0.35,0.01}{\textbf{\textit{#1}}}}
\newcommand{\OtherTok}[1]{\textcolor[rgb]{0.56,0.35,0.01}{#1}}
\newcommand{\FunctionTok}[1]{\textcolor[rgb]{0.00,0.00,0.00}{#1}}
\newcommand{\VariableTok}[1]{\textcolor[rgb]{0.00,0.00,0.00}{#1}}
\newcommand{\ControlFlowTok}[1]{\textcolor[rgb]{0.13,0.29,0.53}{\textbf{#1}}}
\newcommand{\OperatorTok}[1]{\textcolor[rgb]{0.81,0.36,0.00}{\textbf{#1}}}
\newcommand{\BuiltInTok}[1]{#1}
\newcommand{\ExtensionTok}[1]{#1}
\newcommand{\PreprocessorTok}[1]{\textcolor[rgb]{0.56,0.35,0.01}{\textit{#1}}}
\newcommand{\AttributeTok}[1]{\textcolor[rgb]{0.77,0.63,0.00}{#1}}
\newcommand{\RegionMarkerTok}[1]{#1}
\newcommand{\InformationTok}[1]{\textcolor[rgb]{0.56,0.35,0.01}{\textbf{\textit{#1}}}}
\newcommand{\WarningTok}[1]{\textcolor[rgb]{0.56,0.35,0.01}{\textbf{\textit{#1}}}}
\newcommand{\AlertTok}[1]{\textcolor[rgb]{0.94,0.16,0.16}{#1}}
\newcommand{\ErrorTok}[1]{\textcolor[rgb]{0.64,0.00,0.00}{\textbf{#1}}}
\newcommand{\NormalTok}[1]{#1}
\usepackage{longtable,booktabs}
\usepackage{graphicx,grffile}
\makeatletter
\def\maxwidth{\ifdim\Gin@nat@width>\linewidth\linewidth\else\Gin@nat@width\fi}
\def\maxheight{\ifdim\Gin@nat@height>\textheight\textheight\else\Gin@nat@height\fi}
\makeatother
% Scale images if necessary, so that they will not overflow the page
% margins by default, and it is still possible to overwrite the defaults
% using explicit options in \includegraphics[width, height, ...]{}
\setkeys{Gin}{width=\maxwidth,height=\maxheight,keepaspectratio}
\IfFileExists{parskip.sty}{%
\usepackage{parskip}
}{% else
\setlength{\parindent}{0pt}
\setlength{\parskip}{6pt plus 2pt minus 1pt}
}
\setlength{\emergencystretch}{3em}  % prevent overfull lines
\providecommand{\tightlist}{%
  \setlength{\itemsep}{0pt}\setlength{\parskip}{0pt}}
\setcounter{secnumdepth}{5}
% Redefines (sub)paragraphs to behave more like sections
\ifx\paragraph\undefined\else
\let\oldparagraph\paragraph
\renewcommand{\paragraph}[1]{\oldparagraph{#1}\mbox{}}
\fi
\ifx\subparagraph\undefined\else
\let\oldsubparagraph\subparagraph
\renewcommand{\subparagraph}[1]{\oldsubparagraph{#1}\mbox{}}
\fi

%%% Use protect on footnotes to avoid problems with footnotes in titles
\let\rmarkdownfootnote\footnote%
\def\footnote{\protect\rmarkdownfootnote}

%%% Change title format to be more compact
\usepackage{titling}

% Create subtitle command for use in maketitle
\providecommand{\subtitle}[1]{
  \posttitle{
    \begin{center}\large#1\end{center}
    }
}

\setlength{\droptitle}{-2em}

  \title{IQSS Workshops}
    \pretitle{\vspace{\droptitle}\centering\huge}
  \posttitle{\par}
    \author{}
    \preauthor{}\postauthor{}
      \predate{\centering\large\emph}
  \postdate{\par}
    \date{September 2019}

\usepackage{booktabs}

\usepackage{epsfig}
\usepackage{epstopdf}
\usepackage{rotate}
\usepackage{graphicx}
\usepackage{hyperref}
\usepackage{alphalph}
\usepackage{caption}
\usepackage[hang,flushmargin]{footmisc}
\usepackage{framed}
\usepackage{xcolor}
\usepackage{verbatim} 

\usepackage{bm}
\setcounter{MaxMatrixCols}{20}
\newcommand{\Var}{\mathrm{Var}}
\newcommand{\SD}{\mathrm{SD}}
\newcommand{\Cov}{\mathrm{Cov}}
\newcommand{\fx}{f({\bf x})}
\newcommand\R{{\textsf R~}}
\newcommand\Rst{\textsf{RStudio}}

% spacing between environments
\usepackage{amsthm}
\makeatletter
\def\thm@space@setup{%
  \thm@preskip=15pt plus 2pt minus 4pt
  \thm@postskip=\thm@preskip
}
\makeatother


% Title format
\usepackage{titling}
\pretitle{\Huge\sffamily}
\posttitle{\par\vskip 0.5em}
\predate{\LARGE\sffamily}
\postdate{\par}

\urlstyle{tt}

\begin{document}
\maketitle

{
\setcounter{tocdepth}{1}
\tableofcontents
}
\chapter*{Introduction}\label{introduction}
\addcontentsline{toc}{chapter}{Introduction}

\section*{Table of Contents}\label{table-of-contents}
\addcontentsline{toc}{section}{Table of Contents}

Materials for the \href{http://dss.iq.harvard.edu}{Data Science
Services} statistical software workshops from the
\href{http://iq.harvard.edu}{Institute for Quantitative Social Science}
at Harvard.

\begin{enumerate}
\def\labelenumi{\arabic{enumi}.}
\tightlist
\item
  \href{./R/Rintro/Rintro.html}{Introduction to R}
\item
  \href{./R/Rmodels/Rmodels.html}{Regression models in R}
\item
  \href{./R/Rgraphics/Rgraphics.html}{Graphics in R using ggplot2}
\item
  \href{./R/RDataWrangling/RDataWrangling.html}{R data wrangling}
\item
  \href{./Python/PythonIntro/PythonIntro.html}{Introduction to Python}
\item
  \href{./Python/PythonWebScrape/PythonWebScrape.html}{Python
  web-scraping}
\item
  \href{./Stata/StataIntro/StataIntro.html}{Introduction to Stata}
\end{enumerate}

These workshops are a work-in-progress, please provide feedback! Email:
\href{mailto:help@iq.harvard.edu}{\nolinkurl{help@iq.harvard.edu}}

\section*{Authors and Sources}\label{authors-and-sources}
\addcontentsline{toc}{section}{Authors and Sources}

These content of these workshops are almost entirely the work of Ista
Zahn, now at Harvard's School of Public Health. The current workshop
materials have been modified by Steve Worthington, Jinjie Liu, and Yihan
Wang, at Harvard's Institute for Quantitative Social Science.

\chapter{Welcome}\label{welcome}

\section{Materials and setup}\label{materials-and-setup}

\textbf{NOTE: skip this section if you are not running R locally} (e.g.,
if you are running R in your browser using a remote Jupyter server)

You should have R installed --if not:

\begin{itemize}
\tightlist
\item
  Download and install R from \url{http://cran.r-project.org}
\item
  Download and install RStudio from
  \url{https://www.rstudio.com/products/rstudio/download/\#download}
\end{itemize}

Notes and examples for this workshop are available at
\href{http://tutorials.iq.harvard.edu/R/Rintro/Rintro.html}{}

Start RStudio create a new project: - On Windows click the start button
and search for rstudio. On Mac RStudio will be in your applications
folder. - In Rstudio go to \texttt{File\ -\textgreater{}\ New\ Project}.
- Choose \texttt{New\ Directory} and \texttt{New\ Project}. - Choose a
name and location for your new project directory.

\section{Workshop goals and approach}\label{workshop-goals-and-approach}

In this workshop you will

\begin{itemize}
\tightlist
\item
  learn R basics,
\item
  learn about the R package ecosystem,
\item
  practice reading files and manipulating data in R
\end{itemize}

A more general goal is to get you comfortable with R so that it seems
less scary and mystifying than it perhaps does now. Note that this is by
no means a complete or thorough introduction to R! It's just enough to
get you started.

This workshop is relatively informal, example-oriented, and hands-on. We
won't spend much time examining language features in detail. Instead we
will work through an example, and learn some things about the R along
the way.

As an example project we will analyze the popularity of baby names in
the US from 1960 through 2017. Among the questions we will use R to
answer are:

\begin{itemize}
\tightlist
\item
  In which year did your name achieve peak popularity?
\item
  How many children were born each year?
\item
  What are the most popular names overall? For girls? For Boys?
\end{itemize}

\chapter{Graphical User Interfaces
(GUIs)}\label{graphical-user-interfaces-guis}

There are many different ways you can interact with R. See the
\href{http://tutorials.iq.harvard.edu/DataScienceTools/DataScienceTools.html}{Data
Science Tools workshop notes} for details.

For this workshop I encourage you to use RStudio; it is a good
R-specific IDE that mostly just works.

\section{Launch RStudio (skip if not using
Rstudio)}\label{launch-rstudio-skip-if-not-using-rstudio}

\textbf{Note: skip this section if you are not using Rstudio (e.g., if
you are running these examples in a Jupyter notebook).}

\begin{itemize}
\tightlist
\item
  Start the RStudio program
\item
  In RStudio, go to \textbf{File =\textgreater{} New File
  =\textgreater{} R Script}
\end{itemize}

The window in the upper-left is your R script. This is where you will
write instructions for R to carry out.

The window in the lower-left is the R console. This is where results
will be displayed.

\section{Exercise 0}\label{exercise-0}

The purpose of this exercise is to give you an opportunity to explore
the interface provided by RStudio (or whichever GUI you've decided to
use). You may not know how to do these things; that's fine! This is an
opportunity to figure it out.

Also keep in mind that we are living in a golden age of tab completion.
If you don't know the name of an R function, try guessing the first two
or three letters and pressing TAB. If you guessed correctly the function
you are looking for should appear in a pop up!

\begin{center}\rule{0.5\linewidth}{\linethickness}\end{center}

\begin{enumerate}
\def\labelenumi{\arabic{enumi}.}
\tightlist
\item
  Try to get R to add 2 plus 2.
\end{enumerate}

\begin{Shaded}
\begin{Highlighting}[]
\NormalTok{##}
\end{Highlighting}
\end{Shaded}

\begin{enumerate}
\def\labelenumi{\arabic{enumi}.}
\setcounter{enumi}{1}
\tightlist
\item
  Try to calculate the square root of 10.
\end{enumerate}

\begin{Shaded}
\begin{Highlighting}[]
\NormalTok{##}
\end{Highlighting}
\end{Shaded}

\begin{enumerate}
\def\labelenumi{\arabic{enumi}.}
\setcounter{enumi}{2}
\tightlist
\item
  R includes extensive documentation, including a manual named ``An
  introduction to R''. Use the RStudio help pane. to locate this manual.
\end{enumerate}

\section{Exercise 0 solution}\label{exercise-0-solution}

\begin{Shaded}
\begin{Highlighting}[]
\NormalTok{## 1. 2 plus 2}
\DecValTok{2} \OperatorTok{+}\StringTok{ }\DecValTok{2}
\NormalTok{## or}
\KeywordTok{sum}\NormalTok{(}\DecValTok{2}\NormalTok{, }\DecValTok{2}\NormalTok{)}
\end{Highlighting}
\end{Shaded}

\begin{Shaded}
\begin{Highlighting}[]
\NormalTok{## 2. square root of 10:}
\KeywordTok{sqrt}\NormalTok{(}\DecValTok{10}\NormalTok{)}
\NormalTok{## or}
\DecValTok{10}\OperatorTok{^}\NormalTok{(}\DecValTok{1}\OperatorTok{/}\DecValTok{2}\NormalTok{)}
\end{Highlighting}
\end{Shaded}

\begin{Shaded}
\begin{Highlighting}[]
\NormalTok{## 3. Find "An Introduction to R".}
\end{Highlighting}
\end{Shaded}

\begin{Shaded}
\begin{Highlighting}[]
\NormalTok{## Go to the main help page by running 'help.start() or using the GUI}
\NormalTok{## menu, find and click on the link to "An Introduction to R".}
\end{Highlighting}
\end{Shaded}

\chapter{R basics}\label{r-basics}

\section{Function calls}\label{function-calls}

The general form for calling R functions is

\begin{Shaded}
\begin{Highlighting}[]
\NormalTok{## FunctionName(arg.1 = value.1, arg.2 = value.2, ..., arg.n - value.n)}
\end{Highlighting}
\end{Shaded}

Arguments can be matched by name; unnamed arguments will be matched by
position.

\section{Assignment}\label{assignment}

Values can be assigned names and used in subsequent operations

\begin{itemize}
\tightlist
\item
  The ``gets'' \texttt{\textless{}-} operator (less than followed by a
  dash) is used to save values
\item
  The name on the left gets the value on the right.
\end{itemize}

\begin{Shaded}
\begin{Highlighting}[]
\KeywordTok{sqrt}\NormalTok{(}\DecValTok{10}\NormalTok{) ## calculate square root of 10; result is not stored anywhere}
\NormalTok{x <-}\StringTok{ }\KeywordTok{sqrt}\NormalTok{(}\DecValTok{10}\NormalTok{) }\CommentTok{# assign result to a variable named x}
\end{Highlighting}
\end{Shaded}

Names should start with a letter, and contain only letters, numbers,
underscores, and periods.

\section{Asking R for help}\label{asking-r-for-help}

You can ask R for help using the \texttt{help} function, or the
\texttt{?} shortcut.

\begin{Shaded}
\begin{Highlighting}[]
\KeywordTok{help}\NormalTok{(help)}
\end{Highlighting}
\end{Shaded}

The \texttt{help} function can be used to look up the documentation for
a function, or to look up the documentation to a package. We can learn
how to use the \texttt{stats} package by reading its documentation like
this:

\begin{Shaded}
\begin{Highlighting}[]
\KeywordTok{help}\NormalTok{(}\DataTypeTok{package =} \StringTok{"stats"}\NormalTok{)}
\end{Highlighting}
\end{Shaded}

\chapter{Getting data into R}\label{getting-data-into-r}

R has data reading functionality built-in -- see e.g.,
\texttt{help(read.table)}. However, faster and more robust tools are
available, and so to make things easier on ourselves we will use a
\emph{contributed package} called \texttt{readr} instead. This requires
that we learn a little bit about packages in R.

\section{Installing and using R
packages}\label{installing-and-using-r-packages}

A large number of contributed packages are available. If you are looking
for a package for a specific task,
\url{https://cran.r-project.org/web/views/} and \url{https://r-pkg.org}
are good places to start.

You can install a package in R using the \texttt{install.packages()}
function. Once a package is installed you may use the \texttt{library}
function to attach it so that it can be used.

\begin{Shaded}
\begin{Highlighting}[]
\NormalTok{## install.packages("readr")}
\KeywordTok{library}\NormalTok{(readr)}
\end{Highlighting}
\end{Shaded}

\section{Readers for common file
types}\label{readers-for-common-file-types}

In order to read data from a file, you have to know what kind of file it
is. The table below lists functions that can import data from common
plain-text formats.

\begin{longtable}[]{@{}ll@{}}
\toprule
Data Type & Function\tabularnewline
\midrule
\endhead
comma separated & \texttt{read\_csv()}\tabularnewline
tab separated & \texttt{read\_delim()}\tabularnewline
other delimited formats & \texttt{read\_table()}\tabularnewline
fixed width & \texttt{read\_fwf()}\tabularnewline
\bottomrule
\end{longtable}

\textbf{Note} You may be confused by the existence of similar functions,
e.g., \texttt{read.csv} and \texttt{read.delim}. These are legacy
functions that tend to be slower and less robust than the \texttt{readr}
functions. One way to tell them apart is that the faster more robust
versions use underscores in their names (e.g., \texttt{read\_csv}) while
the older functions us dots (e.g., \texttt{read.csv}). My advice is to
use the more robust newer versions, i.e., the ones with underscores.

\section{Baby names data}\label{baby-names-data}

The examples in this workshop use US baby names data retrieved from
\url{https://catalog.data.gov/dataset/baby-names-from-social-security-card-applications-national-level-data}
A cleaned and merged version of these data is available at
\texttt{http://tutorials.iq.harvard.edu/data/babyNames.csv}.

\section{Exercise 1: Reading the baby names
data}\label{exercise-1-reading-the-baby-names-data}

Make sure you have installed the \texttt{readr} package and attached it
with \texttt{library(readr)}.

Baby names data are available at
\texttt{"http://tutorials.iq.harvard.edu/data/babyNames.csv"}.

\begin{enumerate}
\def\labelenumi{\arabic{enumi}.}
\item
  Open the \texttt{read\_csv} help page to determine how to use it to
  read in data.
\item
  Read the baby names data using the \texttt{read\_csv} function and
  assign the result with the name \texttt{baby.names}.
\item
  BONUS (optional): Save the \texttt{baby.names} data as a Stata data
  set \texttt{babynames.dta} and as an R data set
  \texttt{babynames.rds}.
\end{enumerate}

\section{Exercise 1 solution}\label{exercise-1-solution}

\begin{Shaded}
\begin{Highlighting}[]
\NormalTok{## read ?read_csv}
\end{Highlighting}
\end{Shaded}

\begin{Shaded}
\begin{Highlighting}[]
\NormalTok{baby.names <-}\StringTok{ }\KeywordTok{read_csv}\NormalTok{(}\StringTok{"http://tutorials.iq.harvard.edu/data/babyNames.csv"}\NormalTok{)}
\end{Highlighting}
\end{Shaded}

\chapter{Popularity of your name}\label{popularity-of-your-name}

In this section we will pull out specific names and examine changes in
their popularity over time.

The \texttt{baby.names} object we created in the last exercise is a
\texttt{data.frame}. There are many other data structures in R, but for
now we'll focus on working with \texttt{data.frames}.

R has decent data manipulation tools built-in -- see e.g.,
\texttt{help(Extract)}. However, these tools are powerful and complex
and often overwhelm beginners. To make things easier on ourselves we
will use a \emph{contributed package} called \texttt{dplyr} instead.

\begin{Shaded}
\begin{Highlighting}[]
\NormalTok{## install.packages("dplyr")}
\KeywordTok{library}\NormalTok{(dplyr)}
\end{Highlighting}
\end{Shaded}

\section{Filtering and arranging
data}\label{filtering-and-arranging-data}

One way to find the year in which your name was the most popular is to
filter out just the rows corresponding to your name, and then arrange
(sort) by Count.

To demonstrate these techniques we'll try to determine whether ``Alex''"
or ``Jim'' was more popular in 1992. We start by filtering the data so
that we keep only rows where Year is equal to \texttt{1992} and Name is
either ``Alex'' or ``Mark''.

\begin{Shaded}
\begin{Highlighting}[]
\NormalTok{am <-}\StringTok{ }\KeywordTok{filter}\NormalTok{(baby.names, }
\NormalTok{             Year }\OperatorTok{==}\StringTok{ }\DecValTok{1992} \OperatorTok{&}\StringTok{ }\NormalTok{(Name }\OperatorTok{==}\StringTok{ "Alex"} \OperatorTok{|}\StringTok{ }\NormalTok{Name }\OperatorTok{==}\StringTok{ "Mark"}\NormalTok{))}
\NormalTok{am}
\end{Highlighting}
\end{Shaded}

Notice that we can we can combine conditons using \texttt{\&} (AND) and
\texttt{\textbar{}} (OR).

In this case it's pretty easy to see that ``Mark'' is more popular, but
to make it even easier we can arrange the data so that the most popular
name is listed first.

\begin{Shaded}
\begin{Highlighting}[]
\KeywordTok{arrange}\NormalTok{(am, Count)}
\end{Highlighting}
\end{Shaded}

\begin{Shaded}
\begin{Highlighting}[]
\KeywordTok{arrange}\NormalTok{(am, }\KeywordTok{desc}\NormalTok{(Count))}
\end{Highlighting}
\end{Shaded}

\section{Other logical operators}\label{other-logical-operators}

In the previous example we used \texttt{==} to filter rows. Other
relational and logical operators are listed below.

\begin{longtable}[]{@{}ll@{}}
\toprule
Operator & Meaning\tabularnewline
\midrule
\endhead
\texttt{==} & equal to\tabularnewline
\texttt{!=} & not equal to\tabularnewline
\texttt{\textgreater{}} & greater than\tabularnewline
\texttt{\textgreater{}=} & greater than or equal to\tabularnewline
\texttt{\textless{}} & less than\tabularnewline
\texttt{\textless{}=} & less than or equal to\tabularnewline
\texttt{\%in\%} & contained in\tabularnewline
\bottomrule
\end{longtable}

These operators may be combined with \texttt{\&} (and) or
\texttt{\textbar{}} (or).

\section{Exercise 2: Peak popularity of your
name}\label{exercise-2-peak-popularity-of-your-name}

In this exercise you will discover the year your name reached its
maximum popularity.

Read in the ``babyNames.csv'' file if you have not already done so,
assigning the result to \texttt{baby.names}. The file is located at
\texttt{"http://tutorials.iq.harvard.edu/data/babyNames.csv"}

Make sure you have installed the \texttt{dplyr} package and attached it
with \texttt{library(dplyr)}.

\begin{enumerate}
\def\labelenumi{\arabic{enumi}.}
\tightlist
\item
  Use \texttt{filter} to extract data for your name (or another name of
  your choice).
\end{enumerate}

\begin{Shaded}
\begin{Highlighting}[]
\NormalTok{##}
\end{Highlighting}
\end{Shaded}

\begin{enumerate}
\def\labelenumi{\arabic{enumi}.}
\setcounter{enumi}{1}
\tightlist
\item
  Arrange the data you produced in step 1 above by \texttt{Count}. In
  which year was the name most popular?
\end{enumerate}

\begin{Shaded}
\begin{Highlighting}[]
\NormalTok{##}
\end{Highlighting}
\end{Shaded}

\begin{enumerate}
\def\labelenumi{\arabic{enumi}.}
\setcounter{enumi}{2}
\tightlist
\item
  BONUS (optional): Filter the data to extract \emph{only} the row
  containing the most popular boys name in 1999.
\end{enumerate}

\begin{Shaded}
\begin{Highlighting}[]
\NormalTok{##}
\end{Highlighting}
\end{Shaded}

\section{Exercise 2 solution}\label{exercise-2-solution}

\begin{Shaded}
\begin{Highlighting}[]
\CommentTok{# 1.  Use `filter` to extract data for your name (or another name of your choice).  }
\end{Highlighting}
\end{Shaded}

\begin{Shaded}
\begin{Highlighting}[]
\NormalTok{george <-}\StringTok{ }\KeywordTok{filter}\NormalTok{(baby.names, Name }\OperatorTok{==}\StringTok{ "George"}\NormalTok{)}
\end{Highlighting}
\end{Shaded}

\begin{Shaded}
\begin{Highlighting}[]
\CommentTok{# 2.  Arrange the data you produced in step 1 above by `Count`. }
\CommentTok{#     In which year was the name most popular?}
\end{Highlighting}
\end{Shaded}

\begin{Shaded}
\begin{Highlighting}[]
\KeywordTok{arrange}\NormalTok{(george, }\KeywordTok{desc}\NormalTok{(Count))}
\end{Highlighting}
\end{Shaded}

\begin{Shaded}
\begin{Highlighting}[]
\CommentTok{# 3.  BONUS (optional): Filter the data to extract _only_ the }
\CommentTok{#     row containing the most popular boys name in 1999.}
\end{Highlighting}
\end{Shaded}

\begin{Shaded}
\begin{Highlighting}[]
\NormalTok{boys.}\DecValTok{1999}\NormalTok{ <-}\StringTok{ }\KeywordTok{filter}\NormalTok{(baby.names, }
\NormalTok{                    Year }\OperatorTok{==}\StringTok{ }\DecValTok{1999} \OperatorTok{&}\StringTok{ }\NormalTok{Sex }\OperatorTok{==}\StringTok{ "Boys"}\NormalTok{)}
\end{Highlighting}
\end{Shaded}

\begin{Shaded}
\begin{Highlighting}[]
\KeywordTok{filter}\NormalTok{(boys.}\DecValTok{1999}\NormalTok{, Count }\OperatorTok{==}\StringTok{ }\KeywordTok{max}\NormalTok{(Count))}
\end{Highlighting}
\end{Shaded}

\chapter{Plotting baby name trends over
time}\label{plotting-baby-name-trends-over-time}

It can be difficult to spot trends when looking at summary tables.
Plotting the data makes it easier to identify interesting patterns.

R has decent plotting tools built-in -- see e.g., \texttt{help(plot)}.
However, To make things easier on ourselves we will use a
\emph{contributed package} called \texttt{ggplot2} instead.

\begin{Shaded}
\begin{Highlighting}[]
\NormalTok{## install.packages("ggplot2")}
\KeywordTok{library}\NormalTok{(ggplot2)}
\end{Highlighting}
\end{Shaded}

For quick and simple plots we can use the \texttt{qplot} function. For
example, we can plot the number of babies given the name ``Diana'' over
time like this:

\begin{Shaded}
\begin{Highlighting}[]
\NormalTok{diana <-}\StringTok{ }\KeywordTok{filter}\NormalTok{(baby.names, Name }\OperatorTok{==}\StringTok{ "Diana"}\NormalTok{)}
\end{Highlighting}
\end{Shaded}

\begin{Shaded}
\begin{Highlighting}[]
\KeywordTok{qplot}\NormalTok{(}\DataTypeTok{x =}\NormalTok{ Year, }\DataTypeTok{y =}\NormalTok{ Count,}
     \DataTypeTok{data =}\NormalTok{ diana)}
\end{Highlighting}
\end{Shaded}

Interetingly there are usually some gender-atypical names, even for very
strongly gendered names like ``Diana''. Splitting these trends out by
Sex is very easy:

\begin{Shaded}
\begin{Highlighting}[]
\KeywordTok{qplot}\NormalTok{(}\DataTypeTok{x =}\NormalTok{ Year, }\DataTypeTok{y =}\NormalTok{ Count, }\DataTypeTok{color =}\NormalTok{ Sex,}
      \DataTypeTok{data =}\NormalTok{ diana)}
\end{Highlighting}
\end{Shaded}

\section{Exercise 3: Plotting peak popularity of your
name}\label{exercise-3-plotting-peak-popularity-of-your-name}

Make sure the \texttt{ggplot2} package is installed, and that you have
attached it using \texttt{library(ggplot2)}.

\begin{enumerate}
\def\labelenumi{\arabic{enumi}.}
\tightlist
\item
  Use \texttt{filter} to extract data for your name (same as previous
  exercise)
\end{enumerate}

\begin{Shaded}
\begin{Highlighting}[]
\NormalTok{##}
\end{Highlighting}
\end{Shaded}

\begin{enumerate}
\def\labelenumi{\arabic{enumi}.}
\setcounter{enumi}{1}
\tightlist
\item
  Plot the data you produced in step 1 above, with \texttt{Year} on the
  x-axis and \texttt{Count} on the y-axis.
\end{enumerate}

\begin{Shaded}
\begin{Highlighting}[]
\NormalTok{##}
\end{Highlighting}
\end{Shaded}

\begin{enumerate}
\def\labelenumi{\arabic{enumi}.}
\setcounter{enumi}{2}
\tightlist
\item
  Adjust the plot so that is shows boys and girls in different colors.
\end{enumerate}

\begin{Shaded}
\begin{Highlighting}[]
\NormalTok{##}
\end{Highlighting}
\end{Shaded}

\begin{enumerate}
\def\labelenumi{\arabic{enumi}.}
\setcounter{enumi}{3}
\tightlist
\item
  BONUS (Optional): Adust the plot to use lines instead of points.
\end{enumerate}

\section{Exercise 3 solution}\label{exercise-3-solution}

\begin{Shaded}
\begin{Highlighting}[]
\CommentTok{# 1. Use `filter` to extract data for your name (same as previous exercise)  }
\end{Highlighting}
\end{Shaded}

\begin{Shaded}
\begin{Highlighting}[]
\NormalTok{george <-}\StringTok{ }\KeywordTok{filter}\NormalTok{(baby.names, Name }\OperatorTok{==}\StringTok{ "George"}\NormalTok{)}
\end{Highlighting}
\end{Shaded}

\begin{Shaded}
\begin{Highlighting}[]
\CommentTok{# 2.  Plot the data you produced in step 1 above, with `Year` on the x-axis}
\CommentTok{#     and `Count` on the y-axis.}
\end{Highlighting}
\end{Shaded}

\begin{Shaded}
\begin{Highlighting}[]
\KeywordTok{qplot}\NormalTok{(}\DataTypeTok{x =}\NormalTok{ Year, }\DataTypeTok{y =}\NormalTok{ Count, }\DataTypeTok{data =}\NormalTok{ george)}
\end{Highlighting}
\end{Shaded}

\begin{Shaded}
\begin{Highlighting}[]
\CommentTok{# 3. Adjust the plot so that is shows boys and girls in different colors.}
\end{Highlighting}
\end{Shaded}

\begin{Shaded}
\begin{Highlighting}[]
\KeywordTok{qplot}\NormalTok{(}\DataTypeTok{x =}\NormalTok{ Year, }\DataTypeTok{y =}\NormalTok{ Count, }\DataTypeTok{color =}\NormalTok{ Sex, }\DataTypeTok{data =}\NormalTok{ george)}
\end{Highlighting}
\end{Shaded}

\begin{Shaded}
\begin{Highlighting}[]
\CommentTok{# 4.  BONUS (Optional): Adust the plot to use lines instead of points.}
\end{Highlighting}
\end{Shaded}

\begin{Shaded}
\begin{Highlighting}[]
\KeywordTok{qplot}\NormalTok{(}\DataTypeTok{x =}\NormalTok{ Year, }\DataTypeTok{y =}\NormalTok{ Count, }\DataTypeTok{color =}\NormalTok{ Sex, }\DataTypeTok{data =}\NormalTok{ george, }\DataTypeTok{geom =} \StringTok{"line"}\NormalTok{)}
\end{Highlighting}
\end{Shaded}

\chapter{Finding the most popular
names}\label{finding-the-most-popular-names}

Our next goal is to find out which names have been the most popular.

\section{Computing better measures of
popularity}\label{computing-better-measures-of-popularity}

So far we've used \texttt{Count} as a measure of popularity. A better
approach is to use proportion or rank to avoid confounding popularity
with the number of babies born in a given year.

The \texttt{mutate} function makes it easy to add or modify the columns
of a \texttt{data.frame}. For example, we can use it compute the log of
the number of boys and girls given each name in each year:

\begin{Shaded}
\begin{Highlighting}[]
\NormalTok{baby.names <-}\StringTok{ }\KeywordTok{mutate}\NormalTok{(baby.names, }\DataTypeTok{logCount =}\NormalTok{ Count}\OperatorTok{/}\DecValTok{1000}\NormalTok{)}
\NormalTok{baby.names}
\end{Highlighting}
\end{Shaded}

\section{Operating by group}\label{operating-by-group}

Because of the nested nature of out data, we want to compute rank or
proportion within each \texttt{Sex} \texttt{X} \texttt{Year} group. The
\texttt{dplyr} package makes this relatively easy.

\begin{Shaded}
\begin{Highlighting}[]
\NormalTok{baby.names <-}\StringTok{ }\KeywordTok{mutate}\NormalTok{(}\KeywordTok{group_by}\NormalTok{(baby.names, Year, Sex),}
                     \DataTypeTok{Rank =} \KeywordTok{rank}\NormalTok{(Count))}
\end{Highlighting}
\end{Shaded}

Note that the data remains grouped until you change the groups by
running \texttt{group\_by} again or remove grouping information with
\texttt{ungroup}.

\section{Exercise 4: Most popular
names}\label{exercise-4-most-popular-names}

In this exercise your goal is to identify the most popular names for
each year.

\begin{enumerate}
\def\labelenumi{\arabic{enumi}.}
\tightlist
\item
  Use \texttt{mutate} and \texttt{group\_by} to create a column named
  ``Proportion'' where \texttt{Proportion\ =\ Count/sum(Count)} for each
  \texttt{Year\ X\ Sex} group.
\end{enumerate}

\begin{Shaded}
\begin{Highlighting}[]
\NormalTok{##}
\end{Highlighting}
\end{Shaded}

\begin{enumerate}
\def\labelenumi{\arabic{enumi}.}
\setcounter{enumi}{1}
\tightlist
\item
  Use \texttt{mutate} and \texttt{group\_by} to create a column named
  ``Rank'' where \texttt{Rank\ =\ rank(-Count)} for each
  \texttt{Year\ X\ Sex} group.
\end{enumerate}

\begin{Shaded}
\begin{Highlighting}[]
\NormalTok{##}
\end{Highlighting}
\end{Shaded}

\begin{enumerate}
\def\labelenumi{\arabic{enumi}.}
\setcounter{enumi}{2}
\tightlist
\item
  Filter the baby names data to display only the most popular name for
  each \texttt{Year\ X\ Sex} group.
\end{enumerate}

\begin{Shaded}
\begin{Highlighting}[]
\NormalTok{##}
\end{Highlighting}
\end{Shaded}

\begin{enumerate}
\def\labelenumi{\arabic{enumi}.}
\setcounter{enumi}{3}
\tightlist
\item
  Plot the data produced in step 4, putting \texttt{Year} on the x-axis
  and \texttt{Proportion} on the y-axis. How has the proportion of
  babies given the most popular name changed over time?
\end{enumerate}

\begin{Shaded}
\begin{Highlighting}[]
\NormalTok{##}
\end{Highlighting}
\end{Shaded}

\begin{enumerate}
\def\labelenumi{\arabic{enumi}.}
\setcounter{enumi}{4}
\tightlist
\item
  BONUS (optional): Which names are the most popular for both boys and
  girls?
\end{enumerate}

\section{Exercise 4 solution}\label{exercise-4-solution}

\begin{Shaded}
\begin{Highlighting}[]
\NormalTok{## 1.  Use `mutate` and `group_by` to create a column named "Proportion"}
\NormalTok{##     where `Proportion = Count/sum(Count)` for each `Year X Sex` group.}
\end{Highlighting}
\end{Shaded}

\begin{Shaded}
\begin{Highlighting}[]
\NormalTok{baby.names <-}\StringTok{ }\KeywordTok{mutate}\NormalTok{(}\KeywordTok{group_by}\NormalTok{(baby.names, Year, Sex),}
                     \DataTypeTok{Proportion =}\NormalTok{ Count}\OperatorTok{/}\KeywordTok{sum}\NormalTok{(Count))}
\end{Highlighting}
\end{Shaded}

\begin{Shaded}
\begin{Highlighting}[]
\NormalTok{## 2.  Use `mutate` and `group_by` to create a column named "Rank" where }
\NormalTok{##     `Rank = rank(-Count)` for each `Year X Sex` group.}
\end{Highlighting}
\end{Shaded}

\begin{Shaded}
\begin{Highlighting}[]
\NormalTok{baby.names <-}\StringTok{ }\KeywordTok{mutate}\NormalTok{(}\KeywordTok{group_by}\NormalTok{(baby.names, Year, Sex),}
                     \DataTypeTok{Rank =} \KeywordTok{rank}\NormalTok{(}\OperatorTok{-}\NormalTok{Count))}
\end{Highlighting}
\end{Shaded}

\begin{Shaded}
\begin{Highlighting}[]
\NormalTok{## 3.  Filter the baby names data to display only the most popular name }
\NormalTok{##     for each `Year X Sex` group.}
\end{Highlighting}
\end{Shaded}

\begin{Shaded}
\begin{Highlighting}[]
\NormalTok{top1 <-}\StringTok{ }\KeywordTok{filter}\NormalTok{(baby.names, Rank }\OperatorTok{==}\StringTok{ }\DecValTok{1}\NormalTok{)}
\end{Highlighting}
\end{Shaded}

\begin{Shaded}
\begin{Highlighting}[]
\NormalTok{## 4. Plot the data produced in step 3, putting `Year` on the x-axis}
\NormalTok{##    and `Proportion` on the y-axis. How has the proportion of babies}
\NormalTok{##    given the most popular name changed over time?}
\end{Highlighting}
\end{Shaded}

\begin{Shaded}
\begin{Highlighting}[]
\KeywordTok{qplot}\NormalTok{(}\DataTypeTok{x =}\NormalTok{ Year, }\DataTypeTok{y =}\NormalTok{ Proportion, }\DataTypeTok{color =}\NormalTok{ Sex, }
      \DataTypeTok{data =}\NormalTok{ top1, }
      \DataTypeTok{geom =} \StringTok{"line"}\NormalTok{)}
\end{Highlighting}
\end{Shaded}

\begin{Shaded}
\begin{Highlighting}[]
\NormalTok{## 5. BONUS (optional): Which names are the most popular for both boys }
\NormalTok{##    and girls?}
\end{Highlighting}
\end{Shaded}

\begin{Shaded}
\begin{Highlighting}[]
\NormalTok{girls.and.boys <-}\StringTok{ }\KeywordTok{inner_join}\NormalTok{(}\KeywordTok{filter}\NormalTok{(baby.names, Sex }\OperatorTok{==}\StringTok{ "Boys"}\NormalTok{), }
                             \KeywordTok{filter}\NormalTok{(baby.names, Sex }\OperatorTok{==}\StringTok{ "Girls"}\NormalTok{),}
                             \DataTypeTok{by =} \KeywordTok{c}\NormalTok{(}\StringTok{"Year"}\NormalTok{, }\StringTok{"Name"}\NormalTok{))}
\end{Highlighting}
\end{Shaded}

\begin{Shaded}
\begin{Highlighting}[]
\NormalTok{girls.and.boys <-}\StringTok{ }\KeywordTok{mutate}\NormalTok{(girls.and.boys,}
                         \DataTypeTok{Product =}\NormalTok{ Count.x }\OperatorTok{*}\StringTok{ }\NormalTok{Count.y,}
                         \DataTypeTok{Rank =} \KeywordTok{rank}\NormalTok{(}\OperatorTok{-}\NormalTok{Product))}
\end{Highlighting}
\end{Shaded}

\begin{Shaded}
\begin{Highlighting}[]
\KeywordTok{filter}\NormalTok{(girls.and.boys, Rank }\OperatorTok{==}\StringTok{ }\DecValTok{1}\NormalTok{)}
\end{Highlighting}
\end{Shaded}

\chapter{Percent choosing one of the top 10
names}\label{percent-choosing-one-of-the-top-10-names}

You may have noticed that the percentage of babies given the most
popular name of the year appears to have decreases over time. We can
compute a more robust measure of the popularity of the most popular
names by calculating the number of babies given one of the top 10 girl
or boy names of the year.

In order to compute this measure we need to operate within goups, as we
did using \texttt{mutate} above, but this time we need to collapse each
group into a single summary statistic. We can achive this using the
\texttt{summarize} function. For example, we can calculate the number of
babies born each year:

\begin{Shaded}
\begin{Highlighting}[]
\NormalTok{bn.by.year <-}\StringTok{ }\KeywordTok{summarize}\NormalTok{(}\KeywordTok{group_by}\NormalTok{(baby.names, Year),}
                       \DataTypeTok{Total =} \KeywordTok{sum}\NormalTok{(Count))}
\NormalTok{bn.by.year}
\end{Highlighting}
\end{Shaded}

\section{Exercise 4: Popularity of the most popular
names}\label{exercise-4-popularity-of-the-most-popular-names}

In this exercise we will plot trends in the proportion of boys and girls
given one of the 10 most popular names each year.

\begin{enumerate}
\def\labelenumi{\arabic{enumi}.}
\tightlist
\item
  Filter the baby.names data, retaining only the 10 most popular girl
  and boy names for each year.
\end{enumerate}

\begin{Shaded}
\begin{Highlighting}[]
\NormalTok{##}
\end{Highlighting}
\end{Shaded}

\begin{enumerate}
\def\labelenumi{\arabic{enumi}.}
\setcounter{enumi}{1}
\tightlist
\item
  Summarize the data produced in step one to calculate the total
  Proportion of boys and girls given one of the top 10 names each year.
\end{enumerate}

\begin{Shaded}
\begin{Highlighting}[]
\NormalTok{##}
\end{Highlighting}
\end{Shaded}

\begin{enumerate}
\def\labelenumi{\arabic{enumi}.}
\setcounter{enumi}{2}
\tightlist
\item
  Plot the data produced in step 2, with year on the x-axis and total
  proportion on the y axis. Color by sex.
\end{enumerate}

\begin{Shaded}
\begin{Highlighting}[]
\NormalTok{##}
\end{Highlighting}
\end{Shaded}

\section{Exercise 4 solution}\label{exercise-4-solution-1}

\begin{Shaded}
\begin{Highlighting}[]
\NormalTok{## 1.  Filter the baby.names data, retaining only the 10 most }
\NormalTok{##     popular girl and boy names for each year.}
\end{Highlighting}
\end{Shaded}

\begin{Shaded}
\begin{Highlighting}[]
\NormalTok{most.popular <-}\StringTok{ }\KeywordTok{filter}\NormalTok{(}\KeywordTok{group_by}\NormalTok{(baby.names, Year, Sex),}
\NormalTok{                       Rank }\OperatorTok{<=}\StringTok{ }\DecValTok{10}\NormalTok{)}
\end{Highlighting}
\end{Shaded}

\begin{Shaded}
\begin{Highlighting}[]
\NormalTok{## 2.  Summarize the data produced in step one to calculate the total}
\NormalTok{##     Proportion of boys and girls given one of the top 10 names}
\NormalTok{##     each year.}
\end{Highlighting}
\end{Shaded}

\begin{Shaded}
\begin{Highlighting}[]
\NormalTok{top10 <-}\StringTok{ }\KeywordTok{summarize}\NormalTok{(}\KeywordTok{group_by}\NormalTok{(most.popular, Year, Sex),}
                   \DataTypeTok{TotalProportion =} \KeywordTok{sum}\NormalTok{(Proportion))}
\end{Highlighting}
\end{Shaded}

\begin{Shaded}
\begin{Highlighting}[]
\NormalTok{## 3.  Plot the data produced in step 2, with year on the x-axis}
\NormalTok{##     and total proportion on the y axis. Color by sex.}
\end{Highlighting}
\end{Shaded}

\begin{Shaded}
\begin{Highlighting}[]
\KeywordTok{qplot}\NormalTok{(}\DataTypeTok{x =}\NormalTok{ Year, }\DataTypeTok{y =}\NormalTok{ TotalProportion, }\DataTypeTok{color =}\NormalTok{ Sex,}
      \DataTypeTok{data =}\NormalTok{ top10,}
      \DataTypeTok{geom =} \StringTok{"line"}\NormalTok{)}
\end{Highlighting}
\end{Shaded}

\chapter{Saving our Work}\label{saving-our-work}

Now that we have made some changes to our data set, we might want to
save those changes to a file.

\section{Saving individual datasets}\label{saving-individual-datasets}

\begin{Shaded}
\begin{Highlighting}[]
\CommentTok{# write data to a .csv file}
\KeywordTok{write_csv}\NormalTok{(baby.names, }\StringTok{"babyNames.csv"}\NormalTok{)}
\end{Highlighting}
\end{Shaded}

\begin{Shaded}
\begin{Highlighting}[]
\CommentTok{# write data to an R file}
\KeywordTok{write_rds}\NormalTok{(baby.names, }\StringTok{"babyNames.rds"}\NormalTok{)}
\end{Highlighting}
\end{Shaded}

\section{Saving and loading R
workspaces}\label{saving-and-loading-r-workspaces}

In addition to importing individual datasets, R can save and load entire
workspaces

\begin{Shaded}
\begin{Highlighting}[]
\KeywordTok{ls}\NormalTok{() }\CommentTok{# list objects in our workspace}
\KeywordTok{save.image}\NormalTok{(}\DataTypeTok{file=}\StringTok{"myWorkspace.RData"}\NormalTok{) }\CommentTok{# save workspace }
\KeywordTok{rm}\NormalTok{(}\DataTypeTok{list=}\KeywordTok{ls}\NormalTok{()) }\CommentTok{# remove all objects from our workspace }
\KeywordTok{ls}\NormalTok{() }\CommentTok{# list stored objects to make sure they are deleted}
\end{Highlighting}
\end{Shaded}

\begin{Shaded}
\begin{Highlighting}[]
\NormalTok{## Load the "myWorkspace.RData" file and check that it is restored}
\KeywordTok{load}\NormalTok{(}\StringTok{"myWorkspace.RData"}\NormalTok{) }\CommentTok{# load myWorkspace.RData}
\KeywordTok{ls}\NormalTok{() }\CommentTok{# list objects}
\end{Highlighting}
\end{Shaded}

\chapter{Wrap-up}\label{wrap-up}

\section{Help us make this workshop
better!}\label{help-us-make-this-workshop-better}

Please take a moment to fill out a very short feedback form. These
workshops exist for you -- tell us what you need!
\url{http://tinyurl.com/R-intro-feedback}

\section{Additional resources}\label{additional-resources}

\begin{itemize}
\tightlist
\item
  IQSS workshops:
  \url{http://projects.iq.harvard.edu/rtc/filter_by/workshops}
\item
  IQSS statistical consulting: \url{http://dss.iq.harvard.edu}
\item
  Software (all free!):

  \begin{itemize}
  \tightlist
  \item
    R and R package download: \url{http://cran.r-project.org}
  \item
    Rstudio download: \url{http://rstudio.org}
  \item
    ESS (emacs R package): \url{http://ess.r-project.org/}
  \end{itemize}
\item
  Online tutorials

  \begin{itemize}
  \tightlist
  \item
    \url{http://www.codeschool.com/courses/try-r}
  \item
    \url{http://www.datacamp.org}
  \item
    \url{http://swirlstats.com/}
  \item
    \url{http://r4ds.had.co.nz/}
  \end{itemize}
\item
  Getting help:

  \begin{itemize}
  \tightlist
  \item
    Documentation and tutorials:
    \url{http://cran.r-project.org/other-docs.html}
  \item
    Recommended R packages by topic:
    \url{http://cran.r-project.org/web/views/}
  \item
    Mailing list: \url{https://stat.ethz.ch/mailman/listinfo/r-help}
  \item
    StackOverflow: \url{http://stackoverflow.com/questions/tagged/r}
  \end{itemize}
\item
  Coming from\ldots{} Stata :
  \url{http://www.princeton.edu/~otorres/RStata.pdf} SAS/SPSS :
  \url{http://www.et.bs.ehu.es/~etptupaf/pub/R/RforSAS\&SPSSusers.pdf}
  matlab : \url{http://www.math.umaine.edu/~hiebeler/comp/matlabR.pdf}
  Python :
  \url{http://mathesaurus.sourceforge.net/matlab-python-xref.pdf}
\end{itemize}


\end{document}
