\documentclass[]{book}
\usepackage{lmodern}
\usepackage{amssymb,amsmath}
\usepackage{ifxetex,ifluatex}
\usepackage{fixltx2e} % provides \textsubscript
\ifnum 0\ifxetex 1\fi\ifluatex 1\fi=0 % if pdftex
  \usepackage[T1]{fontenc}
  \usepackage[utf8]{inputenc}
\else % if luatex or xelatex
  \ifxetex
    \usepackage{mathspec}
  \else
    \usepackage{fontspec}
  \fi
  \defaultfontfeatures{Ligatures=TeX,Scale=MatchLowercase}
\fi
% use upquote if available, for straight quotes in verbatim environments
\IfFileExists{upquote.sty}{\usepackage{upquote}}{}
% use microtype if available
\IfFileExists{microtype.sty}{%
\usepackage{microtype}
\UseMicrotypeSet[protrusion]{basicmath} % disable protrusion for tt fonts
}{}
\usepackage[margin=1.5in]{geometry}
\usepackage{hyperref}
\hypersetup{unicode=true,
            pdftitle={Data Science Workshops},
            pdfborder={0 0 0},
            breaklinks=true}
\urlstyle{same}  % don't use monospace font for urls
\usepackage{natbib}
\bibliographystyle{apalike}
\usepackage{color}
\usepackage{fancyvrb}
\newcommand{\VerbBar}{|}
\newcommand{\VERB}{\Verb[commandchars=\\\{\}]}
\DefineVerbatimEnvironment{Highlighting}{Verbatim}{commandchars=\\\{\}}
% Add ',fontsize=\small' for more characters per line
\usepackage{framed}
\definecolor{shadecolor}{RGB}{248,248,248}
\newenvironment{Shaded}{\begin{snugshade}}{\end{snugshade}}
\newcommand{\KeywordTok}[1]{\textcolor[rgb]{0.13,0.29,0.53}{\textbf{#1}}}
\newcommand{\DataTypeTok}[1]{\textcolor[rgb]{0.13,0.29,0.53}{#1}}
\newcommand{\DecValTok}[1]{\textcolor[rgb]{0.00,0.00,0.81}{#1}}
\newcommand{\BaseNTok}[1]{\textcolor[rgb]{0.00,0.00,0.81}{#1}}
\newcommand{\FloatTok}[1]{\textcolor[rgb]{0.00,0.00,0.81}{#1}}
\newcommand{\ConstantTok}[1]{\textcolor[rgb]{0.00,0.00,0.00}{#1}}
\newcommand{\CharTok}[1]{\textcolor[rgb]{0.31,0.60,0.02}{#1}}
\newcommand{\SpecialCharTok}[1]{\textcolor[rgb]{0.00,0.00,0.00}{#1}}
\newcommand{\StringTok}[1]{\textcolor[rgb]{0.31,0.60,0.02}{#1}}
\newcommand{\VerbatimStringTok}[1]{\textcolor[rgb]{0.31,0.60,0.02}{#1}}
\newcommand{\SpecialStringTok}[1]{\textcolor[rgb]{0.31,0.60,0.02}{#1}}
\newcommand{\ImportTok}[1]{#1}
\newcommand{\CommentTok}[1]{\textcolor[rgb]{0.56,0.35,0.01}{\textit{#1}}}
\newcommand{\DocumentationTok}[1]{\textcolor[rgb]{0.56,0.35,0.01}{\textbf{\textit{#1}}}}
\newcommand{\AnnotationTok}[1]{\textcolor[rgb]{0.56,0.35,0.01}{\textbf{\textit{#1}}}}
\newcommand{\CommentVarTok}[1]{\textcolor[rgb]{0.56,0.35,0.01}{\textbf{\textit{#1}}}}
\newcommand{\OtherTok}[1]{\textcolor[rgb]{0.56,0.35,0.01}{#1}}
\newcommand{\FunctionTok}[1]{\textcolor[rgb]{0.00,0.00,0.00}{#1}}
\newcommand{\VariableTok}[1]{\textcolor[rgb]{0.00,0.00,0.00}{#1}}
\newcommand{\ControlFlowTok}[1]{\textcolor[rgb]{0.13,0.29,0.53}{\textbf{#1}}}
\newcommand{\OperatorTok}[1]{\textcolor[rgb]{0.81,0.36,0.00}{\textbf{#1}}}
\newcommand{\BuiltInTok}[1]{#1}
\newcommand{\ExtensionTok}[1]{#1}
\newcommand{\PreprocessorTok}[1]{\textcolor[rgb]{0.56,0.35,0.01}{\textit{#1}}}
\newcommand{\AttributeTok}[1]{\textcolor[rgb]{0.77,0.63,0.00}{#1}}
\newcommand{\RegionMarkerTok}[1]{#1}
\newcommand{\InformationTok}[1]{\textcolor[rgb]{0.56,0.35,0.01}{\textbf{\textit{#1}}}}
\newcommand{\WarningTok}[1]{\textcolor[rgb]{0.56,0.35,0.01}{\textbf{\textit{#1}}}}
\newcommand{\AlertTok}[1]{\textcolor[rgb]{0.94,0.16,0.16}{#1}}
\newcommand{\ErrorTok}[1]{\textcolor[rgb]{0.64,0.00,0.00}{\textbf{#1}}}
\newcommand{\NormalTok}[1]{#1}
\usepackage{longtable,booktabs}
\usepackage{graphicx,grffile}
\makeatletter
\def\maxwidth{\ifdim\Gin@nat@width>\linewidth\linewidth\else\Gin@nat@width\fi}
\def\maxheight{\ifdim\Gin@nat@height>\textheight\textheight\else\Gin@nat@height\fi}
\makeatother
% Scale images if necessary, so that they will not overflow the page
% margins by default, and it is still possible to overwrite the defaults
% using explicit options in \includegraphics[width, height, ...]{}
\setkeys{Gin}{width=\maxwidth,height=\maxheight,keepaspectratio}
\IfFileExists{parskip.sty}{%
\usepackage{parskip}
}{% else
\setlength{\parindent}{0pt}
\setlength{\parskip}{6pt plus 2pt minus 1pt}
}
\setlength{\emergencystretch}{3em}  % prevent overfull lines
\providecommand{\tightlist}{%
  \setlength{\itemsep}{0pt}\setlength{\parskip}{0pt}}
\setcounter{secnumdepth}{5}
% Redefines (sub)paragraphs to behave more like sections
\ifx\paragraph\undefined\else
\let\oldparagraph\paragraph
\renewcommand{\paragraph}[1]{\oldparagraph{#1}\mbox{}}
\fi
\ifx\subparagraph\undefined\else
\let\oldsubparagraph\subparagraph
\renewcommand{\subparagraph}[1]{\oldsubparagraph{#1}\mbox{}}
\fi

%%% Use protect on footnotes to avoid problems with footnotes in titles
\let\rmarkdownfootnote\footnote%
\def\footnote{\protect\rmarkdownfootnote}

%%% Change title format to be more compact
\usepackage{titling}

% Create subtitle command for use in maketitle
\providecommand{\subtitle}[1]{
  \posttitle{
    \begin{center}\large#1\end{center}
    }
}

\setlength{\droptitle}{-2em}

  \title{Data Science Workshops}
    \pretitle{\vspace{\droptitle}\centering\huge}
  \posttitle{\par}
    \author{}
    \preauthor{}\postauthor{}
      \predate{\centering\large\emph}
  \postdate{\par}
    \date{October 2019}

\usepackage{booktabs}

\usepackage{graphicx}
\usepackage{hyperref}
\usepackage{caption}
\usepackage[hang,flushmargin]{footmisc}
\usepackage{framed}
\usepackage{xcolor}
\usepackage{verbatim} 

% Title format
\usepackage{titling}
\pretitle{\Huge\sffamily}
\posttitle{\par\vskip 0.5em}
\predate{\LARGE\sffamily}
\postdate{\par}

\urlstyle{tt}

\graphicspath{{R/Rintro/}{R/Rmodels/}{R/Rgraphics/}{R/RDataWrangling/}}

\begin{document}
\maketitle

{
\setcounter{tocdepth}{1}
\tableofcontents
}
\chapter*{Introduction}\label{introduction}
\addcontentsline{toc}{chapter}{Introduction}

\section*{Table of Contents}\label{table-of-contents}
\addcontentsline{toc}{section}{Table of Contents}

Materials for the software workshops held at the
\href{http://iq.harvard.edu}{Institute for Quantitative Social Science}
and \href{https://training.rcs.hbs.org}{Harvard Business School} at
Harvard.

\begin{enumerate}
\def\labelenumi{\arabic{enumi}.}
\tightlist
\item
  \href{./DataScienceTools.html}{Data Science Tools}
\item
  \href{./Rintro.html}{R Introduction}
\item
  \href{./Rmodels.html}{R Regression Models}
\item
  \href{./Rgraphics.html}{R Graphics}
\item
  \href{./RDataWrangling.html}{R Data Wrangling}
\item
  \href{./PythonIntro.html}{Python Introduction}
\item
  \href{./PythonWebScrape.html}{Python Web-Scraping}
\item
  \href{./StataIntro.html}{Stata Introduction}
\item
  \href{./StataDatMan.html}{Stata Data Management}
\item
  \href{./StataModGraph.html}{Stata Modeling \& Graphing}
\end{enumerate}

These workshops are a work-in-progress, please provide feedback! Email:
\href{mailto:help@iq.harvard.edu}{\nolinkurl{help@iq.harvard.edu}}

\section*{Authors and Sources}\label{authors-and-sources}
\addcontentsline{toc}{section}{Authors and Sources}

The contents of these workshops are the result of a collaborative effort
from members of the \href{http://dss.iq.harvard.edu}{Data Science
Services} team at IQSS and the
\href{https://training.rcs.hbs.org}{Research Computing Services} team at
HBS. The main contributors are: Ista Zahn, Steve Worthington, Bob
Freeman, Jinjie Liu, Yihan Wang, and Victoria Liublinska.

\part{General}\label{part-general}

\chapter{Data Science Tools}\label{data-science-tools}

\textbf{Topics}

\begin{itemize}
\tightlist
\item
  Data science tool selection
\item
  Data analysis pipelines
\item
  Programming languages comparison
\item
  Text editor and IDE comparison
\item
  Tools for creating reports
\end{itemize}

\section{Tools for working with data}\label{tools-for-working-with-data}

Working with data effectively requires learning at least one programming
or scripting language. You can get by without this, but it would be like
trying to cook with only a butter knife; not recommended! Compared to
using a menu-driven interface (e.g., SPSS or SAS) or a spreadsheet
(e.g., Excel), using a programming language allows you to:

\begin{itemize}
\tightlist
\item
  reproduce results,
\item
  correct errors and update output,
\item
  reuse code,
\item
  collaborate with others,
\item
  automate repetitive tasks, and
\item
  generate manuscripts, reports, and other documents from your code.
\end{itemize}

So, you need to learn a programming language for working with data, but
which one should you learn? Since you'll be writing code you'll want to
set up a comfortable environment for writing and editing that code.
Which text editors are good for this? You'll probably also want to learn
at least one markup language (e.g., LaTeX, Markdown) so that you can
create reproducible manuscripts. What tools are good for this? These
questions will guide our discussion, the goal of which is to help you
decide which tools you should invest time in learning.

\section{The puzzle pieces}\label{the-puzzle-pieces}

As we've noted, working effectively with data requires using a number of
tools.

\subsection{Data analysis building
blocks}\label{data-analysis-building-blocks}

The basic pieces are:

\begin{itemize}
\tightlist
\item
  a data storage and retrieval system,
\item
  an editor for writing code,
\item
  an interpreter or compiler for executing that code,
\item
  a system for presenting results, and
\item
  some ``glue'' to make all the pieces work together.
\end{itemize}

\section{Examples}\label{examples}

Before looking in detail at each of these building blocks we'll look at
a few examples to get an intuitive feel for the basic elements.

\subsection{Old-school example}\label{old-school-example}

In this example we're going to process data in a text file in a way that
would be familiar to a statistician working forty years ago.
Surprisingly, it's not much different from the way we would do it today.
Programs come and go, but the basic ideas remain pretty much the same!

Specifically, we'll process the data in \texttt{1980\_census.txt} by
writing \textbf{fortran} code in the \textbf{vi} text editor and running
it through the \textbf{fortran} compiler. Then we'll take the results
and put them in to a \textbf{TeX} file, again using the \textbf{vi}
editor to create the report. For ``glue'' we will use a terminal
emulator running the bash shell. All of these tools were available in
1980, though some features have been added since that time.

OLD SCHOOL DEMO:

\begin{longtable}[]{@{}llllll@{}}
\toprule
example & data storage & editor & program & report tool &
glue\tabularnewline
\midrule
\endhead
old school & ASCII text file & vi & fortran & TeX & Bourne (compatable)
shell\tabularnewline
\bottomrule
\end{longtable}

\subsection{Something old \& something
new}\label{something-old-something-new}

Next we're going to do the same basic process, this time using a modern
text editor (\textbf{Atom}), a different programming language
(\textbf{Python}), and a modern report generation system (\textbf{LaTeX}
processed via \textbf{xelatex}). For the glue we're still going to use a
shell.

OLD AND NEW DEMO:

\begin{longtable}[]{@{}llllll@{}}
\toprule
example & data storage & editor & program & report tool &
glue\tabularnewline
\midrule
\endhead
old school & ASCII text file & vi & fortran & TeX & Bourne (compatable)
shell\tabularnewline
old and new & ASCII text file & Atom & python & LaTeX & Bash
shell\tabularnewline
\bottomrule
\end{longtable}

\subsection{A modern version}\label{a-modern-version}

Finally, we'll produce the same report using modern tools. Remember, the
process is basically the same: we're just using different tools.

MODERN DEMO:

\begin{longtable}[]{@{}llllll@{}}
\toprule
example & data storage & editor & program & report tool &
glue\tabularnewline
\midrule
\endhead
old school & ASCII text file & vi & fortran & TeX & Bourne (compatable)
shell\tabularnewline
old and new & ASCII text file & Atom & python & LaTeX & Bash
shell\tabularnewline
modern & SQLite database & Rstudio & R & R Markdown &
Rstudio\tabularnewline
\bottomrule
\end{longtable}

\section{Data storage \& retrieval}\label{data-storage-retrieval}

Data storage and retrieval is a fairly dry topic, so we won't spend too
much time on it. There are roughly four types of technology for storing
and retrieving data.

\subsection{Text files}\label{text-files}

Storing data in text files (e.g., comma separated values, other
delimited text formats) is simple and makes the data easy to access from
just about any program. It is also good for archiving data since no
specialized software is needed to read it. The main downsides are that
retrieval is slow and often all-or-nothing, and the fact that storing
metadata in plain text files is cumbersome.

\subsection{Binary files}\label{binary-files}

Many statistics packages and programming languages have a ``native''
binary data storage format. For example, Stata stores data in
\texttt{.dta} files, and R stores data in \texttt{.rds} or
\texttt{.Rdata} files. These storage formats usually more efficient than
text files, and usually provide faster read/write access. They usually
include a mechanism for storing metadata. The down side is that
specialized software is required to read them (will Stata exist in 50
years? Are you sure?) and the ability to read them using other programs
may be limited.

\subsection{Databases}\label{databases}

Storing data in a database requires more up-front planning and set up,
but has several advantages. Databases provide fast selective retrieval
and facilitate efficient storage and flexible retrieval.

\subsection{Distributed file storage}\label{distributed-file-storage}

Data that is too large to fit on a single hard drive may be stored and
analyzed on a distributed file system or database such as the
\emph{Hadoop Distributed File System} or \emph{Cassandra}. When working
with data on this scale considerable infrastructure and specialized
tools will be required.

\section{Programming languages \& statistics
packages}\label{programming-languages-statistics-packages}

There are tens of programs for statistics and data science available.
Here we will focus only on the more popular programs that offer a wide
range of features. Note that for specific applications a specialized
program may be better, e.g., many people use Mplus for structural
equation models and another program for everything else.

\subsection{Programming language
features}\label{programming-language-features}

Things we want a statistics program to do include:

\begin{itemize}
\tightlist
\item
  read/write data from/to a variety of data storage systems,
\item
  manipulate data,
\item
  perform statistical methods,
\item
  visualize data and results,
\item
  export results in a variety of formats,
\item
  be easy to use,
\item
  be well documented,
\item
  have a large user community.
\end{itemize}

Note that this list is deceptively simple; each item may include a
diversity of complicated features. For example,``read/write data from/to
a variety of data storage systems'' may include reading from databases,
image files, .pdf files, .html and .xml files from a website, and any
number of proprietary data storage formats.

\subsection{Program comparison}\label{program-comparison}

\begin{longtable}[]{@{}lllllll@{}}
\toprule
\begin{minipage}[b]{0.08\columnwidth}\raggedright\strut
Program\strut
\end{minipage} & \begin{minipage}[b]{0.11\columnwidth}\raggedright\strut
Statistics\strut
\end{minipage} & \begin{minipage}[b]{0.13\columnwidth}\raggedright\strut
Visualization\strut
\end{minipage} & \begin{minipage}[b]{0.16\columnwidth}\raggedright\strut
Machine learning\strut
\end{minipage} & \begin{minipage}[b]{0.11\columnwidth}\raggedright\strut
Ease of use\strut
\end{minipage} & \begin{minipage}[b]{0.17\columnwidth}\raggedright\strut
Power/flexibility\strut
\end{minipage} & \begin{minipage}[b]{0.05\columnwidth}\raggedright\strut
Fun\strut
\end{minipage}\tabularnewline
\midrule
\endhead
\begin{minipage}[t]{0.08\columnwidth}\raggedright\strut
Stata\strut
\end{minipage} & \begin{minipage}[t]{0.11\columnwidth}\raggedright\strut
Good\strut
\end{minipage} & \begin{minipage}[t]{0.13\columnwidth}\raggedright\strut
Servicable\strut
\end{minipage} & \begin{minipage}[t]{0.16\columnwidth}\raggedright\strut
Limited\strut
\end{minipage} & \begin{minipage}[t]{0.11\columnwidth}\raggedright\strut
Very easy\strut
\end{minipage} & \begin{minipage}[t]{0.17\columnwidth}\raggedright\strut
Low\strut
\end{minipage} & \begin{minipage}[t]{0.05\columnwidth}\raggedright\strut
Some\strut
\end{minipage}\tabularnewline
\begin{minipage}[t]{0.08\columnwidth}\raggedright\strut
SPSS\strut
\end{minipage} & \begin{minipage}[t]{0.11\columnwidth}\raggedright\strut
OK\strut
\end{minipage} & \begin{minipage}[t]{0.13\columnwidth}\raggedright\strut
Servicable\strut
\end{minipage} & \begin{minipage}[t]{0.16\columnwidth}\raggedright\strut
Limited\strut
\end{minipage} & \begin{minipage}[t]{0.11\columnwidth}\raggedright\strut
Easy\strut
\end{minipage} & \begin{minipage}[t]{0.17\columnwidth}\raggedright\strut
Low\strut
\end{minipage} & \begin{minipage}[t]{0.05\columnwidth}\raggedright\strut
None\strut
\end{minipage}\tabularnewline
\begin{minipage}[t]{0.08\columnwidth}\raggedright\strut
SAS\strut
\end{minipage} & \begin{minipage}[t]{0.11\columnwidth}\raggedright\strut
Good\strut
\end{minipage} & \begin{minipage}[t]{0.13\columnwidth}\raggedright\strut
Not great\strut
\end{minipage} & \begin{minipage}[t]{0.16\columnwidth}\raggedright\strut
Good\strut
\end{minipage} & \begin{minipage}[t]{0.11\columnwidth}\raggedright\strut
Moderate\strut
\end{minipage} & \begin{minipage}[t]{0.17\columnwidth}\raggedright\strut
Moderate\strut
\end{minipage} & \begin{minipage}[t]{0.05\columnwidth}\raggedright\strut
None\strut
\end{minipage}\tabularnewline
\begin{minipage}[t]{0.08\columnwidth}\raggedright\strut
Matlab\strut
\end{minipage} & \begin{minipage}[t]{0.11\columnwidth}\raggedright\strut
Good\strut
\end{minipage} & \begin{minipage}[t]{0.13\columnwidth}\raggedright\strut
Good\strut
\end{minipage} & \begin{minipage}[t]{0.16\columnwidth}\raggedright\strut
Good\strut
\end{minipage} & \begin{minipage}[t]{0.11\columnwidth}\raggedright\strut
Moderate\strut
\end{minipage} & \begin{minipage}[t]{0.17\columnwidth}\raggedright\strut
Good\strut
\end{minipage} & \begin{minipage}[t]{0.05\columnwidth}\raggedright\strut
Some\strut
\end{minipage}\tabularnewline
\begin{minipage}[t]{0.08\columnwidth}\raggedright\strut
R\strut
\end{minipage} & \begin{minipage}[t]{0.11\columnwidth}\raggedright\strut
Excellent\strut
\end{minipage} & \begin{minipage}[t]{0.13\columnwidth}\raggedright\strut
Excellent\strut
\end{minipage} & \begin{minipage}[t]{0.16\columnwidth}\raggedright\strut
Good\strut
\end{minipage} & \begin{minipage}[t]{0.11\columnwidth}\raggedright\strut
Moderate\strut
\end{minipage} & \begin{minipage}[t]{0.17\columnwidth}\raggedright\strut
Excellent\strut
\end{minipage} & \begin{minipage}[t]{0.05\columnwidth}\raggedright\strut
Yes\strut
\end{minipage}\tabularnewline
\begin{minipage}[t]{0.08\columnwidth}\raggedright\strut
Python\strut
\end{minipage} & \begin{minipage}[t]{0.11\columnwidth}\raggedright\strut
Good\strut
\end{minipage} & \begin{minipage}[t]{0.13\columnwidth}\raggedright\strut
Good\strut
\end{minipage} & \begin{minipage}[t]{0.16\columnwidth}\raggedright\strut
Excellent\strut
\end{minipage} & \begin{minipage}[t]{0.11\columnwidth}\raggedright\strut
Moderate\strut
\end{minipage} & \begin{minipage}[t]{0.17\columnwidth}\raggedright\strut
Excellent\strut
\end{minipage} & \begin{minipage}[t]{0.05\columnwidth}\raggedright\strut
Yes\strut
\end{minipage}\tabularnewline
\begin{minipage}[t]{0.08\columnwidth}\raggedright\strut
Julia\strut
\end{minipage} & \begin{minipage}[t]{0.11\columnwidth}\raggedright\strut
OK\strut
\end{minipage} & \begin{minipage}[t]{0.13\columnwidth}\raggedright\strut
Excellent\strut
\end{minipage} & \begin{minipage}[t]{0.16\columnwidth}\raggedright\strut
Good\strut
\end{minipage} & \begin{minipage}[t]{0.11\columnwidth}\raggedright\strut
Hard\strut
\end{minipage} & \begin{minipage}[t]{0.17\columnwidth}\raggedright\strut
Excellent\strut
\end{minipage} & \begin{minipage}[t]{0.05\columnwidth}\raggedright\strut
Yes\strut
\end{minipage}\tabularnewline
\bottomrule
\end{longtable}

\subsection{Examples: Read data from a file \&
summarize}\label{examples-read-data-from-a-file-summarize}

In this example we will compare the syntax for reading and summarizing
data stored in a file.

\begin{itemize}
\tightlist
\item
  Stata
\end{itemize}

\begin{verbatim}
import delimited using "http://tutorials.iq.harvard.edu/R/Rgraphics/dataSets/EconomistData.csv"
sum
\end{verbatim}

\begin{verbatim}
set more off
 "EconomistData.csv"
Picked up _JAVA_OPTIONS: -Dawt.useSystemAAFontSettings=gasp -Dswing.aatext=true -Dsun.java2d.opengl=true
(6 vars, 173 obs)
sum

    Variable |        Obs        Mean    Std. Dev.       Min        Max
-------------+---------------------------------------------------------
          v1 |        173          87    50.08493          1        173
     country |          0
     hdirank |        173    95.28324    55.00767          1        187
         hdi |        173    .6580867    .1755888       .286       .943
         cpi |        173    4.052023    2.116782        1.5        9.5
-------------+---------------------------------------------------------
      region |          0
\end{verbatim}

\begin{itemize}
\tightlist
\item
  R
\end{itemize}

\begin{Shaded}
\begin{Highlighting}[]
\NormalTok{cpi <-}\StringTok{ }\KeywordTok{read.csv}\NormalTok{(}\StringTok{"http://tutorials.iq.harvard.edu/R/Rgraphics/dataSets/EconomistData.csv"}\NormalTok{)}
\KeywordTok{summary}\NormalTok{(cpi)}
\end{Highlighting}
\end{Shaded}

\begin{verbatim}
      X              Country       HDI.Rank           HDI        
Min.   :  1   Afghanistan:  1   Min.   :  1.00   Min.   :0.2860  
1st Qu.: 44   Albania    :  1   1st Qu.: 47.00   1st Qu.:0.5090  
Median : 87   Algeria    :  1   Median : 96.00   Median :0.6980  
Mean   : 87   Angola     :  1   Mean   : 95.28   Mean   :0.6581  
3rd Qu.:130   Argentina  :  1   3rd Qu.:143.00   3rd Qu.:0.7930  
Max.   :173   Armenia    :  1   Max.   :187.00   Max.   :0.9430  
              (Other)    :167                                    
     CPI                      Region  
Min.   :1.500   Americas         :31  
1st Qu.:2.500   Asia Pacific     :30  
Median :3.200   East EU Cemt Asia:18  
Mean   :4.052   EU W. Europe     :30  
3rd Qu.:5.100   MENA             :18  
Max.   :9.500   SSA              :46
\end{verbatim}

\begin{itemize}
\tightlist
\item
  Matlab
\end{itemize}

\begin{Shaded}
\begin{Highlighting}[]
\NormalTok{tmpfile = websave(tempname(), }\StringTok{'http://tutorials.iq.harvard.edu/R/Rgraphics/dataSets/EconomistData.csv'}\NormalTok{);}
\NormalTok{cpi = readtable(tmpfile);}
\NormalTok{summary(cpi)}
\end{Highlighting}
\end{Shaded}

\begin{verbatim}
tmpfile = websave(tempname(), 'http://tutorials.iq.harvard.edu/R/Rgraphics/dataSets/EconomistData.csv');
cpi = readtable(tmpfile);
summary(cpi)

Variables:

    Var1: 173×1 cell array of character vectors

    Country: 173×1 cell array of character vectors

    HDI_Rank: 173×1 double

        Description:  Original column heading: 'HDI.Rank'
        Values:

            Min         1       
            Median     96       
            Max       187       

    HDI: 173×1 double

        Values:

            Min       0.286
            Median    0.698
            Max       0.943

    CPI: 173×1 double

        Values:

            Min       1.5  
            Median    3.2  
            Max       9.5  

    Region: 173×1 cell array of character vectors
'org_babel_eoe'

ans =

    'org_babel_eoe'
\end{verbatim}

\begin{itemize}
\tightlist
\item
  Python
\end{itemize}

\begin{Shaded}
\begin{Highlighting}[]
\ImportTok{import}\NormalTok{ pandas }\ImportTok{as}\NormalTok{ pd}
\NormalTok{cpi }\OperatorTok{=}\NormalTok{ pd.read_csv(}\StringTok{'http://tutorials.iq.harvard.edu/R/Rgraphics/dataSets/EconomistData.csv'}\NormalTok{)}
\NormalTok{cpi.describe(include }\OperatorTok{=} \StringTok{'all'}\NormalTok{)}
\end{Highlighting}
\end{Shaded}

\begin{verbatim}
Python 3.6.2 (default, Jul 20 2017, 03:52:27) 
[GCC 7.1.1 20170630] on linux
Type "help", "copyright", "credits" or "license" for more information.
        Unnamed: 0 Country    HDI.Rank         HDI         CPI Region
count   173.000000     173  173.000000  173.000000  173.000000    173
unique         NaN     173         NaN         NaN         NaN      6
top            NaN    Oman         NaN         NaN         NaN    SSA
freq           NaN       1         NaN         NaN         NaN     46
mean     87.000000     NaN   95.283237    0.658087    4.052023    NaN
std      50.084928     NaN   55.007670    0.175589    2.116782    NaN
min       1.000000     NaN    1.000000    0.286000    1.500000    NaN
25%      44.000000     NaN   47.000000    0.509000    2.500000    NaN
50%      87.000000     NaN   96.000000    0.698000    3.200000    NaN
75%     130.000000     NaN  143.000000    0.793000    5.100000    NaN
max     173.000000     NaN  187.000000    0.943000    9.500000    NaN
\end{verbatim}

\subsection{Examples: Fit a linear
regression}\label{examples-fit-a-linear-regression}

Fitting statistical models is pretty straight-forward in all popular
programs.

\begin{itemize}
\tightlist
\item
  Stata
\end{itemize}

\begin{verbatim}
regress hdi cpi
\end{verbatim}

\begin{verbatim}
regress hdi cpi

      Source |       SS           df       MS      Number of obs   =       173
-------------+----------------------------------   F(1, 171)       =    168.85
       Model |  2.63475703         1  2.63475703   Prob > F        =    0.0000
    Residual |   2.6682467       171  .015603782   R-squared       =    0.4968
-------------+----------------------------------   Adj R-squared   =    0.4939
       Total |  5.30300372       172  .030831417   Root MSE        =    .12492

------------------------------------------------------------------------------
         hdi |      Coef.   Std. Err.      t    P>|t|     [95% Conf. Interval]
-------------+----------------------------------------------------------------
         cpi |   .0584696   .0044996    12.99   0.000     .0495876    .0673515
       _cons |   .4211666   .0205577    20.49   0.000     .3805871    .4617462
------------------------------------------------------------------------------
\end{verbatim}

\begin{itemize}
\tightlist
\item
  R
\end{itemize}

\begin{Shaded}
\begin{Highlighting}[]
\KeywordTok{summary}\NormalTok{(}\KeywordTok{lm}\NormalTok{(HDI }\OperatorTok{~}\StringTok{ }\NormalTok{CPI, }\DataTypeTok{data =}\NormalTok{ cpi))}
\end{Highlighting}
\end{Shaded}

\begin{verbatim}
Call:
lm(formula = HDI ~ CPI, data = cpi)

Residuals:
     Min       1Q   Median       3Q      Max 
-0.28452 -0.08380  0.01372  0.09157  0.24104 

Coefficients:
            Estimate Std. Error t value Pr(>|t|)    
(Intercept)  0.42117    0.02056   20.49   <2e-16 ***
CPI          0.05847    0.00450   12.99   <2e-16 ***
---
Signif. codes:  0 ‘***’ 0.001 ‘**’ 0.01 ‘*’ 0.05 ‘.’ 0.1 ‘ ’ 1

Residual standard error: 0.1249 on 171 degrees of freedom
Multiple R-squared:  0.4968,    Adjusted R-squared:  0.4939 
F-statistic: 168.9 on 1 and 171 DF,  p-value: < 2.2e-16
\end{verbatim}

\begin{itemize}
\tightlist
\item
  Matlab
\end{itemize}

\begin{Shaded}
\begin{Highlighting}[]
\NormalTok{fitlm(cpi, }\StringTok{'HDI~CPI'}\NormalTok{)}
\end{Highlighting}
\end{Shaded}

\begin{verbatim}
fitlm(cpi, 'HDI~CPI')

ans = 


Linear regression model:
    HDI ~ 1 + CPI

Estimated Coefficients:
                   Estimate       SE        tStat       pValue  
                   ________    _________    ______    __________

    (Intercept)    0.42117      0.020558    20.487    6.7008e-48
    CPI            0.05847     0.0044996    12.994    2.6908e-27


Number of observations: 173, Error degrees of freedom: 171
Root Mean Squared Error: 0.125
R-squared: 0.497,  Adjusted R-Squared 0.494
F-statistic vs. constant model: 169, p-value = 2.69e-27
'org_babel_eoe'

ans =

    'org_babel_eoe'
\end{verbatim}

\begin{itemize}
\tightlist
\item
  Python
\end{itemize}

\begin{Shaded}
\begin{Highlighting}[]
\ImportTok{import}\NormalTok{ statsmodels.formula.api }\ImportTok{as}\NormalTok{ model}
\NormalTok{X }\OperatorTok{=}\NormalTok{ cpi[[}\StringTok{'CPI'}\NormalTok{]]}
\NormalTok{Y }\OperatorTok{=}\NormalTok{ cpi[[}\StringTok{'HDI'}\NormalTok{]]}
\NormalTok{model.OLS(Y, X).fit().summary()}
\end{Highlighting}
\end{Shaded}

\begin{verbatim}
<class 'statsmodels.iolib.summary.Summary'>
"""
                            OLS Regression Results                            
==============================================================================
Dep. Variable:                    HDI   R-squared:                       0.885
Model:                            OLS   Adj. R-squared:                  0.884
Method:                 Least Squares   F-statistic:                     1325.
Date:                Thu, 31 Aug 2017   Prob (F-statistic):           9.89e-83
Time:                        23:16:45   Log-Likelihood:                 8.1584
No. Observations:                 173   AIC:                            -14.32
Df Residuals:                     172   BIC:                            -11.16
Df Model:                           1                                         
Covariance Type:            nonrobust                                         
==============================================================================
                 coef    std err          t      P>|t|      [0.025      0.975]
------------------------------------------------------------------------------
CPI            0.1402      0.004     36.401      0.000       0.133       0.148
==============================================================================
Omnibus:                       10.423   Durbin-Watson:                   1.616
Prob(Omnibus):                  0.005   Jarque-Bera (JB):               11.099
Skew:                          -0.599   Prob(JB):                      0.00389
Kurtosis:                       2.674   Cond. No.                         1.00
==============================================================================

Warnings:
[1] Standard Errors assume that the covariance matrix of the errors is correctly specified.
"""
\end{verbatim}

\subsection{Examples: Extract links for .html
file}\label{examples-extract-links-for-.html-file}

Retrieving data from a website is a common task. Here we parse a simple
web page containing links to files we wish to download.

\begin{itemize}
\tightlist
\item
  Stata
\end{itemize}

\begin{verbatim}
disp "Ha ha ha! No, you do not want to use Stata for this!"
\end{verbatim}

\begin{verbatim}
disp "Ha ha ha! No, you do not want to use Stata for this!"
Ha ha ha! No, you do not want to use Stata for this!
\end{verbatim}

\begin{itemize}
\tightlist
\item
  R
\end{itemize}

\begin{Shaded}
\begin{Highlighting}[]
\KeywordTok{library}\NormalTok{(xml2)}
\NormalTok{index_page <-}\StringTok{ }\KeywordTok{read_html}\NormalTok{(}\StringTok{"http://tutorials.iq.harvard.edu/example_data/baby_names/EW/"}\NormalTok{)}
\NormalTok{all_anchors <-}\StringTok{ }\KeywordTok{xml_find_all}\NormalTok{(index_page, }\StringTok{"//a"}\NormalTok{)}
\NormalTok{all_hrefs <-}\StringTok{ }\KeywordTok{xml_attr}\NormalTok{(all_anchors, }\StringTok{"href"}\NormalTok{)}
\NormalTok{data_hrefs <-}\StringTok{ }\KeywordTok{grep}\NormalTok{(}\StringTok{"}\CharTok{\textbackslash{}\textbackslash{}}\StringTok{.csv$"}\NormalTok{, all_hrefs, }\DataTypeTok{value =} \OtherTok{TRUE}\NormalTok{)}
\NormalTok{data_links <-}\StringTok{ }\KeywordTok{paste0}\NormalTok{(}\StringTok{"http://tutorials.iq.harvard.edu/example_data/baby_names/EW/"}\NormalTok{, data_hrefs)}
\NormalTok{data_links}
\end{Highlighting}
\end{Shaded}

\begin{verbatim}
 [1] "http://tutorials.iq.harvard.edu/example_data/baby_names/EW/boys_1996.csv" 
 [2] "http://tutorials.iq.harvard.edu/example_data/baby_names/EW/boys_1997.csv" 
 [3] "http://tutorials.iq.harvard.edu/example_data/baby_names/EW/boys_1998.csv" 
 [4] "http://tutorials.iq.harvard.edu/example_data/baby_names/EW/boys_1999.csv" 
 [5] "http://tutorials.iq.harvard.edu/example_data/baby_names/EW/boys_2000.csv" 
 [6] "http://tutorials.iq.harvard.edu/example_data/baby_names/EW/boys_2001.csv" 
 [7] "http://tutorials.iq.harvard.edu/example_data/baby_names/EW/boys_2002.csv" 
 [8] "http://tutorials.iq.harvard.edu/example_data/baby_names/EW/boys_2003.csv" 
 [9] "http://tutorials.iq.harvard.edu/example_data/baby_names/EW/boys_2004.csv" 
[10] "http://tutorials.iq.harvard.edu/example_data/baby_names/EW/boys_2005.csv" 
[11] "http://tutorials.iq.harvard.edu/example_data/baby_names/EW/boys_2006.csv" 
[12] "http://tutorials.iq.harvard.edu/example_data/baby_names/EW/boys_2007.csv" 
[13] "http://tutorials.iq.harvard.edu/example_data/baby_names/EW/boys_2008.csv" 
[14] "http://tutorials.iq.harvard.edu/example_data/baby_names/EW/boys_2009.csv" 
[15] "http://tutorials.iq.harvard.edu/example_data/baby_names/EW/boys_2010.csv" 
[16] "http://tutorials.iq.harvard.edu/example_data/baby_names/EW/boys_2011.csv" 
[17] "http://tutorials.iq.harvard.edu/example_data/baby_names/EW/boys_2012.csv" 
[18] "http://tutorials.iq.harvard.edu/example_data/baby_names/EW/boys_2013.csv" 
[19] "http://tutorials.iq.harvard.edu/example_data/baby_names/EW/boys_2014.csv" 
[20] "http://tutorials.iq.harvard.edu/example_data/baby_names/EW/boys_2015.csv" 
[21] "http://tutorials.iq.harvard.edu/example_data/baby_names/EW/girls_1996.csv"
[22] "http://tutorials.iq.harvard.edu/example_data/baby_names/EW/girls_1997.csv"
[23] "http://tutorials.iq.harvard.edu/example_data/baby_names/EW/girls_1998.csv"
[24] "http://tutorials.iq.harvard.edu/example_data/baby_names/EW/girls_1999.csv"
[25] "http://tutorials.iq.harvard.edu/example_data/baby_names/EW/girls_2000.csv"
[26] "http://tutorials.iq.harvard.edu/example_data/baby_names/EW/girls_2001.csv"
[27] "http://tutorials.iq.harvard.edu/example_data/baby_names/EW/girls_2002.csv"
[28] "http://tutorials.iq.harvard.edu/example_data/baby_names/EW/girls_2003.csv"
[29] "http://tutorials.iq.harvard.edu/example_data/baby_names/EW/girls_2004.csv"
[30] "http://tutorials.iq.harvard.edu/example_data/baby_names/EW/girls_2005.csv"
[31] "http://tutorials.iq.harvard.edu/example_data/baby_names/EW/girls_2006.csv"
[32] "http://tutorials.iq.harvard.edu/example_data/baby_names/EW/girls_2007.csv"
[33] "http://tutorials.iq.harvard.edu/example_data/baby_names/EW/girls_2008.csv"
[34] "http://tutorials.iq.harvard.edu/example_data/baby_names/EW/girls_2009.csv"
[35] "http://tutorials.iq.harvard.edu/example_data/baby_names/EW/girls_2010.csv"
[36] "http://tutorials.iq.harvard.edu/example_data/baby_names/EW/girls_2011.csv"
[37] "http://tutorials.iq.harvard.edu/example_data/baby_names/EW/girls_2012.csv"
[38] "http://tutorials.iq.harvard.edu/example_data/baby_names/EW/girls_2013.csv"
[39] "http://tutorials.iq.harvard.edu/example_data/baby_names/EW/girls_2014.csv"
[40] "http://tutorials.iq.harvard.edu/example_data/baby_names/EW/girls_2015.csv"
\end{verbatim}

\begin{itemize}
\tightlist
\item
  Matlab
\end{itemize}

\begin{Shaded}
\begin{Highlighting}[]
\NormalTok{index_page = urlread(}\StringTok{'http://tutorials.iq.harvard.edu/example_data/baby_names/EW/'}\NormalTok{);}
\NormalTok{all_hrefs = regexp(index_page, }\StringTok{'<a href="([^"]*\textbackslash{}.csv)">'}\NormalTok{, }\StringTok{'tokens'}\NormalTok{)';}
\NormalTok{all_hrefs = [all_hrefs\{:\}]';}
\NormalTok{all_links = strcat(}\StringTok{'http://tutorials.iq.harvard.edu/example_data/baby_names/EW/'}\NormalTok{, all_hrefs)}
\end{Highlighting}
\end{Shaded}

\begin{verbatim}
index_page = urlread('http://tutorials.iq.harvard.edu/example_data/baby_names/EW/');
all_hrefs = regexp(index_page, '<a href="([^"]*\.csv)">', 'tokens')';
all_hrefs = [all_hrefs{:}]';
all_links = strcat('http://tutorials.iq.harvard.edu/example_data/baby_names/EW/', all_hrefs)

all_links =

  40×1 cell array

    'http://tutorials.iq.harvard.edu/example_data/baby_names/EW/boys_1996.csv'
    'http://tutorials.iq.harvard.edu/example_data/baby_names/EW/boys_1997.csv'
    'http://tutorials.iq.harvard.edu/example_data/baby_names/EW/boys_1998.csv'
    'http://tutorials.iq.harvard.edu/example_data/baby_names/EW/boys_1999.csv'
    'http://tutorials.iq.harvard.edu/example_data/baby_names/EW/boys_2000.csv'
    'http://tutorials.iq.harvard.edu/example_data/baby_names/EW/boys_2001.csv'
    'http://tutorials.iq.harvard.edu/example_data/baby_names/EW/boys_2002.csv'
    'http://tutorials.iq.harvard.edu/example_data/baby_names/EW/boys_2003.csv'
    'http://tutorials.iq.harvard.edu/example_data/baby_names/EW/boys_2004.csv'
    'http://tutorials.iq.harvard.edu/example_data/baby_names/EW/boys_2005.csv'
    'http://tutorials.iq.harvard.edu/example_data/baby_names/EW/boys_2006.csv'
    'http://tutorials.iq.harvard.edu/example_data/baby_names/EW/boys_2007.csv'
    'http://tutorials.iq.harvard.edu/example_data/baby_names/EW/boys_2008.csv'
    'http://tutorials.iq.harvard.edu/example_data/baby_names/EW/boys_2009.csv'
    'http://tutorials.iq.harvard.edu/example_data/baby_names/EW/boys_2010.csv'
    'http://tutorials.iq.harvard.edu/example_data/baby_names/EW/boys_2011.csv'
    'http://tutorials.iq.harvard.edu/example_data/baby_names/EW/boys_2012.csv'
    'http://tutorials.iq.harvard.edu/example_data/baby_names/EW/boys_2013.csv'
    'http://tutorials.iq.harvard.edu/example_data/baby_names/EW/boys_2014.csv'
    'http://tutorials.iq.harvard.edu/example_data/baby_names/EW/boys_2015.csv'
    'http://tutorials.iq.harvard.edu/example_data/baby_names/EW/girls_1996.csv'
    'http://tutorials.iq.harvard.edu/example_data/baby_names/EW/girls_1997.csv'
    'http://tutorials.iq.harvard.edu/example_data/baby_names/EW/girls_1998.csv'
    'http://tutorials.iq.harvard.edu/example_data/baby_names/EW/girls_1999.csv'
    'http://tutorials.iq.harvard.edu/example_data/baby_names/EW/girls_2000.csv'
    'http://tutorials.iq.harvard.edu/example_data/baby_names/EW/girls_2001.csv'
    'http://tutorials.iq.harvard.edu/example_data/baby_names/EW/girls_2002.csv'
    'http://tutorials.iq.harvard.edu/example_data/baby_names/EW/girls_2003.csv'
    'http://tutorials.iq.harvard.edu/example_data/baby_names/EW/girls_2004.csv'
    'http://tutorials.iq.harvard.edu/example_data/baby_names/EW/girls_2005.csv'
    'http://tutorials.iq.harvard.edu/example_data/baby_names/EW/girls_2006.csv'
    'http://tutorials.iq.harvard.edu/example_data/baby_names/EW/girls_2007.csv'
    'http://tutorials.iq.harvard.edu/example_data/baby_names/EW/girls_2008.csv'
    'http://tutorials.iq.harvard.edu/example_data/baby_names/EW/girls_2009.csv'
    'http://tutorials.iq.harvard.edu/example_data/baby_names/EW/girls_2010.csv'
    'http://tutorials.iq.harvard.edu/example_data/baby_names/EW/girls_2011.csv'
    'http://tutorials.iq.harvard.edu/example_data/baby_names/EW/girls_2012.csv'
    'http://tutorials.iq.harvard.edu/example_data/baby_names/EW/girls_2013.csv'
    'http://tutorials.iq.harvard.edu/example_data/baby_names/EW/girls_2014.csv'
    'http://tutorials.iq.harvard.edu/example_data/baby_names/EW/girls_2015.csv'
'org_babel_eoe'

ans =

    'org_babel_eoe'
\end{verbatim}

\begin{itemize}
\tightlist
\item
  Python
\end{itemize}

\begin{Shaded}
\begin{Highlighting}[]
\ImportTok{from}\NormalTok{ lxml }\ImportTok{import}\NormalTok{ etree}
\ImportTok{import}\NormalTok{ requests}

\NormalTok{index_text }\OperatorTok{=}\NormalTok{ requests.get(}\StringTok{'http://tutorials.iq.harvard.edu/example_data/baby_names/EW/'}\NormalTok{).text}
\NormalTok{index_page }\OperatorTok{=}\NormalTok{ etree.HTML(index_text)}
\NormalTok{all_hrefs }\OperatorTok{=}\NormalTok{ [a.values() }\ControlFlowTok{for}\NormalTok{ a }\KeywordTok{in}\NormalTok{ index_page.findall(}\StringTok{".//a"}\NormalTok{)]}
\NormalTok{data_links }\OperatorTok{=}\NormalTok{ [}\StringTok{'http://tutorials.iq.harvard.edu/example_data/baby_names/EW/'} \OperatorTok{+}
\NormalTok{              href[}\DecValTok{0}\NormalTok{] }\ControlFlowTok{for}\NormalTok{ href }\KeywordTok{in}\NormalTok{ all_hrefs }\ControlFlowTok{if} \StringTok{'csv'} \KeywordTok{in}\NormalTok{ href[}\DecValTok{0}\NormalTok{]]}
\ControlFlowTok{for}\NormalTok{ link }\KeywordTok{in}\NormalTok{ data_links:}
    \BuiltInTok{print}\NormalTok{(link)}
\end{Highlighting}
\end{Shaded}

\begin{verbatim}
http://tutorials.iq.harvard.edu/example_data/baby_names/EW/boys_1996.csv
http://tutorials.iq.harvard.edu/example_data/baby_names/EW/boys_1997.csv
http://tutorials.iq.harvard.edu/example_data/baby_names/EW/boys_1998.csv
http://tutorials.iq.harvard.edu/example_data/baby_names/EW/boys_1999.csv
http://tutorials.iq.harvard.edu/example_data/baby_names/EW/boys_2000.csv
http://tutorials.iq.harvard.edu/example_data/baby_names/EW/boys_2001.csv
http://tutorials.iq.harvard.edu/example_data/baby_names/EW/boys_2002.csv
http://tutorials.iq.harvard.edu/example_data/baby_names/EW/boys_2003.csv
http://tutorials.iq.harvard.edu/example_data/baby_names/EW/boys_2004.csv
http://tutorials.iq.harvard.edu/example_data/baby_names/EW/boys_2005.csv
http://tutorials.iq.harvard.edu/example_data/baby_names/EW/boys_2006.csv
http://tutorials.iq.harvard.edu/example_data/baby_names/EW/boys_2007.csv
http://tutorials.iq.harvard.edu/example_data/baby_names/EW/boys_2008.csv
http://tutorials.iq.harvard.edu/example_data/baby_names/EW/boys_2009.csv
http://tutorials.iq.harvard.edu/example_data/baby_names/EW/boys_2010.csv
http://tutorials.iq.harvard.edu/example_data/baby_names/EW/boys_2011.csv
http://tutorials.iq.harvard.edu/example_data/baby_names/EW/boys_2012.csv
http://tutorials.iq.harvard.edu/example_data/baby_names/EW/boys_2013.csv
http://tutorials.iq.harvard.edu/example_data/baby_names/EW/boys_2014.csv
http://tutorials.iq.harvard.edu/example_data/baby_names/EW/boys_2015.csv
http://tutorials.iq.harvard.edu/example_data/baby_names/EW/girls_1996.csv
http://tutorials.iq.harvard.edu/example_data/baby_names/EW/girls_1997.csv
http://tutorials.iq.harvard.edu/example_data/baby_names/EW/girls_1998.csv
http://tutorials.iq.harvard.edu/example_data/baby_names/EW/girls_1999.csv
http://tutorials.iq.harvard.edu/example_data/baby_names/EW/girls_2000.csv
http://tutorials.iq.harvard.edu/example_data/baby_names/EW/girls_2001.csv
http://tutorials.iq.harvard.edu/example_data/baby_names/EW/girls_2002.csv
http://tutorials.iq.harvard.edu/example_data/baby_names/EW/girls_2003.csv
http://tutorials.iq.harvard.edu/example_data/baby_names/EW/girls_2004.csv
http://tutorials.iq.harvard.edu/example_data/baby_names/EW/girls_2005.csv
http://tutorials.iq.harvard.edu/example_data/baby_names/EW/girls_2006.csv
http://tutorials.iq.harvard.edu/example_data/baby_names/EW/girls_2007.csv
http://tutorials.iq.harvard.edu/example_data/baby_names/EW/girls_2008.csv
http://tutorials.iq.harvard.edu/example_data/baby_names/EW/girls_2009.csv
http://tutorials.iq.harvard.edu/example_data/baby_names/EW/girls_2010.csv
http://tutorials.iq.harvard.edu/example_data/baby_names/EW/girls_2011.csv
http://tutorials.iq.harvard.edu/example_data/baby_names/EW/girls_2012.csv
http://tutorials.iq.harvard.edu/example_data/baby_names/EW/girls_2013.csv
http://tutorials.iq.harvard.edu/example_data/baby_names/EW/girls_2014.csv
http://tutorials.iq.harvard.edu/example_data/baby_names/EW/girls_2015.csv
\end{verbatim}

\section{Creating reports}\label{creating-reports}

Once you've analyzed your data you'll most likely want to communicate
your results. For short informal projects this might take the form of a
blog post or an email to your colleagues. For larger more formal
projects you'll likely want to prepare a substantial report or
manuscript for disseminating your findings via a journal publication or
other means. Other common means of reporting research findings include
posters or slides for a conference talk.

Regardless of the type of report, you may choose to use either a
\emph{markup language} or a WYSIWYG application like Microsoft
Word/Powerpoint of a desktop publishing application such as Adobe
InDesign.

\subsection{Markup languages}\label{markup-languages}

A markup language is a system for producing a formatted document from a
text file using information by the markup. A major advantage of markup
languages is that the formatting instructions can be easily generated by
the program you use for analyzing your data.

Markup languages include \emph{HTML}, \emph{LaTeX}, \emph{Markdown} and
many others. \emph{LaTeX} and \emph{Markdown} are currently popular
among data scientists, although others are used as well.

\emph{Markdown} is easy to write and designed to be human-readable. It
is newer and somewhat less feature-full compared to LaTeX. It's main
advantage is simplicity. \emph{LaTeX} is more verbose but provides for
just about any feature you'll ever need.

MARKDOWN DEMO LATEX DEMO

\subsection{Word processors}\label{word-processors}

Modern word processors are largely just graphical user interfaces that
write a markup language (usually XML) for you. They are commonly used
for creating reports, but care must be taken when doing so.

If you use a word processor to produce your reports you should

\begin{itemize}
\tightlist
\item
  use the structured outline feature,
\item
  link rather than embed external resources (figures, tables, etc.),
\item
  use cross-referencing features, and
\item
  use a bibliography management system.
\end{itemize}

WORD PROCESSOR DEMO

\section{Text editors \& Integrated Development
Environments}\label{text-editors-integrated-development-environments}

A text editor edits text obviously. But that is not all! At a minimum, a
text editor will also have a mechanism for reading and writing text
files. Most text editors do much more than this.

An IDE provides tools for working with code, such as syntax
highlighting, code completion, jump-to-definition, execute/compile,
package management, refactoring, etc. Of course an IDE includes a text
editor.

Editors and IDE's are not really separate categories; as you add
features to a text editor it becomes more like an IDE, and a simple IDE
may provide little more than a text editor. For example, Emacs is
commonly referred to as a text editor, but it provides nearly every
feature you would expect an IDE to have.

A more useful distinction is between language-specific editors/IDEs and
general purpose editors/IDEs. The former are typically easier to set up
since they come pre-configured for use with a specific language. General
purpose editors/IDEs typically provide language support via
\emph{plugins} and may require extensive configuration for each
language.

\subsection{Language specific editors \&
IDEs}\label{language-specific-editors-ides}

\begin{longtable}[]{@{}llll@{}}
\toprule
Editor & Features & Ease of use & Language\tabularnewline
\midrule
\endhead
RStudio & Excellent & Easy & R\tabularnewline
Spyder & Excellent & Easy & Python\tabularnewline
Stata do file editor & OK & Easy & Stata\tabularnewline
SPSS syntax editor & OK & Easy & SPSS\tabularnewline
\bottomrule
\end{longtable}

LANGUAGE SPECIFIC IDE DEMO

\subsection{General purpose editors \&
IDEs}\label{general-purpose-editors-ides}

\begin{longtable}[]{@{}llll@{}}
\toprule
Editor & Features & Ease of use & Language support\tabularnewline
\midrule
\endhead
Vim & Excellent & Hard & Good\tabularnewline
Emacs & Excellent & Hard & Excellent\tabularnewline
VS code & Excellent & Easy & Very good\tabularnewline
Atom & Good & Moderate & Good\tabularnewline
Eclipse & Excellent & Easy & Good\tabularnewline
Sublime Text & Good & Easy & Good\tabularnewline
Notepad++ & OK & Easy & OK\tabularnewline
Textmate & Good & Moderate & Good\tabularnewline
Kate & OK & Easy & Good\tabularnewline
\bottomrule
\end{longtable}

GENERAL PURPOSE EDITOR DEMO

\section{Literate programming \&
notebooks}\label{literate-programming-notebooks}

In one of the Early demos we say an example of embedding R code in a
markdown document. A closely related approach is to create a
\emph{notebook} that includes the prose of the report, the code used for
the analysis, and the results produced by that code.

\subsection{Literate programming}\label{literate-programming}

Literate programming is the practice of embedding computer code in a
natural language document. For example, using \emph{RMarkdown} we can
embed R code in a report authored using Markdown. Python and Stata have
their own versions of literate programming using Markdown.

\subsection{Notebooks}\label{notebooks}

Notebooks go one step farther, and include the output produced by the
original program directly in the notebook. Examples include
\emph{Jupyter}, \emph{Appache Zeppelin}, and \emph{Emacs Org Mode}.

NOTEBOOKS DEMO

\section{Big data, annoying data, \& computationally intensive
methods}\label{big-data-annoying-data-computationally-intensive-methods}

Thus far we've discussed popular programming languages, data storage and
retrieval options, text editors, and reporting technology. These are the
basic building blocks I recommend using just about any time you find
yourself working with data. There are times however when more is needed.
For example, you may wish to use distributed computing for large or
resource intensive computations.

\subsection{Computing clusters at
Harvard}\label{computing-clusters-at-harvard}

Harvard provides a number of computing clusters, including Odyssey and
the Research Computing Environment. Using these systems will be much
easier if you know the basic tools well. After all, you're still going
to need data storage/retrieval, you'll still need a text editor write
code, and a programming language to write it in. My advice is to master
these basics, and learn the rest as you need it.

\section{Wrap up}\label{wrap-up}

\subsection{Feedback}\label{feedback}

These workshops are a work-in-progress, please provide any feedback to:
\href{mailto:help@iq.harvard.edu}{\nolinkurl{help@iq.harvard.edu}}

\subsection{Resources}\label{resources}

\begin{itemize}
\tightlist
\item
  IQSS

  \begin{itemize}
  \tightlist
  \item
    Workshops: \url{https://dss.iq.harvard.edu/workshop-materials}
  \item
    Data Science Services: \url{https://dss.iq.harvard.edu/}
  \item
    Research Computing Environment:
    \url{https://iqss.github.io/dss-rce/}
  \end{itemize}
\item
  HBS

  \begin{itemize}
  \tightlist
  \item
    Research Computing Services workshops:
    \url{https://training.rcs.hbs.org/workshops}
  \item
    Other HBS RCS resources:
    \url{https://training.rcs.hbs.org/workshop-materials}
  \item
    RCS consulting email: \url{mailto:research@hbs.edu}
  \end{itemize}
\end{itemize}

\part{R}\label{part-r}

\chapter{R Introduction}\label{r-introduction}

\textbf{Topics}

\begin{itemize}
\tightlist
\item
  Assignment
\item
  Function arguments
\item
  Finding help
\item
  Reading data
\item
  Filtering rows, selecting columns, and arranging data
\item
  Conditional operations
\item
  Saving data
\end{itemize}

\section{Setup}\label{setup}

\subsection{Software \& materials}\label{software-materials}

You should have R and RStudio installed --- if not:

\begin{itemize}
\tightlist
\item
  Download and install R: \url{http://cran.r-project.org}
\item
  Download and install RStudio:
  \url{https://www.rstudio.com/products/rstudio/download/\#download}
\end{itemize}

Download materials:

\begin{itemize}
\tightlist
\item
  Download class materials at
  \url{https://github.com/IQSS/dss-workshops-redux/raw/master/R/Rintro.zip}
\item
  Extract materials from the zipped directory \texttt{Rintro.zip}
  (Right-click =\textgreater{} Extract All on Windows, double-click on
  Mac) and move them to your desktop!
\end{itemize}

Start RStudio and create a new project:

\begin{itemize}
\tightlist
\item
  On Windows click the start button and search for RStudio. On Mac
  RStudio will be in your applications folder.
\item
  In Rstudio go to \texttt{File\ -\textgreater{}\ New\ Project}.
\item
  Choose \texttt{Existing\ Directory} and browse to the \texttt{Rintro}
  directory.
\item
  Choose \texttt{File\ -\textgreater{}\ Open\ File} and select the blank
  version of the \texttt{.Rmd} file.
\end{itemize}

\subsection{Goals}\label{goals}

Class Structure and Organization:

\begin{itemize}
\tightlist
\item
  Ask questions at any time. Really!
\item
  Collaboration is encouraged - please spend a minute introducing
  yourself to your neighbors!
\end{itemize}

This is an introductory R course:

\begin{itemize}
\tightlist
\item
  Assumes no prior knowledge of R
\item
  Relatively slow-paced
\item
  The workshop covers R basics, the R package ecosystem, and practice
  reading files and manipulating data in R
\item
  A more general goal is to get you comfortable with R so that it seems
  less scary and mystifying than it perhaps does now. Note that this is
  by no means a complete or thorough introduction to R! It's just enough
  to get you started.
\end{itemize}

As an example project we will analyze the popularity of baby names in
the US from 1960 through 2017. Among the questions we will answer using
R are:

\begin{itemize}
\tightlist
\item
  In which year did your name achieve peak popularity?
\item
  How many children were born each year?
\item
  What are the most popular names overall? For girls? For Boys?
\end{itemize}

\section{R Interfaces}\label{r-interfaces}

There are many different ways you can interact with R. See the
\href{./DataScienceTools.html}{Data Science Tools workshop notes} for
details.

For this workshop we will use \href{https://rstudio.com/}{RStudio}; it
is a good R-specific integrated development environment (IDE) with many
features.

There are also several different formats for writing code for R. Two of
the most popular are:

\begin{enumerate}
\def\labelenumi{\arabic{enumi}.}
\item
  \textbf{R scripts} --- a type of plain text file that allows you to
  write R code and basic comments about the code:
\item
  \href{https://rmarkdown.rstudio.com/}{\textbf{Rmarkdown}} --- a type
  of text file that allows you to include plain text with R code and
  easily convert the contents into HTML (for a webpage), MS Word, or PDF
  (via LaTeX). Many people write their journal papers, dissertations,
  and statistics/math class notes in Rmarkdown, since it is easy to use
  and to convert into other formats later.
\end{enumerate}

\section{Exercise 0}\label{exercise-0}

The purpose of this exercise is to give you an opportunity to explore
the interface provided by RStudio. You may not know how to do these
things; that's fine! This is an opportunity to figure it out.

Also keep in mind that we are living in a golden age of tab completion.
If you don't know the name of an R function, try guessing the first two
or three letters and pressing TAB. If you guessed correctly the function
you are looking for should appear in a pop up!

\begin{center}\rule{0.5\linewidth}{\linethickness}\end{center}

\begin{enumerate}
\def\labelenumi{\arabic{enumi}.}
\tightlist
\item
  Try to get R to add 2 plus 2.
\end{enumerate}

\begin{Shaded}
\begin{Highlighting}[]
\NormalTok{##}
\end{Highlighting}
\end{Shaded}

\begin{enumerate}
\def\labelenumi{\arabic{enumi}.}
\setcounter{enumi}{1}
\tightlist
\item
  Try to calculate the square root of 10.
\end{enumerate}

\begin{Shaded}
\begin{Highlighting}[]
\NormalTok{##}
\end{Highlighting}
\end{Shaded}

\begin{enumerate}
\def\labelenumi{\arabic{enumi}.}
\setcounter{enumi}{2}
\tightlist
\item
  R includes extensive documentation, including a manual named ``An
  introduction to R''. Use the RStudio help pane. to locate this manual.
\end{enumerate}

\section{R basics}\label{r-basics}

\subsection{Function calls}\label{function-calls}

The general form for calling R functions is

\begin{Shaded}
\begin{Highlighting}[]
\NormalTok{## FunctionName(arg.1 = value.1, arg.2 = value.2, ..., arg.n - value.n)}
\end{Highlighting}
\end{Shaded}

Arguments can be \textbf{matched by name}; unnamed arguments will be
\textbf{matched by position}.

\begin{Shaded}
\begin{Highlighting}[]
\NormalTok{values <-}\StringTok{ }\KeywordTok{c}\NormalTok{(}\FloatTok{1.45}\NormalTok{, }\FloatTok{2.34}\NormalTok{, }\FloatTok{5.68}\NormalTok{)}
\KeywordTok{round}\NormalTok{(}\DataTypeTok{x =}\NormalTok{ values, }\DataTypeTok{digits =} \DecValTok{1}\NormalTok{) }\CommentTok{# match by name}
\KeywordTok{round}\NormalTok{(values, }\DecValTok{1}\NormalTok{) }\CommentTok{# match by position}
\KeywordTok{round}\NormalTok{(}\DecValTok{1}\NormalTok{, values) }\CommentTok{# be careful when matching by position!}
\KeywordTok{round}\NormalTok{(}\DataTypeTok{digits =} \DecValTok{1}\NormalTok{, }\DataTypeTok{x =}\NormalTok{ values) }\CommentTok{# matching by name is safer!}
\end{Highlighting}
\end{Shaded}

\subsection{Assignment}\label{assignment}

Values can be assigned names and used in subsequent operations

\begin{itemize}
\tightlist
\item
  The \textbf{gets} \texttt{\textless{}-} operator (less than followed
  by a dash) is used to save values
\item
  The name on the left \textbf{gets} the value on the right.
\end{itemize}

\begin{Shaded}
\begin{Highlighting}[]
\KeywordTok{sqrt}\NormalTok{(}\DecValTok{10}\NormalTok{) ## calculate square root of 10; result is not stored anywhere}
\NormalTok{x <-}\StringTok{ }\KeywordTok{sqrt}\NormalTok{(}\DecValTok{10}\NormalTok{) }\CommentTok{# assign result to a variable named x}
\end{Highlighting}
\end{Shaded}

Names should start with a letter, and contain only letters, numbers,
underscores, and periods.

\subsection{Asking R for help}\label{asking-r-for-help}

You can ask R for help using the \texttt{help} function, or the
\texttt{?} shortcut.

\begin{Shaded}
\begin{Highlighting}[]
\KeywordTok{help}\NormalTok{(help)}
\NormalTok{?help}
\NormalTok{?sqrt}
\end{Highlighting}
\end{Shaded}

The \texttt{help} function can be used to look up the documentation for
a function, or to look up the documentation to a package. We can learn
how to use the \texttt{stats} package by reading its documentation like
this:

\begin{Shaded}
\begin{Highlighting}[]
\KeywordTok{help}\NormalTok{(}\DataTypeTok{package =} \StringTok{"stats"}\NormalTok{)}
\end{Highlighting}
\end{Shaded}

\section{Getting data into R}\label{getting-data-into-r}

R has data reading functionality built-in -- see e.g.,
\texttt{help(read.table)}. However, faster and more robust tools are
available, and so to make things easier on ourselves we will use a
\emph{contributed package} instead. This requires that we learn a little
bit about packages in R.

\subsection{Installing \& using R
packages}\label{installing-using-r-packages}

A large number of contributed packages are available. If you are looking
for a package for a specific task,
\url{https://cran.r-project.org/web/views/} and \url{https://r-pkg.org}
are good places to start.

You can install a package in R using the \texttt{install.packages()}
function. Once a package is installed you may use the \texttt{library()}
function to attach it so that it can be used.

While R's built-in packages are powerful, in recent years there has been
a big surge in well-designed \emph{contributed packages} for R. In
particular, a collection of R packages called
\href{https://www.tidyverse.org/}{\texttt{tidyverse}} have been designed
specifically for data science. All packages included in
\texttt{tidyverse} share an underlying design philosophy, grammar, and
data structures. We will use \texttt{tidyverse} packages throughout the
workshop, so let's install them now:

\begin{Shaded}
\begin{Highlighting}[]
\NormalTok{## install.packages("tidyverse")}
\KeywordTok{library}\NormalTok{(tidyverse)}
\end{Highlighting}
\end{Shaded}

\begin{figure}
\centering
\includegraphics{R/Rintro/images/tidyverse.png}
\caption{}
\end{figure}

We can also install the \texttt{rmarkdown} package, which will allow us
to combine our text and code into a formatted document at the end of the
workshop:

\begin{Shaded}
\begin{Highlighting}[]
\NormalTok{## install.packages("rmarkdown")}
\KeywordTok{library}\NormalTok{(rmarkdown)}
\end{Highlighting}
\end{Shaded}

\subsection{Readers for common file
types}\label{readers-for-common-file-types}

To read data from a file, you have to know what kind of file it is. The
table below lists functions from the \texttt{readr} package, which is
part of \texttt{tidyverse}, that can import data from common plain-text
formats.

\begin{longtable}[]{@{}ll@{}}
\toprule
Data Type & Function\tabularnewline
\midrule
\endhead
comma separated & \texttt{read\_csv()}\tabularnewline
tab separated & \texttt{read\_delim()}\tabularnewline
other delimited formats & \texttt{read\_table()}\tabularnewline
fixed width & \texttt{read\_fwf()}\tabularnewline
\bottomrule
\end{longtable}

\textbf{Note} You may be confused by the existence of similar functions,
e.g., \texttt{read.csv} and \texttt{read.delim}. These are legacy
functions that tend to be slower and less robust than the \texttt{readr}
functions. One way to tell them apart is that the faster more robust
versions use underscores in their names (e.g., \texttt{read\_csv}) while
the older functions use dots (e.g., \texttt{read.csv}). My advice is to
use the more robust newer versions, i.e., the ones with underscores.

\subsection{Baby names data}\label{baby-names-data}

The examples in this workshop use US baby names data retrieved from
\url{https://catalog.data.gov/dataset/baby-names-from-social-security-card-applications-national-level-data}
A cleaned and merged version of these data is available at
\texttt{http://tutorials.iq.harvard.edu/data/babyNames.csv}.

\section{Exercise 1}\label{exercise-1}

\textbf{Reading the baby names data}

Make sure you have installed the \texttt{tidyverse} suite of packages
and attached them with \texttt{library(tidyverse)}. Baby names data are
available at
\texttt{"http://tutorials.iq.harvard.edu/data/babyNames.csv"}.

\begin{enumerate}
\def\labelenumi{\arabic{enumi}.}
\tightlist
\item
  Open the \texttt{read\_csv} help page to determine how to use it to
  read in data.
\end{enumerate}

\begin{Shaded}
\begin{Highlighting}[]
\NormalTok{##}
\end{Highlighting}
\end{Shaded}

\begin{enumerate}
\def\labelenumi{\arabic{enumi}.}
\setcounter{enumi}{1}
\tightlist
\item
  Read the baby names data using the \texttt{read\_csv} function and
  assign the result with the name \texttt{baby\_names}.
\end{enumerate}

\begin{Shaded}
\begin{Highlighting}[]
\NormalTok{##}
\end{Highlighting}
\end{Shaded}

\section{Popularity of your name}\label{popularity-of-your-name}

In this section we will pull out specific names and examine changes in
their popularity over time.

The \texttt{baby\_names} object we created in the last exercise is a
\texttt{data.frame}. There are many other data structures in R, but for
now we'll focus on working with \texttt{data.frames}.

R has decent data manipulation tools built-in -- see e.g.,
\texttt{help(Extract)}. But, \texttt{tidyverse} packages often provide
more intuitive syntax for accomplishing the same task. In particular, we
will use the \texttt{dplyr} package from \texttt{tidyverse} to filter,
select, and arrange data.

\subsection{Filtering, selecting, \& arranging
data}\label{filtering-selecting-arranging-data}

One way to find the year in which your name was the most popular is to
filter out just the rows corresponding to your name, and then arrange
(sort) by Count.

To demonstrate these techniques we'll try to determine whether ``Alex''"
or ``Mark'' was more popular in 1992. We start by filtering the data so
that we keep only rows where Year is equal to \texttt{1992} and Name is
either ``Alex'' or ``Mark''.

\begin{Shaded}
\begin{Highlighting}[]
\NormalTok{## Read in the baby names data if you haven't already}
\NormalTok{baby_names <-}\StringTok{ }\KeywordTok{read_csv}\NormalTok{(}\StringTok{"babyNames.csv"}\NormalTok{)}
\end{Highlighting}
\end{Shaded}

\begin{Shaded}
\begin{Highlighting}[]
\NormalTok{baby_names_alexmark <-}\StringTok{ }\KeywordTok{filter}\NormalTok{(baby_names, }
\NormalTok{             Year }\OperatorTok{==}\StringTok{ }\DecValTok{1992} \OperatorTok{&}\StringTok{ }\NormalTok{(Name }\OperatorTok{==}\StringTok{ "Alex"} \OperatorTok{|}\StringTok{ }\NormalTok{Name }\OperatorTok{==}\StringTok{ "Mark"}\NormalTok{))}
\NormalTok{baby_names_alexmark}
\end{Highlighting}
\end{Shaded}

Notice that we can combine conditions using \texttt{\&} (AND) and
\texttt{\textbar{}} (OR).

In this case it's pretty easy to see that ``Mark'' is more popular, but
to make it even easier we can arrange the data so that the most popular
name is listed first.

\begin{Shaded}
\begin{Highlighting}[]
\KeywordTok{arrange}\NormalTok{(baby_names_alexmark, Count)}
\end{Highlighting}
\end{Shaded}

\begin{Shaded}
\begin{Highlighting}[]
\KeywordTok{arrange}\NormalTok{(baby_names_alexmark, }\KeywordTok{desc}\NormalTok{(Count))}
\end{Highlighting}
\end{Shaded}

We can also use the \texttt{select()} function to subset the
\texttt{data.frame} by columns. We can then assign the output to a new
object.

\begin{Shaded}
\begin{Highlighting}[]
\NormalTok{baby_names_subset <-}\StringTok{ }\KeywordTok{select}\NormalTok{(baby_names, Name, Count)}
\KeywordTok{head}\NormalTok{(baby_names_subset)}
\end{Highlighting}
\end{Shaded}

\subsection{Other logical operators}\label{other-logical-operators}

In the previous example we used \texttt{==} to filter rows. Other
relational and logical operators are listed below.

\begin{longtable}[]{@{}ll@{}}
\toprule
Operator & Meaning\tabularnewline
\midrule
\endhead
\texttt{==} & equal to\tabularnewline
\texttt{!=} & not equal to\tabularnewline
\texttt{\textgreater{}} & greater than\tabularnewline
\texttt{\textgreater{}=} & greater than or equal to\tabularnewline
\texttt{\textless{}} & less than\tabularnewline
\texttt{\textless{}=} & less than or equal to\tabularnewline
\texttt{\%in\%} & contained in\tabularnewline
\bottomrule
\end{longtable}

These operators may be combined with \texttt{\&} (and) or
\texttt{\textbar{}} (or).

\section{Exercise 2.1}\label{exercise-2.1}

\textbf{Peak popularity of your name}

In this exercise you will discover the year your name reached its
maximum popularity.

Read in the ``babyNames.csv'' file if you have not already done so,
assigning the result to \texttt{baby\_names}. The file is located at
\texttt{"http://tutorials.iq.harvard.edu/data/babyNames.csv"}

Make sure you have installed the \texttt{tidyverse} suite of packages
and attached them with \texttt{library(tidyverse)}.

\begin{enumerate}
\def\labelenumi{\arabic{enumi}.}
\tightlist
\item
  Use \texttt{filter} to extract data for your name (or another name of
  your choice).
\end{enumerate}

\begin{Shaded}
\begin{Highlighting}[]
\NormalTok{##}
\end{Highlighting}
\end{Shaded}

\begin{enumerate}
\def\labelenumi{\arabic{enumi}.}
\setcounter{enumi}{1}
\tightlist
\item
  Arrange the data you produced in step 1 above by \texttt{Count}. In
  which year was the name most popular?
\end{enumerate}

\begin{Shaded}
\begin{Highlighting}[]
\NormalTok{##}
\end{Highlighting}
\end{Shaded}

\begin{enumerate}
\def\labelenumi{\arabic{enumi}.}
\setcounter{enumi}{2}
\tightlist
\item
  BONUS (optional): Filter the data to extract \emph{only} the row
  containing the most popular boys name in 1999.
\end{enumerate}

\begin{Shaded}
\begin{Highlighting}[]
\NormalTok{##}
\end{Highlighting}
\end{Shaded}

\section{Pipe operator in R}\label{pipe-operator-in-r}

There is one very special operator in R called a \textbf{pipe} operator
that looks like this: \texttt{\%\textgreater{}\%}. It allows us to
``chain'' several function calls and, as each function returns an
object, feed it into the next call in a single statement, without
needing extra variables to store the intermediate results. The point of
the pipe is to help you write code in a way that is easier to read and
understand as we will see below.

There is no need to load any additional packages as the operator is made
available via the \texttt{magrittr} package installed as part of
\texttt{tidyverse}. Let's rewrite the sequence of commands to output
ordered counts for names ``Alex'' or ``Mark''.

\begin{Shaded}
\begin{Highlighting}[]
\NormalTok{baby_names }\OperatorTok\StringTok{ }
\StringTok{  }\KeywordTok{filter}\NormalTok{(Year }\OperatorTok{==}\StringTok{ }\DecValTok{1992} \OperatorTok{&}\StringTok{ }\NormalTok{(Name }\OperatorTok{==}\StringTok{ "Alex"} \OperatorTok{|}\StringTok{ }\NormalTok{Name }\OperatorTok{==}\StringTok{ "Mark"}\NormalTok{)) }\OperatorTok
\StringTok{  }\KeywordTok{arrange}\NormalTok{(}\KeywordTok{desc}\NormalTok{(Count))}
\end{Highlighting}
\end{Shaded}

Hint: try pronouncing ``then'' whenever you see
\texttt{\%\textgreater{}\%}. Notice that we avoided creating an
intermediate variable \texttt{baby\_names\_alexmark} and performed the
entire task in just ``one line''!

\section{Exercise 2.2}\label{exercise-2.2}

Rewrite the solution to Exercise 2.1 using pipes. Remember that we were
looking for the year your name reached its maximum popularity. For that,
we filtered the data and then arranged by \texttt{Count.}

\begin{Shaded}
\begin{Highlighting}[]
\NormalTok{##}
\end{Highlighting}
\end{Shaded}

\section{Plotting baby name trends over
time}\label{plotting-baby-name-trends-over-time}

It can be difficult to spot trends when looking at summary tables.
Plotting the data makes it easier to identify interesting patterns.

R has decent plotting tools built-in -- see e.g., \texttt{help(plot)}.
However, again, we will make use of an excellent \emph{contributed
package} from \texttt{tidyverse} called \texttt{ggplot2}.

For quick and simple plots we can use the \texttt{qplot()} function. For
example, we can plot the number of babies given the name ``Diana'' over
time like this:

\begin{Shaded}
\begin{Highlighting}[]
\NormalTok{baby_names_diana <-}\StringTok{ }\KeywordTok{filter}\NormalTok{(baby_names, Name }\OperatorTok{==}\StringTok{ "Diana"}\NormalTok{)}
\end{Highlighting}
\end{Shaded}

\begin{Shaded}
\begin{Highlighting}[]
\KeywordTok{qplot}\NormalTok{(}\DataTypeTok{x =}\NormalTok{ Year, }\DataTypeTok{y =}\NormalTok{ Count,}
     \DataTypeTok{data =}\NormalTok{ baby_names_diana)}
\end{Highlighting}
\end{Shaded}

Interestingly, there are usually some gender-atypical names, even for
very strongly gendered names like ``Diana''. Splitting these trends out
by Sex is very easy:

\begin{Shaded}
\begin{Highlighting}[]
\KeywordTok{qplot}\NormalTok{(}\DataTypeTok{x =}\NormalTok{ Year, }\DataTypeTok{y =}\NormalTok{ Count, }\DataTypeTok{color =}\NormalTok{ Sex,}
      \DataTypeTok{data =}\NormalTok{ baby_names_diana)}
\end{Highlighting}
\end{Shaded}

\section{Exercise 3}\label{exercise-3}

\textbf{Plotting peak popularity of your name}

Make sure the \texttt{tidyverse} suite of packages is installed, and
that you have attached them using \texttt{library(tidyverse)}.

\begin{enumerate}
\def\labelenumi{\arabic{enumi}.}
\tightlist
\item
  Use \texttt{filter} to extract data for your name (same as previous
  exercise)
\end{enumerate}

\begin{Shaded}
\begin{Highlighting}[]
\NormalTok{##}
\end{Highlighting}
\end{Shaded}

\begin{enumerate}
\def\labelenumi{\arabic{enumi}.}
\setcounter{enumi}{1}
\tightlist
\item
  Plot the data you produced in step 1 above, with \texttt{Year} on the
  x-axis and \texttt{Count} on the y-axis.
\end{enumerate}

\begin{Shaded}
\begin{Highlighting}[]
\NormalTok{##}
\end{Highlighting}
\end{Shaded}

\begin{enumerate}
\def\labelenumi{\arabic{enumi}.}
\setcounter{enumi}{2}
\tightlist
\item
  Adjust the plot so that is shows boys and girls in different colors.
\end{enumerate}

\begin{Shaded}
\begin{Highlighting}[]
\NormalTok{##}
\end{Highlighting}
\end{Shaded}

\begin{enumerate}
\def\labelenumi{\arabic{enumi}.}
\setcounter{enumi}{3}
\tightlist
\item
  BONUS (Optional): Adust the plot to use lines instead of points.
\end{enumerate}

\begin{Shaded}
\begin{Highlighting}[]
\NormalTok{##}
\end{Highlighting}
\end{Shaded}

\section{Finding the most popular
names}\label{finding-the-most-popular-names}

Our next goal is to find out which names have been the most popular.

\subsection{Computing better measures of
popularity}\label{computing-better-measures-of-popularity}

So far we've used \texttt{Count} as a measure of popularity. A better
approach is to use proportion or rank to avoid confounding popularity
with the number of babies born in a given year.

The \texttt{mutate()} function makes it easy to add or modify the
columns of a \texttt{data.frame}. For example, we can use it to rescale
the count of each name in each year:

\begin{Shaded}
\begin{Highlighting}[]
\NormalTok{baby_names <-}\StringTok{ }\KeywordTok{mutate}\NormalTok{(baby_names, }\DataTypeTok{Count_1k =}\NormalTok{ Count}\OperatorTok{/}\DecValTok{1000}\NormalTok{)}
\NormalTok{baby_names }\CommentTok{# same as print(baby_names)}
\end{Highlighting}
\end{Shaded}

Notice that executing the second line led to printing all 1 million rows
in our \texttt{data.frame}! If we would just like to glance at the first
6 lines we can use the \texttt{head()} function:

\begin{Shaded}
\begin{Highlighting}[]
\KeywordTok{head}\NormalTok{(baby_names) }
\end{Highlighting}
\end{Shaded}

If we like, we can also \texttt{select()} a subset of columns from the
baby names data:

\begin{Shaded}
\begin{Highlighting}[]
\NormalTok{baby_names }\OperatorTok\StringTok{ }
\StringTok{  }\KeywordTok{select}\NormalTok{(Name, Sex, Year, Count_1k) }\OperatorTok
\StringTok{  }\KeywordTok{head}\NormalTok{()}
\end{Highlighting}
\end{Shaded}

\subsection{Operating by group}\label{operating-by-group}

Because of the nested nature of our data, we want to compute rank or
proportion within each \texttt{Sex} by \texttt{Year} group. The
\texttt{dplyr} package makes this relatively straightforward.

\begin{Shaded}
\begin{Highlighting}[]
\NormalTok{baby_names <-}\StringTok{ }
\StringTok{  }\NormalTok{baby_names }\OperatorTok
\StringTok{  }\KeywordTok{group_by}\NormalTok{(Year, Sex) }\OperatorTok
\StringTok{  }\KeywordTok{mutate}\NormalTok{(}\DataTypeTok{Rank =} \KeywordTok{rank}\NormalTok{(Count_1k)) }\OperatorTok
\StringTok{  }\KeywordTok{ungroup}\NormalTok{()}

\KeywordTok{head}\NormalTok{(baby_names)}
\end{Highlighting}
\end{Shaded}

Note that the data remains grouped until you change the groups by
running \texttt{group\_by()} again or remove grouping information with
\texttt{ungroup()}.

\section{Exercise 4}\label{exercise-4}

\textbf{Most popular names}

In this exercise your goal is to identify the most popular names for
each year.

\begin{enumerate}
\def\labelenumi{\arabic{enumi}.}
\tightlist
\item
  Use \texttt{mutate()} and \texttt{group\_by()} to create a column
  named ``Proportion'' where \texttt{Proportion\ =\ Count/sum(Count)}
  for each \texttt{Year\ X\ Sex} group. Use pipes wherever it makes
  sense.
\end{enumerate}

\begin{Shaded}
\begin{Highlighting}[]
\NormalTok{## }
\end{Highlighting}
\end{Shaded}

\begin{enumerate}
\def\labelenumi{\arabic{enumi}.}
\setcounter{enumi}{1}
\tightlist
\item
  Use \texttt{mutate()} and \texttt{group\_by()} to create a column
  named ``Rank'' where \texttt{Rank\ =\ rank(-Count)} for each
  \texttt{Year\ X\ Sex} group.
\end{enumerate}

\begin{Shaded}
\begin{Highlighting}[]
\NormalTok{##}
\end{Highlighting}
\end{Shaded}

\begin{enumerate}
\def\labelenumi{\arabic{enumi}.}
\setcounter{enumi}{2}
\tightlist
\item
  Filter the baby names data to display only the most popular name for
  each \texttt{Year\ X\ Sex} group. Keep only the columns: Year, Name,
  Sex, and Proportion.
\end{enumerate}

\begin{Shaded}
\begin{Highlighting}[]
\NormalTok{##}
\end{Highlighting}
\end{Shaded}

\begin{enumerate}
\def\labelenumi{\arabic{enumi}.}
\setcounter{enumi}{3}
\tightlist
\item
  Plot the data produced in step 4, putting \texttt{Year} on the x-axis
  and \texttt{Proportion} on the y-axis. How has the proportion of
  babies given the most popular name changed over time?
\end{enumerate}

\begin{Shaded}
\begin{Highlighting}[]
\NormalTok{##}
\end{Highlighting}
\end{Shaded}

\begin{enumerate}
\def\labelenumi{\arabic{enumi}.}
\setcounter{enumi}{4}
\tightlist
\item
  BONUS (optional): Which names are the most popular for both boys and
  girls?
\end{enumerate}

\begin{Shaded}
\begin{Highlighting}[]
\NormalTok{##}
\end{Highlighting}
\end{Shaded}

\section{Percent choosing one of the top 10
names}\label{percent-choosing-one-of-the-top-10-names}

You may have noticed that the percentage of babies given the most
popular name of the year appears to have decreased over time. We can
compute a more robust measure of the popularity of the most popular
names by calculating the number of babies given one of the top 10 girl
or boy names of the year.

To compute this measure we need to operate within groups, as we did
using \texttt{mutate()} above, but this time we need to collapse each
group into a single summary statistic. We can achieve this using the
\texttt{summarize()} function.

First, let's see how this function works without grouping. The following
code outputs the total number of girls and boys in the data:

\begin{Shaded}
\begin{Highlighting}[]
\NormalTok{baby_names }\OperatorTok\StringTok{ }
\StringTok{  }\KeywordTok{summarize}\NormalTok{(}\DataTypeTok{Girls_n =} \KeywordTok{sum}\NormalTok{(Sex}\OperatorTok{==}\StringTok{"Girls"}\NormalTok{),}
            \DataTypeTok{Boys_n =} \KeywordTok{sum}\NormalTok{(Sex}\OperatorTok{==}\StringTok{"Boys"}\NormalTok{))}
\end{Highlighting}
\end{Shaded}

Next, using \texttt{group\_by()} and \texttt{summarize()} together, we
can calculate the number of babies born each year:

\begin{Shaded}
\begin{Highlighting}[]
\NormalTok{bn_by_year <-}
\StringTok{  }\NormalTok{baby_names }\OperatorTok
\StringTok{  }\KeywordTok{group_by}\NormalTok{(Year) }\OperatorTok
\StringTok{  }\KeywordTok{summarize}\NormalTok{(}\DataTypeTok{Total =} \KeywordTok{sum}\NormalTok{(Count))}

\KeywordTok{head}\NormalTok{(bn_by_year)}
\end{Highlighting}
\end{Shaded}

\section{Exercise 5}\label{exercise-5}

\textbf{Popularity of the most popular names}

In this exercise we will plot trends in the proportion of boys and girls
given one of the 10 most popular names each year.

\begin{enumerate}
\def\labelenumi{\arabic{enumi}.}
\tightlist
\item
  Filter the \texttt{baby\_names} data, retaining only the 10 most
  popular girl and boy names for each year.
\end{enumerate}

\begin{Shaded}
\begin{Highlighting}[]
\NormalTok{##}
\end{Highlighting}
\end{Shaded}

\begin{enumerate}
\def\labelenumi{\arabic{enumi}.}
\setcounter{enumi}{1}
\tightlist
\item
  Summarize the data produced in step one to calculate the total
  Proportion of boys and girls given one of the top 10 names each year.
\end{enumerate}

\begin{Shaded}
\begin{Highlighting}[]
\NormalTok{##}
\end{Highlighting}
\end{Shaded}

\begin{enumerate}
\def\labelenumi{\arabic{enumi}.}
\setcounter{enumi}{2}
\tightlist
\item
  Plot the data produced in step 2, with year on the x-axis and total
  proportion on the y axis. Color by sex and notice the trend.
\end{enumerate}

\begin{Shaded}
\begin{Highlighting}[]
\NormalTok{##}
\end{Highlighting}
\end{Shaded}

\section{Saving our Work}\label{saving-our-work}

Now that we have made some changes to our data set, we might want to
save those changes to a file.

\subsection{Saving individual
datasets}\label{saving-individual-datasets}

You might find functions \texttt{write\_csv()} and \texttt{write\_rds()}
from package \texttt{readr} handy!

\begin{Shaded}
\begin{Highlighting}[]
\CommentTok{# write data to a .csv file}
\KeywordTok{write_csv}\NormalTok{(baby_names, }\StringTok{"babyNames.csv"}\NormalTok{)}
\end{Highlighting}
\end{Shaded}

\begin{Shaded}
\begin{Highlighting}[]
\CommentTok{# write data to an R file}
\KeywordTok{write_rds}\NormalTok{(baby_names, }\StringTok{"babyNames.rds"}\NormalTok{)}
\end{Highlighting}
\end{Shaded}

\subsection{Saving multiple datasets}\label{saving-multiple-datasets}

\begin{Shaded}
\begin{Highlighting}[]
\KeywordTok{ls}\NormalTok{() }\CommentTok{# list objects in our workspace}
\KeywordTok{save}\NormalTok{(baby_names_diana, bn_by_year, baby_names_subset, }\DataTypeTok{file=}\StringTok{"myDataFiles.RData"}\NormalTok{)  }
\end{Highlighting}
\end{Shaded}

\begin{Shaded}
\begin{Highlighting}[]
\NormalTok{## Load the "myDataFiles.RData"}
\NormalTok{## load("myDataFiles.RData") }
\end{Highlighting}
\end{Shaded}

\subsection{Saving \& loading R
workspaces}\label{saving-loading-r-workspaces}

In addition to importing individual datasets, R can save and load entire
workspaces

\begin{Shaded}
\begin{Highlighting}[]
\KeywordTok{ls}\NormalTok{() }\CommentTok{# list objects in our workspace}
\KeywordTok{save.image}\NormalTok{(}\DataTypeTok{file=}\StringTok{"myWorkspace.RData"}\NormalTok{) }\CommentTok{# save workspace }
\KeywordTok{rm}\NormalTok{(}\DataTypeTok{list=}\KeywordTok{ls}\NormalTok{()) }\CommentTok{# remove all objects from our workspace }
\KeywordTok{ls}\NormalTok{() }\CommentTok{# list stored objects to make sure they are deleted}
\end{Highlighting}
\end{Shaded}

\begin{Shaded}
\begin{Highlighting}[]
\NormalTok{## Load the "myWorkspace.RData" file and check that it is restored}
\KeywordTok{load}\NormalTok{(}\StringTok{"myWorkspace.RData"}\NormalTok{) }\CommentTok{# load myWorkspace.RData}
\KeywordTok{ls}\NormalTok{() }\CommentTok{# list objects}
\end{Highlighting}
\end{Shaded}

\section{Exercise solutions}\label{exercise-solutions}

\subsection{Ex 0: prototype}\label{ex-0-prototype}

\begin{Shaded}
\begin{Highlighting}[]
\NormalTok{## 1. 2 plus 2}
\DecValTok{2} \OperatorTok{+}\StringTok{ }\DecValTok{2}
\NormalTok{## or}
\KeywordTok{sum}\NormalTok{(}\DecValTok{2}\NormalTok{, }\DecValTok{2}\NormalTok{)}
\end{Highlighting}
\end{Shaded}

\begin{Shaded}
\begin{Highlighting}[]
\NormalTok{## 2. square root of 10:}
\KeywordTok{sqrt}\NormalTok{(}\DecValTok{10}\NormalTok{)}
\NormalTok{## or}
\DecValTok{10}\OperatorTok{^}\NormalTok{(}\DecValTok{1}\OperatorTok{/}\DecValTok{2}\NormalTok{)}
\end{Highlighting}
\end{Shaded}

\begin{Shaded}
\begin{Highlighting}[]
\NormalTok{## 3. Find "An Introduction to R".}
\end{Highlighting}
\end{Shaded}

\begin{Shaded}
\begin{Highlighting}[]
\NormalTok{## Go to the main help page by running 'help.start() or using the GUI}
\NormalTok{## menu, find and click on the link to "An Introduction to R".}
\end{Highlighting}
\end{Shaded}

\begin{Shaded}
\begin{Highlighting}[]
\NormalTok{##}
\end{Highlighting}
\end{Shaded}

\subsection{Ex 1: prototype}\label{ex-1-prototype}

\begin{Shaded}
\begin{Highlighting}[]
\NormalTok{## read ?read_csv}
\end{Highlighting}
\end{Shaded}

\begin{Shaded}
\begin{Highlighting}[]
\NormalTok{baby_names <-}\StringTok{ }\KeywordTok{read_csv}\NormalTok{(}\StringTok{"http://tutorials.iq.harvard.edu/data/babyNames.csv"}\NormalTok{)}
\end{Highlighting}
\end{Shaded}

\subsection{Ex 2.1: prototype}\label{ex-2.1-prototype}

\begin{Shaded}
\begin{Highlighting}[]
\CommentTok{# 1.  Use `filter` to extract data for your name (or another name of your choice).  }
\end{Highlighting}
\end{Shaded}

\begin{Shaded}
\begin{Highlighting}[]
\NormalTok{baby_names_george <-}\StringTok{ }\KeywordTok{filter}\NormalTok{(baby_names, Name }\OperatorTok{==}\StringTok{ "George"}\NormalTok{)}
\end{Highlighting}
\end{Shaded}

\begin{Shaded}
\begin{Highlighting}[]
\CommentTok{# 2.  Arrange the data you produced in step 1 above by `Count`. }
\CommentTok{#     In which year was the name most popular?}
\end{Highlighting}
\end{Shaded}

\begin{Shaded}
\begin{Highlighting}[]
\KeywordTok{arrange}\NormalTok{(baby_names_george, }\KeywordTok{desc}\NormalTok{(Count))}
\end{Highlighting}
\end{Shaded}

\begin{Shaded}
\begin{Highlighting}[]
\CommentTok{# 3.  BONUS (optional): Filter the data to extract _only_ the }
\CommentTok{#     row containing the most popular boys name in 1999.}
\end{Highlighting}
\end{Shaded}

\begin{Shaded}
\begin{Highlighting}[]
\NormalTok{baby_names_boys_}\DecValTok{1999}\NormalTok{ <-}\StringTok{ }\KeywordTok{filter}\NormalTok{(baby_names, }
\NormalTok{                    Year }\OperatorTok{==}\StringTok{ }\DecValTok{1999} \OperatorTok{&}\StringTok{ }\NormalTok{Sex }\OperatorTok{==}\StringTok{ "Boys"}\NormalTok{)}
\end{Highlighting}
\end{Shaded}

\begin{Shaded}
\begin{Highlighting}[]
\KeywordTok{filter}\NormalTok{(baby_names_boys_}\DecValTok{1999}\NormalTok{, Count }\OperatorTok{==}\StringTok{ }\KeywordTok{max}\NormalTok{(Count))}
\end{Highlighting}
\end{Shaded}

\subsection{Ex 2.2: prototype}\label{ex-2.2-prototype}

\begin{Shaded}
\begin{Highlighting}[]
\NormalTok{baby_names }\OperatorTok\StringTok{ }
\StringTok{  }\KeywordTok{filter}\NormalTok{(Name }\OperatorTok{==}\StringTok{ "George"}\NormalTok{) }\OperatorTok
\StringTok{  }\KeywordTok{arrange}\NormalTok{(}\KeywordTok{desc}\NormalTok{(Count))}
\end{Highlighting}
\end{Shaded}

\subsection{Ex 3: prototype}\label{ex-3-prototype}

\begin{Shaded}
\begin{Highlighting}[]
\CommentTok{# 1. Use `filter()` to extract data for your name (same as previous exercise)  }
\end{Highlighting}
\end{Shaded}

\begin{Shaded}
\begin{Highlighting}[]
\NormalTok{baby_names_george <-}\StringTok{ }\KeywordTok{filter}\NormalTok{(baby_names, Name }\OperatorTok{==}\StringTok{ "George"}\NormalTok{)}
\end{Highlighting}
\end{Shaded}

\begin{Shaded}
\begin{Highlighting}[]
\CommentTok{# 2.  Plot the data you produced in step 1 above, with `Year` on the x-axis}
\CommentTok{#     and `Count` on the y-axis.}
\end{Highlighting}
\end{Shaded}

\begin{Shaded}
\begin{Highlighting}[]
\KeywordTok{qplot}\NormalTok{(}\DataTypeTok{x =}\NormalTok{ Year, }\DataTypeTok{y =}\NormalTok{ Count, }\DataTypeTok{data =}\NormalTok{ baby_names_george)}
\end{Highlighting}
\end{Shaded}

\begin{Shaded}
\begin{Highlighting}[]
\CommentTok{# 3. Adjust the plot so that is shows boys and girls in different colors.}
\end{Highlighting}
\end{Shaded}

\begin{Shaded}
\begin{Highlighting}[]
\KeywordTok{qplot}\NormalTok{(}\DataTypeTok{x =}\NormalTok{ Year, }\DataTypeTok{y =}\NormalTok{ Count, }\DataTypeTok{color =}\NormalTok{ Sex, }\DataTypeTok{data =}\NormalTok{ baby_names_george)}
\end{Highlighting}
\end{Shaded}

\begin{Shaded}
\begin{Highlighting}[]
\CommentTok{# 4.  BONUS (Optional): Adjust the plot to use lines instead of points.}
\end{Highlighting}
\end{Shaded}

\begin{Shaded}
\begin{Highlighting}[]
\KeywordTok{qplot}\NormalTok{(}\DataTypeTok{x =}\NormalTok{ Year, }\DataTypeTok{y =}\NormalTok{ Count, }\DataTypeTok{color =}\NormalTok{ Sex, }\DataTypeTok{data =}\NormalTok{ baby_names_george, }\DataTypeTok{geom =} \StringTok{"line"}\NormalTok{)}
\end{Highlighting}
\end{Shaded}

\subsection{Ex 4: prototype}\label{ex-4-prototype}

\begin{Shaded}
\begin{Highlighting}[]
\NormalTok{## 1.  Use `mutate()` and `group_by()` to create a column named "Proportion"}
\NormalTok{##     where `Proportion = Count/sum(Count)` for each `Year X Sex` group.}
\end{Highlighting}
\end{Shaded}

\begin{Shaded}
\begin{Highlighting}[]
\NormalTok{baby_names <-}\StringTok{ }
\StringTok{  }\NormalTok{baby_names }\OperatorTok
\StringTok{  }\KeywordTok{group_by}\NormalTok{(Year, Sex) }\OperatorTok
\StringTok{  }\KeywordTok{mutate}\NormalTok{(}\DataTypeTok{Proportion =}\NormalTok{ Count}\OperatorTok{/}\KeywordTok{sum}\NormalTok{(Count)) }\OperatorTok
\StringTok{  }\KeywordTok{ungroup}\NormalTok{()}

\KeywordTok{head}\NormalTok{(baby_names)  }
\end{Highlighting}
\end{Shaded}

\begin{Shaded}
\begin{Highlighting}[]
\NormalTok{## 2.  Use `mutate()` and `group_by()` to create a column named "Rank" where }
\NormalTok{##     `Rank = rank(-Count)` for each `Year X Sex` group.}
\end{Highlighting}
\end{Shaded}

\begin{Shaded}
\begin{Highlighting}[]
\NormalTok{baby_names <-}\StringTok{ }
\StringTok{  }\NormalTok{baby_names }\OperatorTok
\StringTok{  }\KeywordTok{group_by}\NormalTok{(Year, Sex) }\OperatorTok
\StringTok{  }\KeywordTok{mutate}\NormalTok{(}\DataTypeTok{Rank =} \KeywordTok{rank}\NormalTok{(}\OperatorTok{-}\NormalTok{Count)) }\OperatorTok
\StringTok{  }\KeywordTok{ungroup}\NormalTok{()}

\KeywordTok{head}\NormalTok{(baby_names)   }
\end{Highlighting}
\end{Shaded}

\begin{Shaded}
\begin{Highlighting}[]
\NormalTok{## 3.  Filter the baby names data to display only the most popular name }
\NormalTok{##     for each `Year X Sex` group.}
\end{Highlighting}
\end{Shaded}

\begin{Shaded}
\begin{Highlighting}[]
\NormalTok{top1 <-}\StringTok{ }
\StringTok{  }\NormalTok{baby_names }\OperatorTok
\StringTok{  }\KeywordTok{filter}\NormalTok{(Rank }\OperatorTok{==}\StringTok{ }\DecValTok{1}\NormalTok{) }\OperatorTok
\StringTok{  }\KeywordTok{select}\NormalTok{(Year, Name, Sex, Proportion)}

\KeywordTok{head}\NormalTok{(top1)}
\end{Highlighting}
\end{Shaded}

\begin{Shaded}
\begin{Highlighting}[]
\NormalTok{## 4. Plot the data produced in step 3, putting `Year` on the x-axis}
\NormalTok{##    and `Proportion` on the y-axis. How has the proportion of babies}
\NormalTok{##    given the most popular name changed over time?}
\end{Highlighting}
\end{Shaded}

\begin{Shaded}
\begin{Highlighting}[]
\KeywordTok{qplot}\NormalTok{(}\DataTypeTok{x =}\NormalTok{ Year, }
      \DataTypeTok{y =}\NormalTok{ Proportion, }
      \DataTypeTok{color =}\NormalTok{ Sex, }
      \DataTypeTok{data =}\NormalTok{ top1, }
      \DataTypeTok{geom =} \StringTok{"line"}\NormalTok{)}
\end{Highlighting}
\end{Shaded}

\begin{Shaded}
\begin{Highlighting}[]
\NormalTok{## 5. BONUS (optional): Which names are the most popular for both boys }
\NormalTok{##    and girls?}
\end{Highlighting}
\end{Shaded}

\begin{Shaded}
\begin{Highlighting}[]
\NormalTok{bn_girls <-}\StringTok{ }\NormalTok{baby_names }\OperatorTok\StringTok{ }
\StringTok{  }\KeywordTok{filter}\NormalTok{(Sex }\OperatorTok{==}\StringTok{ "Girls"}\NormalTok{) }\OperatorTok
\StringTok{  }\KeywordTok{select}\NormalTok{(Name, Year, Count)}

\NormalTok{bn_boys <-}\StringTok{ }\NormalTok{baby_names }\OperatorTok\StringTok{ }
\StringTok{  }\KeywordTok{filter}\NormalTok{(Sex }\OperatorTok{==}\StringTok{ "Boys"}\NormalTok{) }\OperatorTok
\StringTok{  }\KeywordTok{select}\NormalTok{(Name, Year, Count)}

\NormalTok{girls_and_boys <-}\StringTok{ }\KeywordTok{inner_join}\NormalTok{(bn_girls, }
\NormalTok{                             bn_boys,}
                             \DataTypeTok{by =} \KeywordTok{c}\NormalTok{(}\StringTok{"Year"}\NormalTok{, }\StringTok{"Name"}\NormalTok{))}
\KeywordTok{head}\NormalTok{(girls_and_boys)}
\end{Highlighting}
\end{Shaded}

\begin{Shaded}
\begin{Highlighting}[]
\NormalTok{girls_and_boys <-}\StringTok{ }
\StringTok{  }\NormalTok{girls_and_boys }\OperatorTok
\StringTok{  }\KeywordTok{mutate}\NormalTok{(}\DataTypeTok{Product =}\NormalTok{ Count.x }\OperatorTok{*}\StringTok{ }\NormalTok{Count.y,}
         \DataTypeTok{Rank =} \KeywordTok{rank}\NormalTok{(}\OperatorTok{-}\NormalTok{Product)) }\OperatorTok
\StringTok{  }\KeywordTok{filter}\NormalTok{(Rank }\OperatorTok{==}\StringTok{ }\DecValTok{1}\NormalTok{)}

\KeywordTok{head}\NormalTok{(girls_and_boys)}
\end{Highlighting}
\end{Shaded}

\subsection{Ex 5: prototype}\label{ex-5-prototype}

\begin{Shaded}
\begin{Highlighting}[]
\NormalTok{## 1.  Filter the baby_names data, retaining only the 10 most }
\NormalTok{##     popular girl and boy names for each year.}
\end{Highlighting}
\end{Shaded}

\begin{Shaded}
\begin{Highlighting}[]
\NormalTok{most_popular <-}\StringTok{ }
\StringTok{  }\NormalTok{baby_names }\OperatorTok\StringTok{ }
\StringTok{  }\KeywordTok{group_by}\NormalTok{(Year, Sex) }\OperatorTok
\StringTok{  }\KeywordTok{filter}\NormalTok{(Rank }\OperatorTok{<=}\StringTok{ }\DecValTok{10}\NormalTok{)}

\KeywordTok{head}\NormalTok{(most_popular, }\DataTypeTok{n =} \DecValTok{10}\NormalTok{)}
\end{Highlighting}
\end{Shaded}

\begin{Shaded}
\begin{Highlighting}[]
\NormalTok{## 2.  Summarize the data produced in step one to calculate the total}
\NormalTok{##     Proportion of boys and girls given one of the top 10 names}
\NormalTok{##     each year.}
\end{Highlighting}
\end{Shaded}

\begin{Shaded}
\begin{Highlighting}[]
\NormalTok{top10 <-}\StringTok{ }
\StringTok{  }\NormalTok{most_popular }\OperatorTok\StringTok{ }\CommentTok{# it is already grouped by Year and Sex}
\StringTok{  }\KeywordTok{summarize}\NormalTok{(}\DataTypeTok{TotalProportion =} \KeywordTok{sum}\NormalTok{(Proportion))}
\end{Highlighting}
\end{Shaded}

\begin{Shaded}
\begin{Highlighting}[]
\NormalTok{## 3.  Plot the data produced in step 2, with year on the x-axis}
\NormalTok{##     and total proportion on the y axis. Color by sex.}
\end{Highlighting}
\end{Shaded}

\begin{Shaded}
\begin{Highlighting}[]
\KeywordTok{qplot}\NormalTok{(}\DataTypeTok{x =}\NormalTok{ Year, }
      \DataTypeTok{y =}\NormalTok{ TotalProportion, }
      \DataTypeTok{color =}\NormalTok{ Sex,}
      \DataTypeTok{data =}\NormalTok{ top10,}
      \DataTypeTok{geom =} \StringTok{"line"}\NormalTok{)}
\end{Highlighting}
\end{Shaded}

\section{Wrap-up}\label{wrap-up-1}

\subsection{Feedback}\label{feedback-1}

These workshops are a work-in-progress, please provide any feedback to:
\href{mailto:help@iq.harvard.edu}{\nolinkurl{help@iq.harvard.edu}}

\subsection{Resources}\label{resources-1}

\begin{itemize}
\tightlist
\item
  IQSS

  \begin{itemize}
  \tightlist
  \item
    Workshops: \url{https://dss.iq.harvard.edu/workshop-materials}
  \item
    Data Science Services: \url{https://dss.iq.harvard.edu/}
  \item
    Research Computing Environment:
    \url{https://iqss.github.io/dss-rce/}
  \end{itemize}
\item
  HBS

  \begin{itemize}
  \tightlist
  \item
    Research Computing Services workshops:
    \url{https://training.rcs.hbs.org/workshops}
  \item
    Other HBS RCS resources:
    \url{https://training.rcs.hbs.org/workshop-materials}
  \item
    RCS consulting email: \url{mailto:research@hbs.edu}
  \end{itemize}
\item
  Software (all free!):

  \begin{itemize}
  \tightlist
  \item
    R and R package download: \url{http://cran.r-project.org}
  \item
    Rstudio download: \url{http://rstudio.org}
  \item
    ESS (emacs R package): \url{http://ess.r-project.org/}
  \end{itemize}
\item
  Online tutorials

  \begin{itemize}
  \tightlist
  \item
    \url{http://www.codeschool.com/courses/try-r}
  \item
    \url{http://www.datacamp.org}
  \item
    \url{https://rmarkdown.rstudio.com/lesson-1.html}
  \item
    \url{http://swirlstats.com/}
  \item
    \url{http://r4ds.had.co.nz/}
  \end{itemize}
\item
  Getting help:

  \begin{itemize}
  \tightlist
  \item
    Documentation and tutorials:
    \url{http://cran.r-project.org/other-docs.html}
  \item
    Recommended R packages by topic:
    \url{http://cran.r-project.org/web/views/}
  \item
    Mailing list: \url{https://stat.ethz.ch/mailman/listinfo/r-help}
  \item
    StackOverflow: \url{http://stackoverflow.com/questions/tagged/r}
  \item
    R-Bloggers: \url{https://www.r-bloggers.com/}
  \end{itemize}
\item
  Coming from \ldots{}

  \begin{itemize}
  \tightlist
  \item
    Stata: \url{http://www.princeton.edu/~otorres/RStata.pdf}
  \item
    SAS/SPSS: \url{http://r4stats.com/books/free-version/}
  \item
    Matlab: \url{http://www.math.umaine.edu/~hiebeler/comp/matlabR.pdf}
  \item
    Python:
    \url{http://mathesaurus.sourceforge.net/matlab-python-xref.pdf}
  \end{itemize}
\end{itemize}

\chapter{R Regression Models}\label{r-regression-models}

\textbf{Topics}

\begin{itemize}
\tightlist
\item
  R formula interface
\item
  Run and interpret variety of regression models in R
\item
  Factor contrasts to test specific hypotheses
\item
  Model comparisons
\item
  Predicted marginal effects
\end{itemize}

\section{Setup}\label{setup-1}

\subsection{Software \& materials}\label{software-materials-1}

You should have R and RStudio installed --- if not:

\begin{itemize}
\tightlist
\item
  Download and install R: \url{http://cran.r-project.org}
\item
  Download and install RStudio:
  \url{https://www.rstudio.com/products/rstudio/download/\#download}
\end{itemize}

Download materials:

\begin{itemize}
\tightlist
\item
  Download class materials at
  \url{https://github.com/IQSS/dss-workshops-redux/raw/master/R/Rmodels.zip}
\item
  Extract materials from the zipped directory \texttt{Rmodels.zip}
  (Right-click =\textgreater{} Extract All on Windows, double-click on
  Mac) and move them to your desktop!
\end{itemize}

Start RStudio and create a new project:

\begin{itemize}
\tightlist
\item
  On Windows click the start button and search for RStudio. On Mac
  RStudio will be in your applications folder.
\item
  In Rstudio go to \texttt{File\ -\textgreater{}\ New\ Project}.
\item
  Choose \texttt{Existing\ Directory} and browse to the \texttt{Rmodels}
  directory.
\item
  Choose \texttt{File\ -\textgreater{}\ Open\ File} and select the blank
  version of the \texttt{.Rmd} file.
\end{itemize}

While R's built-in packages are powerful, in recent years there has been
a big surge in well-designed \emph{contributed packages} for R. In
particular, a collection of R packages called
\href{https://www.tidyverse.org/}{tidyverse} have been designed
specifically for data science. All packages included in
\texttt{tidyverse} share an underlying design philosophy, grammar, and
data structures. We will use \texttt{tidyverse} packages throughout the
workshop, so let's install them now:

\begin{Shaded}
\begin{Highlighting}[]
\CommentTok{# install.packages("tidyverse")}
\KeywordTok{library}\NormalTok{(tidyverse)}
\end{Highlighting}
\end{Shaded}

\subsection{Goals}\label{goals-1}

Class Structure and Organization:

\begin{itemize}
\tightlist
\item
  Ask questions at any time. Really!
\item
  Collaboration is encouraged - please spend a minute introducing
  yourself to your neighbors!
\end{itemize}

This is an intermediate R course:

\begin{itemize}
\tightlist
\item
  Assumes working knowledge of R
\item
  Relatively fast-paced
\item
  This is not a statistics course! We assume you know the theory behind
  the models
\end{itemize}

\section{Before fitting a model}\label{before-fitting-a-model}

\subsection{Load the data}\label{load-the-data}

List the data files we're going to work with:

\begin{Shaded}
\begin{Highlighting}[]
\KeywordTok{list.files}\NormalTok{(}\StringTok{"dataSets"}\NormalTok{)}
\end{Highlighting}
\end{Shaded}

We're going to use the \texttt{states} data first, which originally
appeared in \emph{Statistics with Stata} by Lawrence C. Hamilton.

\begin{Shaded}
\begin{Highlighting}[]
  \CommentTok{# read the states data}
\NormalTok{  states_data <-}\StringTok{ }\KeywordTok{read_rds}\NormalTok{(}\StringTok{"dataSets/states.rds"}\NormalTok{) }
\end{Highlighting}
\end{Shaded}

\begin{longtable}[]{@{}ll@{}}
\toprule
Variable & Description\tabularnewline
\midrule
\endhead
csat & Mean composite SAT score\tabularnewline
expense & Per pupil expenditures\tabularnewline
percent & \% HS graduates taking SAT\tabularnewline
income & Median household income, \$1,000\tabularnewline
region & Geographic region: West, N. East, South, Midwest\tabularnewline
house & House '91 environ. voting, \%\tabularnewline
senate & Senate '91 environ. voting, \%\tabularnewline
energy & Per capita energy consumed, Btu\tabularnewline
metro & Metropolitan area population, \%\tabularnewline
waste & Per capita solid waste, tons\tabularnewline
\bottomrule
\end{longtable}

\subsection{Examine the data}\label{examine-the-data}

Start by examining the data to check for problems.

\begin{Shaded}
\begin{Highlighting}[]
  \CommentTok{# summary of expense and csat columns, all rows}
\NormalTok{  sts_ex_sat <-}\StringTok{ }\KeywordTok{subset}\NormalTok{(states_data, }\DataTypeTok{select =} \KeywordTok{c}\NormalTok{(}\StringTok{"expense"}\NormalTok{, }\StringTok{"csat"}\NormalTok{))}
  \KeywordTok{summary}\NormalTok{(sts.ex.sat)}
  \CommentTok{# correlation between expense and csat}
  \KeywordTok{cor}\NormalTok{(sts_ex_sat) }
\end{Highlighting}
\end{Shaded}

\subsection{Plot the data}\label{plot-the-data}

Plot the data to look for multivariate outliers, non-linear
relationships etc.

\begin{Shaded}
\begin{Highlighting}[]
  \CommentTok{# scatter plot of expense vs csat}
  \KeywordTok{plot}\NormalTok{(sts_ex_sat)}
\end{Highlighting}
\end{Shaded}

\begin{figure}
\centering
\includegraphics{images/statesCorr1.png}
\caption{}
\end{figure}

\section{Models with continuous
outcomes}\label{models-with-continuous-outcomes}

\begin{itemize}
\tightlist
\item
  Ordinary least squares (OLS) regression models can be fit with the
  \texttt{lm()} function
\item
  For example, we can use \texttt{lm} to predict SAT scores based on
  per-pupal expenditures:
\end{itemize}

\begin{Shaded}
\begin{Highlighting}[]
  \CommentTok{# Fit our regression model}
\NormalTok{  sat_mod <-}\StringTok{ }\KeywordTok{lm}\NormalTok{(csat }\OperatorTok{~}\StringTok{ }\NormalTok{expense, }\CommentTok{# regression formula}
                \DataTypeTok{data=}\NormalTok{states_data) }\CommentTok{# data }
                
  \CommentTok{# Summarize and print the results}
  \KeywordTok{summary}\NormalTok{(sat_mod) }\CommentTok{# show regression coefficients table}
\end{Highlighting}
\end{Shaded}

\subsection{\texorpdfstring{Why is the association between expense \&
SAT scores
\emph{negative}?}{Why is the association between expense \& SAT scores negative?}}\label{why-is-the-association-between-expense-sat-scores-negative}

Many people find it surprising that the per-capita expenditure on
students is negatively related to SAT scores. The beauty of multiple
regression is that we can try to pull these apart. What would the
association between expense and SAT scores be if there were no
difference among the states in the percentage of students taking the
SAT?

\begin{Shaded}
\begin{Highlighting}[]
  \KeywordTok{lm}\NormalTok{(csat }\OperatorTok{~}\StringTok{ }\NormalTok{expense }\OperatorTok{+}\StringTok{ }\NormalTok{percent, }\DataTypeTok{data =}\NormalTok{ states_data) }\OperatorTok\StringTok{ }
\StringTok{  }\KeywordTok{summary}\NormalTok{()}
\end{Highlighting}
\end{Shaded}

\subsection{\texorpdfstring{The \texttt{lm} class \&
methods}{The lm class \& methods}}\label{the-lm-class-methods}

OK, we fit our model. Now what?

\begin{itemize}
\tightlist
\item
  Examine the model object:
\end{itemize}

\begin{Shaded}
\begin{Highlighting}[]
  \KeywordTok{class}\NormalTok{(sat_mod)}
  \KeywordTok{str}\NormalTok{(sat_mod)}
  \KeywordTok{names}\NormalTok{(sat_mod)}
  \KeywordTok{methods}\NormalTok{(}\DataTypeTok{class =} \KeywordTok{class}\NormalTok{(sat_mod))}
\end{Highlighting}
\end{Shaded}

\begin{itemize}
\tightlist
\item
  Use function methods to get more information about the fit
\end{itemize}

\begin{Shaded}
\begin{Highlighting}[]
  \KeywordTok{summary}\NormalTok{(sat_mod)}
  \KeywordTok{summary}\NormalTok{(sat_mod) }\OperatorTok\StringTok{ }\KeywordTok{coef}\NormalTok{()}
  \KeywordTok{methods}\NormalTok{(}\StringTok{"summary"}\NormalTok{)}
  \KeywordTok{confint}\NormalTok{(sat_mod)}
\end{Highlighting}
\end{Shaded}

\subsection{OLS regression
assumptions}\label{ols-regression-assumptions}

\begin{itemize}
\tightlist
\item
  OLS regression relies on several assumptions, including that the
  residuals are normally distributed and homoscedastic, the errors are
  independent and the relationships are linear.
\item
  Investigate these assumptions visually by plotting your model:
\end{itemize}

\begin{Shaded}
\begin{Highlighting}[]
  \KeywordTok{par}\NormalTok{(}\DataTypeTok{mfrow =} \KeywordTok{c}\NormalTok{(}\DecValTok{2}\NormalTok{, }\DecValTok{2}\NormalTok{)) }
  \KeywordTok{plot}\NormalTok{(sat_mod)}
\end{Highlighting}
\end{Shaded}

\subsection{Comparing models}\label{comparing-models}

Do congressional voting patterns predict SAT scores over and above
expense? Fit two models and compare them:

\begin{Shaded}
\begin{Highlighting}[]
  \CommentTok{# fit another model, adding house and senate as predictors}
\NormalTok{  sat_voting_mod <-}\StringTok{ }\KeywordTok{lm}\NormalTok{(csat }\OperatorTok{~}\StringTok{ }\NormalTok{expense }\OperatorTok{+}\StringTok{ }\NormalTok{house }\OperatorTok{+}\StringTok{ }\NormalTok{senate,}
                        \DataTypeTok{data =} \KeywordTok{na.omit}\NormalTok{(states_data))}

\NormalTok{  sat_mod <-}\StringTok{ }\KeywordTok{update}\NormalTok{(sat_mod, }\DataTypeTok{data=}\KeywordTok{na.omit}\NormalTok{(states_data))}

  \CommentTok{# compare using the anova() function}
  \KeywordTok{anova}\NormalTok{(sat_mod, sat_voting_mod)}
  \KeywordTok{summary}\NormalTok{(sat_voting_mod) }\OperatorTok\StringTok{ }\KeywordTok{coef}\NormalTok{()}
\end{Highlighting}
\end{Shaded}

\section{Exercise 0}\label{exercise-0-1}

\textbf{Ordinary least squares regression}

Use the \emph{states.rds} data set. Fit a model predicting energy
consumed per capita (energy) from the percentage of residents living in
metropolitan areas (metro). Be sure to

\begin{enumerate}
\def\labelenumi{\arabic{enumi}.}
\tightlist
\item
  Examine/plot the data before fitting the model
\item
  Print and interpret the model \texttt{summary}
\item
  \texttt{plot} the model to look for deviations from modeling
  assumptions
\end{enumerate}

Select one or more additional predictors to add to your model and repeat
steps 1-3. Is this model significantly better than the model with
\emph{metro} as the only predictor?

\section{Interactions \& factors}\label{interactions-factors}

\subsection{Modeling interactions}\label{modeling-interactions}

Interactions allow us assess the extent to which the association between
one predictor and the outcome depends on a second predictor. For
example: Does the association between expense and SAT scores depend on
the median income in the state?

\begin{Shaded}
\begin{Highlighting}[]
    \CommentTok{# Add the interaction to the model}
\NormalTok{  sat_expense_by_percent <-}\StringTok{ }\KeywordTok{lm}\NormalTok{(csat }\OperatorTok{~}\StringTok{ }\NormalTok{expense }\OperatorTok{+}\StringTok{ }\NormalTok{income }\OperatorTok{+}\StringTok{ }\NormalTok{expense }\OperatorTok{:}\StringTok{ }\NormalTok{income, }\DataTypeTok{data=}\NormalTok{states_data)}
\NormalTok{  sat_expense_by_percent <-}\StringTok{ }\KeywordTok{lm}\NormalTok{(csat }\OperatorTok{~}\StringTok{ }\NormalTok{expense }\OperatorTok{*}\StringTok{ }\NormalTok{income, }\DataTypeTok{data=}\NormalTok{states_data) }\CommentTok{# same as above, but shorter syntax}

  \CommentTok{# Show the regression coefficients table}
  \KeywordTok{summary}\NormalTok{(sat_expense_by_percent) }\OperatorTok\StringTok{ }\KeywordTok{coef}\NormalTok{() }
\end{Highlighting}
\end{Shaded}

\subsection{Regression with categorical
predictors}\label{regression-with-categorical-predictors}

Let's try to predict SAT scores from region, a categorical variable.
Note that you must make sure R does not think your categorical variable
is numeric.

\begin{Shaded}
\begin{Highlighting}[]
  \CommentTok{# make sure R knows region is categorical}
  \KeywordTok{str}\NormalTok{(states_data}\OperatorTok{$}\NormalTok{region)}
\NormalTok{  states_data}\OperatorTok{$}\NormalTok{region <-}\StringTok{ }\KeywordTok{factor}\NormalTok{(states_data}\OperatorTok{$}\NormalTok{region)}

  \CommentTok{# Add region to the model}
\NormalTok{  sat_region <-}\StringTok{ }\KeywordTok{lm}\NormalTok{(csat }\OperatorTok{~}\StringTok{ }\NormalTok{region, }\DataTypeTok{data=}\NormalTok{states_data) }

  \CommentTok{# Show the results}
  \KeywordTok{summary}\NormalTok{(sat_region) }\OperatorTok\StringTok{ }\KeywordTok{coef}\NormalTok{() }\CommentTok{# show the regression coefficients table}
  \KeywordTok{anova}\NormalTok{(sat_region) }\CommentTok{# show ANOVA table}
\end{Highlighting}
\end{Shaded}

Again, \textbf{make sure to tell R which variables are categorical by
converting them to factors!}

\subsection{Setting factor reference groups \&
contrasts}\label{setting-factor-reference-groups-contrasts}

In the previous example we use the default contrasts for region. The
default in R is treatment contrasts, with the first level as the
reference. We can change the reference group or use another coding
scheme using the \texttt{C} function.

\begin{Shaded}
\begin{Highlighting}[]
  \CommentTok{# print default contrasts}
  \KeywordTok{contrasts}\NormalTok{(states_data}\OperatorTok{$}\NormalTok{region)}

  \CommentTok{# change the reference group}
\NormalTok{  states_data}\OperatorTok{$}\NormalTok{region <-}\StringTok{ }\KeywordTok{relevel}\NormalTok{(states_data}\OperatorTok{$}\NormalTok{region, }\DataTypeTok{ref =} \StringTok{"Midwest"}\NormalTok{)}
\NormalTok{  m1 <-}\StringTok{ }\KeywordTok{lm}\NormalTok{(csat }\OperatorTok{~}\StringTok{ }\NormalTok{region, }\DataTypeTok{data=}\NormalTok{states_data)}
  \KeywordTok{summary}\NormalTok{(m1) }\OperatorTok\StringTok{ }\KeywordTok{coef}\NormalTok{()}

  \CommentTok{# get all pairwise contrasts between means}
  \CommentTok{# install.packages("emmeans")}
  \KeywordTok{library}\NormalTok{(emmeans)}
\NormalTok{  means <-}\StringTok{ }\KeywordTok{emmeans}\NormalTok{(m1, }\DataTypeTok{specs =} \OperatorTok{~}\StringTok{ }\NormalTok{region)}
\NormalTok{  means}
  \KeywordTok{contrast}\NormalTok{(means, }\DataTypeTok{method =} \StringTok{"pairwise"}\NormalTok{)}

  \CommentTok{# change the coding scheme}
  \KeywordTok{lm}\NormalTok{(csat }\OperatorTok{~}\StringTok{ }\KeywordTok{C}\NormalTok{(region, contr.helmert), }\DataTypeTok{data=}\NormalTok{states_data) }\OperatorTok
\StringTok{  }\KeywordTok{summary}\NormalTok{() }\OperatorTok
\StringTok{  }\KeywordTok{coef}\NormalTok{()}
\end{Highlighting}
\end{Shaded}

See \texttt{?contr.treatment} for other coding schemes and also
\texttt{?contrasts} and \texttt{?relevel}.

\section{Exercise 1}\label{exercise-1-1}

\textbf{Interactions \& factors}

Use the states data set.

\begin{enumerate}
\def\labelenumi{\arabic{enumi}.}
\item
  Add on to the regression equation that you created in exercise 1 by
  generating an interaction term and testing the interaction.
\item
  Try adding region to the model. Are there significant differences
  across the four regions?
\end{enumerate}

\section{Models with binary outcomes}\label{models-with-binary-outcomes}

\subsection{Logistic regression}\label{logistic-regression}

This far we have used the \texttt{lm} function to fit our regression
models. \texttt{lm} is great, but limited--in particular it only fits
models for continuous dependent variables. For categorical dependent
variables we can use the \texttt{glm()} function.

For these models we will use a different dataset, drawn from the
National Health Interview Survey. From the
\href{http://www.cdc.gov/nchs/nhis.htm}{CDC website}:

\begin{quote}
The National Health Interview Survey (NHIS) has monitored the health of
the nation since 1957. NHIS data on a broad range of health topics are
collected through personal household interviews. For over 50 years, the
U.S. Census Bureau has been the data collection agent for the National
Health Interview Survey. Survey results have been instrumental in
providing data to track health status, health care access, and progress
toward achieving national health objectives.
\end{quote}

Load the National Health Interview Survey data:

\begin{Shaded}
\begin{Highlighting}[]
\NormalTok{  NH11 <-}\StringTok{ }\KeywordTok{read_rds}\NormalTok{(}\StringTok{"dataSets/NatHealth2011.rds"}\NormalTok{)}
\end{Highlighting}
\end{Shaded}

\subsection{Logistic regression
example}\label{logistic-regression-example}

Let's predict the probability of being diagnosed with hypertension based
on age, sex, sleep, and bmi

\begin{Shaded}
\begin{Highlighting}[]
  \KeywordTok{str}\NormalTok{(NH11}\OperatorTok{$}\NormalTok{hypev) }\CommentTok{# check stucture of hypev}
  \KeywordTok{levels}\NormalTok{(NH11}\OperatorTok{$}\NormalTok{hypev) }\CommentTok{# check levels of hypev}

  \CommentTok{# collapse all missing values to NA}
\NormalTok{  NH11}\OperatorTok{$}\NormalTok{hypev <-}\StringTok{ }\KeywordTok{factor}\NormalTok{(NH11}\OperatorTok{$}\NormalTok{hypev, }\DataTypeTok{levels=}\KeywordTok{c}\NormalTok{(}\StringTok{"2 No"}\NormalTok{, }\StringTok{"1 Yes"}\NormalTok{))}

  \CommentTok{# run our regression model}
\NormalTok{  hyp_out <-}\StringTok{ }\KeywordTok{glm}\NormalTok{(hypev }\OperatorTok{~}\StringTok{ }\NormalTok{age_p }\OperatorTok{+}\StringTok{ }\NormalTok{sex }\OperatorTok{+}\StringTok{ }\NormalTok{sleep }\OperatorTok{+}\StringTok{ }\NormalTok{bmi,}
                \DataTypeTok{data =}\NormalTok{ NH11, }\DataTypeTok{family =} \KeywordTok{binomial}\NormalTok{(}\DataTypeTok{link =} \StringTok{"logit"}\NormalTok{))}
  \KeywordTok{summary}\NormalTok{(hyp_out) }\OperatorTok\StringTok{ }\KeywordTok{coef}\NormalTok{()}
\end{Highlighting}
\end{Shaded}

\subsection{Logistic regression
coefficients}\label{logistic-regression-coefficients}

Generalized linear models use link functions, so raw coefficients are
difficult to interpret. For example, the age coefficient of .06 in the
previous model tells us that for every one unit increase in age, the log
odds of hypertension diagnosis increases by 0.06. Since most of us are
not used to thinking in log odds this is not too helpful!

One solution is to transform the coefficients to make them easier to
interpret

\begin{Shaded}
\begin{Highlighting}[]
\NormalTok{  hyp_out_tab <-}\StringTok{ }\KeywordTok{summary}\NormalTok{(hyp_out) }\OperatorTok\StringTok{ }\KeywordTok{coef}\NormalTok{()}
\NormalTok{  hyp_out_tab[, }\StringTok{"Estimate"}\NormalTok{] <-}\StringTok{ }\KeywordTok{coef}\NormalTok{(hyp_out) }\OperatorTok\StringTok{ }\KeywordTok{exp}\NormalTok{()}
\NormalTok{  hyp_out_tab}
\end{Highlighting}
\end{Shaded}

\subsection{Packages for computing \& graphing predicted
values}\label{packages-for-computing-graphing-predicted-values}

Instead of doing all this ourselves, we can use the effects package to
compute quantities of interest for us.

\begin{Shaded}
\begin{Highlighting}[]
  \KeywordTok{library}\NormalTok{(effects)}
\NormalTok{  eff <-}\StringTok{ }\KeywordTok{allEffects}\NormalTok{(hyp_out)}
\NormalTok{  eff2 <-}\StringTok{ }\KeywordTok{allEffects}\NormalTok{(hyp_out, }\DataTypeTok{xlevels =} \KeywordTok{list}\NormalTok{(age_p, }\KeywordTok{seq}\NormalTok{(}\DecValTok{20}\NormalTok{, }\DecValTok{80}\NormalTok{, }\DataTypeTok{by =} \DecValTok{5}\NormalTok{)))}
  \KeywordTok{plot}\NormalTok{(eff)}
  \KeywordTok{as.data.frame}\NormalTok{(eff) }\CommentTok{# confidence intervals}
\end{Highlighting}
\end{Shaded}

\begin{figure}
\centering
\includegraphics{images/effects1.png}
\caption{}
\end{figure}

\section{Exercise 2}\label{exercise-2}

\textbf{Logistic regression}

Use the NH11 data set that we loaded earlier.

\begin{enumerate}
\def\labelenumi{\arabic{enumi}.}
\tightlist
\item
  Use glm to conduct a logistic regression to predict ever worked
  (everwrk) using age (age\_p) and marital status (r\_maritl).
\item
  Predict the probability of working for each level of marital status.
\end{enumerate}

Note that the data is not perfectly clean and ready to be modeled. You
will need to clean up at least some of the variables before fitting the
model.

\section{Multilevel modeling}\label{multilevel-modeling}

\subsection{Multilevel modeling
overview}\label{multilevel-modeling-overview}

\begin{itemize}
\tightlist
\item
  Multi-level (AKA hierarchical) models are a type of mixed-effects
  models
\item
  Used to model variation due to group membership where the goal is to
  generalize to a population of groups
\item
  Can model different intercepts and/or slopes for each group
\item
  Mixed-effecs models include two types of predictors: fixed-effects and
  random effects
\item
  Fixed-effects -- observed levels are of direct interest (.e.g, sex,
  political party\ldots{})
\item
  Random-effects -- observed levels not of direct interest: goal is to
  make inferences to a population represented by observed levels
\item
  In R the lme4 package is the most popular for mixed effects models
\item
  Use the \texttt{lmer} function for liner mixed models, \texttt{glmer}
  for generalized mixed models
\end{itemize}

\begin{Shaded}
\begin{Highlighting}[]
  \KeywordTok{library}\NormalTok{(lme4)}
\end{Highlighting}
\end{Shaded}

\subsection{The Exam data}\label{the-exam-data}

The Exam data set contans exam scores of 4,059 students from 65 schools
in Inner London. The variable names are as follows:

\begin{longtable}[]{@{}ll@{}}
\toprule
Variable & Description\tabularnewline
\midrule
\endhead
school & School ID - a factor.\tabularnewline
normexam & Normalized exam score.\tabularnewline
standLRT & Standardised LR test score.\tabularnewline
student & Student id (within school) - a factor\tabularnewline
\bottomrule
\end{longtable}

\begin{Shaded}
\begin{Highlighting}[]
\NormalTok{  Exam <-}\StringTok{ }\KeywordTok{read_rds}\NormalTok{(}\StringTok{"dataSets/Exam.rds"}\NormalTok{)}
\end{Highlighting}
\end{Shaded}

\subsection{The null model \& ICC}\label{the-null-model-icc}

As a preliminary step it is often useful to partition the variance in
the dependent variable into the various levels. This can be accomplished
by running a null model (i.e., a model with a random effects grouping
structure, but no fixed-effects predictors).

\begin{Shaded}
\begin{Highlighting}[]
  \CommentTok{# null model, grouping by school but not fixed effects.}
\NormalTok{  Norm1 <-}\KeywordTok{lmer}\NormalTok{(normexam }\OperatorTok{~}\StringTok{ }\DecValTok{1} \OperatorTok{+}\StringTok{ }\NormalTok{(}\DecValTok{1} \OperatorTok{|}\StringTok{ }\NormalTok{school),}
                \DataTypeTok{data=}\KeywordTok{na.omit}\NormalTok{(Exam), }\DataTypeTok{REML =} \OtherTok{FALSE}\NormalTok{)}
  \KeywordTok{summary}\NormalTok{(Norm1)}
\end{Highlighting}
\end{Shaded}

The is .169/(.169 + .848) = .17: 17\% of the variance is at the school
level.

There is no consensus on how to calculate p-values for MLMs; hence why
they are omitted from the \texttt{lme4} output. But, if you really need
p-values, the \texttt{lmerTest} package will calculate p-values for you
(using the Satterthwaite approximation) and you can use the same model
syntax:

\begin{Shaded}
\begin{Highlighting}[]
  \CommentTok{# install.packages("lmerTest")}
\NormalTok{  Norm1_test <-}\StringTok{ }\NormalTok{lmerTest}\OperatorTok{::}\KeywordTok{lmer}\NormalTok{(normexam }\OperatorTok{~}\StringTok{ }\DecValTok{1} \OperatorTok{+}\StringTok{ }\NormalTok{(}\DecValTok{1} \OperatorTok{|}\StringTok{ }\NormalTok{school),}
                      \DataTypeTok{data=}\KeywordTok{na.omit}\NormalTok{(Exam), }\DataTypeTok{REML =} \OtherTok{FALSE}\NormalTok{)}
  \KeywordTok{summary}\NormalTok{(Norm1_test)}
\end{Highlighting}
\end{Shaded}

\subsection{Adding fixed-effects
predictors}\label{adding-fixed-effects-predictors}

Predict exam scores from student's standardized tests scores

\begin{Shaded}
\begin{Highlighting}[]
\NormalTok{  Norm2 <-}\KeywordTok{lmer}\NormalTok{(normexam }\OperatorTok{~}\StringTok{ }\DecValTok{1} \OperatorTok{+}\StringTok{ }\NormalTok{standLRT }\OperatorTok{+}\StringTok{ }\NormalTok{(}\DecValTok{1} \OperatorTok{|}\StringTok{ }\NormalTok{school),}
               \DataTypeTok{data=}\KeywordTok{na.omit}\NormalTok{(Exam), }\DataTypeTok{REML =} \OtherTok{FALSE}\NormalTok{) }
  \KeywordTok{summary}\NormalTok{(Norm2) }
\end{Highlighting}
\end{Shaded}

\subsection{Multiple degree of freedom
comparisons}\label{multiple-degree-of-freedom-comparisons}

As with \texttt{lm} and \texttt{glm} models, you can compare the two
\texttt{lmer} models using the \texttt{anova} function.

\begin{Shaded}
\begin{Highlighting}[]
  \KeywordTok{anova}\NormalTok{(Norm1, Norm2)}
\end{Highlighting}
\end{Shaded}

\subsection{Random slopes}\label{random-slopes}

Add a random effect of students' standardized test scores as well. Now
in addition to estimating the distribution of intercepts across schools,
we also estimate the distribution of the slope of exam on standardized
test.

\begin{Shaded}
\begin{Highlighting}[]
\NormalTok{  Norm3 <-}\StringTok{ }\KeywordTok{lmer}\NormalTok{(normexam }\OperatorTok{~}\StringTok{ }\DecValTok{1} \OperatorTok{+}\StringTok{ }\NormalTok{standLRT }\OperatorTok{+}\StringTok{ }\NormalTok{(}\DecValTok{1} \OperatorTok{+}\StringTok{ }\NormalTok{standLRT }\OperatorTok{|}\StringTok{ }\NormalTok{school), }
                \DataTypeTok{data =} \KeywordTok{na.omit}\NormalTok{(Exam), }\DataTypeTok{REML =} \OtherTok{FALSE}\NormalTok{) }
  \KeywordTok{summary}\NormalTok{(Norm3) }
\end{Highlighting}
\end{Shaded}

\subsection{Test the significance of the random
slope}\label{test-the-significance-of-the-random-slope}

To test the significance of a random slope just compare models with and
without the random slope term

\begin{Shaded}
\begin{Highlighting}[]
  \KeywordTok{anova}\NormalTok{(Norm2, Norm3) }
\end{Highlighting}
\end{Shaded}

\section{Exercise 3}\label{exercise-3-1}

\textbf{Multilevel modeling}

Use the dataset, bh1996:

\begin{Shaded}
\begin{Highlighting}[]
\KeywordTok{data}\NormalTok{(bh1996, }\DataTypeTok{package=}\StringTok{"multilevel"}\NormalTok{)}
\end{Highlighting}
\end{Shaded}

From the data documentation:

\begin{quote}
Variables are Leadership Climate (LEAD), Well-Being (WBEING), and Work
Hours (HRS). The group identifier is named ``GRP''.
\end{quote}

\begin{enumerate}
\def\labelenumi{\arabic{enumi}.}
\tightlist
\item
  Create a null model predicting wellbeing (``WBEING'')
\item
  Calculate the ICC for your null model
\item
  Run a second multi-level model that adds two individual-level
  predictors, average number of hours worked (``HRS'') and leadership
  skills (``LEAD'') to the model and interpret your output.
\item
  Now, add a random effect of average number of hours worked (``HRS'')
  to the model and interpret your output. Test the significance of this
  random term.
\end{enumerate}

\section{Exercise solutions}\label{exercise-solutions-1}

\subsection{Ex 0: prototype}\label{ex-0-prototype-1}

Use the \emph{states.rds} data set.

\begin{Shaded}
\begin{Highlighting}[]
\NormalTok{  states <-}\StringTok{ }\KeywordTok{read_rds}\NormalTok{(}\StringTok{"dataSets/states.rds"}\NormalTok{)}
\end{Highlighting}
\end{Shaded}

Fit a model predicting energy consumed per capita (energy) from the
percentage of residents living in metropolitan areas (metro). Be sure to

\begin{enumerate}
\def\labelenumi{\arabic{enumi}.}
\tightlist
\item
  Examine/plot the data before fitting the model
\end{enumerate}

\begin{Shaded}
\begin{Highlighting}[]
\NormalTok{  states_en_met <-}\StringTok{ }\KeywordTok{subset}\NormalTok{(states, }\DataTypeTok{select =} \KeywordTok{c}\NormalTok{(}\StringTok{"metro"}\NormalTok{, }\StringTok{"energy"}\NormalTok{))}
  \KeywordTok{summary}\NormalTok{(states_en_met)}
  \KeywordTok{plot}\NormalTok{(states_en_met)}
  \KeywordTok{cor}\NormalTok{(states_en_met, }\DataTypeTok{use=}\StringTok{"pairwise"}\NormalTok{)}
\end{Highlighting}
\end{Shaded}

\begin{enumerate}
\def\labelenumi{\arabic{enumi}.}
\setcounter{enumi}{1}
\tightlist
\item
  Print and interpret the model \texttt{summary}
\end{enumerate}

\begin{Shaded}
\begin{Highlighting}[]
\NormalTok{  mod_en_met <-}\StringTok{ }\KeywordTok{lm}\NormalTok{(energy }\OperatorTok{~}\StringTok{ }\NormalTok{metro, }\DataTypeTok{data =}\NormalTok{ states)}
  \KeywordTok{summary}\NormalTok{(mod_en_met)}
\end{Highlighting}
\end{Shaded}

\begin{enumerate}
\def\labelenumi{\arabic{enumi}.}
\setcounter{enumi}{2}
\tightlist
\item
  \texttt{plot} the model to look for deviations from modeling
  assumptions
\end{enumerate}

\begin{Shaded}
\begin{Highlighting}[]
  \KeywordTok{plot}\NormalTok{(mod_en_met)}
\end{Highlighting}
\end{Shaded}

Select one or more additional predictors to add to your model and repeat
steps 1-3. Is this model significantly better than the model with
\emph{metro} as the only predictor?

\begin{Shaded}
\begin{Highlighting}[]
\NormalTok{  states_en_met_pop_wst <-}\StringTok{ }\KeywordTok{subset}\NormalTok{(states, }\DataTypeTok{select =} \KeywordTok{c}\NormalTok{(}\StringTok{"energy"}\NormalTok{, }\StringTok{"metro"}\NormalTok{, }\StringTok{"pop"}\NormalTok{, }\StringTok{"waste"}\NormalTok{))}
  \KeywordTok{summary}\NormalTok{(states_en_met_pop_wst)}
  \KeywordTok{plot}\NormalTok{(states_en_met_pop_wst)}
  \KeywordTok{cor}\NormalTok{(states_en_met_pop_wst, }\DataTypeTok{use =} \StringTok{"pairwise"}\NormalTok{)}

\NormalTok{  mod_en_met_pop_waste <-}\StringTok{ }\KeywordTok{lm}\NormalTok{(energy }\OperatorTok{~}\StringTok{ }\NormalTok{metro }\OperatorTok{+}\StringTok{ }\NormalTok{pop }\OperatorTok{+}\StringTok{ }\NormalTok{waste, }\DataTypeTok{data =}\NormalTok{ states)}
  \KeywordTok{summary}\NormalTok{(mod_en_met_pop_waste)}
  \KeywordTok{anova}\NormalTok{(mod_en_met, mod_en_met_pop_waste)}
\end{Highlighting}
\end{Shaded}

\subsection{Ex 1: prototype}\label{ex-1-prototype-1}

Use the states data set.

\begin{enumerate}
\def\labelenumi{\arabic{enumi}.}
\tightlist
\item
  Add on to the regression equation that you created in exercise 1 by
  generating an interaction term and testing the interaction.
\end{enumerate}

\begin{Shaded}
\begin{Highlighting}[]
\NormalTok{  mod_en_metro_by_waste <-}\StringTok{ }\KeywordTok{lm}\NormalTok{(energy }\OperatorTok{~}\StringTok{ }\NormalTok{metro }\OperatorTok{*}\StringTok{ }\NormalTok{waste, }\DataTypeTok{data =}\NormalTok{ states)}
\end{Highlighting}
\end{Shaded}

\begin{enumerate}
\def\labelenumi{\arabic{enumi}.}
\setcounter{enumi}{1}
\tightlist
\item
  Try adding a region to the model. Are there significant differences
  across the four regions?
\end{enumerate}

\begin{Shaded}
\begin{Highlighting}[]
\NormalTok{  mod_en_region <-}\StringTok{ }\KeywordTok{lm}\NormalTok{(energy }\OperatorTok{~}\StringTok{ }\NormalTok{metro }\OperatorTok{*}\StringTok{ }\NormalTok{waste }\OperatorTok{+}\StringTok{ }\NormalTok{region, }\DataTypeTok{data =}\NormalTok{ states)}
  \KeywordTok{anova}\NormalTok{(mod_en_region)}
\end{Highlighting}
\end{Shaded}

\subsection{Ex 2: prototype}\label{ex-2-prototype}

Use the NH11 data set that we loaded earlier. Note that the data is not
perfectly clean and ready to be modeled. You will need to clean up at
least some of the variables before fitting the model.

\begin{enumerate}
\def\labelenumi{\arabic{enumi}.}
\tightlist
\item
  Use glm to conduct a logistic regression to predict ever worked
  (everwrk) using age (age\_p) and marital status (r\_maritl).
\end{enumerate}

\begin{Shaded}
\begin{Highlighting}[]
\NormalTok{  nh11_wrk_age_mar <-}\StringTok{ }\KeywordTok{subset}\NormalTok{(NH11, }\DataTypeTok{select =} \KeywordTok{c}\NormalTok{(}\StringTok{"everwrk"}\NormalTok{, }\StringTok{"age_p"}\NormalTok{, }\StringTok{"r_maritl"}\NormalTok{))}
  \KeywordTok{summary}\NormalTok{(nh11_wrk_age_mar)}

\NormalTok{  NH11 <-}\StringTok{ }\KeywordTok{transform}\NormalTok{(NH11,}
                    \DataTypeTok{everwrk =} \KeywordTok{factor}\NormalTok{(everwrk, }\DataTypeTok{levels =} \KeywordTok{c}\NormalTok{(}\StringTok{"1 Yes"}\NormalTok{, }\StringTok{"2 No"}\NormalTok{)),}
                    \DataTypeTok{r_maritl =} \KeywordTok{droplevels}\NormalTok{(r_maritl))}

\NormalTok{  mod_wk_age_mar <-}\StringTok{ }\KeywordTok{glm}\NormalTok{(everwrk }\OperatorTok{~}\StringTok{ }\NormalTok{age_p }\OperatorTok{+}\StringTok{ }\NormalTok{r_maritl, }\DataTypeTok{data =}\NormalTok{ NH11,}
                        \DataTypeTok{family =} \KeywordTok{binomial}\NormalTok{(}\DataTypeTok{link =} \StringTok{"logit"}\NormalTok{))}

  \KeywordTok{summary}\NormalTok{(mod_wk_age_mar)}
\end{Highlighting}
\end{Shaded}

\begin{enumerate}
\def\labelenumi{\arabic{enumi}.}
\setcounter{enumi}{1}
\tightlist
\item
  Predict the probability of working for each level of marital status.
\end{enumerate}

\begin{Shaded}
\begin{Highlighting}[]
  \KeywordTok{library}\NormalTok{(effects)}
  \KeywordTok{data.frame}\NormalTok{(}\KeywordTok{Effect}\NormalTok{(}\StringTok{"r_maritl"}\NormalTok{, mod_wk_age_mar))}
\end{Highlighting}
\end{Shaded}

\subsection{Ex 3: prototype}\label{ex-3-prototype-1}

Use the dataset, bh1996:

\begin{Shaded}
\begin{Highlighting}[]
  \KeywordTok{data}\NormalTok{(bh1996, }\DataTypeTok{package=}\StringTok{"multilevel"}\NormalTok{)}
\end{Highlighting}
\end{Shaded}

From the data documentation:

\begin{quote}
Variables are Cohesion (COHES), Leadership Climate (LEAD), Well-Being
(WBEING) and Work Hours (HRS). The group identifier is named ``GRP''.
\end{quote}

\begin{enumerate}
\def\labelenumi{\arabic{enumi}.}
\tightlist
\item
  Create a null model predicting wellbeing (``WBEING'')
\end{enumerate}

\begin{Shaded}
\begin{Highlighting}[]
  \KeywordTok{library}\NormalTok{(lme4)}
\NormalTok{  mod_grp0 <-}\StringTok{ }\KeywordTok{lmer}\NormalTok{(WBEING }\OperatorTok{~}\StringTok{ }\DecValTok{1} \OperatorTok{+}\StringTok{ }\NormalTok{(}\DecValTok{1} \OperatorTok{|}\StringTok{ }\NormalTok{GRP), }\DataTypeTok{data =}\NormalTok{ bh1996)}
  \KeywordTok{summary}\NormalTok{(mod_grp0)}
\end{Highlighting}
\end{Shaded}

\begin{enumerate}
\def\labelenumi{\arabic{enumi}.}
\setcounter{enumi}{2}
\tightlist
\item
  Run a second multi-level model that adds two individual-level
  predictors, average number of hours worked (``HRS'') and leadership
  skills (``LEAD'') to the model and interpret your output.
\end{enumerate}

\begin{Shaded}
\begin{Highlighting}[]
\NormalTok{  mod_grp1 <-}\StringTok{ }\KeywordTok{lmer}\NormalTok{(WBEING }\OperatorTok{~}\StringTok{ }\NormalTok{HRS }\OperatorTok{+}\StringTok{ }\NormalTok{LEAD }\OperatorTok{+}\StringTok{ }\NormalTok{(}\DecValTok{1} \OperatorTok{|}\StringTok{ }\NormalTok{GRP), }\DataTypeTok{data =}\NormalTok{ bh1996)}
  \KeywordTok{summary}\NormalTok{(mod_grp1)}
\end{Highlighting}
\end{Shaded}

\begin{enumerate}
\def\labelenumi{\arabic{enumi}.}
\setcounter{enumi}{2}
\tightlist
\item
  Now, add a random effect of average number of hours worked (``HRS'')
  to the model and interpret your output. Test the significance of this
  random term.
\end{enumerate}

\begin{Shaded}
\begin{Highlighting}[]
\NormalTok{  mod_grp2 <-}\StringTok{ }\KeywordTok{lmer}\NormalTok{(WBEING }\OperatorTok{~}\StringTok{ }\NormalTok{HRS }\OperatorTok{+}\StringTok{ }\NormalTok{LEAD }\OperatorTok{+}\StringTok{ }\NormalTok{(}\DecValTok{1} \OperatorTok{+}\StringTok{ }\NormalTok{HRS }\OperatorTok{|}\StringTok{ }\NormalTok{GRP), }\DataTypeTok{data =}\NormalTok{ bh1996)}
  \KeywordTok{anova}\NormalTok{(mod_grp1, mod_grp2)}
\end{Highlighting}
\end{Shaded}

\section{Wrap-up}\label{wrap-up-2}

\subsection{Feedback}\label{feedback-2}

These workshops are a work in progress, please provide any feedback to:
\href{mailto:help@iq.harvard.edu}{\nolinkurl{help@iq.harvard.edu}}

\subsection{Resources}\label{resources-2}

\begin{itemize}
\tightlist
\item
  IQSS

  \begin{itemize}
  \tightlist
  \item
    Workshops: \url{https://dss.iq.harvard.edu/workshop-materials}
  \item
    Data Science Services: \url{https://dss.iq.harvard.edu/}
  \item
    Research Computing Environment:
    \url{https://iqss.github.io/dss-rce/}
  \end{itemize}
\item
  HBS

  \begin{itemize}
  \tightlist
  \item
    Research Computing Services workshops:
    \url{https://training.rcs.hbs.org/workshops}
  \item
    Other HBS RCS resources:
    \url{https://training.rcs.hbs.org/workshop-materials}
  \item
    RCS consulting email: \url{mailto:research@hbs.edu}
  \end{itemize}
\end{itemize}

\chapter{R Graphics}\label{r-graphics}

\textbf{Topics}

\begin{itemize}
\tightlist
\item
  R \texttt{ggplot2} package
\item
  Setup basic plots
\item
  Add and modify scales and legends
\item
  Manipulate plot labels
\item
  Change and create plot themes
\end{itemize}

\section{Setup}\label{setup-2}

\subsection{Software \& materials}\label{software-materials-2}

You should have R and RStudio installed --- if not:

\begin{itemize}
\tightlist
\item
  Download and install R: \url{http://cran.r-project.org}
\item
  Download and install RStudio:
  \url{https://www.rstudio.com/products/rstudio/download/\#download}
\end{itemize}

Download materials:

\begin{itemize}
\tightlist
\item
  Download class materials at
  \url{https://github.com/IQSS/dss-workshops-redux/raw/master/R/Rgraphics.zip}
\item
  Extract materials from the zipped directory \texttt{Rgraphics.zip}
  (Right-click =\textgreater{} Extract All on Windows, double-click on
  Mac) and move them to your desktop!
\end{itemize}

Start RStudio and create a new project:

\begin{itemize}
\tightlist
\item
  On Windows click the start button and search for RStudio. On Mac
  RStudio will be in your applications folder.
\item
  In Rstudio go to \texttt{File\ -\textgreater{}\ New\ Project}.
\item
  Choose \texttt{Existing\ Directory} and browse to the
  \texttt{Rgraphics} directory.
\item
  Choose \texttt{File\ -\textgreater{}\ Open\ File} and select the blank
  version of the \texttt{.Rmd} file.
\end{itemize}

While R's built-in packages are powerful, in recent years there has been
a big surge in well-designed \emph{contributed packages} for R. In
particular, a collection of R packages called
\href{https://www.tidyverse.org/}{tidyverse} have been designed
specifically for data science. All packages included in
\texttt{tidyverse} share an underlying design philosophy, grammar, and
data structures. We will use \texttt{tidyverse} packages throughout the
workshop, so let's install them now:

\begin{Shaded}
\begin{Highlighting}[]
\CommentTok{# install.packages("tidyverse")}
\KeywordTok{library}\NormalTok{(tidyverse)}
\end{Highlighting}
\end{Shaded}

\subsection{Goals}\label{goals-2}

Class Structure and Organization:

\begin{itemize}
\tightlist
\item
  Ask questions at any time. Really!
\item
  Collaboration is encouraged - please spend a minute introducing
  yourself to your neighbors!
\end{itemize}

This is an intermediate R course:

\begin{itemize}
\tightlist
\item
  Assumes working knowledge of R
\item
  Relatively fast-paced
\item
  Focus is on \texttt{ggplot2} graphics; other packages will not be
  covered
\end{itemize}

\subsection{Starting at the end}\label{starting-at-the-end}

By the end of the workshop you will be able to reproduce this graphic
from the Economist:

\begin{figure}
\centering
\includegraphics{images/Economist1.png}
\caption{img}
\end{figure}

\section{\texorpdfstring{Why
\texttt{ggplot2}?}{Why ggplot2?}}\label{why-ggplot2}

\texttt{ggplot2} is a package within in the \texttt{tidyverse} suite of
packages. Advantages of \texttt{ggplot2} include:

\begin{itemize}
\tightlist
\item
  consistent underlying \texttt{grammar\ of\ graphics} (Wilkinson, 2005)
\item
  plot specification at a high level of abstraction
\item
  very flexible
\item
  theme system for polishing plot appearance
\item
  mature and complete graphics system
\item
  many users, active mailing list
\end{itemize}

That said, there are some things you cannot (or should not) do with
\texttt{ggplot2}:

\begin{itemize}
\tightlist
\item
  3-dimensional graphics (see the \texttt{rgl} package)
\item
  Graph-theory type graphs (nodes/edges layout; see the \texttt{igraph}
  package)
\item
  Interactive graphics (see the \texttt{ggvis} package)
\end{itemize}

\subsection{What is the Grammar Of
Graphics?}\label{what-is-the-grammar-of-graphics}

The basic idea: independently specify plot building blocks and combine
them to create just about any kind of graphical display you want.
Building blocks of a graph include the following (\textbf{bold denotes
essential elements}):

\begin{itemize}
\tightlist
\item
  \textbf{data}
\item
  \textbf{aesthetic mapping}
\item
  \textbf{geometric object}
\item
  statistical transformations
\item
  scales
\item
  coordinate system
\item
  position adjustments
\item
  faceting
\item
  themes
\end{itemize}

\subsection{Example data: housing
prices}\label{example-data-housing-prices}

Let's look at housing prices.

\begin{Shaded}
\begin{Highlighting}[]
\NormalTok{housing <-}\StringTok{ }\KeywordTok{read_csv}\NormalTok{(}\StringTok{"dataSets/landdata-states.csv"}\NormalTok{)}
\KeywordTok{head}\NormalTok{(housing[}\DecValTok{1}\OperatorTok{:}\DecValTok{5}\NormalTok{])}
\end{Highlighting}
\end{Shaded}

\subsection{\texorpdfstring{\texttt{ggplot2} VS base
graphics}{ggplot2 VS base graphics}}\label{ggplot2-vs-base-graphics}

Compared to base graphics, \texttt{ggplot2}

\begin{itemize}
\tightlist
\item
  is more verbose for simple / canned graphics
\item
  is less verbose for complex / custom graphics
\item
  does not have methods (data should always be in a \texttt{data.frame})
\item
  uses a different system for adding plot elements
\end{itemize}

\subsection{For simple graphs}\label{for-simple-graphs}

Base graphics histogram example:

\begin{Shaded}
\begin{Highlighting}[]
\KeywordTok{hist}\NormalTok{(housing}\OperatorTok{$}\NormalTok{Home_Value)}
\end{Highlighting}
\end{Shaded}

\texttt{ggplot2} histogram example:

\begin{Shaded}
\begin{Highlighting}[]
\KeywordTok{library}\NormalTok{(ggplot2)}
\KeywordTok{ggplot}\NormalTok{(housing, }\KeywordTok{aes}\NormalTok{(}\DataTypeTok{x =}\NormalTok{ Home_Value)) }\OperatorTok{+}
\StringTok{  }\KeywordTok{geom_histogram}\NormalTok{()}
\end{Highlighting}
\end{Shaded}

\subsection{For more complex graphs}\label{for-more-complex-graphs}

Base graphics colored scatter plot example:

\begin{Shaded}
\begin{Highlighting}[]
\KeywordTok{plot}\NormalTok{(Home_Value }\OperatorTok{~}\StringTok{ }\NormalTok{Date,}
     \DataTypeTok{col =} \KeywordTok{factor}\NormalTok{(State),}
     \DataTypeTok{data =} \KeywordTok{filter}\NormalTok{(housing, State }\OperatorTok\StringTok{ }\KeywordTok{c}\NormalTok{(}\StringTok{"MA"}\NormalTok{, }\StringTok{"TX"}\NormalTok{)))}

\KeywordTok{legend}\NormalTok{(}\StringTok{"topleft"}\NormalTok{,}
       \DataTypeTok{legend =} \KeywordTok{c}\NormalTok{(}\StringTok{"MA"}\NormalTok{, }\StringTok{"TX"}\NormalTok{),}
       \DataTypeTok{col =} \KeywordTok{c}\NormalTok{(}\StringTok{"black"}\NormalTok{, }\StringTok{"red"}\NormalTok{),}
       \DataTypeTok{pch =} \DecValTok{1}\NormalTok{)}
\end{Highlighting}
\end{Shaded}

\texttt{ggplot2} colored scatter plot example:

\begin{Shaded}
\begin{Highlighting}[]
\KeywordTok{ggplot}\NormalTok{(}\KeywordTok{filter}\NormalTok{(housing, State }\OperatorTok\StringTok{ }\KeywordTok{c}\NormalTok{(}\StringTok{"MA"}\NormalTok{, }\StringTok{"TX"}\NormalTok{)),}
       \KeywordTok{aes}\NormalTok{(}\DataTypeTok{x=}\NormalTok{Date,}
           \DataTypeTok{y=}\NormalTok{Home_Value,}
           \DataTypeTok{color=}\NormalTok{State))}\OperatorTok{+}
\StringTok{  }\KeywordTok{geom_point}\NormalTok{()}
\end{Highlighting}
\end{Shaded}

\texttt{ggplot2} wins!

\section{Geometric objects \&
aesthetics}\label{geometric-objects-aesthetics}

\subsection{Aesthetic mapping}\label{aesthetic-mapping}

In ggplot land \emph{aesthetic} means ``something you can see''.
Examples include:

\begin{itemize}
\tightlist
\item
  position (i.e., on the x and y axes)
\item
  color (``outside'' color)
\item
  fill (``inside'' color)
\item
  shape (of points)
\item
  linetype
\item
  size
\end{itemize}

Each type of geom accepts only a subset of all aesthetics; refer to the
geom help pages to see what mappings each geom accepts. Aesthetic
mappings are set with the \texttt{aes()} function.

\subsection{\texorpdfstring{Geometric objects
(\texttt{geom})}{Geometric objects (geom)}}\label{geometric-objects-geom}

Geometric objects are the actual marks we put on a plot. Examples
include:

\begin{itemize}
\tightlist
\item
  points (\texttt{geom\_point()}, for scatter plots, dot plots, etc.)
\item
  lines (\texttt{geom\_line()}, for time series, trend lines, etc.)
\item
  boxplot (\texttt{geom\_boxplot()}, for boxplots!)
\end{itemize}

A plot \textbf{must have at least one geom}; there is no upper limit.
You can add a geom to a plot using the \texttt{+} operator.

Each \texttt{geom\_} has a particular set of aesthetic mappings
associated with it. Some examples are provided below, with required
aesthetics in \textbf{bold} and optional aesthetics in plain text:

\begin{longtable}[]{@{}lll@{}}
\toprule
\begin{minipage}[b]{0.13\columnwidth}\raggedright\strut
\texttt{geom\_}\strut
\end{minipage} & \begin{minipage}[b]{0.14\columnwidth}\raggedright\strut
Usage\strut
\end{minipage} & \begin{minipage}[b]{0.64\columnwidth}\raggedright\strut
Aesthetics\strut
\end{minipage}\tabularnewline
\midrule
\endhead
\begin{minipage}[t]{0.13\columnwidth}\raggedright\strut
\texttt{geom\_point()}\strut
\end{minipage} & \begin{minipage}[t]{0.14\columnwidth}\raggedright\strut
Scatter plot\strut
\end{minipage} & \begin{minipage}[t]{0.64\columnwidth}\raggedright\strut
\textbf{\texttt{x}},\textbf{\texttt{y}},\texttt{alpha},\texttt{color},\texttt{fill},\texttt{group},\texttt{shape},\texttt{size},\texttt{stroke}\strut
\end{minipage}\tabularnewline
\begin{minipage}[t]{0.13\columnwidth}\raggedright\strut
\texttt{geom\_line()}\strut
\end{minipage} & \begin{minipage}[t]{0.14\columnwidth}\raggedright\strut
Line plot\strut
\end{minipage} & \begin{minipage}[t]{0.64\columnwidth}\raggedright\strut
\textbf{\texttt{x}},\textbf{\texttt{y}},\texttt{alpha},\texttt{color},\texttt{linetype},\texttt{size}\strut
\end{minipage}\tabularnewline
\begin{minipage}[t]{0.13\columnwidth}\raggedright\strut
\texttt{geom\_bar()}\strut
\end{minipage} & \begin{minipage}[t]{0.14\columnwidth}\raggedright\strut
Bar chart\strut
\end{minipage} & \begin{minipage}[t]{0.64\columnwidth}\raggedright\strut
\textbf{\texttt{x}},\textbf{\texttt{y}},\texttt{alpha},\texttt{color},\texttt{fill},\texttt{group},\texttt{linetype},\texttt{size}\strut
\end{minipage}\tabularnewline
\begin{minipage}[t]{0.13\columnwidth}\raggedright\strut
\texttt{geom\_boxplot()}\strut
\end{minipage} & \begin{minipage}[t]{0.14\columnwidth}\raggedright\strut
Boxplot\strut
\end{minipage} & \begin{minipage}[t]{0.64\columnwidth}\raggedright\strut
\textbf{\texttt{x}},\textbf{\texttt{lower}},\textbf{\texttt{upper}},\textbf{\texttt{middle}},\textbf{\texttt{ymin}},\textbf{\texttt{ymax}},\texttt{alpha},\texttt{color},\texttt{fill}\strut
\end{minipage}\tabularnewline
\begin{minipage}[t]{0.13\columnwidth}\raggedright\strut
\texttt{geom\_density()}\strut
\end{minipage} & \begin{minipage}[t]{0.14\columnwidth}\raggedright\strut
Density plot\strut
\end{minipage} & \begin{minipage}[t]{0.64\columnwidth}\raggedright\strut
\textbf{\texttt{x}},\textbf{\texttt{y}},\texttt{alpha},\texttt{color},\texttt{fill},\texttt{group},\texttt{linetype},\texttt{size},\texttt{weight}\strut
\end{minipage}\tabularnewline
\begin{minipage}[t]{0.13\columnwidth}\raggedright\strut
\texttt{geom\_smooth()}\strut
\end{minipage} & \begin{minipage}[t]{0.14\columnwidth}\raggedright\strut
Conditional means\strut
\end{minipage} & \begin{minipage}[t]{0.64\columnwidth}\raggedright\strut
\textbf{\texttt{x}},\textbf{\texttt{y}},\texttt{alpha},\texttt{color},\texttt{fill},\texttt{group},\texttt{linetype},\texttt{size},\texttt{weight}\strut
\end{minipage}\tabularnewline
\begin{minipage}[t]{0.13\columnwidth}\raggedright\strut
\texttt{geom\_label()}\strut
\end{minipage} & \begin{minipage}[t]{0.14\columnwidth}\raggedright\strut
Text\strut
\end{minipage} & \begin{minipage}[t]{0.64\columnwidth}\raggedright\strut
\textbf{\texttt{x}},\textbf{\texttt{y}},\textbf{\texttt{label}},\texttt{alpha},\texttt{angle},\texttt{color},\texttt{family},\texttt{fontface},\texttt{size}\strut
\end{minipage}\tabularnewline
\bottomrule
\end{longtable}

You can get a list of all available geometric objects and their
associated aesthetics at \url{https://ggplot2.tidyverse.org/reference/}

or simply type \texttt{geom\_\textless{}tab\textgreater{}} in any good R
IDE (such as Rstudio or ESS) to see a list of functions starting with
\texttt{geom\_}.

\subsubsection{Points (scatterplot)}\label{points-scatterplot}

Now that we know about geometric objects and aesthetic mapping, we can
make a \texttt{ggplot()}. \texttt{geom\_point()} requires mappings for x
and y, all others are optional.

\begin{Shaded}
\begin{Highlighting}[]
\NormalTok{hp2001Q1 <-}\StringTok{ }\KeywordTok{filter}\NormalTok{(housing, Date }\OperatorTok{==}\StringTok{ }\FloatTok{2001.25}\NormalTok{) }
\KeywordTok{ggplot}\NormalTok{(hp2001Q1,}
       \KeywordTok{aes}\NormalTok{(}\DataTypeTok{y =}\NormalTok{ Structure.Cost, }\DataTypeTok{x =}\NormalTok{ Land_Value)) }\OperatorTok{+}
\StringTok{  }\KeywordTok{geom_point}\NormalTok{()}
\end{Highlighting}
\end{Shaded}

\begin{Shaded}
\begin{Highlighting}[]
\KeywordTok{ggplot}\NormalTok{(hp2001Q1,}
       \KeywordTok{aes}\NormalTok{(}\DataTypeTok{y =}\NormalTok{ Structure.Cost, }\DataTypeTok{x =} \KeywordTok{log}\NormalTok{(Land_Value))) }\OperatorTok{+}
\StringTok{  }\KeywordTok{geom_point}\NormalTok{()}
\end{Highlighting}
\end{Shaded}

\subsubsection{Lines (prediction line)}\label{lines-prediction-line}

A plot constructed with \texttt{ggplot()} can have more than one geom.
In that case the mappings established in the \texttt{ggplot()} call are
plot defaults that can be added to or overridden. Our plot could use a
regression line:

\begin{Shaded}
\begin{Highlighting}[]
\NormalTok{hp2001Q1}\OperatorTok{$}\NormalTok{pred_SC <-}\StringTok{ }\KeywordTok{lm}\NormalTok{(Structure_Cost }\OperatorTok{~}\StringTok{ }\KeywordTok{log}\NormalTok{(Land_Value), }\DataTypeTok{data =}\NormalTok{ hp2001Q1) }\OperatorTok
\StringTok{  }\KeywordTok{predict}\NormalTok{()}

\NormalTok{p1 <-}\StringTok{ }\KeywordTok{ggplot}\NormalTok{(hp2001Q1, }\KeywordTok{aes}\NormalTok{(}\DataTypeTok{x =} \KeywordTok{log}\NormalTok{(Land_Value), }\DataTypeTok{y =}\NormalTok{ Structure_Cost))}

\NormalTok{p1 }\OperatorTok{+}\StringTok{ }\KeywordTok{geom_point}\NormalTok{(}\KeywordTok{aes}\NormalTok{(}\DataTypeTok{color =}\NormalTok{ Home_Value)) }\OperatorTok{+}
\StringTok{  }\KeywordTok{geom_line}\NormalTok{(}\KeywordTok{aes}\NormalTok{(}\DataTypeTok{y =}\NormalTok{ pred_SC))}
\end{Highlighting}
\end{Shaded}

\subsubsection{Smoothers}\label{smoothers}

Not all geometric objects are simple shapes; the smooth geom includes a
line and a ribbon.

\begin{Shaded}
\begin{Highlighting}[]
\NormalTok{p1 }\OperatorTok{+}
\StringTok{  }\KeywordTok{geom_point}\NormalTok{(}\KeywordTok{aes}\NormalTok{(}\DataTypeTok{color =}\NormalTok{ Home_Value)) }\OperatorTok{+}
\StringTok{  }\KeywordTok{geom_smooth}\NormalTok{()}
\end{Highlighting}
\end{Shaded}

\subsubsection{Text (label points)}\label{text-label-points}

Each geom accepts a particular set of mappings; for example
\texttt{geom\_text()} accepts a \texttt{labels} mapping.

\begin{Shaded}
\begin{Highlighting}[]
\NormalTok{p1 }\OperatorTok{+}\StringTok{ }
\StringTok{  }\KeywordTok{geom_text}\NormalTok{(}\KeywordTok{aes}\NormalTok{(}\DataTypeTok{label=}\NormalTok{State), }\DataTypeTok{size =} \DecValTok{3}\NormalTok{)}
\end{Highlighting}
\end{Shaded}

\begin{Shaded}
\begin{Highlighting}[]
\NormalTok{## install.packages("ggrepel") }
\KeywordTok{library}\NormalTok{(ggrepel)}

\NormalTok{p1 }\OperatorTok{+}\StringTok{ }
\StringTok{  }\KeywordTok{geom_point}\NormalTok{() }\OperatorTok{+}\StringTok{ }
\StringTok{  }\KeywordTok{geom_text_repel}\NormalTok{(}\KeywordTok{aes}\NormalTok{(}\DataTypeTok{label=}\NormalTok{State), }\DataTypeTok{size =} \DecValTok{3}\NormalTok{)}
\end{Highlighting}
\end{Shaded}

\subsection{Aesthetic mapping VS
assignment}\label{aesthetic-mapping-vs-assignment}

Note that variables are mapped to aesthetics with the \texttt{aes()}
function, while fixed aesthetics are set outside the \texttt{aes()}
call. This sometimes leads to confusion, as in this example:

\begin{Shaded}
\begin{Highlighting}[]
\NormalTok{p1 }\OperatorTok{+}
\StringTok{  }\KeywordTok{geom_point}\NormalTok{(}\KeywordTok{aes}\NormalTok{(}\DataTypeTok{size =} \DecValTok{2}\NormalTok{),}\CommentTok{# incorrect! 2 is not a variable}
             \DataTypeTok{color=}\StringTok{"red"}\NormalTok{) }\CommentTok{# this is fine -- all points red}
\end{Highlighting}
\end{Shaded}

\subsection{Mapping variables to other
aesthetics}\label{mapping-variables-to-other-aesthetics}

Other aesthetics are mapped in the same way as x and y in the previous
example.

\begin{Shaded}
\begin{Highlighting}[]
\NormalTok{p1 }\OperatorTok{+}
\StringTok{  }\KeywordTok{geom_point}\NormalTok{(}\KeywordTok{aes}\NormalTok{(}\DataTypeTok{color =}\NormalTok{ Home_Value, }\DataTypeTok{shape =}\NormalTok{ region))}
\end{Highlighting}
\end{Shaded}

\section{Exercise 0}\label{exercise-0-2}

The data for the exercises is available in the
\texttt{dataSets/EconomistData.csv} file. Read it in with

\begin{Shaded}
\begin{Highlighting}[]
\NormalTok{dat <-}\StringTok{ }\KeywordTok{read_csv}\NormalTok{(}\StringTok{"dataSets/EconomistData.csv"}\NormalTok{)}
\end{Highlighting}
\end{Shaded}

Original sources for these data are
\url{http://www.transparency.org/content/download/64476/1031428}
\url{http://hdrstats.undp.org/en/indicators/display_cf_xls_indicator.cfm?indicator_id=103106\&lang=en}

These data consist of \emph{Human Development Index} and
\emph{Corruption Perception Index} scores for several countries.

\begin{enumerate}
\def\labelenumi{\arabic{enumi}.}
\tightlist
\item
  Create a scatter plot with CPI on the x axis and HDI on the y axis.
\item
  Color the points blue.
\item
  Map the color of the the points to Region.
\item
  Make the points bigger by setting size to 2
\item
  Map the size of the points to HDI\_Rank
\end{enumerate}

\section{Statistical transformations}\label{statistical-transformations}

\subsection{Why transform data?}\label{why-transform-data}

Some plot types (such as scatterplots) do not require transformations;
each point is plotted at x and y coordinates equal to the original
value. Other plots, such as boxplots, histograms, prediction lines etc.
require statistical transformations:

\begin{itemize}
\tightlist
\item
  for a boxplot the y values must be transformed to the median and
  1.5(IQR)
\item
  for a smoother the y values must be transformed into predicted values
\end{itemize}

Each geom has a default statistic, but these can be changed. For
example, the default statistic for \texttt{geom\_histogram()} is
\texttt{stat\_bin()}:

\begin{Shaded}
\begin{Highlighting}[]
\KeywordTok{args}\NormalTok{(geom_histogram)}
\KeywordTok{args}\NormalTok{(stat_bin)}
\end{Highlighting}
\end{Shaded}

Here is a list of geoms and their default statistics
\url{https://ggplot2.tidyverse.org/reference/}

\subsection{Setting arguments}\label{setting-arguments}

Arguments to \texttt{stat\_} functions can be passed through
\texttt{geom\_} functions. This can be slightly annoying because in
order to change it you have to first determine which stat the geom uses,
then determine the arguments to that stat.

For example, here is the default histogram of Home.Value:

\begin{Shaded}
\begin{Highlighting}[]
\NormalTok{p2 <-}\StringTok{ }\KeywordTok{ggplot}\NormalTok{(housing, }\KeywordTok{aes}\NormalTok{(}\DataTypeTok{x =}\NormalTok{ Home_Value))}
\NormalTok{p2 }\OperatorTok{+}\StringTok{ }\KeywordTok{geom_histogram}\NormalTok{()}
\end{Highlighting}
\end{Shaded}

can change it by passing the \texttt{binwidth} argument to the
\texttt{stat\_bin()} function:

\begin{Shaded}
\begin{Highlighting}[]
\NormalTok{p2 }\OperatorTok{+}\StringTok{ }\KeywordTok{geom_histogram}\NormalTok{(}\DataTypeTok{stat =} \StringTok{"bin"}\NormalTok{, }\DataTypeTok{binwidth=}\DecValTok{4000}\NormalTok{)}
\end{Highlighting}
\end{Shaded}

\subsection{Changing the
transformation}\label{changing-the-transformation}

Sometimes the default statistical transformation is not what you need.
This is often the case with pre-summarized data:

\begin{Shaded}
\begin{Highlighting}[]
\NormalTok{housing_sum <-}\StringTok{ }
\StringTok{  }\NormalTok{housing }\OperatorTok
\StringTok{  }\KeywordTok{group_by}\NormalTok{(State) }\OperatorTok
\StringTok{  }\KeywordTok{summarize}\NormalTok{(}\DataTypeTok{Home_Value_Mean =} \KeywordTok{mean}\NormalTok{(Home_Value)) }\OperatorTok
\StringTok{  }\KeywordTok{ungroup}\NormalTok{()}

\KeywordTok{rbind}\NormalTok{(}\KeywordTok{head}\NormalTok{(housing_sum), }\KeywordTok{tail}\NormalTok{(housing_sum))}
\end{Highlighting}
\end{Shaded}

\begin{Shaded}
\begin{Highlighting}[]
\KeywordTok{ggplot}\NormalTok{(housing_sum, }\KeywordTok{aes}\NormalTok{(}\DataTypeTok{x=}\NormalTok{State, }\DataTypeTok{y=}\NormalTok{Home_Value_Mean)) }\OperatorTok{+}\StringTok{ }
\StringTok{  }\KeywordTok{geom_bar}\NormalTok{()}
\end{Highlighting}
\end{Shaded}

What is the problem with the previous plot? Basically we take binned and
summarized data and ask ggplot to bin and summarize it again (remember,
\texttt{geom\_bar()} defaults to \texttt{stat\ =\ stat\_count};
obviously this will not work. We can fix it by telling
\texttt{geom\_bar()} to use a different statistical transformation
function:

\begin{Shaded}
\begin{Highlighting}[]
\KeywordTok{ggplot}\NormalTok{(housing_sum, }\KeywordTok{aes}\NormalTok{(}\DataTypeTok{x=}\NormalTok{State, }\DataTypeTok{y=}\NormalTok{Home_Value_Mean)) }\OperatorTok{+}\StringTok{ }
\StringTok{  }\KeywordTok{geom_bar}\NormalTok{(}\DataTypeTok{stat=}\StringTok{"identity"}\NormalTok{)}
\end{Highlighting}
\end{Shaded}

\section{Exercise 1}\label{exercise-1-2}

\begin{enumerate}
\def\labelenumi{\arabic{enumi}.}
\tightlist
\item
  Re-create a scatter plot with CPI on the x axis and HDI on the y axis
  (as you did in the previous exercise).
\item
  Overlay a smoothing line on top of the scatter plot using
  \texttt{geom\_smooth()}.
\item
  Overlay a smoothing line on top of the scatter plot using
  \texttt{geom\_smooth()}, but use a linear model for the predictions.
  Hint: see \texttt{?stat\_smooth}.
\item
  Overlay a smoothing line on top of the scatter plot using the default
  \emph{loess} method for \texttt{geom\_smooth()}, but make it less
  smooth. Hint: see \texttt{?loess}.
\item
  BONUS: Overlay a smoothing line on top of the scatter plot using
  \texttt{geom\_line()}. Hint: change the statistical transformation.
\end{enumerate}

\section{Scales}\label{scales}

\subsection{Controlling aesthetic
mapping}\label{controlling-aesthetic-mapping}

Aesthetic mapping (i.e., with \texttt{aes()}) only says that a variable
should be mapped to an aesthetic. It doesn't say \emph{how} that should
happen. For example, when mapping a variable to \emph{shape} with
\texttt{aes(shape\ =\ x)} you don't say \emph{what} shapes should be
used. Similarly, \texttt{aes(color\ =\ z)} doesn't say \emph{what}
colors should be used. Describing what colors/shapes/sizes etc. to use
is done by modifying the corresponding \emph{scale}. In \texttt{ggplot2}
scales include

\begin{itemize}
\tightlist
\item
  position
\item
  color and fill
\item
  size
\item
  shape
\item
  line type
\end{itemize}

Scales are modified with a series of functions using a
\texttt{scale\_\textless{}aesthetic\textgreater{}\_\textless{}type\textgreater{}}
naming scheme. Try typing \texttt{scale\_\textless{}tab\textgreater{}}
to see a list of scale modification functions.

\subsection{Common scale arguments}\label{common-scale-arguments}

The following arguments are common to most scales in \texttt{ggplot2}:

\begin{itemize}
\tightlist
\item
  \textbf{name:} the first argument gives the axis or legend title
\item
  \textbf{limits:} the minimum and maximum of the scale
\item
  \textbf{breaks:} the points along the scale where labels should appear
\item
  \textbf{labels:} the labels that appear at each break
\end{itemize}

Specific scale functions may have additional arguments; for example, the
\texttt{scale\_color\_continuous()} function has arguments \texttt{low}
and \texttt{high} for setting the colors at the low and high end of the
scale.

\subsection{Scale modification
examples}\label{scale-modification-examples}

Start by constructing a dotplot showing the distribution of home values
by Date and State.

\begin{Shaded}
\begin{Highlighting}[]
\NormalTok{p4 <-}\StringTok{ }\KeywordTok{ggplot}\NormalTok{(housing, }\KeywordTok{aes}\NormalTok{(}\DataTypeTok{x =}\NormalTok{ State, }\DataTypeTok{y =}\NormalTok{ Home_Price_Index)) }\OperatorTok{+}\StringTok{ }
\StringTok{    }\KeywordTok{geom_point}\NormalTok{(}\KeywordTok{aes}\NormalTok{(}\DataTypeTok{color =}\NormalTok{ Date), }\DataTypeTok{alpha =} \FloatTok{0.5}\NormalTok{, }\DataTypeTok{size =} \FloatTok{1.5}\NormalTok{,}
               \DataTypeTok{position =} \KeywordTok{position_jitter}\NormalTok{(}\DataTypeTok{width =} \FloatTok{0.25}\NormalTok{, }\DataTypeTok{height =} \DecValTok{0}\NormalTok{))}\ErrorTok{)}
\end{Highlighting}
\end{Shaded}

Now modify the breaks for the color scales

\begin{Shaded}
\begin{Highlighting}[]
\NormalTok{p4 }\OperatorTok{+}\StringTok{ }
\StringTok{  }\KeywordTok{scale_color_continuous}\NormalTok{(}\DataTypeTok{name=}\StringTok{""}\NormalTok{,}
                         \DataTypeTok{breaks =} \KeywordTok{c}\NormalTok{(}\DecValTok{1976}\NormalTok{, }\DecValTok{1994}\NormalTok{, }\DecValTok{2013}\NormalTok{),}
                         \DataTypeTok{labels =} \KeywordTok{c}\NormalTok{(}\StringTok{"'76"}\NormalTok{, }\StringTok{"'94"}\NormalTok{, }\StringTok{"'13"}\NormalTok{))}
\end{Highlighting}
\end{Shaded}

Next change the low and high values to blue and red:

\begin{Shaded}
\begin{Highlighting}[]
\NormalTok{p4 }\OperatorTok{+}
\StringTok{  }\KeywordTok{scale_color_continuous}\NormalTok{(}\DataTypeTok{name=}\StringTok{""}\NormalTok{,}
                         \DataTypeTok{breaks =} \KeywordTok{c}\NormalTok{(}\DecValTok{1976}\NormalTok{, }\DecValTok{1994}\NormalTok{, }\DecValTok{2013}\NormalTok{),}
                         \DataTypeTok{labels =} \KeywordTok{c}\NormalTok{(}\StringTok{"'76"}\NormalTok{, }\StringTok{"'94"}\NormalTok{, }\StringTok{"'13"}\NormalTok{),}
                         \DataTypeTok{low =} \StringTok{"blue"}\NormalTok{, }\DataTypeTok{high =} \StringTok{"red"}\NormalTok{)}
\end{Highlighting}
\end{Shaded}

Now mute the colors:

\begin{Shaded}
\begin{Highlighting}[]
\KeywordTok{library}\NormalTok{(scales)}

\NormalTok{p4 }\OperatorTok{+}
\StringTok{  }\KeywordTok{scale_color_continuous}\NormalTok{(}\DataTypeTok{name=}\StringTok{""}\NormalTok{,}
                         \DataTypeTok{breaks =} \KeywordTok{c}\NormalTok{(}\DecValTok{1976}\NormalTok{, }\DecValTok{1994}\NormalTok{, }\DecValTok{2013}\NormalTok{),}
                         \DataTypeTok{labels =} \KeywordTok{c}\NormalTok{(}\StringTok{"'76"}\NormalTok{, }\StringTok{"'94"}\NormalTok{, }\StringTok{"'13"}\NormalTok{),}
                         \DataTypeTok{low =} \KeywordTok{muted}\NormalTok{(}\StringTok{"blue"}\NormalTok{), }\DataTypeTok{high =} \KeywordTok{muted}\NormalTok{(}\StringTok{"red"}\NormalTok{))}
\end{Highlighting}
\end{Shaded}

\subsection{Using different color
scales}\label{using-different-color-scales}

\texttt{ggplot2} has a wide variety of color scales; here is an example
using \texttt{scale\_color\_gradient2()} to interpolate between three
different colors.

\begin{Shaded}
\begin{Highlighting}[]
\NormalTok{p4 }\OperatorTok{+}
\StringTok{  }\KeywordTok{scale_color_gradient2}\NormalTok{(}\DataTypeTok{name=}\StringTok{""}\NormalTok{,}
                        \DataTypeTok{breaks =} \KeywordTok{c}\NormalTok{(}\DecValTok{1976}\NormalTok{, }\DecValTok{1994}\NormalTok{, }\DecValTok{2013}\NormalTok{),}
                        \DataTypeTok{labels =} \KeywordTok{c}\NormalTok{(}\StringTok{"'76"}\NormalTok{, }\StringTok{"'94"}\NormalTok{, }\StringTok{"'13"}\NormalTok{),}
                        \DataTypeTok{low =} \KeywordTok{muted}\NormalTok{(}\StringTok{"blue"}\NormalTok{),}
                        \DataTypeTok{high =} \KeywordTok{muted}\NormalTok{(}\StringTok{"red"}\NormalTok{),}
                        \DataTypeTok{mid =} \StringTok{"gray60"}\NormalTok{,}
                        \DataTypeTok{midpoint =} \DecValTok{1994}\NormalTok{)}
\end{Highlighting}
\end{Shaded}

\subsection{Available scales}\label{available-scales}

\begin{itemize}
\tightlist
\item
  Partial combination matrix of available scales
\end{itemize}

\begin{longtable}[]{@{}lll@{}}
\toprule
\texttt{scale\_} & Types & Examples\tabularnewline
\midrule
\endhead
\texttt{scale\_color\_} & \texttt{identity} &
\texttt{scale\_fill\_continuous()}\tabularnewline
\texttt{scale\_fill\_} & \texttt{manual} &
\texttt{scale\_color\_discrete()}\tabularnewline
\texttt{scale\_size\_} & \texttt{continuous} &
\texttt{scale\_size\_manual()}\tabularnewline
& \texttt{discrete} & \texttt{scale\_size\_discrete()}\tabularnewline
& &\tabularnewline
\texttt{scale\_shape\_} & \texttt{discrete} &
\texttt{scale\_shape\_discrete()}\tabularnewline
\texttt{scale\_linetype\_} & \texttt{identity} &
\texttt{scale\_shape\_manual()}\tabularnewline
& \texttt{manual} & \texttt{scale\_linetype\_discrete()}\tabularnewline
& &\tabularnewline
\texttt{scale\_x\_} & \texttt{continuous} &
\texttt{scale\_x\_continuous()}\tabularnewline
\texttt{scale\_y\_} & \texttt{discrete} &
\texttt{scale\_y\_discrete()}\tabularnewline
& \texttt{reverse} & \texttt{scale\_x\_log()}\tabularnewline
& \texttt{log} & \texttt{scale\_y\_reverse()}\tabularnewline
& \texttt{date} & \texttt{scale\_x\_date()}\tabularnewline
& \texttt{datetime} & \texttt{scale\_y\_datetime()}\tabularnewline
\bottomrule
\end{longtable}

Note that in RStudio you can type \texttt{scale\_} followed by
\texttt{tab} to get the whole list of available scales. For a complete
list of available scales see
\url{https://ggplot2.tidyverse.org/reference/}

\section{Exercise 2}\label{exercise-2-1}

\begin{enumerate}
\def\labelenumi{\arabic{enumi}.}
\tightlist
\item
  Create a scatter plot with CPI on the x axis and HDI on the y axis.
  Color the points to indicate region.
\item
  Modify the x, y, and color scales so that they have more
  easily-understood names (e.g., spell out ``Human development Index''
  instead of ``HDI''). Hint: see \texttt{?scale\_x\_discrete}.
\item
  Modify the color scale to use specific values of your choosing. Hint:
  see \texttt{?scale\_color\_manual}.
\end{enumerate}

\section{Faceting}\label{faceting}

\subsection{What is faceting?}\label{what-is-faceting}

\begin{itemize}
\tightlist
\item
  Faceting is \texttt{ggplot2} parlance for \textbf{small multiples}
\item
  The idea is to create separate graphs for subsets of data
\item
  \texttt{ggplot2} offers two functions for creating small multiples:

  \begin{enumerate}
  \def\labelenumi{\arabic{enumi}.}
  \tightlist
  \item
    \texttt{facet\_wrap()}: define subsets as the levels of a single
    grouping variable
  \item
    \texttt{facet\_grid()}: define subsets as the crossing of two
    grouping variables
  \end{enumerate}
\item
  Facilitates comparison among plots, not just of geoms within a plot
\end{itemize}

\subsection{What is the trend in housing prices in each
state?}\label{what-is-the-trend-in-housing-prices-in-each-state}

\begin{itemize}
\tightlist
\item
  Start by using a technique we already know; map State to color:
\end{itemize}

\begin{Shaded}
\begin{Highlighting}[]
\NormalTok{p5 <-}\StringTok{ }\KeywordTok{ggplot}\NormalTok{(housing, }\KeywordTok{aes}\NormalTok{(}\DataTypeTok{x =}\NormalTok{ Date, }\DataTypeTok{y =}\NormalTok{ Home_Value))}
\NormalTok{p5 }\OperatorTok{+}\StringTok{ }\KeywordTok{geom_line}\NormalTok{(}\KeywordTok{aes}\NormalTok{(}\DataTypeTok{color =}\NormalTok{ State))  }
\end{Highlighting}
\end{Shaded}

There are two problems here; there are too many states to distinguish
each one by color, and the lines obscure one another.

\subsection{Faceting to the rescue}\label{faceting-to-the-rescue}

We can remedy the deficiencies of the previous plot by faceting by state
rather than mapping state to color.

\begin{Shaded}
\begin{Highlighting}[]
\NormalTok{(p5 <-}\StringTok{ }\NormalTok{p5 }\OperatorTok{+}\StringTok{ }\KeywordTok{geom_line}\NormalTok{() }\OperatorTok{+}
\StringTok{   }\KeywordTok{facet_wrap}\NormalTok{(}\OperatorTok{~}\StringTok{ }\NormalTok{State, }\DataTypeTok{ncol =} \DecValTok{10}\NormalTok{))}
\end{Highlighting}
\end{Shaded}

There is also a \texttt{facet\_grid()} function for faceting in two
dimensions.

\section{Themes}\label{themes}

\subsection{What are themes?}\label{what-are-themes}

The \texttt{ggplot2} theme system handles non-data plot elements such as

\begin{itemize}
\tightlist
\item
  Axis labels
\item
  Plot background
\item
  Facet label background
\item
  Legend appearance
\end{itemize}

Built-in themes include:

\begin{itemize}
\tightlist
\item
  \texttt{theme\_gray()} (default)
\item
  \texttt{theme\_bw()}
\item
  \texttt{theme\_classic()}
\end{itemize}

\begin{Shaded}
\begin{Highlighting}[]
\NormalTok{p5 }\OperatorTok{+}\StringTok{ }\KeywordTok{theme_linedraw}\NormalTok{()}
\end{Highlighting}
\end{Shaded}

\begin{Shaded}
\begin{Highlighting}[]
\NormalTok{p5 }\OperatorTok{+}\StringTok{ }\KeywordTok{theme_light}\NormalTok{()}
\end{Highlighting}
\end{Shaded}

You can see a list of available built-in themes here
\url{https://ggplot2.tidyverse.org/reference/}

\subsection{Overriding theme defaults}\label{overriding-theme-defaults}

Specific theme elements can be overridden using \texttt{theme()}. For
example:

\begin{Shaded}
\begin{Highlighting}[]
\NormalTok{p5 }\OperatorTok{+}\StringTok{ }\KeywordTok{theme_minimal}\NormalTok{() }\OperatorTok{+}
\StringTok{  }\KeywordTok{theme}\NormalTok{(}\DataTypeTok{text =} \KeywordTok{element_text}\NormalTok{(}\DataTypeTok{color =} \StringTok{"turquoise"}\NormalTok{))}
\end{Highlighting}
\end{Shaded}

All theme options are documented in \texttt{?theme}.

\subsection{Creating \& saving new
themes}\label{creating-saving-new-themes}

You can create new themes, as in the following example:

\begin{Shaded}
\begin{Highlighting}[]
\NormalTok{theme_new <-}\StringTok{ }\KeywordTok{theme_bw}\NormalTok{() }\OperatorTok{+}
\StringTok{  }\KeywordTok{theme}\NormalTok{(}\DataTypeTok{plot.background =} \KeywordTok{element_rect}\NormalTok{(}\DataTypeTok{size =} \DecValTok{1}\NormalTok{, }\DataTypeTok{color =} \StringTok{"blue"}\NormalTok{, }\DataTypeTok{fill =} \StringTok{"black"}\NormalTok{),}
        \DataTypeTok{text=}\KeywordTok{element_text}\NormalTok{(}\DataTypeTok{size =} \DecValTok{12}\NormalTok{, }\DataTypeTok{family =} \StringTok{"Serif"}\NormalTok{, }\DataTypeTok{color =} \StringTok{"ivory"}\NormalTok{),}
        \DataTypeTok{axis.text.y =} \KeywordTok{element_text}\NormalTok{(}\DataTypeTok{colour =} \StringTok{"purple"}\NormalTok{),}
        \DataTypeTok{axis.text.x =} \KeywordTok{element_text}\NormalTok{(}\DataTypeTok{colour =} \StringTok{"red"}\NormalTok{),}
        \DataTypeTok{panel.background =} \KeywordTok{element_rect}\NormalTok{(}\DataTypeTok{fill =} \StringTok{"pink"}\NormalTok{),}
        \DataTypeTok{strip.background =} \KeywordTok{element_rect}\NormalTok{(}\DataTypeTok{fill =} \KeywordTok{muted}\NormalTok{(}\StringTok{"orange"}\NormalTok{)))}

\NormalTok{p5 }\OperatorTok{+}\StringTok{ }\NormalTok{theme_new}
\end{Highlighting}
\end{Shaded}

You can see all the plot elements that can be changed by
\texttt{theme()} using:

\begin{Shaded}
\begin{Highlighting}[]
\KeywordTok{names}\NormalTok{(}\KeywordTok{theme_get}\NormalTok{())}

\CommentTok{# see all arguments for each plot element}
\KeywordTok{theme_get}\NormalTok{()}
\end{Highlighting}
\end{Shaded}

\section{The \#1 FAQ}\label{the-1-faq}

\subsection{Map aesthetic to different
columns}\label{map-aesthetic-to-different-columns}

The most frequently asked question goes something like this: \emph{I
have two variables in my data.frame, and I'd like to plot them as
separate points, with different color depending on which variable it is.
How do I do that?}

\textbf{Wrong}

\begin{Shaded}
\begin{Highlighting}[]
\NormalTok{housing_byyear <-}\StringTok{ }
\StringTok{  }\NormalTok{housing }\OperatorTok
\StringTok{  }\KeywordTok{group_by}\NormalTok{(Date) }\OperatorTok
\StringTok{  }\KeywordTok{summarize}\NormalTok{(}\DataTypeTok{Home_Value_Mean =} \KeywordTok{mean}\NormalTok{(Home_Value),}
            \DataTypeTok{Land_Value_Mean =} \KeywordTok{mean}\NormalTok{(Land_Value)) }\OperatorTok
\StringTok{  }\KeywordTok{ungroup}\NormalTok{()}

\KeywordTok{ggplot}\NormalTok{(housing_byyear, }\KeywordTok{aes}\NormalTok{(}\DataTypeTok{x=}\NormalTok{Date)) }\OperatorTok{+}
\StringTok{  }\KeywordTok{geom_line}\NormalTok{(}\KeywordTok{aes}\NormalTok{(}\DataTypeTok{y=}\NormalTok{Home_Value_Mean), }\DataTypeTok{color=}\StringTok{"red"}\NormalTok{) }\OperatorTok{+}
\StringTok{  }\KeywordTok{geom_line}\NormalTok{(}\KeywordTok{aes}\NormalTok{(}\DataTypeTok{y=}\NormalTok{Land_Value_Mean), }\DataTypeTok{color=}\StringTok{"blue"}\NormalTok{)}
\end{Highlighting}
\end{Shaded}

\textbf{Right}

\begin{Shaded}
\begin{Highlighting}[]
\NormalTok{home_land_byyear <-}\StringTok{ }\KeywordTok{gather}\NormalTok{(housing_byyear,}
                           \DataTypeTok{value =} \StringTok{"value"}\NormalTok{,}
                           \DataTypeTok{key =} \StringTok{"type"}\NormalTok{,}
\NormalTok{                           Home_Value, Land_Value)}

\KeywordTok{ggplot}\NormalTok{(home_land_byyear, }\KeywordTok{aes}\NormalTok{(}\DataTypeTok{x=}\NormalTok{Date, }\DataTypeTok{y=}\NormalTok{value, }\DataTypeTok{color=}\NormalTok{type)) }\OperatorTok{+}
\StringTok{  }\KeywordTok{geom_line}\NormalTok{()}
\end{Highlighting}
\end{Shaded}

\section{Putting it all together}\label{putting-it-all-together}

\subsection{\texorpdfstring{Challenge: recreate this \texttt{Economist}
graph}{Challenge: recreate this Economist graph}}\label{challenge-recreate-this-economist-graph}

\begin{figure}
\centering
\includegraphics{images/Economist1.png}
\caption{img}
\end{figure}

Graph source: \url{http://www.economist.com/node/21541178}

Building off of the graphics you created in the previous exercises, put
the finishing touches to make it as close as possible to the original
economist graph.

\subsection{Challenge solution:}\label{challenge-solution}

Lets start by creating the basic scatter plot, then we can make a list
of things that need to be added or changed. The basic plot looks like
this:

\begin{Shaded}
\begin{Highlighting}[]
\NormalTok{dat <-}\StringTok{ }\KeywordTok{read_csv}\NormalTok{(}\StringTok{"dataSets/EconomistData.csv"}\NormalTok{)}

\NormalTok{pc1 <-}\StringTok{ }\KeywordTok{ggplot}\NormalTok{(dat, }\KeywordTok{aes}\NormalTok{(}\DataTypeTok{x =}\NormalTok{ CPI, }\DataTypeTok{y =}\NormalTok{ HDI, }\DataTypeTok{color =}\NormalTok{ Region))}
\NormalTok{pc1 }\OperatorTok{+}\StringTok{ }\KeywordTok{geom_point}\NormalTok{()}
\end{Highlighting}
\end{Shaded}

To complete this graph we need to:

\begin{itemize}
\tightlist
\item
  {[} {]} add a trend line
\item
  {[} {]} change the point shape to open circle
\item
  {[} {]} change the order and labels of Region
\item
  {[} {]} label select points
\item
  {[} {]} fix up the tick marks and labels
\item
  {[} {]} move color legend to the top
\item
  {[} {]} title, label axes, remove legend title
\item
  {[} {]} theme the graph with no vertical guides
\item
  {[} {]} add model R2 (hard)
\item
  {[} {]} add sources note (hard)
\item
  {[} {]} final touches to make it perfect (use image editor for this)
\end{itemize}

\subsubsection{Adding the trend line}\label{adding-the-trend-line}

Adding the trend line is not too difficult, though we need to guess at
the model being displyed on the graph. A little bit of trial and error
leads to

\begin{Shaded}
\begin{Highlighting}[]
\NormalTok{pc2 <-}\StringTok{ }\NormalTok{pc1 }\OperatorTok{+}
\StringTok{  }\KeywordTok{geom_smooth}\NormalTok{(}\DataTypeTok{mapping =} \KeywordTok{aes}\NormalTok{(}\DataTypeTok{linetype =} \StringTok{"r2"}\NormalTok{),   }\CommentTok{# "r2" is a placeholder for where we'll later put R^2 values}
              \DataTypeTok{method =} \StringTok{"lm"}\NormalTok{,}
              \DataTypeTok{formula =}\NormalTok{ y }\OperatorTok{~}\StringTok{ }\NormalTok{x }\OperatorTok{+}\StringTok{ }\KeywordTok{log}\NormalTok{(x), }\DataTypeTok{se =} \OtherTok{FALSE}\NormalTok{,}
              \DataTypeTok{color =} \StringTok{"red"}\NormalTok{)}
\NormalTok{pc2 }\OperatorTok{+}\StringTok{ }\KeywordTok{geom_point}\NormalTok{()}
\end{Highlighting}
\end{Shaded}

Notice that we put the \texttt{geom\_line} layer first so that it will
be plotted underneath the points, as was done on the original graph.

\subsubsection{Use open points}\label{use-open-points}

This one is a little tricky. We know that we can change the shape with
the \texttt{shape} argument, what value do we set shape to? The example
shown in \texttt{?shape} can help us:

\begin{Shaded}
\begin{Highlighting}[]
\NormalTok{## A look at all 25 symbols}
\NormalTok{df2 <-}\StringTok{ }\KeywordTok{data.frame}\NormalTok{(}\DataTypeTok{x =} \DecValTok{1}\OperatorTok{:}\DecValTok{5}\NormalTok{ , }\DataTypeTok{y =} \DecValTok{1}\OperatorTok{:}\DecValTok{25}\NormalTok{, }\DataTypeTok{z =} \DecValTok{1}\OperatorTok{:}\DecValTok{25}\NormalTok{)}

\NormalTok{s <-}\StringTok{ }\KeywordTok{ggplot}\NormalTok{(df2, }\KeywordTok{aes}\NormalTok{(}\DataTypeTok{x =}\NormalTok{ x, }\DataTypeTok{y =}\NormalTok{ y))}
\NormalTok{s }\OperatorTok{+}\StringTok{ }\KeywordTok{geom_point}\NormalTok{(}\KeywordTok{aes}\NormalTok{(}\DataTypeTok{shape =}\NormalTok{ z), }\DataTypeTok{size =} \DecValTok{4}\NormalTok{) }\OperatorTok{+}\StringTok{ }\KeywordTok{scale_shape_identity}\NormalTok{()}
\end{Highlighting}
\end{Shaded}

This shows us that \emph{shape 1} is an open circle, so

\begin{Shaded}
\begin{Highlighting}[]
\NormalTok{pc2 }\OperatorTok{+}
\StringTok{  }\KeywordTok{geom_point}\NormalTok{(}\DataTypeTok{shape =} \DecValTok{1}\NormalTok{, }\DataTypeTok{size =} \DecValTok{4}\NormalTok{)}
\end{Highlighting}
\end{Shaded}

That is better, but unfortunately the size of the line around the points
is much narrower than on the original.

\begin{Shaded}
\begin{Highlighting}[]
\NormalTok{(pc3 <-}\StringTok{ }\NormalTok{pc2 }\OperatorTok{+}\StringTok{ }\KeywordTok{geom_point}\NormalTok{(}\DataTypeTok{shape =} \DecValTok{1}\NormalTok{, }\DataTypeTok{size =} \FloatTok{2.5}\NormalTok{, }\DataTypeTok{stroke =} \FloatTok{1.25}\NormalTok{))}
\end{Highlighting}
\end{Shaded}

\subsubsection{Labelling points}\label{labelling-points}

This one is tricky in a couple of ways. First, there is no attribute in
the data that separates points that should be labelled from points that
should not be. So the first step is to identify those points.

\begin{Shaded}
\begin{Highlighting}[]
\NormalTok{pointsToLabel <-}\StringTok{ }\KeywordTok{c}\NormalTok{(}\StringTok{"Russia"}\NormalTok{, }\StringTok{"Venezuela"}\NormalTok{, }\StringTok{"Iraq"}\NormalTok{, }\StringTok{"Myanmar"}\NormalTok{, }\StringTok{"Sudan"}\NormalTok{,}
                   \StringTok{"Afghanistan"}\NormalTok{, }\StringTok{"Congo"}\NormalTok{, }\StringTok{"Greece"}\NormalTok{, }\StringTok{"Argentina"}\NormalTok{, }\StringTok{"Brazil"}\NormalTok{,}
                   \StringTok{"India"}\NormalTok{, }\StringTok{"Italy"}\NormalTok{, }\StringTok{"China"}\NormalTok{, }\StringTok{"South Africa"}\NormalTok{, }\StringTok{"Spane"}\NormalTok{,}
                   \StringTok{"Botswana"}\NormalTok{, }\StringTok{"Cape Verde"}\NormalTok{, }\StringTok{"Bhutan"}\NormalTok{, }\StringTok{"Rwanda"}\NormalTok{, }\StringTok{"France"}\NormalTok{,}
                   \StringTok{"United States"}\NormalTok{, }\StringTok{"Germany"}\NormalTok{, }\StringTok{"Britain"}\NormalTok{, }\StringTok{"Barbados"}\NormalTok{, }\StringTok{"Norway"}\NormalTok{, }\StringTok{"Japan"}\NormalTok{,}
                   \StringTok{"New Zealand"}\NormalTok{, }\StringTok{"Singapore"}\NormalTok{)}
\end{Highlighting}
\end{Shaded}

Now we can label these points using \texttt{geom\_text}, like this:

\begin{Shaded}
\begin{Highlighting}[]
\NormalTok{(pc4 <-}\StringTok{ }\NormalTok{pc3 }\OperatorTok{+}
\StringTok{  }\KeywordTok{geom_text}\NormalTok{(}\KeywordTok{aes}\NormalTok{(}\DataTypeTok{label =}\NormalTok{ Country),}
            \DataTypeTok{color =} \StringTok{"gray20"}\NormalTok{,}
            \DataTypeTok{data =} \KeywordTok{filter}\NormalTok{(dat, Country }\OperatorTok\StringTok{ }\NormalTok{pointsToLabel)))}
\end{Highlighting}
\end{Shaded}

This more or less gets the information across, but the labels overlap in
a most unpleasing fashion. We can use the \texttt{ggrepel} package to
make things better, but if you want perfection you will probably have to
do some hand-adjustment.

\begin{Shaded}
\begin{Highlighting}[]
\KeywordTok{library}\NormalTok{(ggrepel)}

\NormalTok{(pc4 <-}\StringTok{ }\NormalTok{pc3 }\OperatorTok{+}
\StringTok{   }\KeywordTok{geom_text_repel}\NormalTok{(}\KeywordTok{aes}\NormalTok{(}\DataTypeTok{label =}\NormalTok{ Country),}
                   \DataTypeTok{color =} \StringTok{"gray20"}\NormalTok{,}
                   \DataTypeTok{data =} \KeywordTok{filter}\NormalTok{(dat, Country }\OperatorTok\StringTok{ }\NormalTok{pointsToLabel),}
                   \DataTypeTok{force =} \DecValTok{10}\NormalTok{))}
\end{Highlighting}
\end{Shaded}

\subsubsection{Change the region labels \&
order}\label{change-the-region-labels-order}

Things are starting to come together. There are just a couple more
things we need to add, and then all that will be left are theme changes.

Comparing our graph to the original we notice that the labels and order
of the Regions in the color legend differ. To correct this we need to
change both the labels and order of the Region variable. We can do this
with the \texttt{factor} function.

\begin{Shaded}
\begin{Highlighting}[]
\NormalTok{dat}\OperatorTok{$}\NormalTok{Region <-}\StringTok{ }\KeywordTok{factor}\NormalTok{(dat}\OperatorTok{$}\NormalTok{Region,}
                     \DataTypeTok{levels =} \KeywordTok{c}\NormalTok{(}\StringTok{"EU W. Europe"}\NormalTok{,}
                                \StringTok{"Americas"}\NormalTok{,}
                                \StringTok{"Asia Pacific"}\NormalTok{,}
                                \StringTok{"East EU Cemt Asia"}\NormalTok{,}
                                \StringTok{"MENA"}\NormalTok{,}
                                \StringTok{"SSA"}\NormalTok{),}
                     \DataTypeTok{labels =} \KeywordTok{c}\NormalTok{(}\StringTok{"OECD"}\NormalTok{,}
                                \StringTok{"Americas"}\NormalTok{,}
                                \StringTok{"Asia &}\CharTok{\textbackslash{}n}\StringTok{Oceania"}\NormalTok{,}
                                \StringTok{"Central &}\CharTok{\textbackslash{}n}\StringTok{Eastern Europe"}\NormalTok{,}
                                \StringTok{"Middle East &}\CharTok{\textbackslash{}n}\StringTok{north Africa"}\NormalTok{,}
                                \StringTok{"Sub-Saharan}\CharTok{\textbackslash{}n}\StringTok{Africa"}\NormalTok{))}
\end{Highlighting}
\end{Shaded}

Now when we construct the plot using these data the order should appear
as it does in the original.

\begin{Shaded}
\begin{Highlighting}[]
\NormalTok{pc4}\OperatorTok{$}\NormalTok{data <-}\StringTok{ }\NormalTok{dat}
\NormalTok{pc4}
\end{Highlighting}
\end{Shaded}

\subsubsection{Add title \& format axes}\label{add-title-format-axes}

The next step is to add the title and format the axes. We do that using
the \texttt{scales} system in \texttt{ggplot2}.

\begin{Shaded}
\begin{Highlighting}[]
\KeywordTok{library}\NormalTok{(grid)}

\NormalTok{(pc5 <-}\StringTok{ }\NormalTok{pc4 }\OperatorTok{+}
\StringTok{  }\KeywordTok{scale_x_continuous}\NormalTok{(}\DataTypeTok{name =} \StringTok{"Corruption Perceptions Index, 2011 (10=least corrupt)"}\NormalTok{,}
                     \DataTypeTok{limits =} \KeywordTok{c}\NormalTok{(.}\DecValTok{9}\NormalTok{, }\FloatTok{10.5}\NormalTok{),}
                     \DataTypeTok{breaks =} \DecValTok{1}\OperatorTok{:}\DecValTok{10}\NormalTok{) }\OperatorTok{+}
\StringTok{  }\KeywordTok{scale_y_continuous}\NormalTok{(}\DataTypeTok{name =} \StringTok{"Human Development Index, 2011 (1=Best)"}\NormalTok{,}
                     \DataTypeTok{limits =} \KeywordTok{c}\NormalTok{(}\FloatTok{0.2}\NormalTok{, }\FloatTok{1.0}\NormalTok{),}
                     \DataTypeTok{breaks =} \KeywordTok{seq}\NormalTok{(}\FloatTok{0.2}\NormalTok{, }\FloatTok{1.0}\NormalTok{, }\DataTypeTok{by =} \FloatTok{0.1}\NormalTok{)) }\OperatorTok{+}
\StringTok{  }\KeywordTok{scale_color_manual}\NormalTok{(}\DataTypeTok{name =} \StringTok{""}\NormalTok{,}
                     \DataTypeTok{values =} \KeywordTok{c}\NormalTok{(}\StringTok{"#24576D"}\NormalTok{,}
                                \StringTok{"#099DD7"}\NormalTok{,}
                                \StringTok{"#28AADC"}\NormalTok{,}
                                \StringTok{"#248E84"}\NormalTok{,}
                                \StringTok{"#F2583F"}\NormalTok{,}
                                \StringTok{"#96503F"}\NormalTok{)) }\OperatorTok{+}
\StringTok{  }\KeywordTok{ggtitle}\NormalTok{(}\StringTok{"Corruption and Human development"}\NormalTok{))}
\end{Highlighting}
\end{Shaded}

\subsubsection{Theme tweaks}\label{theme-tweaks}

Our graph is almost there. To finish up, we need to adjust some of the
theme elements, and label the axes and legends. This part usually
involves some trial and error as you figure out where things need to be
positioned. To see what these various theme settings do you can change
them and observe the results.

\begin{Shaded}
\begin{Highlighting}[]
\KeywordTok{library}\NormalTok{(grid) }\CommentTok{# for the `unit()` function}

\NormalTok{(pc6 <-}\StringTok{ }\NormalTok{pc5 }\OperatorTok{+}
\StringTok{  }\KeywordTok{theme_minimal}\NormalTok{() }\OperatorTok{+}\StringTok{ }\CommentTok{# start with a minimal theme and add what we need}
\StringTok{  }\KeywordTok{theme}\NormalTok{(}\DataTypeTok{text =} \KeywordTok{element_text}\NormalTok{(}\DataTypeTok{color =} \StringTok{"gray20"}\NormalTok{),}
        \DataTypeTok{legend.position =} \KeywordTok{c}\NormalTok{(}\StringTok{"top"}\NormalTok{), }\CommentTok{# position the legend in the upper left }
        \DataTypeTok{legend.direction =} \StringTok{"horizontal"}\NormalTok{,}
        \DataTypeTok{legend.justification =} \FloatTok{0.1}\NormalTok{, }\CommentTok{# anchor point for legend.position.}
        \DataTypeTok{legend.text =} \KeywordTok{element_text}\NormalTok{(}\DataTypeTok{size =} \DecValTok{11}\NormalTok{, }\DataTypeTok{color =} \StringTok{"gray10"}\NormalTok{),}
        \DataTypeTok{axis.text =} \KeywordTok{element_text}\NormalTok{(}\DataTypeTok{face =} \StringTok{"italic"}\NormalTok{),}
        \DataTypeTok{axis.title.x =} \KeywordTok{element_text}\NormalTok{(}\DataTypeTok{vjust =} \OperatorTok{-}\DecValTok{1}\NormalTok{), }\CommentTok{# move title away from axis}
        \DataTypeTok{axis.title.y =} \KeywordTok{element_text}\NormalTok{(}\DataTypeTok{vjust =} \DecValTok{2}\NormalTok{), }\CommentTok{# move away for axis}
        \DataTypeTok{axis.ticks.y =} \KeywordTok{element_blank}\NormalTok{(), }\CommentTok{# element_blank() is how we remove elements}
        \DataTypeTok{axis.line =} \KeywordTok{element_line}\NormalTok{(}\DataTypeTok{color =} \StringTok{"gray40"}\NormalTok{, }\DataTypeTok{size =} \FloatTok{0.5}\NormalTok{),}
        \DataTypeTok{axis.line.y =} \KeywordTok{element_blank}\NormalTok{(),}
        \DataTypeTok{panel.grid.major =} \KeywordTok{element_line}\NormalTok{(}\DataTypeTok{color =} \StringTok{"gray50"}\NormalTok{, }\DataTypeTok{size =} \FloatTok{0.5}\NormalTok{),}
        \DataTypeTok{panel.grid.major.x =} \KeywordTok{element_blank}\NormalTok{()}
\NormalTok{        ))}
\end{Highlighting}
\end{Shaded}

\subsubsection{Add model R2 \& source
note}\label{add-model-r2-source-note}

The last bit of information that we want to have on the graph is the
variance explained by the model represented by the trend line. Lets fit
that model and pull out the R2 first, then think about how to get it
onto the graph.

\begin{Shaded}
\begin{Highlighting}[]
\NormalTok{mR2 <-}\StringTok{ }\KeywordTok{summary}\NormalTok{(}\KeywordTok{lm}\NormalTok{(HDI }\OperatorTok{~}\StringTok{ }\NormalTok{CPI }\OperatorTok{+}\StringTok{ }\KeywordTok{log}\NormalTok{(CPI), }\DataTypeTok{data =}\NormalTok{ dat))}\OperatorTok{$}\NormalTok{r.squared}
\NormalTok{mR2 <-}\StringTok{ }\KeywordTok{paste0}\NormalTok{(}\KeywordTok{format}\NormalTok{(mR2, }\DataTypeTok{digits =} \DecValTok{2}\NormalTok{), }\StringTok{"%"}\NormalTok{)}
\end{Highlighting}
\end{Shaded}

OK, now that we've calculated the values, let's think about how to get
them on the graph. ggplot2 has an \texttt{annotate()} function, but this
is not convenient for adding elements outside the plot area. The
\texttt{grid} package has nice functions for doing this, so we'll use
those.

And here it is, our final version!

\begin{Shaded}
\begin{Highlighting}[]
\KeywordTok{png}\NormalTok{(}\DataTypeTok{file =} \StringTok{"images/econScatter10.png"}\NormalTok{, }\DataTypeTok{width =} \DecValTok{700}\NormalTok{, }\DataTypeTok{height =} \DecValTok{500}\NormalTok{)}
\NormalTok{p <-}\StringTok{ }\KeywordTok{ggplot}\NormalTok{(dat,}
            \DataTypeTok{mapping =} \KeywordTok{aes}\NormalTok{(}\DataTypeTok{x =}\NormalTok{ CPI, }\DataTypeTok{y =}\NormalTok{ HDI)) }\OperatorTok{+}
\StringTok{  }\KeywordTok{geom_smooth}\NormalTok{(}\DataTypeTok{mapping =} \KeywordTok{aes}\NormalTok{(}\DataTypeTok{linetype =} \StringTok{"r2"}\NormalTok{),}
              \DataTypeTok{method =} \StringTok{"lm"}\NormalTok{,}
              \DataTypeTok{formula =}\NormalTok{ y }\OperatorTok{~}\StringTok{ }\NormalTok{x }\OperatorTok{+}\StringTok{ }\KeywordTok{log}\NormalTok{(x), }\DataTypeTok{se =} \OtherTok{FALSE}\NormalTok{,}
              \DataTypeTok{color =} \StringTok{"red"}\NormalTok{) }\OperatorTok{+}
\StringTok{  }\KeywordTok{geom_point}\NormalTok{(}\DataTypeTok{mapping =} \KeywordTok{aes}\NormalTok{(}\DataTypeTok{color =}\NormalTok{ Region),}
             \DataTypeTok{shape =} \DecValTok{1}\NormalTok{,}
             \DataTypeTok{size =} \DecValTok{4}\NormalTok{,}
             \DataTypeTok{stroke =} \FloatTok{1.5}\NormalTok{) }\OperatorTok{+}
\StringTok{  }\KeywordTok{geom_text_repel}\NormalTok{(}\DataTypeTok{mapping =} \KeywordTok{aes}\NormalTok{(}\DataTypeTok{label =}\NormalTok{ Country, }\DataTypeTok{alpha =}\NormalTok{ labels),}
                  \DataTypeTok{color =} \StringTok{"gray20"}\NormalTok{,}
                  \DataTypeTok{data =} \KeywordTok{transform}\NormalTok{(dat,}
                                   \DataTypeTok{labels =}\NormalTok{ Country }\OperatorTok\StringTok{ }\KeywordTok{c}\NormalTok{(}\StringTok{"Russia"}\NormalTok{,}
                                                           \StringTok{"Venezuela"}\NormalTok{,}
                                                           \StringTok{"Iraq"}\NormalTok{,}
                                                           \StringTok{"Mayanmar"}\NormalTok{,}
                                                           \StringTok{"Sudan"}\NormalTok{,}
                                                           \StringTok{"Afghanistan"}\NormalTok{,}
                                                           \StringTok{"Congo"}\NormalTok{,}
                                                           \StringTok{"Greece"}\NormalTok{,}
                                                           \StringTok{"Argentinia"}\NormalTok{,}
                                                           \StringTok{"Italy"}\NormalTok{,}
                                                           \StringTok{"Brazil"}\NormalTok{,}
                                                           \StringTok{"India"}\NormalTok{,}
                                                           \StringTok{"China"}\NormalTok{,}
                                                           \StringTok{"South Africa"}\NormalTok{,}
                                                           \StringTok{"Spain"}\NormalTok{,}
                                                           \StringTok{"Cape Verde"}\NormalTok{,}
                                                           \StringTok{"Bhutan"}\NormalTok{,}
                                                           \StringTok{"Rwanda"}\NormalTok{,}
                                                           \StringTok{"France"}\NormalTok{,}
                                                           \StringTok{"Botswana"}\NormalTok{,}
                                                           \StringTok{"France"}\NormalTok{,}
                                                           \StringTok{"US"}\NormalTok{,}
                                                           \StringTok{"Germany"}\NormalTok{,}
                                                           \StringTok{"Britain"}\NormalTok{,}
                                                           \StringTok{"Barbados"}\NormalTok{,}
                                                           \StringTok{"Japan"}\NormalTok{,}
                                                           \StringTok{"Norway"}\NormalTok{,}
                                                           \StringTok{"New Zealand"}\NormalTok{,}
                                                           \StringTok{"Sigapore"}\NormalTok{))) }\OperatorTok{+}
\StringTok{  }\KeywordTok{scale_x_continuous}\NormalTok{(}\DataTypeTok{name =} \StringTok{"Corruption Perception Index, 2011 (10=least corrupt)"}\NormalTok{,}
                     \DataTypeTok{limits =} \KeywordTok{c}\NormalTok{(}\FloatTok{1.0}\NormalTok{, }\FloatTok{10.0}\NormalTok{),}
                     \DataTypeTok{breaks =} \DecValTok{1}\OperatorTok{:}\DecValTok{10}\NormalTok{) }\OperatorTok{+}
\StringTok{  }\KeywordTok{scale_y_continuous}\NormalTok{(}\DataTypeTok{name =} \StringTok{"Human Development Index, 2011 (1=best)"}\NormalTok{,}
                     \DataTypeTok{limits =} \KeywordTok{c}\NormalTok{(}\FloatTok{0.2}\NormalTok{, }\FloatTok{1.0}\NormalTok{),}
                     \DataTypeTok{breaks =} \KeywordTok{seq}\NormalTok{(}\FloatTok{0.2}\NormalTok{, }\FloatTok{1.0}\NormalTok{, }\DataTypeTok{by =} \FloatTok{0.1}\NormalTok{)) }\OperatorTok{+}
\StringTok{  }\KeywordTok{scale_color_manual}\NormalTok{(}\DataTypeTok{name =} \StringTok{""}\NormalTok{,}
                     \DataTypeTok{values =} \KeywordTok{c}\NormalTok{(}\StringTok{"#24576D"}\NormalTok{,}
                                \StringTok{"#099DD7"}\NormalTok{,}
                                \StringTok{"#28AADC"}\NormalTok{,}
                                \StringTok{"#248E84"}\NormalTok{,}
                                \StringTok{"#F2583F"}\NormalTok{,}
                                \StringTok{"#96503F"}\NormalTok{),}
                     \DataTypeTok{guide =} \KeywordTok{guide_legend}\NormalTok{(}\DataTypeTok{nrow =} \DecValTok{1}\NormalTok{, }\DataTypeTok{order=}\DecValTok{1}\NormalTok{)) }\OperatorTok{+}
\StringTok{  }\KeywordTok{scale_alpha_discrete}\NormalTok{(}\DataTypeTok{range =} \KeywordTok{c}\NormalTok{(}\DecValTok{0}\NormalTok{, }\DecValTok{1}\NormalTok{),}
                       \DataTypeTok{guide =} \OtherTok{FALSE}\NormalTok{) }\OperatorTok{+}
\StringTok{  }\KeywordTok{scale_linetype}\NormalTok{(}\DataTypeTok{name =} \StringTok{""}\NormalTok{,}
                 \DataTypeTok{breaks =} \StringTok{"r2"}\NormalTok{,}
                 \DataTypeTok{labels =} \KeywordTok{list}\NormalTok{(}\KeywordTok{bquote}\NormalTok{(R}\OperatorTok{^}\DecValTok{2}\OperatorTok{==}\NormalTok{.(mR2))),}
                 \DataTypeTok{guide =} \KeywordTok{guide_legend}\NormalTok{(}\DataTypeTok{override.aes =} \KeywordTok{list}\NormalTok{(}\DataTypeTok{linetype =} \DecValTok{1}\NormalTok{, }\DataTypeTok{size =} \DecValTok{2}\NormalTok{, }\DataTypeTok{color =} \StringTok{"red"}\NormalTok{), }\DataTypeTok{order=}\DecValTok{2}\NormalTok{)) }\OperatorTok{+}
\StringTok{  }\KeywordTok{ggtitle}\NormalTok{(}\StringTok{"Corruption and human development"}\NormalTok{) }\OperatorTok{+}
\StringTok{  }\KeywordTok{labs}\NormalTok{(}\DataTypeTok{caption=}\StringTok{"Sources: Transparency International; UN Human Development Report"}\NormalTok{) }\OperatorTok{+}
\StringTok{  }\KeywordTok{theme_bw}\NormalTok{() }\OperatorTok{+}
\StringTok{  }\KeywordTok{theme}\NormalTok{(}\DataTypeTok{panel.border =} \KeywordTok{element_blank}\NormalTok{(),}
        \DataTypeTok{panel.grid =} \KeywordTok{element_blank}\NormalTok{(),}
        \DataTypeTok{panel.grid.major.y =} \KeywordTok{element_line}\NormalTok{(}\DataTypeTok{color =} \StringTok{"gray"}\NormalTok{),}
        \DataTypeTok{text =} \KeywordTok{element_text}\NormalTok{(}\DataTypeTok{color =} \StringTok{"gray20"}\NormalTok{),}
        \DataTypeTok{axis.title.x =} \KeywordTok{element_text}\NormalTok{(}\DataTypeTok{face=}\StringTok{"italic"}\NormalTok{),}
        \DataTypeTok{axis.title.y =} \KeywordTok{element_text}\NormalTok{(}\DataTypeTok{face=}\StringTok{"italic"}\NormalTok{),}
        \DataTypeTok{legend.position =} \StringTok{"top"}\NormalTok{,}
        \DataTypeTok{legend.direction =} \StringTok{"horizontal"}\NormalTok{,}
        \DataTypeTok{legend.box =} \StringTok{"horizontal"}\NormalTok{,}
        \DataTypeTok{legend.text =} \KeywordTok{element_text}\NormalTok{(}\DataTypeTok{size =} \DecValTok{12}\NormalTok{),}
        \DataTypeTok{plot.caption =} \KeywordTok{element_text}\NormalTok{(}\DataTypeTok{hjust=}\DecValTok{0}\NormalTok{),}
        \DataTypeTok{plot.title =} \KeywordTok{element_text}\NormalTok{(}\DataTypeTok{size =} \DecValTok{16}\NormalTok{, }\DataTypeTok{face =} \StringTok{"bold"}\NormalTok{))}
\NormalTok{p}

\KeywordTok{dev.off}\NormalTok{()}
\end{Highlighting}
\end{Shaded}

Comparing it to the original suggests that we've got most of the
important elements.

\section{Exercise solutions}\label{exercise-solutions-2}

\subsection{Ex 0: prototype}\label{ex-0-prototype-2}

\begin{enumerate}
\def\labelenumi{\arabic{enumi}.}
\tightlist
\item
  Create a scatter plot with CPI on the x axis and HDI on the y axis.
\end{enumerate}

\begin{Shaded}
\begin{Highlighting}[]
\KeywordTok{ggplot}\NormalTok{(dat, }\KeywordTok{aes}\NormalTok{(}\DataTypeTok{x =}\NormalTok{ CPI, }\DataTypeTok{y =}\NormalTok{ HDI)) }\OperatorTok{+}
\StringTok{  }\KeywordTok{geom_point}\NormalTok{()}
\end{Highlighting}
\end{Shaded}

\begin{enumerate}
\def\labelenumi{\arabic{enumi}.}
\setcounter{enumi}{1}
\tightlist
\item
  Color the points in the previous plot blue.
\end{enumerate}

\begin{Shaded}
\begin{Highlighting}[]
\KeywordTok{ggplot}\NormalTok{(dat, }\KeywordTok{aes}\NormalTok{(}\DataTypeTok{x =}\NormalTok{ CPI, }\DataTypeTok{y =}\NormalTok{ HDI)) }\OperatorTok{+}
\StringTok{  }\KeywordTok{geom_point}\NormalTok{(}\DataTypeTok{color =} \StringTok{"blue"}\NormalTok{)}
\end{Highlighting}
\end{Shaded}

\begin{enumerate}
\def\labelenumi{\arabic{enumi}.}
\setcounter{enumi}{2}
\tightlist
\item
  Color the points in the previous plot according to \emph{Region}.
\end{enumerate}

\begin{Shaded}
\begin{Highlighting}[]
\KeywordTok{ggplot}\NormalTok{(dat, }\KeywordTok{aes}\NormalTok{(}\DataTypeTok{x =}\NormalTok{ CPI, }\DataTypeTok{y =}\NormalTok{ HDI)) }\OperatorTok{+}
\StringTok{  }\KeywordTok{geom_point}\NormalTok{(}\KeywordTok{aes}\NormalTok{(}\DataTypeTok{color =}\NormalTok{ Region))}
\end{Highlighting}
\end{Shaded}

\begin{enumerate}
\def\labelenumi{\arabic{enumi}.}
\setcounter{enumi}{3}
\tightlist
\item
  Make the points bigger by setting size to 2
\end{enumerate}

\begin{Shaded}
\begin{Highlighting}[]
\KeywordTok{ggplot}\NormalTok{(dat, }\KeywordTok{aes}\NormalTok{(}\DataTypeTok{x =}\NormalTok{ CPI, }\DataTypeTok{y =}\NormalTok{ HDI)) }\OperatorTok{+}
\StringTok{  }\KeywordTok{geom_point}\NormalTok{(}\KeywordTok{aes}\NormalTok{(}\DataTypeTok{color =}\NormalTok{ Region), }\DataTypeTok{size =} \DecValTok{2}\NormalTok{)}
\end{Highlighting}
\end{Shaded}

\begin{enumerate}
\def\labelenumi{\arabic{enumi}.}
\setcounter{enumi}{4}
\tightlist
\item
  Map the size of the points to HDI.Rank
\end{enumerate}

\begin{Shaded}
\begin{Highlighting}[]
\KeywordTok{ggplot}\NormalTok{(dat, }\KeywordTok{aes}\NormalTok{(}\DataTypeTok{x =}\NormalTok{ CPI, }\DataTypeTok{y =}\NormalTok{ HDI)) }\OperatorTok{+}
\KeywordTok{geom_point}\NormalTok{(}\KeywordTok{aes}\NormalTok{(}\DataTypeTok{color =}\NormalTok{ Region, }\DataTypeTok{size =}\NormalTok{  HDI_Rank))}
\end{Highlighting}
\end{Shaded}

\subsection{Ex 1: prototype}\label{ex-1-prototype-2}

\begin{enumerate}
\def\labelenumi{\arabic{enumi}.}
\tightlist
\item
  Re-create a scatter plot with CPI on the x axis and HDI on the y axis
  (as you did in the previous exercise).
\end{enumerate}

\begin{Shaded}
\begin{Highlighting}[]
\KeywordTok{ggplot}\NormalTok{(dat, }\KeywordTok{aes}\NormalTok{(}\DataTypeTok{x =}\NormalTok{ CPI, }\DataTypeTok{y =}\NormalTok{ HDI)) }\OperatorTok{+}
\StringTok{  }\KeywordTok{geom_point}\NormalTok{()}
\end{Highlighting}
\end{Shaded}

\begin{enumerate}
\def\labelenumi{\arabic{enumi}.}
\setcounter{enumi}{1}
\tightlist
\item
  Overlay a smoothing line on top of the scatter plot using
  \texttt{geom\_smooth}
\end{enumerate}

\begin{Shaded}
\begin{Highlighting}[]
\KeywordTok{ggplot}\NormalTok{(dat, }\KeywordTok{aes}\NormalTok{(}\DataTypeTok{x =}\NormalTok{ CPI, }\DataTypeTok{y =}\NormalTok{ HDI)) }\OperatorTok{+}
\StringTok{  }\KeywordTok{geom_point}\NormalTok{() }\OperatorTok{+}
\StringTok{  }\KeywordTok{geom_smooth}\NormalTok{()}
\end{Highlighting}
\end{Shaded}

\begin{enumerate}
\def\labelenumi{\arabic{enumi}.}
\setcounter{enumi}{2}
\tightlist
\item
  Overlay a smoothing line on top of the scatter plot using
  \texttt{geom\_smooth}, but use a linear model for the predictions.
  Hint: see \texttt{?stat\_smooth}.
\end{enumerate}

\begin{Shaded}
\begin{Highlighting}[]
\KeywordTok{ggplot}\NormalTok{(dat, }\KeywordTok{aes}\NormalTok{(}\DataTypeTok{x =}\NormalTok{ CPI, }\DataTypeTok{y =}\NormalTok{ HDI)) }\OperatorTok{+}
\StringTok{  }\KeywordTok{geom_point}\NormalTok{() }\OperatorTok{+}
\StringTok{  }\KeywordTok{geom_smooth}\NormalTok{(}\DataTypeTok{method =} \StringTok{"lm"}\NormalTok{)}
\end{Highlighting}
\end{Shaded}

\begin{enumerate}
\def\labelenumi{\arabic{enumi}.}
\setcounter{enumi}{3}
\tightlist
\item
  Overlay a loess (method = ``loess'') smoothling line on top of the
  scatter plot using \texttt{geom\_line}. Hint: change the statistical
  transformation.
\end{enumerate}

\begin{Shaded}
\begin{Highlighting}[]
\KeywordTok{ggplot}\NormalTok{(dat, }\KeywordTok{aes}\NormalTok{(}\DataTypeTok{x =}\NormalTok{ CPI, }\DataTypeTok{y =}\NormalTok{ HDI)) }\OperatorTok{+}
\StringTok{  }\KeywordTok{geom_point}\NormalTok{() }\OperatorTok{+}
\StringTok{  }\KeywordTok{geom_line}\NormalTok{(}\DataTypeTok{stat =} \StringTok{"smooth"}\NormalTok{, }\DataTypeTok{method =} \StringTok{"loess"}\NormalTok{)}
\end{Highlighting}
\end{Shaded}

\begin{enumerate}
\def\labelenumi{\arabic{enumi}.}
\setcounter{enumi}{3}
\tightlist
\item
  BONUS: Overlay a smoothing line on top of the scatter plot using the
  \emph{loess} method, but make it less smooth. Hint: see
  \texttt{?loess}.
\end{enumerate}

\begin{Shaded}
\begin{Highlighting}[]
\KeywordTok{ggplot}\NormalTok{(dat, }\KeywordTok{aes}\NormalTok{(}\DataTypeTok{x =}\NormalTok{ CPI, }\DataTypeTok{y =}\NormalTok{ HDI)) }\OperatorTok{+}
\StringTok{  }\KeywordTok{geom_point}\NormalTok{() }\OperatorTok{+}
\StringTok{  }\KeywordTok{geom_smooth}\NormalTok{(}\DataTypeTok{span =}\NormalTok{ .}\DecValTok{4}\NormalTok{)}
\end{Highlighting}
\end{Shaded}

\subsection{Ex 2: prototype}\label{ex-2-prototype-1}

\begin{enumerate}
\def\labelenumi{\arabic{enumi}.}
\tightlist
\item
  Create a scatter plot with CPI on the x axis and HDI on the y axis.
  Color the points to indicate region.
\end{enumerate}

\begin{Shaded}
\begin{Highlighting}[]
\KeywordTok{ggplot}\NormalTok{(dat, }\KeywordTok{aes}\NormalTok{(}\DataTypeTok{x =}\NormalTok{ CPI, }\DataTypeTok{y =}\NormalTok{ HDI, }\DataTypeTok{color =}\NormalTok{ Region)) }\OperatorTok{+}
\StringTok{  }\KeywordTok{geom_point}\NormalTok{()}
\end{Highlighting}
\end{Shaded}

\begin{enumerate}
\def\labelenumi{\arabic{enumi}.}
\setcounter{enumi}{1}
\tightlist
\item
  Modify the x, y, and color scales so that they have more
  easily-understood names (e.g., spell out ``Human development Index''
  instead of ``HDI'').
\end{enumerate}

\begin{Shaded}
\begin{Highlighting}[]
\KeywordTok{ggplot}\NormalTok{(dat, }\KeywordTok{aes}\NormalTok{(}\DataTypeTok{x =}\NormalTok{ CPI, }\DataTypeTok{y =}\NormalTok{ HDI, }\DataTypeTok{color =}\NormalTok{ Region)) }\OperatorTok{+}
\KeywordTok{geom_point}\NormalTok{() }\OperatorTok{+}
\KeywordTok{scale_x_continuous}\NormalTok{(}\DataTypeTok{name =} \StringTok{"Corruption Perception Index"}\NormalTok{) }\OperatorTok{+}
\KeywordTok{scale_y_continuous}\NormalTok{(}\DataTypeTok{name =} \StringTok{"Human Development Index"}\NormalTok{) }\OperatorTok{+}
\KeywordTok{scale_color_discrete}\NormalTok{(}\DataTypeTok{name =} \StringTok{"Region of the world"}\NormalTok{)}
\end{Highlighting}
\end{Shaded}

\begin{enumerate}
\def\labelenumi{\arabic{enumi}.}
\setcounter{enumi}{2}
\tightlist
\item
  Modify the color scale to use specific values of your choosing. Hint:
  see \texttt{?scale\_color\_manual}.
\end{enumerate}

\begin{Shaded}
\begin{Highlighting}[]
\KeywordTok{ggplot}\NormalTok{(dat, }\KeywordTok{aes}\NormalTok{(}\DataTypeTok{x =}\NormalTok{ CPI, }\DataTypeTok{y =}\NormalTok{ HDI, }\DataTypeTok{color =}\NormalTok{ Region)) }\OperatorTok{+}
\KeywordTok{geom_point}\NormalTok{() }\OperatorTok{+}
\KeywordTok{scale_x_continuous}\NormalTok{(}\DataTypeTok{name =} \StringTok{"Corruption Perception Index"}\NormalTok{) }\OperatorTok{+}
\KeywordTok{scale_y_continuous}\NormalTok{(}\DataTypeTok{name =} \StringTok{"Human Development Index"}\NormalTok{) }\OperatorTok{+}
\StringTok{  }\KeywordTok{scale_color_manual}\NormalTok{(}\DataTypeTok{name =} \StringTok{"Region of the world"}\NormalTok{,}
                     \DataTypeTok{values =} \KeywordTok{c}\NormalTok{(}\StringTok{"#24576D"}\NormalTok{,}
                                \StringTok{"#099DD7"}\NormalTok{,}
                                \StringTok{"#28AADC"}\NormalTok{,}
                                \StringTok{"#248E84"}\NormalTok{,}
                                \StringTok{"#F2583F"}\NormalTok{,}
                                \StringTok{"#96503F"}\NormalTok{))}
\end{Highlighting}
\end{Shaded}

\section{Wrap-up}\label{wrap-up-3}

\subsection{Feedback}\label{feedback-3}

These workshops are a work in progress, please provide any feedback to:
\href{mailto:help@iq.harvard.edu}{\nolinkurl{help@iq.harvard.edu}}

\subsection{Resources}\label{resources-3}

\begin{itemize}
\tightlist
\item
  IQSS

  \begin{itemize}
  \tightlist
  \item
    Workshops: \url{https://dss.iq.harvard.edu/workshop-materials}
  \item
    Data Science Services: \url{https://dss.iq.harvard.edu/}
  \item
    Research Computing Environment:
    \url{https://iqss.github.io/dss-rce/}
  \end{itemize}
\item
  HBS

  \begin{itemize}
  \tightlist
  \item
    Research Computing Services workshops:
    \url{https://training.rcs.hbs.org/workshops}
  \item
    Other HBS RCS resources:
    \url{https://training.rcs.hbs.org/workshop-materials}
  \item
    RCS consulting email: \url{mailto:research@hbs.edu}
  \end{itemize}
\item
  ggplot2

  \begin{itemize}
  \tightlist
  \item
    Reference: \url{https://ggplot2.tidyverse.org/reference/}
  \item
    Cheatsheets:
    \url{https://rstudio.com/wp-content/uploads/2019/01/Cheatsheets_2019.pdf}
  \item
    Mailing list: \url{http://groups.google.com/group/ggplot2}
  \item
    Wiki: \url{https://github.com/hadley/ggplot2/wiki}
  \item
    Website: \url{http://had.co.nz/ggplot2/}
  \item
    StackOverflow:
    \url{http://stackoverflow.com/questions/tagged/ggplot}
  \end{itemize}
\end{itemize}

\chapter{R Data Wrangling}\label{r-data-wrangling}

\textbf{Topics}

\begin{itemize}
\tightlist
\item
  Iterating over files
\item
  Filtering with regular expressions (regex)
\item
  Writing your own functions
\item
  Reshaping data
\item
  Loading Excel worksheets
\end{itemize}

\section{Setup}\label{setup-3}

\subsection{Software \& materials}\label{software-materials-3}

You should have R and RStudio installed --- if not:

\begin{itemize}
\tightlist
\item
  Download and install R: \url{http://cran.r-project.org}
\item
  Download and install RStudio:
  \url{https://www.rstudio.com/products/rstudio/download/\#download}
\end{itemize}

Download materials:

\begin{itemize}
\tightlist
\item
  Download class materials at
  \url{https://github.com/IQSS/dss-workshops-redux/raw/master/R/RDataWrangling.zip}
\item
  Extract materials from the zipped directory
  \texttt{RDataWrangling.zip} (Right-click =\textgreater{} Extract All
  on Windows, double-click on Mac) and move them to your desktop!
\end{itemize}

Start RStudio and create a new project:

\begin{itemize}
\tightlist
\item
  On Windows click the start button and search for RStudio. On Mac
  RStudio will be in your applications folder.
\item
  In Rstudio go to \texttt{File\ -\textgreater{}\ New\ Project}.
\item
  Choose \texttt{Existing\ Directory} and browse to the
  \texttt{RDataWrangling} directory.
\item
  Choose \texttt{File\ -\textgreater{}\ Open\ File} and select the blank
  version of the \texttt{.Rmd} file.
\end{itemize}

While R's built-in packages are powerful, in recent years there has been
a big surge in well-designed \emph{contributed packages} for R. In
particular, a collection of R packages called
\href{https://www.tidyverse.org/}{tidyverse} have been designed
specifically for data science. All packages included in
\texttt{tidyverse} share an underlying design philosophy, grammar, and
data structures. We will use \texttt{tidyverse} packages throughout the
workshop, so let's install them now:

\begin{Shaded}
\begin{Highlighting}[]
\CommentTok{# install.packages("tidyverse")}
\KeywordTok{library}\NormalTok{(tidyverse)}
\end{Highlighting}
\end{Shaded}

\subsection{Goals}\label{goals-3}

Class Structure and Organization:

\begin{itemize}
\tightlist
\item
  Ask questions at any time. Really!
\item
  Collaboration is encouraged - please spend a minute introducing
  yourself to your neighbors!
\end{itemize}

This is an intermediate R course:

\begin{itemize}
\tightlist
\item
  Assumes working knowledge of R
\item
  Relatively fast-paced
\item
  Data scientists are known and celebrated for modeling and visually
  displaying information, but down in the data science engine room there
  is a lot of less glamorous work to be done. Before data can be used
  effectively it must often be cleaned, corrected, and reformatted. This
  workshop introduces the basic tools needed to make your data behave,
  including data reshaping, regular expressions and other text
  manipulation tools.
\end{itemize}

\section{Example project}\label{example-project}

It is common for data to be made available on a website somewhere,
either by a government agency, research group, or other organizations
and entities. Often the data you want is spread over many files, and
retrieving it all one file at a time is tedious and time consuming. Such
is the case with the baby names data we will be using today.

The UK \href{https://www.ons.gov.uk}{Office for National Statistics}
provides yearly data on the most popular baby names going back to 1996.
The data is provided separately for boys and girls and is stored in
Excel spreadsheets.

I have downloaded all the excel files containing boys names data from
\url{https://www.ons.gov.uk/peoplepopulationandcommunity/birthsdeathsandmarriages/livebirths/datasets/babynamesenglandandwalesbabynamesstatisticsboys}
and made them available at
\url{http://tutorials.iq.harvard.edu/R/RDataManagement/data/boysNames.zip}.

Our mission is to extract and graph the \textbf{top 100} boys names in
England and Wales for every year since 1996. There are several things
that make this challenging.

\subsection{Problems with the data}\label{problems-with-the-data}

While it was good of the UK Office for National Statistics to provide
baby name data, they were not very diligent about arranging it in a
convenient or consistent format.

\section{Exercise 0}\label{exercise-0-3}

Our mission is to extract and graph the \textbf{top 100} boys names in
England and Wales for every year since 1996. There are several things
that make this challenging.

\begin{enumerate}
\def\labelenumi{\arabic{enumi}.}
\item
  Locate the file named \texttt{1996boys\_tcm77-254026.xlsx} and open it
  in a spreadsheet. (If you don't have a spreadsheet program installed
  on your computer you can downloads one from
  \url{https://www.libreoffice.org/download/download/}). What issues can
  you identify that might make working with these data more difficult?
\item
  Locate the file named \texttt{2015boysnamesfinal.xlsx} and open it in
  a spreadsheet. In what ways is the format different than the format of
  \texttt{1996boys\_tcm77-254026.xlsx}? How might these differences make
  it more difficult to work with these data?
\end{enumerate}

\section{Working with Excel
worksheets}\label{working-with-excel-worksheets}

As you can see, the data is in quite a messy state. Note that this is
not a contrived example; this is exactly the way the data came to us
from the UK government website! Let's start cleaning and organizing it.

Each Excel file contains a worksheet with the baby names data we want.
Each file also contains additional supplemental worksheets that we are
not currently interested in. As noted above, the worksheet of interest
differs from year to year, but always has ``Table 1'' in the sheet name.

The first step is to get a vector of file names.

\begin{Shaded}
\begin{Highlighting}[]
\NormalTok{boy.file.names <-}\StringTok{ }\KeywordTok{list.files}\NormalTok{(}\StringTok{"dataSets/boys"}\NormalTok{, }\DataTypeTok{full.names =} \OtherTok{TRUE}\NormalTok{)}
\end{Highlighting}
\end{Shaded}

Now that we've told R the names of the data files we can start working
with them. For example, the first file is

\begin{Shaded}
\begin{Highlighting}[]
\NormalTok{boy.file.names[[}\DecValTok{1}\NormalTok{]]}
\end{Highlighting}
\end{Shaded}

and we can use the \texttt{excel\_sheets()} function from the
\emph{readxl} package to list the worksheet names from this file.

\begin{Shaded}
\begin{Highlighting}[]
\KeywordTok{library}\NormalTok{(readxl)}

\KeywordTok{excel_sheets}\NormalTok{(boy.file.names[[}\DecValTok{1}\NormalTok{]])}
\end{Highlighting}
\end{Shaded}

\subsection{\texorpdfstring{Iterating over file names with
\texttt{map()}}{Iterating over file names with map()}}\label{iterating-over-file-names-with-map}

Now that we know how to retrieve the names of the worksheets in an Excel
file we could start writing code to extract the sheet names from each
file, e.g.,

\begin{Shaded}
\begin{Highlighting}[]
\KeywordTok{excel_sheets}\NormalTok{(boy.file.names[[}\DecValTok{1}\NormalTok{]])}
\KeywordTok{excel_sheets}\NormalTok{(boy.file.names[[}\DecValTok{2}\NormalTok{]])}
\NormalTok{## ...}
\KeywordTok{excel_sheets}\NormalTok{(boy.file.names[[}\DecValTok{20}\NormalTok{]])}
\end{Highlighting}
\end{Shaded}

This is not a terrible idea for a small number of files, but it is more
convenient to let R do the iteration for us. We could use a
\texttt{for\ loop}, or \texttt{sapply()}, but the \texttt{map()} family
of functions from the \texttt{purrr} package gives us a more consistent
alternative, so we'll use that.

\begin{Shaded}
\begin{Highlighting}[]
\KeywordTok{library}\NormalTok{(purrr)}
\KeywordTok{map}\NormalTok{(boy.file.names, excel_sheets)}
\end{Highlighting}
\end{Shaded}

\subsection{Filtering strings using regular
expressions}\label{filtering-strings-using-regular-expressions}

In order extract the correct worksheet names we need a way to extract
strings containing ``Table 1''. Base R provides some string manipulation
capabilities (see \texttt{?regex}, \texttt{?sub} and \texttt{?grep}),
but we will use the \emph{stringr} package because it is more
user-friendly.

The \emph{stringr} package provides functions to \emph{detect},
\emph{locate}, \emph{extract}, \emph{match}, \emph{replace},
\emph{combine} and \emph{split} strings (among other things).

Here we want to detect the pattern ``Table 1'', and only return elements
with this pattern. We can do that using the \texttt{str\_subset()}
function. The first argument to \texttt{str\_subset()} is character
vector we want to search in. The second argument is a \emph{regular
expression} matching the pattern we want to retain.

If you are not familiar with regular expressions,
\url{http://www.regexr.com/} is a good place to start.

Now that we know how to filter character vectors using
\texttt{str\_subset()} we can identify the correct sheet in a particular
Excel file. For example,

\begin{Shaded}
\begin{Highlighting}[]
\KeywordTok{library}\NormalTok{(stringr)}
\KeywordTok{str_subset}\NormalTok{(}\KeywordTok{excel_sheets}\NormalTok{(boy.file.names[[}\DecValTok{1}\NormalTok{]]), }\StringTok{"Table 1"}\NormalTok{)}
\end{Highlighting}
\end{Shaded}

\subsection{Writing your own
functions}\label{writing-your-own-functions}

The \texttt{map*} functions are useful when you want to apply a function
to a list or vector of inputs and obtain the return values. This is very
convenient when a function already exists that does exactly what you
want. In the examples above we mapped the \texttt{excel\_sheets()}
function to the elements of a vector containing file names. But now
there is no function that both retrieves worksheet names and subsets
them. Fortunately, writing functions in R is easy.

\begin{Shaded}
\begin{Highlighting}[]
\NormalTok{get.data.sheet.name <-}\StringTok{ }\ControlFlowTok{function}\NormalTok{(file, pattern) \{}
    \KeywordTok{str_subset}\NormalTok{(}\KeywordTok{excel_sheets}\NormalTok{(file), pattern)}
\NormalTok{\}}
\end{Highlighting}
\end{Shaded}

Now we can map this new function over our vector of file names.

\begin{Shaded}
\begin{Highlighting}[]
\KeywordTok{map}\NormalTok{(boy.file.names,}
\NormalTok{    get.data.sheet.name,}
    \DataTypeTok{pattern =} \StringTok{"Table 1"}\NormalTok{)}
\end{Highlighting}
\end{Shaded}

\section{Reading Excel data files}\label{reading-excel-data-files}

Now that we know the correct worksheet from each file we can actually
read those data into R. We can do that using the \texttt{read\_excel()}
function.

We'll start by reading the data from the first file, just to check that
it works. Recall that the actual data starts on row 7, so we want to
skip the first 6 rows.

\begin{Shaded}
\begin{Highlighting}[]
\NormalTok{tmp <-}\StringTok{ }\KeywordTok{read_excel}\NormalTok{(}
\NormalTok{    boy.file.names[}\DecValTok{1}\NormalTok{],}
    \DataTypeTok{sheet =} \KeywordTok{get.data.sheet.name}\NormalTok{(boy.file.names[}\DecValTok{1}\NormalTok{],}
                                \DataTypeTok{pattern =} \StringTok{"Table 1"}\NormalTok{),}
    \DataTypeTok{skip =} \DecValTok{6}\NormalTok{)}

\KeywordTok{library}\NormalTok{(dplyr, }\DataTypeTok{quietly=}\OtherTok{TRUE}\NormalTok{)}
\KeywordTok{glimpse}\NormalTok{(tmp)}
\end{Highlighting}
\end{Shaded}

\section{Exercise 1}\label{exercise-1-3}

\begin{enumerate}
\def\labelenumi{\arabic{enumi}.}
\item
  Write a function that takes a file name as an argument and reads the
  worksheet containing ``Table 1'' from that file. Don't forget to skip
  the first 6 rows.
\item
  Test your function by using it to read \emph{one} of the boys names
  Excel files.
\item
  Use the \texttt{map()} function to read data from all the Excel files,
  using the function you wrote in step 1.
\end{enumerate}

\section{Data cleanup}\label{data-cleanup}

Now that we've read in the data we still have some cleanup to do.
Specifically, we need to:

\begin{enumerate}
\def\labelenumi{\arabic{enumi}.}
\tightlist
\item
  fix column names
\item
  get rid of blank row and the top and the notes at the bottom
\item
  get rid of extraneous ``changes in rank'' columns if they exist
\item
  transform the side-by-side tables layout to a single table.
\end{enumerate}

In short, we want to go from this:

\begin{figure}
\centering
\includegraphics{images/messy.png}
\caption{messy}
\end{figure}

to this:

\begin{figure}
\centering
\includegraphics{images/clean.png}
\caption{tidy}
\end{figure}

There are many ways to do this kind of data manipulation in R. We're
going to use the \emph{dplyr} and \emph{tidyr} packages to make our
lives easier. (Both packages were installed as dependencies of the
\emph{tidyverse} package.)

\subsection{Selecting columns}\label{selecting-columns}

Next we want to retain just the \texttt{Name}, \texttt{Name\_\_1} and
\texttt{Count}, \texttt{Count\_\_1} columns. We can do that using the
\texttt{select()} function:

\begin{Shaded}
\begin{Highlighting}[]
\NormalTok{boysNames[[}\DecValTok{1}\NormalTok{]]}

\NormalTok{boysNames[[}\DecValTok{1}\NormalTok{]] <-}\StringTok{ }\KeywordTok{select}\NormalTok{(boysNames[[}\DecValTok{1}\NormalTok{]], Name, Name__}\DecValTok{1}\NormalTok{, Count, Count__}\DecValTok{1}\NormalTok{)}
\NormalTok{boysNames[[}\DecValTok{1}\NormalTok{]]}
\end{Highlighting}
\end{Shaded}

\subsection{Dropping missing values}\label{dropping-missing-values}

Next we want to remove blank rows and rows used for notes. An easy way
to do that is to use \texttt{drop\_na()} to remove rows with missing
values.

\begin{Shaded}
\begin{Highlighting}[]
\NormalTok{boysNames[[}\DecValTok{1}\NormalTok{]]}

\NormalTok{boysNames[[}\DecValTok{1}\NormalTok{]] <-}\StringTok{ }\KeywordTok{drop_na}\NormalTok{(boysNames[[}\DecValTok{1}\NormalTok{]])}
\NormalTok{boysNames[[}\DecValTok{1}\NormalTok{]]}
\end{Highlighting}
\end{Shaded}

Finally, we will want to filter out missing do this for all the elements
in \texttt{boysNames}, a task I leave to you.

\section{Exercise 2}\label{exercise-2-2}

\begin{enumerate}
\def\labelenumi{\arabic{enumi}.}
\item
  Write a function that takes a \texttt{data.frame} as an argument and
  returns a modified version including only columns named \texttt{Name},
  \texttt{Name\_\_1}, \texttt{Count}, or \texttt{Count\_\_1}.
\item
  Test your function on the first \texttt{data.frame} in the list of
  baby names data.
\item
  Use the \texttt{map()} function to each \texttt{data.frame} in the
  list of baby names data.
\end{enumerate}

\subsection{Re-arranging into a single
table}\label{re-arranging-into-a-single-table}

Our final task is to re-arrange to data so that it is all in a single
table instead of in two side-by-side tables. For many similar tasks the
\texttt{gather()} function in the \emph{tidyr} package is useful, but in
this case we will be better off using a combination of \texttt{select()}
and \texttt{bind\_rows()}.

\begin{Shaded}
\begin{Highlighting}[]
\NormalTok{boysNames[[}\DecValTok{1}\NormalTok{]]}
\KeywordTok{bind_rows}\NormalTok{(}\KeywordTok{select}\NormalTok{(boysNames[[}\DecValTok{1}\NormalTok{]], Name, Count),}
          \KeywordTok{select}\NormalTok{(boysNames[[}\DecValTok{1}\NormalTok{]], }\DataTypeTok{Name =}\NormalTok{ Name__}\DecValTok{1}\NormalTok{, }\DataTypeTok{Count =}\NormalTok{ Count__}\DecValTok{1}\NormalTok{))}
\end{Highlighting}
\end{Shaded}

\section{Exercise 3}\label{exercise-3-2}

\textbf{Cleanup all the data}

In the previous examples we learned how to drop empty rows with
\texttt{filter()}, select only relevant columns with \texttt{select()},
and re-arrange our data with \texttt{select()} and
\texttt{bind\_rows()}. In each case we applied the changes only to the
first element of our \texttt{boysNames} list.

Your task now is to use the \texttt{map()} function to apply each of
these transformations to all the elements in \texttt{boysNames}.

\section{Data organization \& storage}\label{data-organization-storage}

Now that we have the data cleaned up and augmented, we can turn our
attention to organizing and storing the data.

\subsection{One table for each year}\label{one-table-for-each-year}

Right now we have a list of tables, one for each year. This is not a bad
way to go. It has the advantage of making it easy to work with
individual years; it has the disadvantage of making it more difficult to
examine questions that require data from multiple years. To make the
arrangement of the data clearer it helps to name each element of the
list with the year it corresponds too.

\begin{Shaded}
\begin{Highlighting}[]
\KeywordTok{glimpse}\NormalTok{(}\KeywordTok{head}\NormalTok{(boysNames))}
\end{Highlighting}
\end{Shaded}

\begin{Shaded}
\begin{Highlighting}[]
\NormalTok{years <-}\StringTok{ }\KeywordTok{str_extract}\NormalTok{(boy.file.names, }\StringTok{"[0-9]\{4\}"}\NormalTok{)}
\NormalTok{boysNames <-}\StringTok{ }\KeywordTok{setNames}\NormalTok{(boysNames, years)}
\KeywordTok{glimpse}\NormalTok{(}\KeywordTok{head}\NormalTok{(boysNames))}
\end{Highlighting}
\end{Shaded}

\subsection{One big table}\label{one-big-table}

While storing the data in separate tables by year makes some sense, many
operations will be easier if the data is simply stored in one big table.
We've already seen how to turn a list of data.frames into a single
data.frame using \texttt{bind\_rows()}, but there is a problem; The year
information is stored in the names of the list elements, and so
flattening the tables into one will result in losing the year
information! Fortunately it is not too much trouble to add the year
information to each table before flattening.

\begin{Shaded}
\begin{Highlighting}[]
\NormalTok{boysNames <-}\StringTok{ }\KeywordTok{imap}\NormalTok{(boysNames,}
                  \ControlFlowTok{function}\NormalTok{(data, name) \{}
                      \KeywordTok{mutate}\NormalTok{(data, }\DataTypeTok{Year =} \KeywordTok{as.integer}\NormalTok{(name))}
\NormalTok{                      \})}
\NormalTok{boysNames <-}\StringTok{ }\KeywordTok{bind_rows}\NormalTok{(boysNames)}

\KeywordTok{glimpse}\NormalTok{(boysNames)}
\end{Highlighting}
\end{Shaded}

\section{Exercise 4}\label{exercise-4-1}

\textbf{Make one big table}

Turn the list of boys names data.frames into a single table.

Create a directory under \texttt{data/all} and write the data to a
\texttt{.csv} file.

Finally, repeat the previous exercise, this time working with the data
in one big table.

\section{Exercise solutions}\label{exercise-solutions-3}

\subsection{Ex 0: prototype}\label{ex-0-prototype-3}

\begin{quote}
\begin{enumerate}
\def\labelenumi{\arabic{enumi}.}
\tightlist
\item
  Locate the file named \texttt{1996boys\_tcm77-254026.xlsx} and open it
  in a spreadsheet. (If you don't have a spreadsheet program installed
  on your computer you can downloads one from
  \url{https://www.libreoffice.org/download/download/}). What issues can
  you identify that might make working with these data more difficult?
\end{enumerate}
\end{quote}

The data does not start on row one. Headers are on row 7, followed by a
blank line, followed by the actual data.

The data is stored in an inconvenient way, with ranks 1-50 in the first
set of columns and ranks 51-100 in a separate set of columns.

There are notes below the data.

\begin{quote}
\begin{enumerate}
\def\labelenumi{\arabic{enumi}.}
\setcounter{enumi}{2}
\tightlist
\item
  Locate the file named \texttt{2015boysnamesfinal.xlsx} and open it in
  a spreadsheet. In what ways is the format different than the format of
  \texttt{1996boys\_tcm77-254026.xlsx}? How might these differences make
  it more difficult to work with these data?
\end{enumerate}
\end{quote}

The worksheet containing the data of interest is in different positions
and has different names from one year to the next. However, it always
includes ``Table 1'' in the worksheet name.

Some years include columns for ``changes in rank'', others do not.

These differences will make it more difficult to automate re-arranging
the data since we have to write code that can handle different input
formats.

\subsection{Ex 1: prototype}\label{ex-1-prototype-3}

\begin{Shaded}
\begin{Highlighting}[]
\NormalTok{  ## 1. Write a function that takes a file name as an argument and reads}
\NormalTok{  ##    the worksheet containing "Table 1" from that file.}
\NormalTok{  read.baby.names <-}\StringTok{ }\ControlFlowTok{function}\NormalTok{(file) \{}
\NormalTok{      sheet.name <-}\StringTok{ }\KeywordTok{str_subset}\NormalTok{(}\KeywordTok{excel_sheets}\NormalTok{(file), }\StringTok{"Table 1"}\NormalTok{)}
      \KeywordTok{read_excel}\NormalTok{(file, }\DataTypeTok{sheet =}\NormalTok{ sheet.name, }\DataTypeTok{skip =} \DecValTok{6}\NormalTok{)}
\NormalTok{  \}}
  
\NormalTok{  ## 2. Test your function by using it to read *one* of the boys names}
\NormalTok{  ##    Excel files.}
  \KeywordTok{glimpse}\NormalTok{(}\KeywordTok{read.baby.names}\NormalTok{(boy.file.names[}\DecValTok{1}\NormalTok{]))}
     
\NormalTok{  ## 3. Use the `map` function to read data from all the Excel files,}
\NormalTok{  ##    using the function you wrote in step 1.}
\NormalTok{  boysNames <-}\StringTok{ }\KeywordTok{map}\NormalTok{(boy.file.names, read.baby.names)}
\end{Highlighting}
\end{Shaded}

\subsection{Ex 2: prototype}\label{ex-2-prototype-2}

\begin{Shaded}
\begin{Highlighting}[]
\NormalTok{  ## 1. Write a function that takes a `data.frame` as an argument and}
\NormalTok{  ##    returns a modified version including only columns named `Name`,}
\NormalTok{  ##    `Name__1`, `Count`, or `Count__1`.}

\NormalTok{  namecount <-}\StringTok{ }\ControlFlowTok{function}\NormalTok{(data) \{}
      \KeywordTok{select}\NormalTok{(data, Name, Name__}\DecValTok{1}\NormalTok{, Count, Count__}\DecValTok{1}\NormalTok{)}
\NormalTok{  \}}
     
\NormalTok{  ## 2. Test your function on the first `data.frame` in the list of baby}
\NormalTok{  ##    names data.}

  \KeywordTok{namecount}\NormalTok{(boysNames[[}\DecValTok{1}\NormalTok{]])}
  
\NormalTok{  ## 3. Use the `map` function to each `data.frame` in the list of baby}
\NormalTok{  ##    names data.}

\NormalTok{  babyNames <-}\StringTok{ }\KeywordTok{map}\NormalTok{(boysNames, namecount)}
\end{Highlighting}
\end{Shaded}

\subsection{Ex 3: prototype}\label{ex-3-prototype-2}

There are different ways you can go about it. Here is one:

\begin{Shaded}
\begin{Highlighting}[]
\NormalTok{## write a function that does all the cleanup}
\NormalTok{cleanupNamesData <-}\StringTok{ }\ControlFlowTok{function}\NormalTok{(x) \{}
\NormalTok{    filtered <-}\StringTok{ }\KeywordTok{filter}\NormalTok{(x, }\OperatorTok{!}\KeywordTok{is.na}\NormalTok{(Name)) }\CommentTok{# drop rows with no Name value}
\NormalTok{    selected <-}\StringTok{ }\KeywordTok{select}\NormalTok{(filtered, Name, Count, Name__}\DecValTok{1}\NormalTok{, Count__}\DecValTok{1}\NormalTok{) }\CommentTok{# select just Name and Count columns}
    \KeywordTok{bind_rows}\NormalTok{(}\KeywordTok{select}\NormalTok{(selected, Name,  Count), }\CommentTok{# re-arrange into two columns}
              \KeywordTok{select}\NormalTok{(selected, }\DataTypeTok{Name =}\NormalTok{ Name__}\DecValTok{1}\NormalTok{, }\DataTypeTok{Count =}\NormalTok{ Count__}\DecValTok{1}\NormalTok{))}
\NormalTok{\}}

\NormalTok{## test it out on the second data.frame in the list}
\KeywordTok{glimpse}\NormalTok{(boysNames[[}\DecValTok{2}\NormalTok{]]) }\CommentTok{# before cleanup}
\KeywordTok{glimpse}\NormalTok{(}\KeywordTok{cleanupNamesData}\NormalTok{(boysNames[[}\DecValTok{2}\NormalTok{]])) }\CommentTok{# after cleanup}

\NormalTok{## apply the cleanup function to all the data.frames in the list}
\NormalTok{boysNames <-}\StringTok{ }\KeywordTok{map}\NormalTok{(boysNames, cleanupNamesData)}
\end{Highlighting}
\end{Shaded}

\subsection{Ex 4: prototype}\label{ex-4-prototype-1}

Working with the data in one big table is often easier.

\begin{Shaded}
\begin{Highlighting}[]
\NormalTok{boysNames <-}\StringTok{ }\KeywordTok{bind_rows}\NormalTok{(boysNames)}

\KeywordTok{dir.create}\NormalTok{(}\StringTok{"data/all"}\NormalTok{)}

\KeywordTok{write_csv}\NormalTok{(boysNames, }\StringTok{"data/all/boys_names.csv"}\NormalTok{)}

\NormalTok{## What where the five most popular names in 2013?}
\KeywordTok{slice}\NormalTok{(}\KeywordTok{arrange}\NormalTok{(}\KeywordTok{filter}\NormalTok{(boysNames, Year }\OperatorTok{==}\StringTok{ }\DecValTok{2013}\NormalTok{),}
              \KeywordTok{desc}\NormalTok{(Count)),}
      \DecValTok{1}\OperatorTok{:}\DecValTok{5}\NormalTok{)}

\NormalTok{## How has the popularity of the name "ANDREW" changed over time?}
\NormalTok{andrew <-}\StringTok{ }\KeywordTok{filter}\NormalTok{(boysNames, Name }\OperatorTok{==}\StringTok{ "ANDREW"}\NormalTok{)}

\KeywordTok{ggplot}\NormalTok{(andrew, }\KeywordTok{aes}\NormalTok{(}\DataTypeTok{x =}\NormalTok{ Year, }\DataTypeTok{y =}\NormalTok{ Count)) }\OperatorTok{+}
\StringTok{    }\KeywordTok{geom_line}\NormalTok{() }\OperatorTok{+}
\StringTok{    }\KeywordTok{ggtitle}\NormalTok{(}\StringTok{"Popularity of }\CharTok{\textbackslash{}"}\StringTok{Andrew}\CharTok{\textbackslash{}"}\StringTok{, over time"}\NormalTok{)}
\end{Highlighting}
\end{Shaded}

\section{Wrap-up}\label{wrap-up-4}

\subsection{Feedback}\label{feedback-4}

These workshops are a work in progress, please provide any feedback to:
\href{mailto:help@iq.harvard.edu}{\nolinkurl{help@iq.harvard.edu}}

\subsection{Resources}\label{resources-4}

\begin{itemize}
\tightlist
\item
  IQSS

  \begin{itemize}
  \tightlist
  \item
    Workshops: \url{https://dss.iq.harvard.edu/workshop-materials}
  \item
    Data Science Services: \url{https://dss.iq.harvard.edu/}
  \item
    Research Computing Environment:
    \url{https://iqss.github.io/dss-rce/}
  \end{itemize}
\item
  HBS

  \begin{itemize}
  \tightlist
  \item
    Research Computing Services workshops:
    \url{https://training.rcs.hbs.org/workshops}
  \item
    Other HBS RCS resources:
    \url{https://training.rcs.hbs.org/workshop-materials}
  \item
    RCS consulting email: \url{mailto:research@hbs.edu}
  \end{itemize}
\item
  R

  \begin{itemize}
  \tightlist
  \item
    Learn from the best: \url{http://adv-r.had.co.nz/};
    \url{http://r4ds.had.co.nz/}
  \item
    R documentation: \url{http://cran.r-project.org/manuals.html}
  \item
    Collection of R tutorials:
    \url{http://cran.r-project.org/other-docs.html}
  \item
    R for Programmers (by Norman Matloff, UC--Davis)
    \url{http://heather.cs.ucdavis.edu/~matloff/R/RProg.pdf}
  \item
    Calling C and Fortran from R (by Charles Geyer, UMinn)
    \url{http://www.stat.umn.edu/~charlie/rc/}
  \item
    State of the Art in Parallel Computing with R (Schmidberger et al.)
    \url{http://www.jstatso}\textbar{}.org/v31/i01/paper
  \end{itemize}
\end{itemize}

\part{Python}\label{part-python}

\chapter{Python Introduction}\label{python-introduction}

\textbf{Topics}

\begin{itemize}
\tightlist
\item
  Reading data
\item
  Basic functions
\item
  Finding help
\item
  Indexing data objects
\item
  Working with text data
\item
  Conditional operations
\item
  Iterating over data structures
\item
  Lists and dictionaries
\end{itemize}

\section{Setup}\label{setup-4}

\subsection{Software \& Materials}\label{software-materials-4}

\subsubsection{Install the Anaconda Python
distribution}\label{install-the-anaconda-python-distribution}

If using your own computer please install the Anaconda Python
distribution from \url{https://www.anaconda.com/download/}. (Note that
Python version\(\leq\) 3.0 differs considerably from more recent
releases. For this workshop you will need version\(\geq\) 3.6.x)

Accepting the defaults proposed by the Anaconda installer is generally
recommended.

\subsubsection{Download workshop
materials}\label{download-workshop-materials}

\begin{itemize}
\tightlist
\item
  Download class materials at
  \url{https://github.com/IQSS/dss-workshops-redux/raw/master/Python/PythonIntro.zip}
\item
  Extract materials from the zipped directory \texttt{PythonIntro.zip}
  (Right-click =\textgreater{} Extract All on Windows, double-click on
  Mac) and move them to your desktop!
\end{itemize}

\subsubsection{Launch Jupyter Notebook}\label{launch-jupyter-notebook}

Start the \texttt{Anaconda\ Navigator} program in the usual way. Click
the or \texttt{Launch} button under \texttt{Jupyter\ Notebook}.

\subsection{Goals}\label{goals-4}

In this workshop you will * learn about the python package and
application ecosystem, * learn python language basics and common idioms,
and, * practice reading files and manipulating data in python.

A more general goal is to get you comfortable with Python so that it
seems less scary and mystifying than it perhaps does now. Note that this
is by no means a complete or thorough introduction to Python! It's just
enough to get by.

This workshop is relatively \emph{informal}, \emph{example-oriented},
and \emph{hands-on}. We won't spend much time examining language
features in detail. Instead we will work through an example, and learn
some things about the language along the way.

As an example project we will analyze the text of Lewis Carroll's
\emph{Alice's Adventures in Wonderland}. Among the questions we will use
Python to answer are: * How many total and unique words are there? * How
many chapters and paragraphs? * How many words are in each chapter, and
what is the average words per chapter? * How many times is each main
character mentioned?

\section{What is Python?}\label{what-is-python}

Python is a relatively easy to learn general purpose programming
language. People use Python to manipulate, analyze, and visualize data,
make web sites, write games, and much more. Youtube, DropBox, and
BitTorrent are among the things people used python to make.

Like most popular open source programming languages, Python can be
thought of as a \emph{platform} that runs a huge number and variety of
packages. The language itself is mostly valuable because it makes it
easy to create and use a large number of useful packages.

\section{How can I interact with
Python?}\label{how-can-i-interact-with-python}

A number of interfaces designed to make it easy to interact with Python
are available. The Anaconda distribution that we installed earlier
includes both a web-based \emph{Jupyter Notebook} and a more
conventional Integrated Development Environment called \emph{Spyder}.
For this workshop I encourage you to use \emph{Jupyter Notebook}. In
real life you should experiment and choose the interface that you find
most comfortable.

To get started, start the \emph{Jupyter Notebook} application, and
navigate to the \emph{PythonIntro} directory you downloaded and
extracted earlier. Start a new notebook by clicking
\texttt{New\ =\textgreater{}\ Python\ 3} as shown below.

\begin{figure}
\centering
\includegraphics{Python/PythonIntro/images/notebook_new.png}
\caption{notebook\_new.png}
\end{figure}

A Jupyter Notebook contains one or more \emph{cells} containing notes or
code. To insert a new cell click the \texttt{+} button in the upper
left. To execute a cell, select it and press \texttt{Control+Enter} or
click the \texttt{Run} button at the top.

\section{Reading the text of Alice in Wonderland from a
file}\label{reading-the-text-of-alice-in-wonderland-from-a-file}

Reading information from a file is the first step in many projects, so
we'll start there. The workshop materials you downloaded earlier include
a file named \texttt{Alice\_in\_wonderland.txt} which contains the text
of Lewis Carroll's \emph{Alice's Adventures in Wonderland}.

We can open a connection to a file using the \emph{open} function, and
store the result using the \texttt{=} operator.

\begin{Shaded}
\begin{Highlighting}[]
\NormalTok{alice_file }\OperatorTok{=} \BuiltInTok{open}\NormalTok{(}\StringTok{"Alice_in_wonderland.txt"}\NormalTok{)}
\end{Highlighting}
\end{Shaded}

The name on the left of the equals sign (\texttt{alice\_file}) is one
that we chose. When choosing names, \emph{start with a letter}, and use
only \emph{letters}, \emph{numbers} and \emph{underscores}.

The \texttt{alice\_file} object we just created does \emph{not} contain
the contents of \texttt{Alice\_in\_wonderland.txt}. It a representation
in Python of the \emph{file itself} rather than the \emph{contents} of
the file.

The \texttt{alice\_file} object provides \emph{methods} that we can use
to do things with it. Methods are invoked using syntax that looks like
\texttt{ObjectName.method()}. You can see the methods available for
acting on an object by typing the object's name followed by a \texttt{.}
and pressing the \texttt{tab} key. For example, typing
\texttt{alice\_file.} and pressing \texttt{tab} will display a list of
methods as shown below.
\includegraphics{Python/PythonIntro/images/notebook_file_completion.png}.

Among the methods we have for doing things with our \texttt{alice\_file}
object is one named \texttt{read}. We can use the \texttt{help} function
to learn more about it.

\begin{Shaded}
\begin{Highlighting}[]
\BuiltInTok{help}\NormalTok{(alice_file.read)}
\end{Highlighting}
\end{Shaded}

Since \texttt{alice\_file.read} looks promising, we will invoke this
method and see what it does.

\begin{Shaded}
\begin{Highlighting}[]
\NormalTok{alice_txt }\OperatorTok{=}\NormalTok{ alice_file.read()}
\BuiltInTok{print}\NormalTok{(alice_txt[:}\DecValTok{500}\NormalTok{]) }\CommentTok{# the [:500] gets the first 500 character -- more on this later.}
\end{Highlighting}
\end{Shaded}

That's all there is to it! We've read the contents of
\texttt{Alice\_in\_wonderland.txt} and stored this text in a Python
object we named \texttt{alice\_txt}. Now let's start to explore this
object, and learn some more things about Python along the way.

\section{Counting chapters, lines, \&
words}\label{counting-chapters-lines-words}

Now that we have the text we can start answering some questions about
it. To begin with, how many words does it contain? To answer this
question we can split the text up so there is one element per word, and
then count the number of words.

\subsection{Splitting a string into a list of
words}\label{splitting-a-string-into-a-list-of-words}

How do we figure out how to split strings in Python? By asking Python
what our \texttt{alice\_txt} object is and what methods it provides. We
can ask Python what things are using the \texttt{type} function, like
this:

\begin{Shaded}
\begin{Highlighting}[]
\BuiltInTok{type}\NormalTok{(alice_txt)}
\end{Highlighting}
\end{Shaded}

Python tells us that \texttt{alice\_txt} is of type \texttt{str} (i.e.,
it is a string). We can find out what methods are available for working
strings by typing \texttt{alice\_txt.} and pressing \texttt{tab}. We'll
see that among the methods is one named \texttt{split}, as shown below.
\includegraphics{Python/PythonIntro/images/notebook_string_completion.png}
To learn how to use this method we can check the documentation.

\begin{Shaded}
\begin{Highlighting}[]
\BuiltInTok{help}\NormalTok{(alice_txt.split)}
\end{Highlighting}
\end{Shaded}

Since the default is to split on whitespace (spaces, newlines, tabs) we
can get a reasonable word count simply by calling the split method and
counting the number of elements in the result.

\begin{Shaded}
\begin{Highlighting}[]
\NormalTok{alice_words }\OperatorTok{=}\NormalTok{ alice_txt.split()}
\BuiltInTok{len}\NormalTok{(alice_words)}
\end{Highlighting}
\end{Shaded}

\subsection{Using sets to calculate the number of unique
words}\label{using-sets-to-calculate-the-number-of-unique-words}

According to our computation above, there are about 26 thousand total
words in \emph{Alice's Adventures in Wonderland}. But how many
\emph{unique} words are there? Python has a special data structure
called a \emph{set} that makes it easy to find out. A \emph{set} drops
all duplicates, giving a collection of the unique elements.

\begin{Shaded}
\begin{Highlighting}[]
\BuiltInTok{len}\NormalTok{(}\BuiltInTok{set}\NormalTok{(alice_words))}
\end{Highlighting}
\end{Shaded}

There are 5295 unique words in the text.

\section{Exercise 0}\label{exercise-0-4}

\textbf{Reading text from a file \& splitting}

\emph{Alice's Adventures in Wonderland} is full of memorable characters.
The main characters from the story are listed, one-per-line, in the file
named \texttt{Characters.txt}.

NOTE: we will not always explicitly demonstrate everything you need to
know in order to complete an exercise. Instead we focus on teaching you
how to discover available methods and how use the help function to learn
how to use them. It is expected that you will spend some time during the
exercises looking for appropriate methods and perhaps reading
documentation.

\begin{enumerate}
\def\labelenumi{\arabic{enumi}.}
\item
  Open the \texttt{Characters.txt} file and read its contents.
\item
  Split text on newlines to produce a list with one element per line.
  Store the result as ``alice\_characters''.
\end{enumerate}

```

\subsection{Working with lists}\label{working-with-lists}

The \texttt{split} methods we used to break up the text of \emph{Alice
in Wonderland} into words produced a \emph{list}. A lot of the
techniques we'll use later to analyze this text also produce lists, so
its worth taking a minute to learn more about them.

It is always a good idea to know what type of things you're working with
in Python. As you gain experience, you won't have to look this things up
as often, but even experienced Python programmers use the \texttt{type}
function to learn about the objects they are working with.

\begin{Shaded}
\begin{Highlighting}[]
\BuiltInTok{type}\NormalTok{(alice_words)}
\end{Highlighting}
\end{Shaded}

A \emph{list} in Python is used to store a collection of items. As with
other types in Python, you can get a list of methods by typing the name
of the object followed by a \texttt{.} and pressing \texttt{tab}.

\subsection{Extracting subsets from
lists}\label{extracting-subsets-from-lists}

Among the things you can do with a list is extract subsets using bracket
indexing notation. This is useful in many situations, including the
current one where we want to inspect a long list without printing out
the whole thing.

The examples below show how indexing works in Python.

\begin{Shaded}
\begin{Highlighting}[]
\NormalTok{alice_words[}\DecValTok{0}\NormalTok{] }\CommentTok{# first word (yes, we count from zero!)}
\end{Highlighting}
\end{Shaded}

\begin{Shaded}
\begin{Highlighting}[]
\NormalTok{alice_words[}\DecValTok{1}\NormalTok{] }\CommentTok{# second word}
\end{Highlighting}
\end{Shaded}

\begin{Shaded}
\begin{Highlighting}[]
\NormalTok{alice_words[:}\DecValTok{10}\NormalTok{] }\CommentTok{# first 10 words}
\end{Highlighting}
\end{Shaded}

\begin{Shaded}
\begin{Highlighting}[]
\NormalTok{alice_words[}\DecValTok{10}\NormalTok{:}\DecValTok{20}\NormalTok{] }\CommentTok{# words 11 through 20}
\end{Highlighting}
\end{Shaded}

\begin{Shaded}
\begin{Highlighting}[]
\NormalTok{alice_words[}\OperatorTok{-}\DecValTok{1}\NormalTok{] }\CommentTok{# the last word}
\end{Highlighting}
\end{Shaded}

\begin{Shaded}
\begin{Highlighting}[]
\NormalTok{alice_words[}\OperatorTok{-}\DecValTok{10}\NormalTok{:] }\CommentTok{# the last 10 words}
\end{Highlighting}
\end{Shaded}

Note that the displayed representation of lists and other data
structures in python often closely matches the syntax used to create
them. For example, we can create a list using square brackets, just as
we see when we print a list:

\begin{Shaded}
\begin{Highlighting}[]
\NormalTok{[}\StringTok{'her'}\NormalTok{,}
 \StringTok{'own'}\NormalTok{,}
 \StringTok{'child-life,'}\NormalTok{,}
 \StringTok{'and'}\NormalTok{,}
 \StringTok{'the'}\NormalTok{,}
 \StringTok{'happy'}\NormalTok{,}
 \StringTok{'summer'}\NormalTok{,}
 \StringTok{'days.'}\NormalTok{,}
 \StringTok{'THE'}\NormalTok{,}
 \StringTok{'END'}\NormalTok{]}
\end{Highlighting}
\end{Shaded}

\subsection{Sorting \& other in-place
methods}\label{sorting-other-in-place-methods}

There are many other things we can do with lists besides extracting
subsets using bracket indexing. For example, there are methods to append
and remove elements from a list. When using a list method that you are
unfamiliar with, it is always a good idea to read the documentation.

Note that many methods modify the object \emph{in place}. For example,
if we wanted to sort the last 10 words in \texttt{alice\_words} we would
do it like this:

\begin{Shaded}
\begin{Highlighting}[]
\NormalTok{last_10 }\OperatorTok{=}\NormalTok{ alice_words[}\OperatorTok{-}\DecValTok{10}\NormalTok{:]}
\BuiltInTok{print}\NormalTok{(last_10)}
\NormalTok{last_10.sort()}
\BuiltInTok{print}\NormalTok{(last_10)}
\end{Highlighting}
\end{Shaded}

\subsection{Counting chapters \&
paragraphs}\label{counting-chapters-paragraphs}

Now that we know how to split a string and how to work with the
resulting list, we can split on chapter markers to count the number of
chapters. All we need to do is specify the string to split on. Since
each chapter is marked with the string
\texttt{\textquotesingle{}CHAPTER\ \textquotesingle{}} followed by the
chapter number, we can split the text up into chapters using this as the
separator.

\begin{Shaded}
\begin{Highlighting}[]
\NormalTok{alice_chapters }\OperatorTok{=}\NormalTok{ alice_txt.split(}\StringTok{"CHAPTER "}\NormalTok{)}
\BuiltInTok{len}\NormalTok{(alice_chapters)}
\end{Highlighting}
\end{Shaded}

Since the first element contains the material \emph{before} the first
chapter, this tells us there are twelve chapters in the book.

We can count paragraphs in a similar way. Paragraphs are indicated by a
blank line, i.e., two newlines in a row. When working with strings we
can represent newlines with \texttt{\textbackslash{}n}, so our basic
paragraph separator is \texttt{\textbackslash{}n\textbackslash{}n}.

\begin{Shaded}
\begin{Highlighting}[]
\NormalTok{alice_paragraphs }\OperatorTok{=}\NormalTok{ alice_txt.split(}\StringTok{"}\CharTok{\textbackslash{}n\textbackslash{}n}\StringTok{"}\NormalTok{)}
\end{Highlighting}
\end{Shaded}

Before counting the number of paragraphs, I want to inspect the result
to see if it looks correct:

\begin{Shaded}
\begin{Highlighting}[]
\BuiltInTok{print}\NormalTok{(alice_paragraphs[}\DecValTok{0}\NormalTok{], }\StringTok{"}\CharTok{\textbackslash{}n}\StringTok{=========="}\NormalTok{)}
\BuiltInTok{print}\NormalTok{(alice_paragraphs[}\DecValTok{1}\NormalTok{], }\StringTok{"}\CharTok{\textbackslash{}n}\StringTok{=========="}\NormalTok{)}
\BuiltInTok{print}\NormalTok{(alice_paragraphs[}\DecValTok{2}\NormalTok{], }\StringTok{"}\CharTok{\textbackslash{}n}\StringTok{=========="}\NormalTok{)}
\BuiltInTok{print}\NormalTok{(alice_paragraphs[}\DecValTok{3}\NormalTok{], }\StringTok{"}\CharTok{\textbackslash{}n}\StringTok{=========="}\NormalTok{)}
\BuiltInTok{print}\NormalTok{(alice_paragraphs[}\DecValTok{4}\NormalTok{], }\StringTok{"}\CharTok{\textbackslash{}n}\StringTok{=========="}\NormalTok{)}
\BuiltInTok{print}\NormalTok{(alice_paragraphs[}\DecValTok{5}\NormalTok{], }\StringTok{"}\CharTok{\textbackslash{}n}\StringTok{=========="}\NormalTok{)}
\end{Highlighting}
\end{Shaded}

We're counting the title, author, and chapter lines as paragraphs, but
this will do for a rough count.

\begin{Shaded}
\begin{Highlighting}[]
\BuiltInTok{len}\NormalTok{(alice_paragraphs)}
\end{Highlighting}
\end{Shaded}

\section{Exercise 1}\label{exercise-1-4}

\textbf{Count the number of main characters}

So far we've learned that there are 12 chapters, around 830 paragraphs,
and about 26 thousand words in \emph{Alice's Adventures in Wonderland}.
Along the way we've also learned how to open a file and read its
contents, split strings, calculate the length of objects, discover
methods for string and list objects, and index/subset lists in Python.
Now it is time for you to put these skills to use to learn something
about the main characters in the story.

\begin{enumerate}
\def\labelenumi{\arabic{enumi}.}
\item
  Count the number of main characters in the story (i.e., get the length
  of the list you created in previous exercise).
\item
  Extract and print just the first character from the list you created
  in the previous exercise.
\item
  (BONUS, optional): Sort the list you created in step 2 alphabetically,
  and then extract the last element.
\end{enumerate}

\section{Working with nested
structures}\label{working-with-nested-structures}

\textbf{Words within paragraphs within chapters}

This far our analysis as treated the text as a ``flat'' data structure.
For example, when we counted words we just counted words in the whole
document, rather than counting the number of words in each chapter. If
we want to treat our document as a nested structure, with words forming
sentences, sentences forming paragraphs, paragraphs forming chapters,
and chapters forming the book, we need to learn some additional tools.
Specifically, we need to learn how to iterate over lists (or other
collections) and do things with each element in a collection.

There are several ways to iterate in Python, of which we will focus on
\emph{for loops} and \emph{list comprehensions}.

\subsection{Iterating over paragraphs using
for-loops}\label{iterating-over-paragraphs-using-for-loops}

A \emph{for loop} is a way of cycling through the elements of a
collection and doing something with each one. As a simple example, we
can cycle through the first 6 paragraphs and print each one. Cycling
through with a loop makes it easy to insert a separator between the
paragraphs, making it much easier to read the output.

\begin{Shaded}
\begin{Highlighting}[]
\ControlFlowTok{for}\NormalTok{ paragraph }\KeywordTok{in}\NormalTok{ alice_paragraphs[:}\DecValTok{6}\NormalTok{]:}
    \BuiltInTok{print}\NormalTok{(paragraph)}
    \BuiltInTok{print}\NormalTok{(}\StringTok{'=================================='}\NormalTok{)}
\BuiltInTok{print}\NormalTok{(}\StringTok{'DONE.'}\NormalTok{)}
\end{Highlighting}
\end{Shaded}

Notice that the syntax of a for-loop is

\begin{verbatim}
for <thing> in <collection>:
    do stuff with <thing>
\end{verbatim}

Notice also that the body of the for-loop is indented. This is
important, because it is this indentation that defines the body of the
loop. Notice that ``DONE.'' is only printed once, since
\texttt{print(\textquotesingle{}DONE.\textquotesingle{})} is not
indented and is therefore outside of the body of the loop.

Loops in Python are great because the syntax is relatively simple, and
because they are very powerful. Inside of the body of a loop you can use
all the tools you use elsewhere in python.

Here is one more example of a loop, this time iterating over all the
chapters and calculating the number of paragraphs in each chapter.

\begin{Shaded}
\begin{Highlighting}[]
\ControlFlowTok{for}\NormalTok{ chapter }\KeywordTok{in}\NormalTok{ alice_chapters[}\DecValTok{1}\NormalTok{:]:}
\NormalTok{    paragraphs }\OperatorTok{=}\NormalTok{ chapter.split(}\StringTok{"}\CharTok{\textbackslash{}n\textbackslash{}n}\StringTok{"}\NormalTok{)}
    \BuiltInTok{print}\NormalTok{(}\BuiltInTok{len}\NormalTok{(paragraphs))}
\end{Highlighting}
\end{Shaded}

\subsection{Iterating \& collecting paragraphs per chapter using list
comprehension}\label{iterating-collecting-paragraphs-per-chapter-using-list-comprehension}

We could use for-loops to fill in lists of values, but there is a
special syntax in Python that is often better for this use case. This
special syntax is called a \emph{list comprehension} and it looks like
this:

\begin{Shaded}
\begin{Highlighting}[]
\NormalTok{paragraphs_per_chapter }\OperatorTok{=}\NormalTok{ [}\BuiltInTok{len}\NormalTok{(chapter.split(}\StringTok{"}\CharTok{\textbackslash{}n\textbackslash{}n}\StringTok{"}\NormalTok{)) }
                          \ControlFlowTok{for}\NormalTok{ chapter }\KeywordTok{in}\NormalTok{ alice_chapters[}\DecValTok{1}\NormalTok{:]]}
\BuiltInTok{print}\NormalTok{(paragraphs_per_chapter)}
\end{Highlighting}
\end{Shaded}

Notice that \emph{list comprehension} is very similar to a \emph{for
loop}, though the order is different. In a \emph{for-loop} the
\texttt{for} part comes first and the expressions that make up the body
come second and are indented. In a \emph{list comprehension} the
expression comes first and the \texttt{for} part comes afterward. Notice
also the square brackets surrounding the whole thing -- these brackets
are what tells Python that you want a list.

Here is another list comprehension that counts the number of times the
name ``Alice'' appears in each chapter.

\begin{Shaded}
\begin{Highlighting}[]
\NormalTok{alices_per_chapter }\OperatorTok{=}\NormalTok{ [chapter.count(}\StringTok{"Alice"}\NormalTok{) }\ControlFlowTok{for}\NormalTok{ chapter }\KeywordTok{in}\NormalTok{ alice_chapters]}
\BuiltInTok{print}\NormalTok{(alices_per_chapter)}
\end{Highlighting}
\end{Shaded}

\subsection{Organizing results in
dictionaries}\label{organizing-results-in-dictionaries}

Our code for calculating the number of of times ``Alice'' was mentioned
per chapter worked, but with a little effort we can make it much easier
to interpret by associating each count with the chapter it corresponds
to. In Python we can use a \texttt{dict} (i.e., ``dictionary'') to store
key-value pairs.

First, we can iterate over each chapter and grab just the first line
(that is, the chapter titles). These will become our keys.

\begin{Shaded}
\begin{Highlighting}[]
\NormalTok{chapter_names }\OperatorTok{=}\NormalTok{ [chapter.splitlines()[}\DecValTok{0}\NormalTok{] }\ControlFlowTok{for}\NormalTok{ chapter }\KeywordTok{in}\NormalTok{ alice_chapters[}\DecValTok{1}\NormalTok{:]]}
\BuiltInTok{print}\NormalTok{(chapter_names)}
\end{Highlighting}
\end{Shaded}

Finally we can combine the chapter titles and counts and convert them to
a dictionary.

\begin{Shaded}
\begin{Highlighting}[]
\BuiltInTok{dict}\NormalTok{(}\BuiltInTok{zip}\NormalTok{(chapter_names, }
\NormalTok{         [chapter.count(}\StringTok{"Alice"}\NormalTok{) }
          \ControlFlowTok{for}\NormalTok{ chapter }\KeywordTok{in}\NormalTok{ alice_chapters]))}
\end{Highlighting}
\end{Shaded}

\section{Exercise 2}\label{exercise-2-3}

\textbf{Iterating \& counting things}

Now that we know how to iterate using for-loops and list comprehensions
the possibilities really start to open up. For example, we can use these
techniques to count the number of times each character appears in the
story.

\begin{enumerate}
\def\labelenumi{\arabic{enumi}.}
\tightlist
\item
  Make sure you have both the text and the list of characters.
\end{enumerate}

Open and read both ``Alice\_in\_wonderland.txt'' and ``Characters.txt''
if you have not already done so.

\begin{enumerate}
\def\labelenumi{\arabic{enumi}.}
\setcounter{enumi}{1}
\tightlist
\item
  Which chapter has the most words?
\end{enumerate}

Split the text into chaptes (i.e., split on ``CHAPTER'') and use a
for-loop or list comprehension to iterate over the chapters. For each
chapter, split it into words and calculate the length.

\begin{enumerate}
\def\labelenumi{\arabic{enumi}.}
\setcounter{enumi}{2}
\tightlist
\item
  How many times is each character mentioned in the text?
\end{enumerate}

Iterate over the list of characters using a for-loop or list
comprehension. For each character, call the count method with that
character as the argument.

\begin{enumerate}
\def\labelenumi{\arabic{enumi}.}
\setcounter{enumi}{3}
\tightlist
\item
  (BONUS, optional): Put the character counts computed above in a
  dictionary with character names as the keys and counts as the values.
\item
  (BONUS, optional): Use a nested list comprehension to calculate the
  number of times each character is mentioned in each chapter.
\end{enumerate}

\section{Importing numpy \& calculating simple
statistics}\label{importing-numpy-calculating-simple-statistics}

Now that we know how to iterate over lists and calculate numbers for
each element, we may wish to do some simple math using these numbers.
For example, we may want to calculate the mean and standard deviation of
the distribution of the number of paragraphs in each chapter. Python has
a handful of math functions built-in (e.g., \texttt{min} and
\texttt{max}) but built-in math support is pretty limited.

When you find that something isn't available in Python itself, its time
to look for a package that does it. Although it is somewhat overkill for
simply calculating a mean we're going to use a popular package called
\emph{numpy} for this. The \emph{numpy} package is included in the
Anaconda Python distribution we are using, so we don't need to install
it separately.

In order to use \emph{numpy} or other packages, you must first import
them. We can import numpy as follows:

\begin{Shaded}
\begin{Highlighting}[]
\ImportTok{import}\NormalTok{ numpy}
\end{Highlighting}
\end{Shaded}

The \emph{numpy} package is very popular and includes a lot of useful
functions. For example, we can use it to calculate means and standard
deviations:

\begin{Shaded}
\begin{Highlighting}[]
\BuiltInTok{print}\NormalTok{(numpy.mean(paragraphs_per_chapter))}
\BuiltInTok{print}\NormalTok{(numpy.std(paragraphs_per_chapter))}
\end{Highlighting}
\end{Shaded}

and compute correlations:

\begin{Shaded}
\begin{Highlighting}[]
\NormalTok{words_per_chapter }\OperatorTok{=}\NormalTok{ [}\BuiltInTok{len}\NormalTok{(chapter.split()) }\ControlFlowTok{for}\NormalTok{ chapter }\KeywordTok{in}\NormalTok{ alice_chapters]}
\NormalTok{alices_per_chapter }\OperatorTok{=}\NormalTok{ [chapter.count(}\StringTok{"Alice"}\NormalTok{) }\ControlFlowTok{for}\NormalTok{ chapter }\KeywordTok{in}\NormalTok{ alice_chapters]}

\BuiltInTok{print}\NormalTok{(numpy.corrcoef(words_per_chapter, alices_per_chapter))}
\end{Highlighting}
\end{Shaded}

\section{Wrap-up}\label{wrap-up-5}

\subsection{Feedback}\label{feedback-5}

These workshops are a work in progress, please provide any feedback to:
\href{mailto:help@iq.harvard.edu}{\nolinkurl{help@iq.harvard.edu}}

\subsection{Resources}\label{resources-5}

\begin{itemize}
\tightlist
\item
  IQSS

  \begin{itemize}
  \tightlist
  \item
    Workshops: \url{https://dss.iq.harvard.edu/workshop-materials}
  \item
    Data Science Services: \url{https://dss.iq.harvard.edu/}
  \item
    Research Computing Environment:
    \url{https://iqss.github.io/dss-rce/}
  \end{itemize}
\item
  HBS

  \begin{itemize}
  \tightlist
  \item
    Research Computing Services workshops:
    \url{https://training.rcs.hbs.org/workshops}
  \item
    Other HBS RCS resources:
    \url{https://training.rcs.hbs.org/workshop-materials}
  \item
    RCS consulting email: \url{mailto:research@hbs.edu}
  \end{itemize}
\item
  Graphics

  \begin{itemize}
  \tightlist
  \item
    matplotlib: \url{https://matplotlib.org/}
  \item
    seaborn: \url{https://seaborn.pydata.org/}
  \item
    plotly: \url{https://plot.ly/python/}
  \end{itemize}
\item
  Quantitative Data Analysis

  \begin{itemize}
  \tightlist
  \item
    numpy: \url{http://www.numpy.org/}
  \item
    scipy: \url{https://www.scipy.org/}
  \item
    pandas: \url{https://pandas.pydata.org/}
  \item
    scikit-learn: \url{http://scikit-learn.org/stable/}
  \item
    statsmodels: \url{http://www.statsmodels.org/stable/}
  \end{itemize}
\item
  Text analysis

  \begin{itemize}
  \tightlist
  \item
    textblob: \url{https://textblob.readthedocs.io/en/dev/}
  \item
    nltk: \url{http://www.nltk.org/}
  \item
    Gensim: \url{https://radimrehurek.com/gensim/}
  \end{itemize}
\item
  Webscraping

  \begin{itemize}
  \tightlist
  \item
    scrapy: \url{https://scrapy.org/}
  \item
    requests: \url{http://docs.python-requests.org/en/master/}
  \item
    lxml: \url{https://lxml.de/}
  \item
    BeautifulSoup: \url{https://www.crummy.com/software/BeautifulSoup/}
  \end{itemize}
\item
  Social Network Analysis

  \begin{itemize}
  \tightlist
  \item
    networkx: \url{https://networkx.github.io/}
  \item
    graph-tool: \url{https://graph-tool.skewed.de/}
  \end{itemize}
\end{itemize}

\chapter{Python Web-Scraping}\label{python-web-scraping}

\textbf{Topics}

\begin{itemize}
\tightlist
\item
  Web basics
\item
  Making web requests
\item
  Inspecting web sites
\item
  Retrieving web data
\item
  Using Xpaths to retrieve \texttt{html} content
\item
  Parsing \texttt{html} content
\item
  Cleaning and storing text from \texttt{html}
\end{itemize}

\section{Setup}\label{setup-5}

\subsection{Software \& Materials}\label{software-materials-5}

\subsubsection{Install the Anaconda Python
distribution}\label{install-the-anaconda-python-distribution-1}

If using your own computer please install the Anaconda Python
distribution from \url{https://www.anaconda.com/download/}. (Note that
Python version \(\leq\) 3.0 differs considerably from more recent
releases. For this workshop you will need version \(\geq\) 3.6.x)

Accepting the defaults proposed by the Anaconda installer is generally
recommended.

\subsubsection{Workshop notes}\label{workshop-notes}

\begin{itemize}
\tightlist
\item
  Download class materials at
  \url{https://github.com/IQSS/dss-workshops-redux/raw/master/Python/PythonWebScrape.zip}
\item
  Extract materials from the zipped directory
  \texttt{PythonWebScrape.zip} (Right-click =\textgreater{} Extract All
  on Windows, double-click on Mac) and move them to your desktop!
\end{itemize}

Start the \texttt{Jupyter\ Notebook} application and open the
\texttt{PythonWebScrape.ipynb} file in the \texttt{PythonWebScrape}
folder you downloaded previously.

\subsection{Goals}\label{goals-5}

In this workshop you will

\begin{itemize}
\tightlist
\item
  learn basic web scraping principles and techniques,
\item
  learn how to use the \texttt{requests} package in Python,
\item
  practice making requests and manipulating responses from the server.
\end{itemize}

This workshop is relatively \emph{informal}, \emph{example-oriented},
and \emph{hands-on}. We will learn by working through an example web
scraping project.

Note that this is \textbf{not} an introductory workshop. Familiarity
with Python, including but not limited to knowledge of lists and
dictionaries, indexing, and loops and / or comprehensions is assumed. If
you need an introduction to Python or a refresher, we recommend the
\href{https://dss.iq.harvard.edu/workshop-materials\#widget-0}{IQSS
Introduction to Python}.

Note also that this workshop will not teach you everything you need to
know in order to retrieve data from any web service you might wish to
scrape. You can expect to learn just enough to be dangerous.

\section{Preliminary questions}\label{preliminary-questions}

\subsection{What is web scraping?}\label{what-is-web-scraping}

Web scraping is the activity of automating retrieval of information from
a web service designed for human interaction.

\subsection{Is web scraping legal? Is it
ethical?}\label{is-web-scraping-legal-is-it-ethical}

It depends. If you have legal questions seek legal counsel. You can
mitigate some ethical issues by building delays and restrictions into
your web scraping program so as to avoid impacting the availability of
the web service for other users or the cost of hosting the service for
the service provider.

\section{Example project}\label{example-project-1}

In this workshop I will demonstrate web scraping techniques using the
Collections page at \url{https://www.harvardartmuseums.org/collections}
and let you use the skills you'll learn to retrieve information from
other parts of the Harvard Art Museums website.

The basic strategy is pretty much the same for most scraping projects.
We will use our web browser (Chrome or Firefox recommended) to examine
the page you wish to retrieve data from, and copy/paste information from
your web browser into your scraping program.

\section{Take shortcuts if you can}\label{take-shortcuts-if-you-can}

We wish to extract information from
\url{https://www.harvardartmuseums.org/collections}. Like most modern
web pages, a lot goes on behind the scenes to produce the page we see in
our browser. Our goal is to pull back the curtain to see what the
website does when we interact with it. Once we see how the website works
we can start retrieving data from it. If we are lucky we'll find a
resource that returns the data we're looking for in a structured format
like \href{https://json.org/}{JSON} or
\href{https://en.wikipedia.org/wiki/XML}{XML}.

\subsection{Examining the structure of our target web
service}\label{examining-the-structure-of-our-target-web-service}

We start by opening the collections web page in a web browser and
inspecting it.

\begin{figure}
\centering
\includegraphics{Python/PythonWebScrape/images/dev_tools.png}
\caption{}
\end{figure}

\begin{figure}
\centering
\includegraphics{Python/PythonWebScrape/images/dev_tools_pane.png}
\caption{}
\end{figure}

If we scroll down to the bottom of the Collections page, we'll see a
button that says ``Load More''. Let's see what happens when we click on
that button. To do so, click on ``Network'' in the developer tools
window, then click the ``Load More Collections'' button. You should see
a list of requests that were made as a result of clicking that button,
as shown below.

\begin{figure}
\centering
\includegraphics{Python/PythonWebScrape/images/dev_tools_network.png}
\caption{}
\end{figure}

If we look at that second request, the one to a script named
\texttt{browse}, we'll see that it returns all the information we need,
in a convenient format called \texttt{JSON}. All we need to retrieve
collection data is call make \texttt{GET} requests to
\url{https://www.harvardartmuseums.org/browse} with the correct
parameters.

\subsection{Making requests using
Python}\label{making-requests-using-python}

The URL we want to retrieve data from has the following structure

\begin{verbatim}
scheme                    domain    path  parameters
 https www.harvardartmuseums.org  browse  load_amount=10&offset=0
\end{verbatim}

It is often convenient to create variables containing the domain(s) and
path(s) you'll be working with, as this allows you to swap out paths and
parameters as needed. Note that the path is separated from the domain
with \texttt{/} and the parameters are separated from the path with
\texttt{?}. If there are multiple parameters they are separated from
each other with a \texttt{\&}.

For example, we can define the domain and path of the collections URL as
follows:

\begin{Shaded}
\begin{Highlighting}[]
\NormalTok{museum_domain }\OperatorTok{=} \StringTok{'https://www.harvardartmuseums.org'}
\NormalTok{collection_path }\OperatorTok{=} \StringTok{'browse'}

\NormalTok{collection_url }\OperatorTok{=}\NormalTok{ (museum_domain}
                  \OperatorTok{+} \StringTok{"/"}
                  \OperatorTok{+}\NormalTok{ collection_path)}

\BuiltInTok{print}\NormalTok{(collection_url)}
\end{Highlighting}
\end{Shaded}

Note that we omit the parameters here because it is usually easier to
pass them as a \texttt{dict} when using the \texttt{requests} library in
Python. This will become clearer shortly.

Now that we've constructed the URL we wish interact with we're ready to
make our first request in Python.

\begin{Shaded}
\begin{Highlighting}[]
\ImportTok{import}\NormalTok{ requests}

\NormalTok{collections1 }\OperatorTok{=}\NormalTok{ requests.get(}
\NormalTok{    collection_url,}
\NormalTok{    params }\OperatorTok{=}\NormalTok{ \{}\StringTok{'load_amount'}\NormalTok{: }\DecValTok{10}\NormalTok{,}
                  \StringTok{'offset'}\NormalTok{: }\DecValTok{0}\NormalTok{\}}
\NormalTok{)}
\end{Highlighting}
\end{Shaded}

\begin{Shaded}
\begin{Highlighting}[]
\CommentTok{# }\AlertTok{###}\CommentTok{ Parsing JSON data}
\CommentTok{# We already know from inspecting network traffic in our web}
\CommentTok{# browser that this URL returns JSON, but we can use Python to verify}
\CommentTok{# this assumption.}
\NormalTok{collections1.headers[}\StringTok{'Content-Type'}\NormalTok{]}
\end{Highlighting}
\end{Shaded}

Since JSON is a structured data format, parsing it into python data
structures is easy. In fact, there's a method for that!

\begin{Shaded}
\begin{Highlighting}[]
\NormalTok{collections1 }\OperatorTok{=}\NormalTok{ collections1.json()}
\BuiltInTok{print}\NormalTok{(collections1)}
\end{Highlighting}
\end{Shaded}

That's it. Really, we are done here. Everyone go home!

OK not really, there is still more we can lean. But you have to admit
that was pretty easy. If you can identify a service that returns the
data you want in structured from, web scraping becomes a pretty trivial
enterprise. We'll discuss several other scenarios and topics, but for
some web scraping tasks this is really all you need to know.

\subsection{Organizing \& saving the
data}\label{organizing-saving-the-data}

The records we retrieved from
\texttt{https://www.harvardartmuseums.org/browse} are arranged as a list
of dictionaries. We can easily select the fields of arrange these data
into a pandas \texttt{DataFrame} to facilitate subsequent analysis.

\begin{Shaded}
\begin{Highlighting}[]
\ImportTok{import}\NormalTok{ pandas }\ImportTok{as}\NormalTok{ pd}
\end{Highlighting}
\end{Shaded}

\begin{Shaded}
\begin{Highlighting}[]
\NormalTok{records1 }\OperatorTok{=}\NormalTok{ pd.DataFrame.from_records(collections1[}\StringTok{'records'}\NormalTok{])}
\end{Highlighting}
\end{Shaded}

\begin{Shaded}
\begin{Highlighting}[]
\BuiltInTok{print}\NormalTok{(records1)}
\end{Highlighting}
\end{Shaded}

and write the data to a file.

\begin{Shaded}
\begin{Highlighting}[]
\NormalTok{records1.to_csv(}\StringTok{"records1.csv"}\NormalTok{)}
\end{Highlighting}
\end{Shaded}

\subsection{Iterating to retrieve all the
data}\label{iterating-to-retrieve-all-the-data}

Of course we don't want just the first page of collections. How can we
retrieve all of them?

Now that we know the web service works, and how to make requests in
Python, we can iterate in the usual way.

\begin{Shaded}
\begin{Highlighting}[]
\NormalTok{records }\OperatorTok{=}\NormalTok{ []}
\ControlFlowTok{for}\NormalTok{ offset }\KeywordTok{in} \BuiltInTok{range}\NormalTok{(}\DecValTok{0}\NormalTok{, }\DecValTok{50}\NormalTok{, }\DecValTok{10}\NormalTok{):}
\NormalTok{    param_values }\OperatorTok{=}\NormalTok{ \{}\StringTok{'load_amount'}\NormalTok{: }\DecValTok{10}\NormalTok{, }\StringTok{'offset'}\NormalTok{: offset\}}
\NormalTok{    current_request }\OperatorTok{=}\NormalTok{ requests.get(collection_url, params }\OperatorTok{=}\NormalTok{ param_values)}
\NormalTok{    records }\OperatorTok{+=}\NormalTok{ current_request.json()[}\StringTok{'records'}\NormalTok{]}
\end{Highlighting}
\end{Shaded}

\begin{Shaded}
\begin{Highlighting}[]
\CommentTok{## convert list of dicts to a `DataFrame`}
\NormalTok{records_final }\OperatorTok{=}\NormalTok{ pd.DataFrame.from_records(records)}
\end{Highlighting}
\end{Shaded}

\begin{Shaded}
\begin{Highlighting}[]
\CommentTok{# write the data to a file.}
\NormalTok{records_final.to_csv(}\StringTok{"records_final.csv"}\NormalTok{)}
\end{Highlighting}
\end{Shaded}

\begin{Shaded}
\begin{Highlighting}[]
\BuiltInTok{print}\NormalTok{(records_final)}
\end{Highlighting}
\end{Shaded}

\section{Exercise 0}\label{exercise-0-5}

\textbf{Retrieve exhibits data}

In this exercise you will retrieve information about the art exhibitions
at Harvard Art Museums from
\texttt{https://www.harvardartmuseums.org/visit/exhibitions}

\begin{enumerate}
\def\labelenumi{\arabic{enumi}.}
\tightlist
\item
  Using a web browser (Firefox or Chrome recommended) inspect the page
  at \texttt{https://www.harvardartmuseums.org/visit/exhibitions}.
  Examine the network traffic as you interact with the page. Try to find
  where the data displayed on that page comes from.
\item
  Make a \texttt{get} request in Python to retrieve the data from the
  URL identified in step1.
\item
  Write a \emph{loop} or \emph{list comprehension} in Python to retrieve
  data for the first 5 pages of exhibitions data.
\item
  Bonus (optional): Convert the data you retrieved into a pandas
  \texttt{DataFrame} and save it to a \texttt{.csv} file.
\end{enumerate}

\section{Parse HTML if you have to}\label{parse-html-if-you-have-to}

As we've seen, you can often inspect network traffic or other sources to
locate the source of the data you are interested in and the API used to
retrieve it. You should always start by looking for these shortcuts and
using them where possible. If you are really lucky, you'll find a
shortcut that returns the data as JSON or XML. If you are not quite so
lucky, you will have to parse HTML to retrieve the information you need.

For example, when I inspected the network traffic while interacting with
\url{https://www.harvardartmuseums.org/visit/calendar} I didn't see any
requests that returned JSON data. The best we can do appears to be
\url{https://www.harvardartmuseums.org/visit/calendar?date=}, which
unfortunately returns HTML.

\subsection{Retrieving HTML}\label{retrieving-html}

The first step is the same as before: we make at \texttt{GET} request.

\begin{Shaded}
\begin{Highlighting}[]
\NormalTok{calendar_path }\OperatorTok{=} \StringTok{'visit/calendar'}

\NormalTok{calendar_url }\OperatorTok{=}\NormalTok{ (museum_domain }\CommentTok{# recall that we defined museum_domain earlier}
                  \OperatorTok{+} \StringTok{"/"}
                  \OperatorTok{+}\NormalTok{ calendar_path)}

\BuiltInTok{print}\NormalTok{(calendar_url)}

\NormalTok{events0 }\OperatorTok{=}\NormalTok{ requests.get(calendar_url, params }\OperatorTok{=}\NormalTok{ \{}\StringTok{'date'}\NormalTok{: }\StringTok{'2018-11'}\NormalTok{\})}
\end{Highlighting}
\end{Shaded}

As before we can check the headers to see what type of content we
received in response to our request.

\begin{Shaded}
\begin{Highlighting}[]
\NormalTok{events0.headers[}\StringTok{'Content-Type'}\NormalTok{]}
\end{Highlighting}
\end{Shaded}

\subsection{Parsing HTML using the lxml
library}\label{parsing-html-using-the-lxml-library}

Like JSON, HTML is structured; unlike JSON it is designed to be rendered
into a human-readable page rather than simply to store and exchange data
in a computer-readable format. Consequently, parsing HTML and extracting
information from it is somewhat more difficult than parsing JSON.

While JSON parsing is built into the Python \texttt{requests} library,
parsing HTML requires a separate library. I recommend using the HTML
parser from the \texttt{lxml} library; others prefer an alternative
called \texttt{BeautyfulSoup}.

\begin{Shaded}
\begin{Highlighting}[]
\ImportTok{from}\NormalTok{ lxml }\ImportTok{import}\NormalTok{ html}

\NormalTok{events_html }\OperatorTok{=}\NormalTok{ html.fromstring(events0.text)}
\end{Highlighting}
\end{Shaded}

\subsection{Using xpath to extract content from
HTML}\label{using-xpath-to-extract-content-from-html}

\texttt{XPath} is a tool for identifying particular elements withing a
HTML document. The developer tools built into modern web browsers make
it easy to generate \texttt{XPath}s that can used to identify the
elements of a web page that we wish to extract.

We can open the html document we retrieved and inspect it using our web
browser.

\begin{Shaded}
\begin{Highlighting}[]
\NormalTok{html.open_in_browser(events_html, encoding }\OperatorTok{=} \StringTok{'UTF-8'}\NormalTok{)}
\end{Highlighting}
\end{Shaded}

\begin{figure}
\centering
\includegraphics{Python/PythonWebScrape/images/dev_tools_right_click.png}
\caption{}
\end{figure}

\begin{figure}
\centering
\includegraphics{Python/PythonWebScrape/images/dev_tools_inspect.png}
\caption{}
\end{figure}

Once we identify the element containing the information of interest we
can use our web browser to copy the \texttt{XPath} that uniquely
identifies that element.

\begin{figure}
\centering
\includegraphics{Python/PythonWebScrape/images/dev_tools_xpath.png}
\caption{}
\end{figure}

Next we can use python to extract the element of interest:

\begin{Shaded}
\begin{Highlighting}[]
\NormalTok{events_list_html }\OperatorTok{=}\NormalTok{ events_html.xpath(}\StringTok{'//*[@id="events_list"]'}\NormalTok{)[}\DecValTok{0}\NormalTok{]}
\end{Highlighting}
\end{Shaded}

Once again we can use a web browser to inspect the HTML we're currently
working with, and to figure out what we want to extract from it. Let's
look at the first element in our events list.

\begin{Shaded}
\begin{Highlighting}[]
\NormalTok{first_event_html }\OperatorTok{=}\NormalTok{ events_list_html[}\DecValTok{0}\NormalTok{]}
\NormalTok{html.open_in_browser(first_event_html, encoding }\OperatorTok{=} \StringTok{'UTF-8'}\NormalTok{)}
\end{Highlighting}
\end{Shaded}

As before we can use our browser to find the xpath of the elements we
want.

\begin{figure}
\centering
\includegraphics{Python/PythonWebScrape/images/dev_tools_figcaption.png}
\caption{}
\end{figure}

(Note that the \texttt{html.open\_in\_browser} function adds enclosing
\texttt{html} and \texttt{body} tags in order to create a complete web
page for viewing. This requires that we adjust the \texttt{xpath}
accordingly.)

By repeating this process for each element we want, we can build a list
of the xpaths to those elements.

\begin{Shaded}
\begin{Highlighting}[]
\NormalTok{elements_we_want }\OperatorTok{=}\NormalTok{ \{}\StringTok{'figcaption'}\NormalTok{: }\StringTok{'div/figure/div/figcaption'}\NormalTok{,}
                    \StringTok{'date'}\NormalTok{: }\StringTok{'div/div/header/time'}\NormalTok{,}
                    \StringTok{'title'}\NormalTok{: }\StringTok{'div/div/header/h2/a'}\NormalTok{,}
                    \StringTok{'time'}\NormalTok{: }\StringTok{'div/div/div/p[1]/time'}\NormalTok{,}
                    \StringTok{'localtion1'}\NormalTok{: }\StringTok{'div/div/div/p[2]/span/span[1]'}\NormalTok{,}
                    \StringTok{'location2'}\NormalTok{: }\StringTok{'div/div/div/p[2]/span/span[2]'}
\NormalTok{                    \}}
\end{Highlighting}
\end{Shaded}

Finally, we can iterate over the elements we want and extract them.

\begin{Shaded}
\begin{Highlighting}[]
\NormalTok{first_event_values }\OperatorTok{=}\NormalTok{ \{\}}
\ControlFlowTok{for}\NormalTok{ key }\KeywordTok{in}\NormalTok{ elements_we_want.keys():}
\NormalTok{    element }\OperatorTok{=}\NormalTok{ first_event_html.xpath(elements_we_want[key])[}\DecValTok{0}\NormalTok{]}
\NormalTok{    first_event_values[key] }\OperatorTok{=}\NormalTok{ element.text_content().strip()}

\BuiltInTok{print}\NormalTok{(first_event_values)}
\end{Highlighting}
\end{Shaded}

\subsection{Iterating to retrieve content from a list of HTML
elements}\label{iterating-to-retrieve-content-from-a-list-of-html-elements}

So far we've retrieved information only for the first event. To retrieve
data for all the events listed on the page we need to iterate over the
events. If we are very lucky, each event will have exactly the same
information structured in exactly the same way and we can simply extend
the code we wrote above to iterate over the events list.

Unfortunately not all these elements are available for every event, so
we need to take care to handle the case where one or more of these
elements is not available. We can do that by defining a function that
tries to retrieve a value and returns an empty string if it fails.

\begin{Shaded}
\begin{Highlighting}[]
\KeywordTok{def}\NormalTok{ get_event_info(event, path):}
    \ControlFlowTok{try}\NormalTok{:}
\NormalTok{        info }\OperatorTok{=}\NormalTok{ event.xpath(path)[}\DecValTok{0}\NormalTok{].text.strip()}
    \ControlFlowTok{except}\NormalTok{:}
\NormalTok{        info }\OperatorTok{=} \StringTok{''}
    \ControlFlowTok{return}\NormalTok{ info}
\end{Highlighting}
\end{Shaded}

Armed with this function we can iterate over the list of events and
extract the available information for each one.

\begin{Shaded}
\begin{Highlighting}[]
\NormalTok{all_event_values }\OperatorTok{=}\NormalTok{ \{\}}
\ControlFlowTok{for}\NormalTok{ key }\KeywordTok{in}\NormalTok{ elements_we_want.keys():}
\NormalTok{    key_values }\OperatorTok{=}\NormalTok{ []}
    \ControlFlowTok{for}\NormalTok{ event }\KeywordTok{in}\NormalTok{ events_list_html: }
\NormalTok{        key_values.append(get_event_info(event, elements_we_want[key]))}
\NormalTok{    all_event_values[key] }\OperatorTok{=}\NormalTok{ key_values}
\end{Highlighting}
\end{Shaded}

For convenience we can arrange these values in a pandas
\texttt{DataFrame} and save them as .csv files, just as we did with our
exhibitions data earlier.

\begin{Shaded}
\begin{Highlighting}[]
\NormalTok{all_event_values }\OperatorTok{=}\NormalTok{ pd.DataFrame.from_dict(all_event_values)}

\NormalTok{all_event_values.to_csv(}\StringTok{"all_event_values.csv"}\NormalTok{)}

\BuiltInTok{print}\NormalTok{(all_event_values)}
\end{Highlighting}
\end{Shaded}

\section{Exercise 1}\label{exercise-1-5}

\textbf{parsing HTML}

In this exercise you will retrieve information about the physical layout
of the Harvard Art Museums. The web page at
\url{https://www.harvardartmuseums.org/visit/floor-plan} contains this
information in HTML from.

\begin{enumerate}
\def\labelenumi{\arabic{enumi}.}
\tightlist
\item
  Using a web browser (Firefox or Chrome recommended) inspect the page
  at \texttt{https://www.harvardartmuseums.org/visit/floor-plan}. Copy
  the \texttt{XPath} to the element containing the list of level
  information. (HINT: the element if interest is a \texttt{ul}, i.e.,
  \texttt{unordered\ list}.)
\item
  Make a \texttt{get} request in Python to retrieve the web page at
  \url{https://www.harvardartmuseums.org/visit/floor-plan}. Extract the
  content from your request object and parse it using
  \texttt{html.fromstring} from the \texttt{lxml} library.
\item
  Use your web browser to find the \texttt{XPath}s to the facilities
  housed on level one. Use Python to extract the text from those
  \texttt{Xpath}s.
\item
  Bonus (optional): Write a \emph{loop} or \emph{list comprehension} in
  Python to retrieve data for all the levels.
\end{enumerate}

\section{\texorpdfstring{\texttt{Scrapy}: for large / complex
projects}{Scrapy: for large / complex projects}}\label{scrapy-for-large-complex-projects}

Scraping websites using the \texttt{requests} library to make GET and
POST requests, and the \texttt{lxml} library to process HTML is a good
way to learn basic web scraping techniques. It is a good choice for
small to medium size projects. For very large or complicated scraping
tasks the \texttt{scrapy} library offers a number of conveniences,
including asynchronously retrieval, session management, convenient
methods for extracting and storing values, and more. More information
about \texttt{scrapy} can be found at \url{https://doc.scrapy.org}.

\section{Browser drivers: a last
resort}\label{browser-drivers-a-last-resort}

It is sometimes necessary (or sometimes just easier) to use a web
browser as an intermediary rather than communicating directly with a web
service. This method has the advantage of being about to use the
javascript engine and session management features of a web browser; the
main disadvantage is that it is slower and tends to be more fragile than
using \texttt{requests} or \texttt{scrapy} to make requests directly
from python. For small scraping projects involving complicated sites
with CAPTHAs or lots of complicated javascript using a browser driver
can be a good option. More information is available at
\url{https://www.seleniumhq.org/docs/03_webdriver.jsp}.

\section{Wrap-up}\label{wrap-up-6}

\subsection{Feedback}\label{feedback-6}

These workshops are a work in progress, please provide any feedback to:
\href{mailto:help@iq.harvard.edu}{\nolinkurl{help@iq.harvard.edu}}

\subsection{Resources}\label{resources-6}

\begin{itemize}
\tightlist
\item
  IQSS

  \begin{itemize}
  \tightlist
  \item
    Workshops: \url{https://dss.iq.harvard.edu/workshop-materials}
  \item
    Data Science Services: \url{https://dss.iq.harvard.edu/}
  \item
    Research Computing Environment:
    \url{https://iqss.github.io/dss-rce/}
  \end{itemize}
\item
  HBS

  \begin{itemize}
  \tightlist
  \item
    Research Computing Services workshops:
    \url{https://training.rcs.hbs.org/workshops}
  \item
    Other HBS RCS resources:
    \url{https://training.rcs.hbs.org/workshop-materials}
  \item
    RCS consulting email: \url{mailto:research@hbs.edu}
  \end{itemize}
\end{itemize}

\part{Stata}\label{part-stata}

\chapter{Stata Introduction}\label{stata-introduction}

\textbf{Topics}

\begin{itemize}
\tightlist
\item
  Stata interface and Do-files
\item
  Finding help
\item
  Reading and writing data
\item
  Basic summary statistics
\item
  Basic graphs
\item
  Basic data management
\end{itemize}

\section{Setup}\label{setup-6}

\subsection{Software \& Materials}\label{software-materials-6}

Laptop users: you will need a copy of Stata installed on your machine.
Harvard FAS affiliates can install a licensed version from
\url{http://downloads.fas.harvard.edu/download}

\begin{itemize}
\tightlist
\item
  Download class materials at
  \url{https://github.com/IQSS/dss-workshops-redux/raw/master/Stata/StataIntro.zip}
\item
  Extract materials from the zipped directory \texttt{StataIntro.zip}
  (Right-click =\textgreater{} Extract All on Windows, double-click on
  Mac) and move them to your desktop!
\end{itemize}

\subsection{Organization}\label{organization}

\begin{itemize}
\tightlist
\item
  Please feel free to ask questions at any point if they are relevant to
  the current topic (or if you are lost!)
\item
  Collaboration is encouraged - please introduce yourself to your
  neighbors!
\item
  If you are using a laptop, you will need to adjust file paths
  accordingly
\item
  Make comments in your Do-file - save on flash drive or email to
  yourself
\end{itemize}

\subsection{Goals}\label{goals-6}

\begin{itemize}
\tightlist
\item
  This is an \textbf{introduction} to Stata
\item
  Assumes no/very little knowledge of Stata
\item
  Not appropriate for people already familiar with Stata
\item
  Learning Objectives:

  \begin{itemize}
  \tightlist
  \item
    Familiarize yourself with the Stata interface
  \item
    Get data in and out of Stata
  \item
    Compute statistics and construct graphical displays
  \item
    Compute new variables and transformations
  \end{itemize}
\end{itemize}

\section{Why stata?}\label{why-stata}

\begin{itemize}
\tightlist
\item
  Used in a variety of disciplines
\item
  User-friendly
\item
  Great guides available on web
\item
  Excellent modeling capabilities
\item
  Student and other discount packages available at reasonable cost
\end{itemize}

\subsection{Stata interface}\label{stata-interface}

\begin{figure}
\centering
\includegraphics{Stata/StataIntro/images/StataInterface.png}
\caption{}
\end{figure}

\begin{itemize}
\tightlist
\item
  Review and Variable windows can be closed (user preference)
\item
  Command window can be shortened (recommended)
\end{itemize}

\subsection{Do-files}\label{do-files}

\begin{itemize}
\tightlist
\item
  You can type all the same commands into the Do-file that you would
  type into the command window
\item
  BUT\ldots{}the Do-file allows you to \textbf{save} your commands
\item
  Your Do-file should contain ALL commands you executed -- at least all
  the ``correct'' commands!
\item
  I recommend never using the command window or menus to make CHANGES to
  data
\item
  Saving commands in Do-file allows you to keep a written record of
  everything you have done to your data

  \begin{itemize}
  \tightlist
  \item
    Allows easy replication
  \item
    Allows you to go back and re-run commands, analyses and make
    modifications
  \end{itemize}
\end{itemize}

\subsection{Stata help}\label{stata-help}

To get help in Stata type \texttt{help} followed by topic or command,
e.g., \texttt{help\ codebook}.

\subsection{General Stata command
syntax}\label{general-stata-command-syntax}

Most Stata commands follow the same basic syntax:
\texttt{Command\ varlist,\ options}.

\subsection{Commenting \& formatting
syntax}\label{commenting-formatting-syntax}

Start with comment describing your Do-file and use comments throughout

\begin{verbatim}
* Use '*' to comment a line and '//' for in-line comments

* Make Stata say hello:
disp "Hello " "World!" // 'disp' is short for 'display'
\end{verbatim}

\begin{itemize}
\tightlist
\item
  Use \texttt{///} to break varlists over multiple lines:
\end{itemize}

\begin{verbatim}
disp "Hello" ///
     " World!"
\end{verbatim}

\subsection{Let's get started}\label{lets-get-started}

\begin{itemize}
\tightlist
\item
  Launch the Stata program (MP or SE, does not matter unless doing
  computationally intensive work)

  \begin{itemize}
  \tightlist
  \item
    Open up a new Do-file
  \item
    Run our first Stata code!
  \end{itemize}
\end{itemize}

\begin{verbatim}
* change directory
// cd "C://Users/dataclass/Desktop/StataIntro"
\end{verbatim}

\section{Getting data into Stata}\label{getting-data-into-stata}

\subsection{Data file commands}\label{data-file-commands}

\begin{itemize}
\tightlist
\item
  Next, we want to open our data file
\item
  Open/save data sets with ``use'' and ``save'':
\end{itemize}

\begin{verbatim}
cd dataSets

// open the gss.dta data set
use gss.dta, clear

// save data file:
save newgss.dta, replace // "replace" option means OK to overwrite existing file
\end{verbatim}

\subsection{A note about path names}\label{a-note-about-path-names}

\begin{itemize}
\tightlist
\item
  If your path has no spaces in the name (that means all directories,
  folders, file names, etc. can have no spaces), you can write the path
  as is
\item
  If there are spaces, you need to put your pathname in quotes
\item
  Best to get in the habit of quoting paths
\end{itemize}

\subsection{Where's my data?}\label{wheres-my-data}

\begin{itemize}
\tightlist
\item
  Data editor (\textbf{browse})
\item
  Data editor (\textbf{edit})

  \begin{itemize}
  \tightlist
  \item
    Using the data editor is discouraged (why?)
  \end{itemize}
\item
  Always keep any changes to your data in your Do-file
\item
  Avoid temptation of making manual changes by viewing data via the
  browser rather than editor
\end{itemize}

\subsection{What if my data is not a Stata
file?}\label{what-if-my-data-is-not-a-stata-file}

\begin{itemize}
\tightlist
\item
  Import delimited text files
\end{itemize}

\begin{verbatim}
* import data from a .csv file
import delimited gss.csv, clear

* save data to a .csv file
export delimited gss_new.csv, replace
\end{verbatim}

\begin{itemize}
\tightlist
\item
  Import data from SAS
\end{itemize}

\begin{verbatim}
* import/export SAS xport files
clear
import sasxport gss.xpt
export sasxport gss_new, replace
\end{verbatim}

\begin{itemize}
\tightlist
\item
  Import data from Excel
\end{itemize}

\begin{verbatim}
* import/export Excel files
clear
import excel gss.xlsx
export excel gss_new, replace
\end{verbatim}

\subsection{What if my data is from another statistical software
program?}\label{what-if-my-data-is-from-another-statistical-software-program}

\begin{itemize}
\tightlist
\item
  SPSS/PASW will allow you to save your data as a Stata file

  \begin{itemize}
  \tightlist
  \item
    Go to: file -\textgreater{} save as -\textgreater{} Stata (use most
    recent version available)
  \item
    Then you can just go into Stata and open it
  \end{itemize}
\item
  Another option is \textbf{StatTransfer}, a program that converts data
  from/to many common formats, including SAS, SPSS, Stata, and many more
\end{itemize}

\section{Exercise 0}\label{exercise-0-6}

\textbf{Importing data}

\begin{enumerate}
\def\labelenumi{\arabic{enumi}.}
\tightlist
\item
  Save any work you've done so far. Close down Stata and open a new
  session.
\item
  Start Stata and open your \texttt{.do} file.
\item
  Change directory (\texttt{cd}) to the \texttt{dataSets} folder.
\item
  Try opening the following files:

  \begin{itemize}
  \tightlist
  \item
    A comma separated value file: \texttt{gss.csv}
  \item
    An Excel file: \texttt{gss.xlsx}
  \end{itemize}
\end{enumerate}

\section{Statistics \& graphs}\label{statistics-graphs}

\subsection{Frequently used commands}\label{frequently-used-commands}

\begin{itemize}
\tightlist
\item
  Commands for reviewing and inspecting data:

  \begin{itemize}
  \tightlist
  \item
    describe // labels, storage type etc.
  \item
    sum // statistical summary (mean, sd, min/max etc.)
  \item
    codebook // storage type, unique values, labels
  \item
    list // print actuall values
  \item
    tab // (cross) tabulate variables
  \item
    browse // view the data in a spreadsheet-like window
  \end{itemize}
\end{itemize}

First, let's ask Stata for help about these commands:

\begin{verbatim}
help sum
\end{verbatim}

\begin{verbatim}
use gss.dta, clear

sum educ // statistical summary of education
\end{verbatim}

\begin{verbatim}
codebook region // information about how region is coded
\end{verbatim}

\begin{verbatim}
tab sex // numbers of male and female participants
\end{verbatim}

\begin{itemize}
\tightlist
\item
  If you run these commands without specifying variables, Stata will
  produce output for every variable
\end{itemize}

\subsection{Basic graphing commands}\label{basic-graphing-commands}

\begin{itemize}
\tightlist
\item
  Univariate distribution(s) using \textbf{hist}
\end{itemize}

\begin{verbatim}
  /* Histograms */
  hist educ
\end{verbatim}

\begin{verbatim}
  // histogram with normal curve; see "help hist" for other options
  hist age, normal  
\end{verbatim}

\begin{itemize}
\tightlist
\item
  View bivariate distributions with scatterplots
\end{itemize}

\begin{verbatim}
   /* scatterplots */
   twoway (scatter educ age)
\end{verbatim}

\begin{verbatim}
graph matrix educ age inc
\end{verbatim}

\subsection{\texorpdfstring{The \texttt{by}
command}{The by command}}\label{the-by-command}

\begin{itemize}
\tightlist
\item
  Sometimes, you'd like to generate output based on different categories
  of a grouping variable
\item
  The ``by'' command does just this
\end{itemize}

\begin{verbatim}
* By Processing
bysort sex: tab happy // tabulate happy separately for men and women
\end{verbatim}

\begin{verbatim}
bysort marital: sum educ // summarize eudcation by marital status
\end{verbatim}

\section{Exercise 1}\label{exercise-1-6}

\textbf{Descriptive statistics}

\begin{enumerate}
\def\labelenumi{\arabic{enumi}.}
\tightlist
\item
  Use the dataset, \texttt{gss.dta}
\item
  Examine a few selected variables using the describe, sum and codebook
  commands
\item
  Tabulate the variable, ``marital,'' with and without labels
\item
  Summarize the variable, ``income'' by marital status
\item
  Cross-tabulate marital with region
\item
  Summarize the variable \texttt{happy} for married individuals only
\end{enumerate}

\section{Basic data management}\label{basic-data-management}

\subsection{Labels}\label{labels}

\begin{itemize}
\tightlist
\item
  You never know why and when your data may be reviewed
\item
  ALWAYS label every variable no matter how insignificant it may seem
\item
  Stata uses two sets of labels: \textbf{variable labels} and
  \textbf{value labels}
\item
  Variable labels are very easy to use -- value labels are a little more
  complicated
\end{itemize}

\subsection{Variable \& value labels}\label{variable-value-labels}

\begin{itemize}
\tightlist
\item
  Variable labels
\end{itemize}

\begin{verbatim}
  /* Labelling and renaming */
  // Label variable inc "household income"
  label var inc "household income"

  // change the name 'educ' to 'education'
  rename educ education

  // you can search names and labels with 'lookfor' 
  lookfor household
\end{verbatim}

\begin{itemize}
\tightlist
\item
  Value labels are a two step process: define a value label, then assign
  defined label to variable(s)
\end{itemize}

\begin{verbatim}
  /*define a value label for sex */
  label define mySexLabel 1 "Male" 2 "Female"

  /* assign our label set to the sex variable*/
  label val sex  mySexLabel
\end{verbatim}

\section{Exercise 2}\label{exercise-2-4}

\textbf{Variable labels \& value labels}

\begin{enumerate}
\def\labelenumi{\arabic{enumi}.}
\tightlist
\item
  Open the data set \texttt{gss.csv}
\item
  Familiarize yourself with the data using describe, sum, etc.
\item
  Rename and label variables using the following codebook:
\end{enumerate}

\begin{longtable}[]{@{}lll@{}}
\toprule
Var & Rename to & Label with\tabularnewline
\midrule
\endhead
v1 & marital & marital status\tabularnewline
v2 & age & age of respondent\tabularnewline
v3 & educ & education\tabularnewline
v4 & sex & respondent's sex\tabularnewline
v5 & inc & household income\tabularnewline
v6 & happy & general happiness\tabularnewline
v7 & region & region of interview\tabularnewline
\bottomrule
\end{longtable}

\begin{enumerate}
\def\labelenumi{\arabic{enumi}.}
\tightlist
\item
  Add value labels to your \texttt{marital} variable using this
  codebook:
\end{enumerate}

\begin{longtable}[]{@{}ll@{}}
\toprule
Value & Label\tabularnewline
\midrule
\endhead
1 & ``married''\tabularnewline
2 & ``widowed''\tabularnewline
3 & ``divorced''\tabularnewline
4 & ``separated''\tabularnewline
5 & ``never married''\tabularnewline
\bottomrule
\end{longtable}

\section{Working on subsets}\label{working-on-subsets}

\begin{itemize}
\tightlist
\item
  It is often useful to select just those rows of your data where some
  condition holds--for example select only rows where sex is 1 (male)
\item
  The following operators allow you to do this:
\end{itemize}

\begin{longtable}[]{@{}ll@{}}
\toprule
Operator & Meaning\tabularnewline
\midrule
\endhead
== & equal to\tabularnewline
!= & not equal to\tabularnewline
\textgreater{} & greater than\tabularnewline
\textgreater{}= & greater than or equal to\tabularnewline
\textless{} & less than\tabularnewline
\textless{}= & less than or equal to\tabularnewline
\& & and\tabularnewline
\textbar{} & or\tabularnewline
\bottomrule
\end{longtable}

\begin{itemize}
\tightlist
\item
  Note the double equals signs for testing equality
\end{itemize}

\section{Generating \& replacing
variables}\label{generating-replacing-variables}

\begin{itemize}
\tightlist
\item
  Create new variables using \texttt{gen}
\end{itemize}

\begin{verbatim}
  // create a new variable named mc_inc
  //   equal to inc minus the mean of inc
  gen mc_inc = inc - 15.37  
\end{verbatim}

\begin{itemize}
\tightlist
\item
  Sometimes useful to start with blank values and fill them in based on
  values of existing variables
\end{itemize}

\begin{verbatim}
  /* the 'generate and replace' strategy */ 
  // generate a column of missings
  gen age_wealth = .

  // Next, start adding your qualifications
  replace age_wealth=1 if age<30 & inc < 10
  replace age_wealth=2 if age<30 & inc > 10
  replace age_wealth=3 if age>30 & inc < 10
  replace age_wealth=4 if age>30 & inc > 10

  // conditions can also be combined with "or"
  gen young=0
  replace young=1 if age_wealth==1 | age_wealth==2
\end{verbatim}

\section{Exercise 3}\label{exercise-3-3}

\textbf{Manipulating variables}

\begin{enumerate}
\def\labelenumi{\arabic{enumi}.}
\tightlist
\item
  Use the dataset, \texttt{gss.dta}
\item
  Generate a new variable, \texttt{age2} equal to \texttt{age} squared
\item
  Generate a new \texttt{high\_income} variable that will take on a
  value of ``1'' if a person has an income value greater than ``15'' and
  ``0'' otherwise
\item
  Generate a new \texttt{divorced\_separated} dummy variable that will
  take on a value of ``1'' if a person is either divorced or separated
  and ``0'' otherwise
\end{enumerate}

\section{Exercise solutions}\label{exercise-solutions-4}

\subsection{Ex 0: prototype}\label{ex-0-prototype-4}

\subsection{Ex 1: prototype}\label{ex-1-prototype-4}

\subsection{Ex 2: prototype}\label{ex-2-prototype-3}

\subsection{Ex 3: prototype}\label{ex-3-prototype-3}

\section{Wrap-up}\label{wrap-up-7}

\subsection{Feedback}\label{feedback-7}

These workshops are a work in progress, please provide any feedback to:
\href{mailto:help@iq.harvard.edu}{\nolinkurl{help@iq.harvard.edu}}

\subsection{Resources}\label{resources-7}

\begin{itemize}
\tightlist
\item
  IQSS

  \begin{itemize}
  \tightlist
  \item
    Workshops: \url{https://dss.iq.harvard.edu/workshop-materials}
  \item
    Data Science Services: \url{https://dss.iq.harvard.edu/}
  \item
    Research Computing Environment:
    \url{https://iqss.github.io/dss-rce/}
  \end{itemize}
\item
  HBS

  \begin{itemize}
  \tightlist
  \item
    Research Computing Services workshops:
    \url{https://training.rcs.hbs.org/workshops}
  \item
    Other HBS RCS resources:
    \url{https://training.rcs.hbs.org/workshop-materials}
  \item
    RCS consulting email: \url{mailto:research@hbs.edu}
  \end{itemize}
\item
  Stata

  \begin{itemize}
  \tightlist
  \item
    UCLA website: \url{http://www.ats.ucla.edu/stat/Stata/}
  \item
    Stata website: \url{http://www.stata.com/help.cgi?contents}
  \item
    Email list: \url{http://www.stata.com/statalist/}
  \end{itemize}
\end{itemize}

\chapter{Stata Data Management}\label{stata-data-management}

\textbf{Topics}

\begin{itemize}
\tightlist
\item
  Generating and replacing variables
\item
  Processing with \texttt{by} statements
\item
  Missing values
\item
  Variable types and conversion
\item
  Merging, appending, and joining
\item
  Creating summarized data sets
\end{itemize}

\section{Setup}\label{setup-7}

\subsection{Software \& Materials}\label{software-materials-7}

Laptop users: you will need a copy of Stata installed on your machine.
Harvard FAS affiliates can install a licensed version from
\url{http://downloads.fas.harvard.edu/download}

\begin{itemize}
\tightlist
\item
  Download class materials at
  \url{https://github.com/IQSS/dss-workshops-redux/raw/master/Stata/StataDatMan.zip}
\item
  Extract materials from the zipped directory \texttt{StataDatMan.zip}
  (Right-click =\textgreater{} Extract All on Windows, double-click on
  Mac) and move them to your desktop!
\end{itemize}

\subsection{Organization}\label{organization-1}

\begin{itemize}
\tightlist
\item
  Please feel free to ask questions at any point if they are relevant to
  the current topic (or if you are lost!)
\item
  Collaboration is encouraged - please introduce yourself to your
  neighbors!
\item
  If you are using a laptop, you will need to adjust file paths
  accordingly
\item
  Make comments in your Do-file - save on flash drive or email to
  yourself
\end{itemize}

\subsection{Goals}\label{goals-7}

\begin{itemize}
\tightlist
\item
  This is an introduction to data management in Stata
\item
  Assumes basic knowledge of Stata
\item
  Not appropriate for people already familiar with Stata
\item
  If you are catching on before the rest of the class, experiment with
  command features described in help files
\item
  Learning Objectives:

  \begin{itemize}
  \tightlist
  \item
    Basic data manipulation commands
  \item
    Processing with \texttt{by} statements
  \item
    Dealing with missing values
  \item
    Variable types and conversion
  \item
    Merging and appending datasets
  \end{itemize}
\end{itemize}

\section{Opening Files in Stata}\label{opening-files-in-stata}

\begin{itemize}
\tightlist
\item
  Look at bottom left hand corner of Stata screen

  \begin{itemize}
  \tightlist
  \item
    This is the directory Stata is currently reading from
  \end{itemize}
\item
  Files are located in the StataDatMan folder in your home directory
\item
  Start by telling Stata where to look for these
\end{itemize}

\begin{verbatim}
  // change directory
  cd "~/Desktop/Stata/StataDatMan"

  // Use dir to see what is in the directory:
  dir
  dir dataSets

  // use the gss data set
  use dataSets/gss.dta
\end{verbatim}

\begin{verbatim}
set more off

cd "~/Desktop/Stata/StataDatMan"
/nfs/www/edu-harvard-iq-tutorials/Stata/StataDatMan


dir

total 100
drwxrwsr-x. 2 apache tutorwww  4096 Oct  9 08:44 dataSets/
-rwxrwxr-x. 1 izahn  tutorwww  1302 Oct  9 08:44 Exercises.do*
drwxrwsr-x. 2 apache tutorwww  4096 Oct  9 08:44 images/
drwxrwsr-x. 4 apache tutorwww  4096 Oct  9 08:44 StataDatMan/
-rwxrwxr-x. 1 izahn  tutorwww 17446 Oct  9 08:44 StataDatMan.do*
-rwxrwxr-x. 1 izahn  tutorwww 38153 Oct  9 08:44 StataDatMan.html*
-rwxrwxr-x. 1 izahn  tutorwww 20463 Oct  9 08:44 StataDatMan.org*
dir dataSets

total 2644
-rwxrwxr-x. 1 izahn tutorwww 275705 Oct  9 08:44 gss1.dta*
-rwxrwxr-x. 1 izahn tutorwww 263324 Oct  9 08:44 gss2.dta*
-rwxrwxr-x. 1 izahn tutorwww 532880 Oct  9 08:44 gssAddObserve.dta*
-rwxrwxr-x. 1 izahn tutorwww 527005 Oct  9 08:44 gssAppend.dta*
-rwxrwxr-x. 1 izahn tutorwww 527005 Oct  9 08:44 gsscompare1.dta*
-rwxrwxr-x. 1 izahn tutorwww 538755 Oct  9 08:44 gss.dta*
-rwxrwxr-x. 1 izahn tutorwww   1139 Oct  9 08:44 marital.dta*


use dataSets/gss.dta
\end{verbatim}

\section{Generating \& replacing
variables}\label{generating-replacing-variables-1}

\subsection{Data Manipulation
Commands}\label{data-manipulation-commands}

Basic commands you'll use for generating new variables or recoding
existing variables:

\begin{itemize}
\tightlist
\item
  gen
\item
  egen
\item
  replace
\item
  recode
\end{itemize}

Many different means of accomplishing the same thing in Stata -- find
what is comfortable (and easy) for you!

\subsection{Generate \& Replace}\label{generate-replace}

The \texttt{replace} command is often used with logic statements.
Available logical operators include the following:

\begin{longtable}[]{@{}ll@{}}
\toprule
Operator & Meaning\tabularnewline
\midrule
\endhead
== & equal to\tabularnewline
!= & not equal to\tabularnewline
\textgreater{} & greater than\tabularnewline
\textgreater{}= & greater than or equal to\tabularnewline
\textless{} & less than\tabularnewline
\textless{}= & less than or equal to\tabularnewline
\& & and\tabularnewline
&\tabularnewline
\bottomrule
\end{longtable}

For example:

\begin{verbatim}
  //create "hapnew" variable
  gen hapnew = .
  //set to 0 if happy equals 1
  replace hapnew=0 if happy==1 
  //set to 1 if happy both and hapmar are greater than 3
  replace hapnew=1 if happy>3 & hapmar>3 
  // tabulate the new 
  tab hapnew
\end{verbatim}

\begin{verbatim}

gen hapnew = .
(1,419 missing values generated)

replace hapnew=0 if happy==1 
(435 real changes made)

replace hapnew=1 if happy>3 & hapmar>3 
(4 real changes made)

tab hapnew

     hapnew |      Freq.     Percent        Cum.
------------+-----------------------------------
          0 |        435       99.09       99.09
          1 |          4        0.91      100.00
------------+-----------------------------------
      Total |        439      100.00
\end{verbatim}

\subsection{Recode}\label{recode}

The \texttt{recode} command is basically generate and replace combined.
You can recode an existing variable OR use recode to create a new
variable (via the \texttt{gen} option).

\begin{verbatim}
  // recode the wrkstat variable 
  recode wrkstat (1=8) (2=7) (3=6) (4=5) (5=4) (6=3) (7=2) (8=1)
  // recode wrkstat into a new variable named wrkstat2
  recode wrkstat (1=8), gen(wrkstat2)
  // tabulate workstat
  tab wrkstat
\end{verbatim}

\begin{verbatim}

recode wrkstat (1=8) (2=7) (3=6) (4=5) (5=4) (6=3) (7=2) (8=1)
(wrkstat: 1419 changes made)

recode wrkstat (1=8), gen(wrkstat2)
(32 differences between wrkstat and wrkstat2)

tab wrkstat

      LABOR FRCE |
          STATUS |      Freq.     Percent        Cum.
-----------------+-----------------------------------
WORKING FULLTIME |         32        2.26        2.26
WORKING PARTTIME |        155       10.92       13.18
TEMP NOT WORKING |         34        2.40       15.57
UNEMPL, LAID OFF |        214       15.08       30.66
         RETIRED |         29        2.04       32.70
          SCHOOL |         35        2.47       35.17
   KEEPING HOUSE |        146       10.29       45.45
           OTHER |        774       54.55      100.00
-----------------+-----------------------------------
           Total |      1,419      100.00
\end{verbatim}

The table below illustrates common forms of recoding

\begin{longtable}[]{@{}lll@{}}
\toprule
\begin{minipage}[b]{0.20\columnwidth}\raggedright\strut
Rule\strut
\end{minipage} & \begin{minipage}[b]{0.19\columnwidth}\raggedright\strut
Example\strut
\end{minipage} & \begin{minipage}[b]{0.33\columnwidth}\raggedright\strut
Meaning\strut
\end{minipage}\tabularnewline
\midrule
\endhead
\begin{minipage}[t]{0.20\columnwidth}\raggedright\strut
\#=\#\strut
\end{minipage} & \begin{minipage}[t]{0.19\columnwidth}\raggedright\strut
3=1\strut
\end{minipage} & \begin{minipage}[t]{0.33\columnwidth}\raggedright\strut
3 recoded to 1\strut
\end{minipage}\tabularnewline
\begin{minipage}[t]{0.20\columnwidth}\raggedright\strut
\#\#=\#\strut
\end{minipage} & \begin{minipage}[t]{0.19\columnwidth}\raggedright\strut
2.
\texttt{9\ \ \ \ \ \ \ \ \textbar{}\ 2\ and\ .\ recoded\ to\ 9\ \ \ \ \textbar{}\ \ \ \ \ \ \ \ \ \ \ \ \ \ \ \ \ \ \ \ \ \ \ \ \ \ \ \textbar{}\ \#/\#}
\#\strut
\end{minipage} & \begin{minipage}[t]{0.33\columnwidth}\raggedright\strut
1/5=4\strut
\end{minipage}\tabularnewline
\begin{minipage}[t]{0.20\columnwidth}\raggedright\strut
nonmissing=\#\strut
\end{minipage} & \begin{minipage}[t]{0.19\columnwidth}\raggedright\strut
nonmiss=8\strut
\end{minipage} & \begin{minipage}[t]{0.33\columnwidth}\raggedright\strut
nonmissing recoded to 8\strut
\end{minipage}\tabularnewline
\begin{minipage}[t]{0.20\columnwidth}\raggedright\strut
missing=\#\strut
\end{minipage} & \begin{minipage}[t]{0.19\columnwidth}\raggedright\strut
miss=9\strut
\end{minipage} & \begin{minipage}[t]{0.33\columnwidth}\raggedright\strut
missing recoded to 9\strut
\end{minipage}\tabularnewline
\bottomrule
\end{longtable}

\subsection{egen}\label{egen}

The \texttt{egen} command (``extensions'' to the \texttt{gen} command)
provides convenient methods for performing many common data manipulation
tasks.

For example, we can use \texttt{egen} to create a new variable that
counts the number of ``yes'' responses on computer, email and internet
use:

\begin{verbatim}
  // count number of yes on use comp email and net 
  egen compuser= anycount(usecomp usemail usenet), values(1)
  tab compuser
\end{verbatim}

\begin{verbatim}

egen compuser= anycount(usecomp usemail usenet), values(1)
tab compuser

    usecomp |
    usemail |
usenet == 1 |      Freq.     Percent        Cum.
------------+-----------------------------------
          0 |        623       43.90       43.90
          1 |        142       10.01       53.91
          2 |         78        5.50       59.41
          3 |        576       40.59      100.00
------------+-----------------------------------
      Total |      1,419      100.00
\end{verbatim}

Here are some additional examples of \texttt{egen} in action:

\begin{verbatim}
  // assess how much missing data each participant has:
  egen countmiss = rowmiss(age-wifeft)
  codebook countmiss
  // compare values on multiple variables
  egen ftdiff=diff(wkftwife wkfthusb)
  codebook ftdiff
\end{verbatim}

\begin{verbatim}

egen countmiss = rowmiss(age-wifeft)
codebook countmiss

-------------------------------------------------------------------------------
countmiss                                                           (unlabeled)
-------------------------------------------------------------------------------

                  type:  numeric (float)

                 range:  [0,7]                        units:  1
         unique values:  6                        missing .:  0/1,419

            tabulation:  Freq.  Value
                           296  0
                           215  1
                           113  2
                             7  3
                           782  6
                             6  7

egen ftdiff=diff(wkftwife wkfthusb)
codebook ftdiff

-------------------------------------------------------------------------------
ftdiff                                                   diff wkftwife wkfthusb
-------------------------------------------------------------------------------

                  type:  numeric (float)

                 range:  [0,1]                        units:  1
         unique values:  2                        missing .:  0/1,419

            tabulation:  Freq.  Value
                         1,169  0
                           250  1
\end{verbatim}

You will need to refer to the documentation to discover what else
\texttt{egen} can do: type ``help egen'' in Stata to get a complete list
of functions.

\section{Exercise 0}\label{exercise-0-7}

\textbf{Generate, Replace, Recode \& Egen}

Open the gss.dta data.

\begin{enumerate}
\def\labelenumi{\arabic{enumi}.}
\tightlist
\item
  Generate a new variable that represents the squared value of age.
\item
  Generate a new variable equal to ``1'' if income is greater than
  ``19''.
\item
  Create a new variable that counts the number of missing responses for
  each respondent. What is the maximum number of missing variables?
\end{enumerate}

\section{By processing}\label{by-processing}

\subsection{\texorpdfstring{The \texttt{bysort}
Command}{The bysort Command}}\label{the-bysort-command}

Sometimes, you'd like to create variables based on different categories
of a single variable. For example, say you want to look at happiness
based on whether an individual is male or female. The ``bysort'' prefix
does just this:

\begin{verbatim}
  // tabulate happy separately for male and female 
  bysort sex: tab happy
  // generate summary statistics using bysort 
  bysort state: egen stateincome = mean(income)
  bysort degree: egen degreeincome = mean(income)
  bysort marital: egen marincomesd = sd(income)
\end{verbatim}

\begin{verbatim}

bysort sex: tab happy

-------------------------------------------------------------------------------
-> sex = Male

      GENERAL |
    HAPPINESS |      Freq.     Percent        Cum.
--------------+-----------------------------------
   VERY HAPPY |        189       30.39       30.39
 PRETTY HAPPY |        350       56.27       86.66
NOT TOO HAPPY |         73       11.74       98.39
           NA |         10        1.61      100.00
--------------+-----------------------------------
        Total |        622      100.00

-------------------------------------------------------------------------------
-> sex = Female

      GENERAL |
    HAPPINESS |      Freq.     Percent        Cum.
--------------+-----------------------------------
   VERY HAPPY |        246       30.87       30.87
 PRETTY HAPPY |        447       56.09       86.95
NOT TOO HAPPY |         84       10.54       97.49
           DK |          1        0.13       97.62
           NA |         19        2.38      100.00
--------------+-----------------------------------
        Total |        797      100.00

bysort state: egen stateincome = mean(income)
variable state not found
r(111);
bysort degree: egen degreeincome = mean(income)
bysort marital: egen marincomesd = sd(income)
\end{verbatim}

\subsection{\texorpdfstring{\texttt{by} prefix vs. \texttt{by}
options}{by prefix vs. by options}}\label{by-prefix-vs.-by-options}

Some commands won't work with by prefix, but instead have a \texttt{by}
option:

\begin{verbatim}
  // generate separate histograms for female and male 
  hist nethrs, by(sex)
\end{verbatim}

\section{Missing values}\label{missing-values}

You always need to consider how missing values are coded when recoding
variables.

\begin{itemize}
\tightlist
\item
  Stata's symbol for a missing value is \texttt{.}
\item
  Stata interprets \texttt{.} as a large value
\item
  Easy to make mistakes!
\end{itemize}

To identify highly educated women, we might use the command:

\begin{verbatim}
  // generate and replace without considering missing values
  gen hi_ed=0
  replace hi_ed=1 if wifeduc>15
  // What happens to our missing values?
  tab hi_ed, mi nola
\end{verbatim}

\begin{verbatim}

gen hi_ed=0
replace hi_ed=1 if wifeduc>15
(944 real changes made)

tab hi_ed, mi nola

      hi_ed |      Freq.     Percent        Cum.
------------+-----------------------------------
          0 |        475       33.47       33.47
          1 |        944       66.53      100.00
------------+-----------------------------------
      Total |      1,419      100.00
\end{verbatim}

It looks like around 66\% have higher education, but look closer:

\begin{verbatim}
  // gen hi_ed2, but don't set a value if wifeduc is missing
  gen hi_ed2 = 0 if wifeduc != . 
  // only replace non-missing
  replace hi_ed2=1 if wifeduc >15 & wifeduc !=. 
  //check to see that missingness is preserved
  tab hi_ed2, mi
\end{verbatim}

\begin{verbatim}

gen hi_ed2 = 0 if wifeduc != . 
(797 missing values generated)

replace hi_ed2=1 if wifeduc >15 & wifeduc !=. 
(147 real changes made)

 |        797       56.17      100.00
------------+-----------------------------------
      Total |      1,419      100.00
\end{verbatim}

The correct value is 10\%. Moral of the story? Be careful with missing
values and remember that Stata considers missing values to be large!

\subsection{Bulk Conversion to Missing
Values}\label{bulk-conversion-to-missing-values}

Often the data collection/generating procedure will have used some other
value besides \texttt{.} to represent missing values. The
\texttt{mvdecode} command will convert all these values to missing. For
example:

\begin{verbatim}
  mvdecode _all, mv(999)
\end{verbatim}

\begin{verbatim}
mvdecode _all, mv(999)
\end{verbatim}

\begin{itemize}
\tightlist
\item
  The ``\_all'' command tells Stata to do this to all variables
\item
  Use this command carefully!

  \begin{itemize}
  \tightlist
  \item
    If you have any variables where ``999'' is a legitimate value, Stata
    is going to recode it to missing
  \item
    As an alternative, you could list var names separately rather than
    using ``\_all''
  \end{itemize}
\end{itemize}

\section{Variable types}\label{variable-types}

Stata uses two main types of variables: String and Numeric. To be able
to perform any mathematical operations, your variables need to be in a
numeric format. Stata can store numbers with differing levels of
precision, as described in the table below.

\begin{longtable}[]{@{}lllll@{}}
\toprule
type & Minimum & Maximum & being 0 & bytes\tabularnewline
\midrule
\endhead
byte & -127 & 100 & +/-1 & 1\tabularnewline
int & -32,767 & 32,740 & +/-1 & 2\tabularnewline
long & -2,147,483,647 & 2,147,483,620 & +/-1 & 4\tabularnewline
float & -1.70141173319*1038 & 1.70141173319*1038 & +/-10-38 &
4\tabularnewline
double & -8.9884656743*10307 & 8.9884656743*10307 & +/-10-323 &
8\tabularnewline
\bottomrule
\end{longtable}

\begin{itemize}
\tightlist
\item
  Precision for float is 3.795x10-8.
\item
  Precision for double is 1.414x10-16.
\end{itemize}

\subsection{Converting to \& from
Strings}\label{converting-to-from-strings}

Stata provides several ways to convert to and from strings. You can use
\texttt{tostring} and \texttt{destring} to convert from one type to the
other:

\begin{verbatim}
  // convert degree to a string
  tostring degree, gen(degree_s)
  // and back to a number
  destring degree_s, gen(degree_n)
\end{verbatim}

\begin{verbatim}

tostring degree, gen(degree_s)
degree_s generated as str1

destring degree_s, gen(degree_n)
degree_s has all characters numeric; degree_n generated as byte
\end{verbatim}

Use \texttt{decode} and \texttt{encode} to convert to/from variable
labels:

\begin{verbatim}
  // convert degree to a descriptive string
  decode degree, gen(degree_s2)
  // and back to a number with labels
  encode degree_s2, gen(degree_n2)
\end{verbatim}

\begin{verbatim}

decode degree, gen(degree_s2)

encode degree_s2, gen(degree_n2)
\end{verbatim}

\subsection{Converting Strings to
Date/Time}\label{converting-strings-to-datetime}

Often date/time variables start out as strings -- You'll need to convert
them to numbers using one of the conversion functions listed below.

\begin{longtable}[]{@{}lll@{}}
\toprule
Format & Meaning & String-to-numeric conversion function\tabularnewline
\midrule
\endhead
\%tc & milliseconds & clock(string, mask)\tabularnewline
\%td & days & date(string, mask)\tabularnewline
\%tw & weeks & weekly(string, mask)\tabularnewline
\%tm & months & monthly(string, mask)\tabularnewline
\%tq & quarters & quarterly(string, mask)\tabularnewline
\%ty & years & yearly(string, mask)\tabularnewline
\bottomrule
\end{longtable}

Date/time variables are stored as the number of units elapsed since
01jan1960 00:00:00.000. For example, the \texttt{date} function returns
the number of days since that time, and the \texttt{clock} function
returns the number of milliseconds since that time.

\begin{verbatim}
  // create string variable and convert to date
  gen date = "November 9 2020"
  gen date1 = date(date, "MDY")
  list date1 in 1/5
\end{verbatim}

\begin{verbatim}

gen date = "November 9 2020"
gen date1 = date(date, "MDY")
list date1 in 1/5

     +-------+
     | date1 |
     |-------|
  1. | 22228 |
  2. | 22228 |
  3. | 22228 |
  4. | 22228 |
  5. | 22228 |
     +-------+
\end{verbatim}

\subsection{Formatting Numbers as
Dates}\label{formatting-numbers-as-dates}

Once you have converted the string to a number you can format it for
display. You can simply accept the defaults used by your formatting
string or provide details to customize it.

\begin{verbatim}
  // format so humans can read the date
  format date1 %d
  list date1 in 1/5
  // format with detail
  format date1 %tdMonth_dd,_CCYY
  list date1 in 1/5
\end{verbatim}

\begin{verbatim}

format date1 %d
list date1 in 1/5

     +-----------+
     |     date1 |
     |-----------|
  1. | 09nov2020 |
  2. | 09nov2020 |
  3. | 09nov2020 |
  4. | 09nov2020 |
  5. | 09nov2020 |
     +-----------+

format date1 %tdMonth_dd,_CCYY
list date1 in 1/5

     +------------------+
     |            date1 |
     |------------------|
  1. | November 9, 2020 |
  2. | November 9, 2020 |
  3. | November 9, 2020 |
  4. | November 9, 2020 |
  5. | November 9, 2020 |
     +------------------+
\end{verbatim}

\section{Exercise 1}\label{exercise-1-7}

\textbf{Missing Values, String Conversion, \& by Processing}

\begin{enumerate}
\def\labelenumi{\arabic{enumi}.}
\tightlist
\item
  Recode values ``99'' and ``98'' on the variable, ``hrs1'' as
  ``missing.''
\item
  Recode the marital variable into a ``string'' variable and then back
  into a numeric variable.
\item
  Create a new variable that associates each individual with the average
  number of hours worked among individuals with matching educational
  degrees (see the last ``by'' example for inspiration).
\end{enumerate}

\section{Merging, appending, \&
joining}\label{merging-appending-joining}

\subsection{Appending Datasets}\label{appending-datasets}

Sometimes you have observations in two different datasets, or you'd like
to add observations to an existing dataset. In this case you can use the
\texttt{append} command to add observations to the end of the
observations in the master dataset. For example:

\begin{verbatim}
  clear
  // from the append help file
  webuse even
  list
  webuse odd
  list
  // Append even data to the end of the odd data
  append using "http://www.stata-press.com/data/r14/even"
  list
  clear
\end{verbatim}

\begin{verbatim}
clear

webuse even
(6th through 8th even numbers)
list

     +---------------+
     | number   even |
     |---------------|
  1. |      6     12 |
  2. |      7     14 |
  3. |      8     16 |
     +---------------+
webuse odd
(First five odd numbers)
list

     +--------------+
     | number   odd |
     |--------------|
  1. |      1     1 |
  2. |      2     3 |
  3. |      3     5 |
  4. |      4     7 |
  5. |      5     9 |
     +--------------+

append using "http://www.stata-press.com/data/r14/even"
list

     +---------------------+
     | number   odd   even |
     |---------------------|
  1. |      1     1      . |
  2. |      2     3      . |
  3. |      3     5      . |
  4. |      4     7      . |
  5. |      5     9      . |
     |---------------------|
  6. |      6     .     12 |
  7. |      7     .     14 |
  8. |      8     .     16 |
     +---------------------+
clear
\end{verbatim}

To keep track of where observations came from, use the \texttt{generate}
option as shown below:

\begin{verbatim}
  webuse odd
  append using "http://www.stata-press.com/data/r14/even", generate(observesource)
  list
  clear
\end{verbatim}

\begin{verbatim}
webuse odd
(First five odd numbers)
 ce)
list

     +--------------------------------+
     | number   odd   observ~e   even |
     |--------------------------------|
  1. |      1     1          0      . |
  2. |      2     3          0      . |
  3. |      3     5          0      . |
  4. |      4     7          0      . |
  5. |      5     9          0      . |
     |--------------------------------|
  6. |      6     .          1     12 |
  7. |      7     .          1     14 |
  8. |      8     .          1     16 |
     +--------------------------------+
clear
\end{verbatim}

There is a ``force'' option will allow for data type mismatches, but
again this is not recommended.

Remember, \texttt{append} is for adding observations (i.e., rows) from a
second data set.

\subsection{Merging Datasets}\label{merging-datasets}

You can \texttt{merge} variables from a second dataset to the dataset
you're currently working with.

\begin{itemize}
\tightlist
\item
  Current active dataset = master dataset
\item
  Dataset you'd like to merge with master = using dataset
\end{itemize}

There are different ways that you might be interested in merging data:

\begin{itemize}
\tightlist
\item
  Two datasets with same participant pool, one row per participant (1:1)
\item
  A dataset with one participant per row with a dataset with multiple
  rows per participant (1:many or many:1)
\end{itemize}

Before you begin:

\begin{itemize}
\tightlist
\item
  Identify the ``ID'' that you will use to merge your two datasets
\item
  Determine which variables you'd like to merge
\item
  In Stata \textgreater{}= 11, data does NOT have to be sorted
\item
  Variable types must match across datasets (there is a ``force'' option
  to get around this, but not recommended)
\end{itemize}

\begin{verbatim}
  // Adapted from the merge help page
  webuse autosize 
  list
  webuse autoexpense
  list

  webuse autosize
  merge 1:1 make using "http://www.stata-press.com/data/r14/autoexpense"
  list
  clear

  // keep only the matches (AKA "inner join")
  webuse autosize, clear
  merge 1:1 make using "http://www.stata-press.com/data/r14/autoexpense", keep(match) nogen
  list
  clear
\end{verbatim}

\begin{verbatim}

webuse autosize 
(1978 Automobile Data)
list

     +------------------------------------+
     | make               weight   length |
     |------------------------------------|
  1. | Toyota Celica       2,410      174 |
  2. | BMW 320i            2,650      177 |
  3. | Cad. Seville        4,290      204 |
  4. | Pont. Grand Prix    3,210      201 |
  5. | Datsun 210          2,020      165 |
     |------------------------------------|
  6. | Plym. Arrow         3,260      170 |
     +------------------------------------+
webuse autoexpense
(1978 Automobile Data)
list

     +---------------------------------+
     | make                price   mpg |
     |---------------------------------|
  1. | Toyota Celica       5,899    18 |
  2. | BMW 320i            9,735    25 |
  3. | Cad. Seville       15,906    21 |
  4. | Pont. Grand Prix    5,222    19 |
  5. | Datsun 210          4,589    35 |
     +---------------------------------+

webuse autosize
(1978 Automobile Data)
merge 1:1 make using "http://www.stata-press.com/data/r14/autoexpense"

    Result                           # of obs.
    -----------------------------------------
    not matched                             1
        from master                         1  (_merge==1)
        from using                          0  (_merge==2)

    matched                                 5  (_merge==3)
    -----------------------------------------
list

     +---------------------------------------------------------------------+
     | make               weight   length    price   mpg            _merge |
     |---------------------------------------------------------------------|
  1. | BMW 320i            2,650      177    9,735    25       matched (3) |
  2. | Cad. Seville        4,290      204   15,906    21       matched (3) |
  3. | Datsun 210          2,020      165    4,589    35       matched (3) |
  4. | Plym. Arrow         3,260      170        .     .   master only (1) |
  5. | Pont. Grand Prix    3,210      201    5,222    19       matched (3) |
     |---------------------------------------------------------------------|
  6. | Toyota Celica       2,410      174    5,899    18       matched (3) |
     +---------------------------------------------------------------------+
clear


webuse autosize, clear
(1978 Automobile Data)
 match) nogen

    Result                           # of obs.
    -----------------------------------------
    not matched                             0
    matched                                 5  
    -----------------------------------------
list

     +---------------------------------------------------+
     | make               weight   length    price   mpg |
     |---------------------------------------------------|
  1. | BMW 320i            2,650      177    9,735    25 |
  2. | Cad. Seville        4,290      204   15,906    21 |
  3. | Datsun 210          2,020      165    4,589    35 |
  4. | Pont. Grand Prix    3,210      201    5,222    19 |
  5. | Toyota Celica       2,410      174    5,899    18 |
     +---------------------------------------------------+
clear
\end{verbatim}

Remember, \texttt{merge} is for adding variables (i.e., columns) from a
second data set.

\subsection{Merge Options}\label{merge-options}

There are several options that provide more fine-grain control over what
happens to non-id columns contained in both data sets. If you've
carefully cleaned and prepared the data prior to merging this shouldn't
be an issue, but here are some details about how stata handles this
situation.

\begin{itemize}
\tightlist
\item
  In standard merge, the master dataset is the authority and WON'T
  CHANGE
\item
  If your master dataset has missing data and some of those values are
  not missing in your using dataset, specify ``update'' -- this will
  fill in missing data in master
\item
  If you want data from your using dataset to overwrite that in your
  master, specify ``replace update'' -- this will replace master data
  with using data UNLESS the value is missing in the using dataset
\end{itemize}

\subsection{Many-to-many merges}\label{many-to-many-merges}

Stata allows you to specify merges like
\texttt{merge\ m:m\ id\ using\ newdata.dta}, but I have never seen this
do anything useful. To quote the official
\href{https://www.stata.com/manuals13/dmerge.pdf}{Stata manual}:

\begin{quote}
\texttt{m:m} specifies a many-to-many merge and is a \textbf{bad idea}.
In an \texttt{m:m} merge, observations are matched within equal values
of the key variable(s), with the first observation being matched to the
first; the second, to the second; and so on. If the master and using
have an unequal number of observations within the group, then the last
observation of the shorter group is used repeatedly to match with
subsequent observations of the longer group. Thus \textbf{m:m merges are
dependent on the current sort order---something which should never
happen}. \textbf{Because m:m merges are such a bad idea, we are not
going to show you an example}. If you think that you need an m:m merge,
then you probably need to work with your data so that you can use a 1:m
or m:1 merge. Tips for this are given in Troubleshooting m:m merges
below
\end{quote}

(emphasis added).

If you are thinking about using \texttt{merge\ m:m} chances are good
that you actually need \texttt{joinby}. Here is a quick example,
modified from the \texttt{joinby} help page.

\begin{verbatim}
  clear
  webuse parent
  list
  webuse children
  list
  // Complete and utter nonsense!
  merge m:m family_id using http://www.stata-press.com/data/r14/parent 
  // You want joinby instead
  clear
  webuse children
  joinby family_id using http://www.stata-press.com/data/r14/parent 
\end{verbatim}

Remember, \texttt{merge\ m:m} is old and broken; \textbf{do not use}.
Anytime you think you might want \texttt{m:m} you should use
\texttt{joinby} instead.

\section{Creating summarized data
sets}\label{creating-summarized-data-sets}

\subsection{Collapse}\label{collapse}

Collapse will take master data and create a new dataset of summary
statistics

\begin{itemize}
\tightlist
\item
  Useful in hierarchical linear modeling if you'd like to create
  aggregate, summary statistics
\item
  Can generate group summary data for many descriptive stats
\item
  Can also attach weights
\end{itemize}

Before you collapse:

\begin{itemize}
\tightlist
\item
  Save your master dataset and then save it again under a new name (this
  will prevent collapse from writing over your original data\_
\item
  Consider issues of missing data. Do you want Stata to use all possible
  observations? If not, the \texttt{cw} (casewise) option will make
  casewise deletions
\end{itemize}

\begin{verbatim}
  // Adapted from the collapse help page
  clear
  webuse college
  list
  // mean and sd by hospital
  collapse (mean) mean_gpa = gpa mean_hour = hour (sd) sd_gpa = gpa sd_hour = hour, by(year)
  list
  clear
\end{verbatim}

\begin{verbatim}

clear
webuse college
list

     +----------------------------+
     | gpa   hour   year   number |
     |----------------------------|
  1. | 3.2     30      1        3 |
  2. | 3.5     34      1        2 |
  3. | 2.8     28      1        9 |
  4. | 2.1     30      1        4 |
  5. | 3.8     29      2        3 |
     |----------------------------|
  6. | 2.5     30      2        4 |
  7. | 2.9     35      2        5 |
  8. | 3.7     30      3        4 |
  9. | 2.2     35      3        2 |
 10. | 3.3     33      3        3 |
     |----------------------------|
 11. | 3.4     32      4        5 |
 12. | 2.9     31      4        2 |
     +----------------------------+

 our, by(year)
list

     +--------------------------------------------------+
     | year   mean_gpa   mean_h~r     sd_gpa    sd_hour |
     |--------------------------------------------------|
  1. |    1        2.9       30.5   .6055301   2.516612 |
  2. |    2   3.066667   31.33333   .6658328    3.21455 |
  3. |    3   3.066667   32.66667   .7767453   2.516612 |
  4. |    4       3.15       31.5   .3535534   .7071068 |
     +--------------------------------------------------+
clear
\end{verbatim}

You could also generate different statistics for multiple variables

\section{Exercise 2}\label{exercise-2-5}

\textbf{Merge, Append, \& Collapse}

Open the gss2.dta dataset. This dataset contains only half of the
variables that are in the complete gss dataset.

\begin{enumerate}
\def\labelenumi{\arabic{enumi}.}
\tightlist
\item
  Merge dataset gss1.dta with dataset gss2.dta. The identification
  variable is ``id.''
\item
  Open the gss.dta dataset and merge in data from the ``marital.dta''
  dataset, which includes income information grouped by individuals'
  marital status. The marital dataset contains collapsed data regarding
  average statistics of individuals based on their marital status.
\item
  Open the gssAppend.dta dataset and Create a new dataset that combines
  the observations in gssAppend.dta with those in gssAddObserve.dta.
\item
  Open the gss.dta dataset. Create a new dataset that summarizes mean
  and standard deviation of income based on individuals' degree status
  (``degree''). In the process of creating this new dataset, rename your
  three new variables.
\end{enumerate}

\section{Exercise Solutions}\label{exercise-solutions-5}

\subsection{Ex 0: prototype}\label{ex-0-prototype-5}

Open the gss.dta data.

\begin{enumerate}
\def\labelenumi{\arabic{enumi}.}
\tightlist
\item
  Generate a new variable that represents the squared value of age.
\end{enumerate}

\begin{verbatim}
     use dataSets/gss.dta, clear
     gen age2 = age^2
\end{verbatim}

\begin{enumerate}
\def\labelenumi{\arabic{enumi}.}
\setcounter{enumi}{1}
\tightlist
\item
  Generate a new variable equal to ``1'' if income is greater than
  ``19''.
\end{enumerate}

\begin{verbatim}
     describe income
     label list income
     recode income (99=.) (98=.)
     gen highincome =0 if income != .
     replace highincome=1 if income>19
     sum highincome
\end{verbatim}

\begin{enumerate}
\def\labelenumi{\arabic{enumi}.}
\setcounter{enumi}{2}
\tightlist
\item
  Create a new variable that counts the number of missing responses for
  each respondent. What is the maximum number of missing variables?
\end{enumerate}

\begin{verbatim}
  egen nmissing = rowmiss(_all)
  sum nmissing
\end{verbatim}

\subsection{Ex 1: prototype}\label{ex-1-prototype-5}

\begin{enumerate}
\def\labelenumi{\arabic{enumi}.}
\tightlist
\item
  Recode values ``99'' and ``98'' on the variable, ``hrs1'' as
  ``missing.''
\end{enumerate}

\begin{verbatim}
  use dataSets/gss.dta, clear
  sum hrs1
  recode hrs1 (99=.) (98=.) 
  sum hrs1
\end{verbatim}

\begin{enumerate}
\def\labelenumi{\arabic{enumi}.}
\setcounter{enumi}{1}
\tightlist
\item
  Recode the marital variable into a ``string'' variable and then back
  into a numeric variable.
\end{enumerate}

\begin{verbatim}
  tostring marital, gen(marstring)
  destring marstring, gen(mardstring)
  //compare with
  decode marital, gen(marital_s)
  encode marital_s, gen(marital_n)

  describe marital marstring mardstring marital_s marital_n
  sum marital marstring mardstring marital_s marital_n
\end{verbatim}

\begin{enumerate}
\def\labelenumi{\arabic{enumi}.}
\setcounter{enumi}{2}
\tightlist
\item
  Create a new variable that associates each individual with the average
  number of hours worked among individuals with matching educational
  degrees (see the last ``by'' example for inspiration).
\end{enumerate}

\begin{verbatim}
  bysort degree: egen hrsdegree = mean(hrs1)
  tab hrsdegree
  tab hrsdegree degree 
\end{verbatim}

\subsection{Ex 2: prototype}\label{ex-2-prototype-4}

Open the gss2.dta dataset. This dataset contains only half of the
variables that are in the complete gss dataset.

\begin{enumerate}
\def\labelenumi{\arabic{enumi}.}
\tightlist
\item
  Merge dataset gss1.dta with dataset gss2.dta. The identification
  variable is ``id.''
\end{enumerate}

\begin{verbatim}
  use dataSets/gss2.dta, clear
  merge 1:1 id using dataSets/gss1.dta
  save gss3.dta, replace
\end{verbatim}

\begin{enumerate}
\def\labelenumi{\arabic{enumi}.}
\setcounter{enumi}{1}
\tightlist
\item
  Open the gss.dta dataset and merge in data from the ``marital.dta''
  dataset, which includes income information grouped by individuals'
  marital status. The marital dataset contains collapsed data regarding
  average statistics of individuals based on their marital status.
\end{enumerate}

\begin{verbatim}
  use dataSets/gss.dta, clear
  merge m:1 marital using dataSets/marital.dta, nogenerate replace update
  save gss4.dta, replace
\end{verbatim}

\begin{enumerate}
\def\labelenumi{\arabic{enumi}.}
\setcounter{enumi}{2}
\tightlist
\item
  Open the gssAppend.dta dataset and Create a new dataset that combines
  the observations in gssAppend.dta with those in gssAddObserve.dta.
\end{enumerate}

\begin{verbatim}
  use dataSets/gssAppend.dta, clear
  append using dataSets/gssAddObserve, generate(observe) 
\end{verbatim}

\begin{enumerate}
\def\labelenumi{\arabic{enumi}.}
\setcounter{enumi}{3}
\tightlist
\item
  Open the gss.dta dataset. Create a new dataset that summarizes mean
  and standard deviation of income based on individuals' degree status
  (``degree''). In the process of creating this new dataset, rename your
  three new variables.
\end{enumerate}

\begin{verbatim}
  use dataSets/gss.dta, clear
  save collapse2.dta, replace
  use collapse2.dta, clear
  collapse (mean) meaninc=income (sd) sdinc=income, by(marital)
\end{verbatim}

\section{Wrap-up}\label{wrap-up-8}

\subsection{Feedback}\label{feedback-8}

These workshops are a work-in-progress, please provide any feedback to:
\href{mailto:help@iq.harvard.edu}{\nolinkurl{help@iq.harvard.edu}}

\subsection{Resources}\label{resources-8}

\begin{itemize}
\tightlist
\item
  IQSS

  \begin{itemize}
  \tightlist
  \item
    Workshops: \url{https://dss.iq.harvard.edu/workshop-materials}
  \item
    Data Science Services: \url{https://dss.iq.harvard.edu/}
  \item
    Research Computing Environment:
    \url{https://iqss.github.io/dss-rce/}
  \end{itemize}
\item
  HBS

  \begin{itemize}
  \tightlist
  \item
    Research Computing Services workshops:
    \url{https://training.rcs.hbs.org/workshops}
  \item
    Other HBS RCS resources:
    \url{https://training.rcs.hbs.org/workshop-materials}
  \item
    RCS consulting email: \url{mailto:research@hbs.edu}
  \end{itemize}
\item
  Stata

  \begin{itemize}
  \tightlist
  \item
    UCLA website: \url{http://www.ats.ucla.edu/stat/Stata/}
  \item
    Stata website: \url{http://www.stata.com/help.cgi?contents}
  \item
    Email list: \url{http://www.stata.com/statalist/}
  \end{itemize}
\end{itemize}

\chapter{Stata Modeling \& Graphing}\label{stata-modeling-graphing}

\textbf{Topics}

\begin{itemize}
\tightlist
\item
  Stata modeling

  \begin{itemize}
  \tightlist
  \item
    Simple regression
  \item
    Multiple regression
  \item
    Interactions
  \item
    Exporting regression tables
  \item
    Testing model assumptions
  \end{itemize}
\item
  Stata graphing

  \begin{itemize}
  \tightlist
  \item
    Univariate graphs
  \item
    Bivariate graphs
  \end{itemize}
\end{itemize}

\section{Setup}\label{setup-8}

\subsection{Software \& Materials}\label{software-materials-8}

Laptop users: you will need a copy of Stata installed on your machine.
Harvard FAS affiliates can install a licensed version from
\url{http://downloads.fas.harvard.edu/download}

\begin{itemize}
\tightlist
\item
  Download class materials at
  \url{https://github.com/IQSS/dss-workshops-redux/raw/master/Stata/StataModGraph.zip}
\item
  Extract materials from the zipped directory \texttt{StataModGraph.zip}
  (Right-click =\textgreater{} Extract All on Windows, double-click on
  Mac) and move them to your desktop!
\end{itemize}

\subsection{Organization}\label{organization-2}

\begin{itemize}
\tightlist
\item
  Please feel free to ask questions at any point if they are relevant to
  the current topic (or if you are lost!)
\item
  Collaboration is encouraged - please introduce yourself to your
  neighbors!
\item
  If you are using a laptop, you will need to adjust file paths
  accordingly
\item
  Make comments in your Do-file - save on flash drive or email to
  yourself
\end{itemize}

\subsection{Goals}\label{goals-8}

\begin{itemize}
\tightlist
\item
  This is an introduction to modeling and visualization in Stata
\item
  Assumes basic knowledge of Stata
\item
  Not appropriate for people already familiar with Stata
\item
  If you are catching on before the rest of the class, experiment with
  command features described in help files
\item
  Learning Objectives:

  \begin{itemize}
  \tightlist
  \item
    Fit models in Stata
  \item
    Test modeling assumptions
  \item
    Plot basic graphs in Stata
  \item
    Plot two-way graphs
  \end{itemize}
\end{itemize}

\section{Fitting models in Stata}\label{fitting-models-in-stata}

\subsection{Today's Dataset}\label{todays-dataset}

\begin{itemize}
\tightlist
\item
  We have data on a variety of variables for all 50 states
\item
  Population, density, energy use, voting tendencies, graduation rates,
  income, etc.
\item
  We're going to be predicting SAT scores
\item
  Univariate Regression: SAT scores and Education Expenditures
\item
  Does the amount of money spent on education affect the mean SAT score
  in a state?
\item
  Dependent variable: csat
\item
  Independent variable: expense
\end{itemize}

\subsection{Opening Files in Stata}\label{opening-files-in-stata-1}

\begin{itemize}
\tightlist
\item
  Look at bottom left hand corner of Stata screen

  \begin{itemize}
  \tightlist
  \item
    This is the directory Stata is currently reading from
  \end{itemize}
\item
  Files are located in the StataStatistics folder on the Desktop
\item
  Start by telling Stata where to look for these
\end{itemize}

\begin{verbatim}
  // change directory
  cd "~/Desktop/Stata/StataStatGraph"
\end{verbatim}

\begin{verbatim}
set more off

cd "~/Desktop/Stata/StataStatGraph"
/nfs/www/edu-harvard-iq-tutorials/Stata/StataStatGraph
\end{verbatim}

\begin{itemize}
\tightlist
\item
  Use dir to see what is in the directory:
\end{itemize}

\begin{verbatim}
  dir
  cd dataSets
  dir
  cd ..
\end{verbatim}

\begin{verbatim}
dir

total 8
drwxr-sr-x. 2 izahn tutorwww 4096 Oct 22 21:59 dataSets/
drwxr-sr-x. 3 izahn tutorwww 4096 Oct 22 21:59 images/
cd dataSets
/nfs/www/edu-harvard-iq-tutorials/Stata/StataStatGraph/dataSets
dir

total 21008
-rwxr-xr-x. 1 izahn tutorwww 21103444 Oct 22 21:59 NatNeighCrimeStudy.dta*
-rwxr-xr-x. 1 izahn tutorwww     8977 Oct 22 21:59 states.dta*
-rwxr-xr-x. 1 izahn tutorwww   298191 Oct 22 21:59 TimePollPubSchools.dta*
cd ..
/nfs/www/edu-harvard-iq-tutorials/Stata/StataStatGraph
\end{verbatim}

\begin{itemize}
\tightlist
\item
  Load the data
\end{itemize}

\begin{verbatim}
  // use the states data set
  use dataSets/states.dta
\end{verbatim}

\begin{verbatim}

use dataSets/states.dta
(U.S. states data 1990-91)
\end{verbatim}

\section{Simple regression}\label{simple-regression}

\subsection{Steps for Running
Regression}\label{steps-for-running-regression}

\begin{enumerate}
\def\labelenumi{\arabic{enumi}.}
\tightlist
\item
  Examine descriptive statistics
\item
  Look at relationship graphically and test correlation(s)
\item
  Run and interpret regression
\item
  Test regression assumptions
\end{enumerate}

\subsection{Preliminaries}\label{preliminaries}

\begin{itemize}
\tightlist
\item
  We want to predict csat scores from expense
\item
  First, let's look at some descriptives
\end{itemize}

\begin{verbatim}
  // generate summary statistics for csat and expense
  sum csat expense
\end{verbatim}

\begin{verbatim}

sum csat expense

    Variable |        Obs        Mean    Std. Dev.       Min        Max
-------------+---------------------------------------------------------
        csat |         51     944.098    66.93497        832       1093
     expense |         51    5235.961    1401.155       2960       9259
\end{verbatim}

\begin{itemize}
\tightlist
\item
  We want to predict csat scores from expense
\item
  First, let's look at some descriptives
\end{itemize}

\begin{verbatim}
  // look at codebok
  codebook csat expense
\end{verbatim}

\begin{verbatim}

codebook csat expense

-------------------------------------------------------------------------------
csat                                                   Mean composite SAT score
-------------------------------------------------------------------------------

                  type:  numeric (int)

                 range:  [832,1093]                   units:  1
         unique values:  45                       missing .:  0/51

                  mean:   944.098
              std. dev:    66.935

           percentiles:        10%       25%       50%       75%       90%
                               874       886       926       997      1024

-------------------------------------------------------------------------------
expense                                         Per pupil expenditures prim&sec
-------------------------------------------------------------------------------

                  type:  numeric (int)

                 range:  [2960,9259]                  units:  1
         unique values:  51                       missing .:  0/51

                  mean:   5235.96
              std. dev:   1401.16

           percentiles:        10%       25%       50%       75%       90%
                              3782      4351      5000      5865      6738
\end{verbatim}

\begin{itemize}
\tightlist
\item
  Next, view relationship graphically
\item
  Scatterplots work well for univariate relationships
\end{itemize}

\begin{verbatim}
  // graph expense by csat
  twoway scatter expense csat
\end{verbatim}

\begin{itemize}
\tightlist
\item
  Next look at the correlation matrix
\end{itemize}

\begin{verbatim}
  // correlate csat and expense
  pwcorr csat expense, star(.05)
\end{verbatim}

\begin{verbatim}

pwcorr csat expense, star(.05)

             |     csat  expense
-------------+------------------
        csat |   1.0000 
     expense |  -0.4663*  1.0000
\end{verbatim}

\begin{itemize}
\tightlist
\item
  Not very interesting with only one predictor
\end{itemize}

\subsection{SAT scores \& Education
Expenditures}\label{sat-scores-education-expenditures}

\begin{verbatim}
  regress csat expense
\end{verbatim}

\begin{verbatim}
regress csat expense

      Source |       SS           df       MS      Number of obs   =        51
-------------+----------------------------------   F(1, 49)        =     13.61
       Model |  48708.3001         1  48708.3001   Prob > F        =    0.0006
    Residual |   175306.21        49  3577.67775   R-squared       =    0.2174
-------------+----------------------------------   Adj R-squared   =    0.2015
       Total |   224014.51        50   4480.2902   Root MSE        =    59.814

------------------------------------------------------------------------------
        csat |      Coef.   Std. Err.      t    P>|t|     [95% Conf. Interval]
-------------+----------------------------------------------------------------
     expense |  -.0222756   .0060371    -3.69   0.001    -.0344077   -.0101436
       _cons |   1060.732    32.7009    32.44   0.000     995.0175    1126.447
------------------------------------------------------------------------------
\end{verbatim}

\subsection{OLS Assumptions}\label{ols-assumptions}

\begin{itemize}
\tightlist
\item
  Assumption 1: Specification is appropriate (i.e., no relevant omitted
  variables)
\item
  Assumption 2: Homoscedasticity (The variance around the regression
  model is the same for all values of the predictor variable)
\item
  Assumption 3: Errors are independent
\item
  Assumption 4: Relationships are linear
\item
  Assumption 5: Normal Distribution (only needed for inference)
\item
  The errors of regression equation are normally distributed
\end{itemize}

\subsubsection{Specification}\label{specification}

The model specification should be informed by theory - i.e., our
substantive knowledge of the subject matter. It's important to include
all relevant predictors in the model, otherwise our estimates will be
biased.

\begin{itemize}
\tightlist
\item
  Goodness of fit
\end{itemize}

\subsubsection{Homoscedasticity}\label{homoscedasticity}

\begin{verbatim}
  rvfplot
\end{verbatim}

\begin{verbatim}
rvfplot
\end{verbatim}

\subsubsection{Normality}\label{normality}

\begin{itemize}
\tightlist
\item
  A simple histogram of the residuals can be informative
\end{itemize}

\begin{verbatim}
  // graph the residual values of csat
  predict resid, residual
  histogram resid, normal 
\end{verbatim}

\begin{verbatim}

predict resid, residual
histogram resid, normal
(bin=7, start=-131.81111, width=38.329487)
\end{verbatim}

\section{Multiple Regression}\label{multiple-regression}

\begin{itemize}
\tightlist
\item
  Just keep adding predictors
\item
  Let's try adding some predictors to the model of SAT scores
\item
  income :: \% students taking SATs
\item
  percent :: \% adults with HS diploma (high)
\end{itemize}

\subsection{Preliminaries}\label{preliminaries-1}

\begin{itemize}
\tightlist
\item
  As before, start with descriptive statistics and correlations
\end{itemize}

\begin{verbatim}
  // descriptive statistics and correlations
  sum income percent high
  pwcorr csat expense income percent high
\end{verbatim}

\begin{verbatim}

sum income percent high

    Variable |        Obs        Mean    Std. Dev.       Min        Max
-------------+---------------------------------------------------------
      income |         51    33.95657    6.423134     23.465     48.618
     percent |         51    35.76471    26.19281          4         81
        high |         51    76.26078    5.588741       64.3       86.6
pwcorr csat expense income percent high

             |     csat  expense   income  percent     high
-------------+---------------------------------------------
        csat |   1.0000 
     expense |  -0.4663   1.0000 
      income |  -0.4713   0.6784   1.0000 
     percent |  -0.8758   0.6509   0.6733   1.0000 
        high |   0.0858   0.3133   0.5099   0.1413   1.0000
\end{verbatim}

\begin{itemize}
\tightlist
\item
  regress csat on exense, income, percent, and high
\end{itemize}

\begin{verbatim}
  regress csat expense income percent high
\end{verbatim}

\begin{verbatim}
regress csat expense income percent high

      Source |       SS           df       MS      Number of obs   =        51
-------------+----------------------------------   F(4, 46)        =     51.86
       Model |  183354.603         4  45838.6508   Prob > F        =    0.0000
    Residual |  40659.9067        46  883.911016   R-squared       =    0.8185
-------------+----------------------------------   Adj R-squared   =    0.8027
       Total |   224014.51        50   4480.2902   Root MSE        =    29.731

------------------------------------------------------------------------------
        csat |      Coef.   Std. Err.      t    P>|t|     [95% Conf. Interval]
-------------+----------------------------------------------------------------
     expense |   .0045604    .004384     1.04   0.304    -.0042641     .013385
      income |   .4437858   1.138947     0.39   0.699    -1.848795    2.736367
     percent |  -2.533084   .2454477   -10.32   0.000    -3.027145   -2.039024
        high |   2.086599   .9246023     2.26   0.029     .2254712    3.947727
       _cons |   836.6197   58.33238    14.34   0.000     719.2027    954.0366
------------------------------------------------------------------------------
\end{verbatim}

\section{Exercise 0}\label{exercise-0-8}

\textbf{Multiple Regression}

Open the datafile, states.dta.

\begin{enumerate}
\def\labelenumi{\arabic{enumi}.}
\tightlist
\item
  Select a few variables to use in a multiple regression of your own.
  Before running the regression, examine descriptive of the variables
  and generate a few scatterplots.
\item
  Run your regression
\item
  Examine the plausibility of the assumptions of normality and
  homogeneity
\end{enumerate}

\section{Interactions}\label{interactions}

\begin{itemize}
\tightlist
\item
  What if we wanted to test an interaction between percent \& high?
\item
  Option 1: generate product terms by hand
\end{itemize}

\begin{verbatim}
  // generate product of percent and high
  gen percenthigh = percent*high 
  regress csat expense income percent high percenthigh
\end{verbatim}

\begin{verbatim}

gen percenthigh = percent*high
regress csat expense income percent high percenthigh

      Source |       SS           df       MS      Number of obs   =        51
-------------+----------------------------------   F(5, 45)        =     46.11
       Model |  187430.401         5  37486.0801   Prob > F        =    0.0000
    Residual |  36584.1091        45  812.980201   R-squared       =    0.8367
-------------+----------------------------------   Adj R-squared   =    0.8185
       Total |   224014.51        50   4480.2902   Root MSE        =    28.513

------------------------------------------------------------------------------
        csat |      Coef.   Std. Err.      t    P>|t|     [95% Conf. Interval]
-------------+----------------------------------------------------------------
     expense |   .0045575   .0042044     1.08   0.284    -.0039107    .0130256
      income |   .0887856    1.10374     0.08   0.936    -2.134261    2.311832
     percent |  -8.143002   2.516509    -3.24   0.002    -13.21151   -3.074493
        high |   .4240906   1.156545     0.37   0.716    -1.905311    2.753492
 percenthigh |   .0740926   .0330909     2.24   0.030     .0074441    .1407411
       _cons |    972.525    82.5457    11.78   0.000     806.2695    1138.781
------------------------------------------------------------------------------
\end{verbatim}

\begin{itemize}
\tightlist
\item
  What if we wanted to test an interaction between percent \& high?
\item
  Option 2: Let Stata do your dirty work
\end{itemize}

\begin{verbatim}
  // use the # sign to represent interactions 
  regress csat percent high c.percent#c.high
  // same as . regress csat c.percent##high
\end{verbatim}

\begin{verbatim}

regress csat percent high c.percent#c.high

      Source |       SS           df       MS      Number of obs   =        51
-------------+----------------------------------   F(3, 47)        =     77.39
       Model |  186302.091         3  62100.6971   Prob > F        =    0.0000
    Residual |  37712.4186        47  802.391885   R-squared       =    0.8317
-------------+----------------------------------   Adj R-squared   =    0.8209
       Total |   224014.51        50   4480.2902   Root MSE        =    28.327

------------------------------------------------------------------------------
        csat |      Coef.   Std. Err.      t    P>|t|     [95% Conf. Interval]
-------------+----------------------------------------------------------------
     percent |   -8.15717   2.488388    -3.28   0.002    -13.16316   -3.151179
        high |   .6674578   1.082615     0.62   0.541    -1.510482    2.845398
             |
   c.percent#|
      c.high |   .0764271   .0324919     2.35   0.023     .0110619    .1417924
             |
       _cons |   974.9354   81.98078    11.89   0.000     810.0113    1139.859
------------------------------------------------------------------------------
\end{verbatim}

\subsection{Categorical Predictors}\label{categorical-predictors}

\begin{itemize}
\tightlist
\item
  For categorical variables, we first need to dummy code
\item
  Use region as example

  \begin{itemize}
  \tightlist
  \item
    Option 1: create dummy codes before fitting regression model
  \end{itemize}
\end{itemize}

\begin{verbatim}
  // create region dummy codes using tab 
  tab region, gen(region)

  //regress csat on region
  regress csat region1 region2 region3
\end{verbatim}

\begin{verbatim}

tab region, gen(region)

Geographica |
   l region |      Freq.     Percent        Cum.
------------+-----------------------------------
       West |         13       26.00       26.00
    N. East |          9       18.00       44.00
      South |         16       32.00       76.00
    Midwest |         12       24.00      100.00
------------+-----------------------------------
      Total |         50      100.00


regress csat region1 region2 region3

      Source |       SS           df       MS      Number of obs   =        50
-------------+----------------------------------   F(3, 46)        =      9.61
       Model |  82049.4719         3   27349.824   Prob > F        =    0.0000
    Residual |  130911.908        46  2845.91105   R-squared       =    0.3853
-------------+----------------------------------   Adj R-squared   =    0.3452
       Total |   212961.38        49  4346.15061   Root MSE        =    53.347

------------------------------------------------------------------------------
        csat |      Coef.   Std. Err.      t    P>|t|     [95% Conf. Interval]
-------------+----------------------------------------------------------------
     region1 |  -63.77564   21.35592    -2.99   0.005    -106.7629    -20.7884
     region2 |  -120.5278   23.52385    -5.12   0.000    -167.8788   -73.17672
     region3 |  -80.08333   20.37225    -3.93   0.000    -121.0906   -39.07611
       _cons |   1010.083   15.39998    65.59   0.000     979.0848    1041.082
------------------------------------------------------------------------------
\end{verbatim}

\begin{itemize}
\tightlist
\item
  For categorical variables, we first need to dummy code
\item
  Use region as example

  \begin{itemize}
  \tightlist
  \item
    Option 2: Let Stata do it for you
  \end{itemize}
\end{itemize}

\begin{verbatim}
  // regress csat on region using fvvarlist syntax
  // see help fvvarlist for details
  regress csat i.region
\end{verbatim}

\begin{verbatim}


regress csat i.region

      Source |       SS           df       MS      Number of obs   =        50
-------------+----------------------------------   F(3, 46)        =      9.61
       Model |  82049.4719         3   27349.824   Prob > F        =    0.0000
    Residual |  130911.908        46  2845.91105   R-squared       =    0.3853
-------------+----------------------------------   Adj R-squared   =    0.3452
       Total |   212961.38        49  4346.15061   Root MSE        =    53.347

------------------------------------------------------------------------------
        csat |      Coef.   Std. Err.      t    P>|t|     [95% Conf. Interval]
-------------+----------------------------------------------------------------
      region |
    N. East  |  -56.75214   23.13285    -2.45   0.018    -103.3161   -10.18813
      South  |  -16.30769   19.91948    -0.82   0.417    -56.40353    23.78814
    Midwest  |   63.77564   21.35592     2.99   0.005      20.7884    106.7629
             |
       _cons |   946.3077   14.79582    63.96   0.000     916.5253    976.0901
------------------------------------------------------------------------------
\end{verbatim}

\section{Exercise 1}\label{exercise-1-8}

\textbf{Regression, Categorical Predictors, \& Interactions}

Open the datafile, states.dta.

\begin{enumerate}
\def\labelenumi{\arabic{enumi}.}
\tightlist
\item
  Add on to the regression equation that you created in exercise 1 by
  generating an interaction term and testing the interaction.
\item
  Try adding a categorical variable to your regression (remember, it
  will need to be dummy coded). You could use region or generate a new
  categorical variable from one of the continuous variables in the
  dataset.
\end{enumerate}

\section{Exporting \& saving results}\label{exporting-saving-results}

\subsection{Regression tables}\label{regression-tables}

\begin{itemize}
\tightlist
\item
  Usually when we're running regression, we'll be testing multiple
  models at a time
\item
  Can be difficult to compare results
\item
  Stata offers several user-friendly options for storing and viewing
  regression output from multiple models
\item
  First, download the necessary packages:
\end{itemize}

\begin{verbatim}
  // install outreg2 package
  findit outreg2
\end{verbatim}

\subsection{Saving \& replaying}\label{saving-replaying}

\begin{itemize}
\tightlist
\item
  You can store regression model results in Stata
\end{itemize}

\begin{verbatim}
  // fit two regression models and store the results
  regress csat expense income percent high
  estimates store Model1
  regress csat expense income percent high i.region
  estimates store Model2
\end{verbatim}

\begin{verbatim}

regress csat expense income percent high

      Source |       SS           df       MS      Number of obs   =        51
-------------+----------------------------------   F(4, 46)        =     51.86
       Model |  183354.603         4  45838.6508   Prob > F        =    0.0000
    Residual |  40659.9067        46  883.911016   R-squared       =    0.8185
-------------+----------------------------------   Adj R-squared   =    0.8027
       Total |   224014.51        50   4480.2902   Root MSE        =    29.731

------------------------------------------------------------------------------
        csat |      Coef.   Std. Err.      t    P>|t|     [95% Conf. Interval]
-------------+----------------------------------------------------------------
     expense |   .0045604    .004384     1.04   0.304    -.0042641     .013385
      income |   .4437858   1.138947     0.39   0.699    -1.848795    2.736367
     percent |  -2.533084   .2454477   -10.32   0.000    -3.027145   -2.039024
        high |   2.086599   .9246023     2.26   0.029     .2254712    3.947727
       _cons |   836.6197   58.33238    14.34   0.000     719.2027    954.0366
------------------------------------------------------------------------------
estimates store Model1
regress csat expense income percent high i.region

      Source |       SS           df       MS      Number of obs   =        50
-------------+----------------------------------   F(7, 42)        =     51.07
       Model |  190570.293         7  27224.3275   Prob > F        =    0.0000
    Residual |  22391.0874        42  533.121128   R-squared       =    0.8949
-------------+----------------------------------   Adj R-squared   =    0.8773
       Total |   212961.38        49  4346.15061   Root MSE        =    23.089

------------------------------------------------------------------------------
        csat |      Coef.   Std. Err.      t    P>|t|     [95% Conf. Interval]
-------------+----------------------------------------------------------------
     expense |   -.004375   .0044603    -0.98   0.332    -.0133763    .0046263
      income |   1.306164    .950279     1.37   0.177    -.6115765    3.223905
     percent |  -2.965514   .2496481   -11.88   0.000    -3.469325   -2.461704
        high |   3.544804   1.075863     3.29   0.002     1.373625    5.715983
             |
      region |
    N. East  |   80.81334    15.4341     5.24   0.000     49.66607    111.9606
      South  |   33.61225   13.94521     2.41   0.020     5.469676    61.75483
    Midwest  |   32.15421   10.20145     3.15   0.003     11.56686    52.74157
             |
       _cons |   724.8289   79.25065     9.15   0.000     564.8946    884.7631
------------------------------------------------------------------------------
estimates store Model2
\end{verbatim}

\begin{itemize}
\tightlist
\item
  Stored models can be recalled
\end{itemize}

\begin{verbatim}
  // Display Model1
  estimates replay Model1
\end{verbatim}

\begin{verbatim}

estimates replay Model1

-------------------------------------------------------------------------------
Model Model1
-------------------------------------------------------------------------------

      Source |       SS           df       MS      Number of obs   =        51
-------------+----------------------------------   F(4, 46)        =     51.86
       Model |  183354.603         4  45838.6508   Prob > F        =    0.0000
    Residual |  40659.9067        46  883.911016   R-squared       =    0.8185
-------------+----------------------------------   Adj R-squared   =    0.8027
       Total |   224014.51        50   4480.2902   Root MSE        =    29.731

------------------------------------------------------------------------------
        csat |      Coef.   Std. Err.      t    P>|t|     [95% Conf. Interval]
-------------+----------------------------------------------------------------
     expense |   .0045604    .004384     1.04   0.304    -.0042641     .013385
      income |   .4437858   1.138947     0.39   0.699    -1.848795    2.736367
     percent |  -2.533084   .2454477   -10.32   0.000    -3.027145   -2.039024
        high |   2.086599   .9246023     2.26   0.029     .2254712    3.947727
       _cons |   836.6197   58.33238    14.34   0.000     719.2027    954.0366
------------------------------------------------------------------------------
\end{verbatim}

\begin{itemize}
\tightlist
\item
  Stored models can be compared
\end{itemize}

\begin{verbatim}
  // Compare Model1 and Model2 coefficients
  estimates table Model1 Model2
\end{verbatim}

\begin{verbatim}

estimates table Model1 Model2

----------------------------------------
    Variable |   Model1       Model2    
-------------+--------------------------
     expense |  .00456044   -.00437502  
      income |  .44378583    1.3061642  
     percent | -2.5330843   -2.9655142  
        high |  2.0865991    3.5448038  
             |
      region |
    N. East  |               80.813342  
      South  |               33.612251  
    Midwest  |               32.154215  
             |
       _cons |  836.61966    724.82886  
----------------------------------------
\end{verbatim}

\subsection{Exporting to Excel}\label{exporting-to-excel}

\begin{itemize}
\tightlist
\item
  Avoid human error when transferring coefficients into tables
\item
  Excel can be used to format publication-ready tables
\end{itemize}

\begin{verbatim}
  outreg2 [Model1 Model2] using csatprediction.xls, replace
\end{verbatim}

\begin{verbatim}
outreg2 [Model1 Model2] using csatprediction.xls, replace
~/ado/plus/o/outreg2.ado
csatprediction.xls
dir : seeout
\end{verbatim}

\begin{center}\rule{0.5\linewidth}{\linethickness}\end{center}

\section{Graphing in Stata}\label{graphing-in-stata}

\subsection{Graphing Strategies}\label{graphing-strategies}

\begin{itemize}
\tightlist
\item
  Keep it simple
\item
  Labels, labels, labels!!
\item
  Avoid cluttered graphs
\item
  Every part of the graph should be meaningful
\item
  Avoid:

  \begin{itemize}
  \tightlist
  \item
    Shading
  \item
    Distracting colors
  \item
    Decoration
  \end{itemize}
\item
  Always know what you're working with before you get started

  \begin{itemize}
  \tightlist
  \item
    Recognize scale of data
  \item
    If you're using multiple variables -- how do their scales align?
  \end{itemize}
\item
  Before any graphing procedure review variables with \texttt{codebook},
  \texttt{sum}, \texttt{tab}, etc.
\item
  HELPFUL STATA HINT: If you want your command to go on multiple lines
  use \texttt{///} at end of each line
\end{itemize}

\subsection{Terrible Graph}\label{terrible-graph}

\begin{figure}
\centering
\includegraphics{Stata/StataModGraph/images/Terrible.png}
\caption{}
\end{figure}

\subsection{Much Better Graph}\label{much-better-graph}

\begin{figure}
\centering
\includegraphics{Stata/StataModGraph/images/Good.png}
\caption{}
\end{figure}

\section{Univariate Graphics}\label{univariate-graphics}

\subsection{Our First Dataset}\label{our-first-dataset}

\begin{itemize}
\tightlist
\item
  Time Magazine Public School Poll

  \begin{itemize}
  \tightlist
  \item
    Based on survey of 1,000 adults in U.S.
  \item
    Conducted in August 2010
  \item
    Questions regarding feelings about parental involvement, teachers
    union, current potential for reform
  \end{itemize}
\item
  Open Stata and call up the datafile for today
\end{itemize}

\begin{verbatim}
  // Step 1: tell Stata where to find data:
  cd "~/StataGraphics/dataSets"
  // Step 2: call up our dataset:
  use TimePollPubSchools.dta
\end{verbatim}

\subsection{Single Continuous
Variables}\label{single-continuous-variables}

\textbf{Example: Histograms}

\begin{itemize}
\tightlist
\item
  Stata assumes you're working with continuous data
\item
  Very simple syntax:

  \begin{itemize}
  \tightlist
  \item
    \texttt{hist\ varname}
  \end{itemize}
\item
  Put a comma after your varname and start adding options

  \begin{itemize}
  \tightlist
  \item
    \texttt{bin(\#)} : change the number of bars that the graph displays
  \item
    \texttt{normal} : overlay normal curve
  \item
    \texttt{addlabels} : add actual values to bars
  \end{itemize}
\end{itemize}

\textbf{Histogram Options}

\begin{itemize}
\tightlist
\item
  To change the numeric depiction of your data add these options after
  the comma

  \begin{itemize}
  \tightlist
  \item
    Choose one: density fraction frequency percent
  \end{itemize}
\item
  Be sure to properly describe your histogram:

  \begin{itemize}
  \tightlist
  \item
    \texttt{title(insert\ name\ of\ graph)}
  \item
    \texttt{subtitle(insert\ subtitle\ of\ graph)}
  \item
    \texttt{note(insert\ note\ to\ appear\ at\ bottom\ of\ graph)}
  \item
    \texttt{caption(insert\ caption\ to\ appear\ below\ notes)}
  \end{itemize}
\end{itemize}

\textbf{Histogram Example}

\begin{verbatim}
  hist F1, bin(10) percent title(TITLE) ///
    subtitle(SUBTITLE) caption(CAPTION) note(NOTES)
\end{verbatim}

\begin{figure}
\centering
\includegraphics{Stata/StataModGraph/images/hist1.png}
\caption{}
\end{figure}

\textbf{Axis Titles \& Labels}

\begin{itemize}
\tightlist
\item
  Axis title options (default is variable label):

  \begin{itemize}
  \tightlist
  \item
    \texttt{xtitle(insert\ x\ axis\ name)}
  \item
    \texttt{ytitle(insert\ y\ axis\ name)}
  \end{itemize}
\item
  Don't want axis titles?

  \begin{itemize}
  \tightlist
  \item
    \texttt{xtitle("")}
  \item
    \texttt{ytitle("")}
  \end{itemize}
\item
  Add labels to X or Y axis:

  \begin{itemize}
  \tightlist
  \item
    xlabel(insert x axis label)
  \item
    ylabel(insert y axis label)
  \end{itemize}
\item
  Tell Stata how to scale each axis

  \begin{itemize}
  \tightlist
  \item
    xlabel(start\#(increment)end\#)
  \item
    xlabel(0(5)100)
  \end{itemize}
\item
  This would label x-axis from 0-100 in increments of 5
\end{itemize}

\textbf{Axis Labels Example}

\begin{verbatim}
  hist F1, bin(10) percent title(TITLE) subtitle(SUBTITLE) ///
      caption(CAPTION) note(NOTES) ///
      xtitle(Here's your x-axis title) ///
  ytitle(here's your y-axis title)
\end{verbatim}

\begin{figure}
\centering
\includegraphics{Stata/StataModGraph/images/hist2.png}
\caption{}
\end{figure}

\subsection{Single Categorical
Variables}\label{single-categorical-variables}

\begin{itemize}
\tightlist
\item
  We can also use the \texttt{hist} command for bar graphs

  \begin{itemize}
  \tightlist
  \item
    Simply specify ``discrete'' with options
  \end{itemize}
\item
  Stata will produce one bar for each level (i.e.~category) of variable
\item
  Use \texttt{xlabel} command to insert names of individual categories
\end{itemize}

\begin{verbatim}
  hist F4, title(Racial breakdown of Time Poll Sample) xtitle(Race) ///
  ytitle(Percent) xlabel(1 "White" 2 "Black" 3 "Asian" 4 "Hispanic" ///
   5 "Other") discrete percent addlabels
\end{verbatim}

\begin{figure}
\centering
\includegraphics{Stata/StataModGraph/images/bargraph.png}
\caption{}
\end{figure}

\section{Exercise 2}\label{exercise-2-6}

\textbf{Histograms Bar Graphs}

\begin{enumerate}
\def\labelenumi{\arabic{enumi}.}
\tightlist
\item
  Open the datafile, NatNeighCrimeStudy.dta.
\item
  Create a histogram of the tract-level poverty rate (variable name:
  \texttt{T\_POVRTY}).
\item
  Insert the normal curve over the histogram
\item
  Change the numeric representation on the Y-axis to ``percent''
\item
  Add appropriate titles to the overall graph and the x axis and y axis.
  Also, add a note that states the source of this data.
\item
  Open the datafile, TimePollPubSchools.dta
\item
  Create a histogram of the question, ``What grade would you give your
  child's school'' (variable name: Q11). Be sure to tell Stata that this
  is a categorical variable.
\item
  Format this graph so that the axes have proper titles and labels.
  Also, add an appropriate title to the overall graph that goes onto two
  lines. Add a note stating the source of the data.
\end{enumerate}

\section{Bivariate Graphics}\label{bivariate-graphics}

\subsection{Next Dataset:}\label{next-dataset}

\begin{itemize}
\tightlist
\item
  National Neighborhood Crime Study (NNCS)

  \begin{itemize}
  \tightlist
  \item
    N=9,593 census tracts in 2000
  \item
    Explore sources of variation in crime for communities in the United
    States
  \item
    Tract-level data: crime, social disorganization, disadvantage,
    socioeconomic inequality
  \item
    City-level data: labor market, socioeconomic inequality, population
    change
  \end{itemize}
\end{itemize}

\subsection{The Twoway Family}\label{the-twoway-family}

\begin{itemize}
\tightlist
\item
  \texttt{twoway} is basic Stata command for all twoway graphs
\item
  Use \texttt{twoway} anytime you want to make comparisons among
  variables
\item
  Can be used to combine graphs (i.e., overlay one graph with another

  \begin{itemize}
  \tightlist
  \item
    e.g., insert line of best fit over a scatter plot
  \end{itemize}
\item
  Some basic examples:
\end{itemize}

\begin{verbatim}
  use NatNeighCrimeStudy.dta
  twoway scatter T_PERCAP T_VIOLNT
  twoway dropline T_PERCAP T_VIOLNT
  twoway  lfitci T_PERCAP T_VIOLNT
\end{verbatim}

\textbf{Twoway \& the \texttt{by} Statement}

\begin{verbatim}
  twoway scatter T_PERCAP T_VIOLNT, by(DIVISION)
\end{verbatim}

\begin{figure}
\centering
\includegraphics{Stata/StataModGraph/images/twowayby.png}
\caption{}
\end{figure}

\textbf{Twoway Title Options}

\begin{itemize}
\tightlist
\item
  Same title options as with histogram

  \begin{itemize}
  \tightlist
  \item
    \texttt{title(insert\ name\ of\ graph)}
  \item
    \texttt{subtitle(insert\ subtitle\ of\ graph)}
  \item
    \texttt{note(insert\ note\ to\ appear\ at\ bottom\ of\ graph)}
  \item
    \texttt{caption(insert\ caption\ to\ appear\ below\ notes)}
  \end{itemize}
\end{itemize}

\textbf{Twoway Title Options Example}

\begin{verbatim}
  twoway scatter T_PERCAP T_VIOLNT, ///
      title(Comparison of Per Capita Income ///
            and Violent Crime Rate at Tract level) ///
  xtitle(Violent Crime Rate) ytitle(Per Capita Income) ///
      note(Source: National Neighborhood Crime Study 2000) 
\end{verbatim}

\begin{itemize}
\tightlist
\item
  The title is a bit cramped--let's fix that:
\end{itemize}

\begin{verbatim}
  twoway scatter T_PERCAP T_VIOLNT, ///
      title("Comparison of Per Capita Income" ///
  "and Violent Crime Rate at Tract level") ///
  xtitle(Violent Crime Rate) ytitle(Per Capita Income) ///
  note(Source: National Neighborhood Crime Study 2000) 
\end{verbatim}

\textbf{Twoway Symbol Options}

\begin{itemize}
\tightlist
\item
  A variety of symbol shapes are available: use
  \texttt{palette\ symbolpalette} to seem them and \texttt{msymbol()} to
  set them
\end{itemize}

\begin{figure}
\centering
\includegraphics{Stata/StataModGraph/images/Symbol.png}
\caption{}
\end{figure}

\textbf{Twoway Symbol Options}

\begin{verbatim}
  twoway scatter T_PERCAP T_VIOLNT, ///
      title("Comparison of Per Capita Income" ///
  "and Violent Crime Rate at Tract level") ///
  xtitle(Violent Crime Rate) ytitle(Per Capita Income) ///
  note(Source: National Neighborhood Crime Study 2000) ///
  msymbol(Sh) mcolor("red")
\end{verbatim}

\begin{figure}
\centering
\includegraphics{Stata/StataModGraph/images/msymbol_mcolor.png}
\caption{}
\end{figure}

\subsection{Overlaying Twoway Graphs}\label{overlaying-twoway-graphs}

\begin{itemize}
\tightlist
\item
  Very simple to combine multiple graphs\ldots{}just put each graph
  command in parentheses

  \begin{itemize}
  \tightlist
  \item
    \texttt{twoway\ (scatter\ var1\ var2)\ (lfit\ var1\ var2)}
  \end{itemize}
\item
  Add individual options to each graph within the parentheses
\item
  Add overall graph options as usual following the comma

  \begin{itemize}
  \tightlist
  \item
    \texttt{twoway\ (scatter\ var1\ var2)\ (lfit\ var1\ var2),\ options}
  \end{itemize}
\end{itemize}

\textbf{Overlaying Points \& Lines}

\begin{verbatim}
  twoway (scatter T_PERCAP T_VIOLNT) ///
      (lfit T_PERCAP T_VIOLNT), ///
      title("Comparison of Per Capita Income" ///
            "and Violent Crime Rate at Tract level") ///
      xtitle(Violent Crime Rate) ytitle(Per Capita Income) ///
      note(Source: National  Neighborhood Crime Study 2000)
\end{verbatim}

\textbf{Overlaying Points \& Labels}

\begin{verbatim}
  twoway (scatter T_PERCAP T_VIOLNT if T_VIOLNT==1976, ///
          mlabel(CITY)) (scatter T_PERCAP T_VIOLNT), ///
      title("Comparison of Per Capita Income" ///
            "and Violent Crime Rate at Tract level") ///
      xlabel(0(200)2400) note(Source: National Neighborhood ///
                              Crime Study 2000) legend(off)
\end{verbatim}

\section{Exercise 3}\label{exercise-3-4}

\textbf{The TwoWay Family}

Open the datafile, NatNeighCrimeStudy.dta.

\begin{enumerate}
\def\labelenumi{\arabic{enumi}.}
\tightlist
\item
  Create a basic twoway scatterplot that compares the city unemployment
  rate (\texttt{C\_UNEMP}) to the percent secondary sector low-wage jobs
  (\texttt{C\_SSLOW})
\item
  Generate the same scatterplot, but this time, divide the plot by the
  dummy variable indicating whether the city is located in the south or
  not (\texttt{C\_SOUTH})
\item
  Change the color of the symbol that you use in this scatter plot
\item
  Change the type of symbol you use to a marker of your choice
\item
  Notice in your scatterplot that is broken down by \texttt{C\_SOUTH}
  that there is an outlier in the upper right hand corner of the ``Not
  South'' graph. Add the city name label to this marker.
\item
  Review the options available under ``help twowayoptions'' and change
  one aspect of your graph using an option that we haven't already
  reviewed
\end{enumerate}

\section{Twoway Line Graphs}\label{twoway-line-graphs}

\begin{itemize}
\tightlist
\item
  Line graphs helpful for a variety of data

  \begin{itemize}
  \tightlist
  \item
    Especially any type of time series data
  \end{itemize}
\item
  We'll use data on US life expectancy from 1900-1999

  \begin{itemize}
  \tightlist
  \item
    \texttt{webuse\ uslifeexp,\ clear}
  \end{itemize}
\end{itemize}

\begin{verbatim}
  webuse uslifeexp, clear
  twoway (line le_wm year, mcolor("red")) ///
      (line le_bm year, mcolor("green"))
\end{verbatim}

\begin{figure}
\centering
\includegraphics{Stata/StataModGraph/images/lineGraph1.png}
\caption{}
\end{figure}

\begin{verbatim}
  twoway (line (le_wfemale le_wmale le_bf le_bm) year, ///
      lpattern(dot solid dot solid))
\end{verbatim}

\begin{figure}
\centering
\includegraphics{Stata/StataModGraph/images/linegraph2.png}
\caption{}
\end{figure}

\textbf{Stata Graphing Lines}

\begin{verbatim}
  palette linepalette
\end{verbatim}

\begin{figure}
\centering
\includegraphics{Stata/StataModGraph/images/linepalette.png}
\caption{}
\end{figure}

\section{Exporting Graphs}\label{exporting-graphs}

\begin{itemize}
\tightlist
\item
  From Stata, right click on image and select ``save as'' or try syntax:

  \begin{itemize}
  \tightlist
  \item
    \texttt{graph\ export\ myfig.esp,\ replace}
  \end{itemize}
\item
  In Microsoft Word: insert -\textgreater{} picture -\textgreater{} from
  file

  \begin{itemize}
  \tightlist
  \item
    Or, right click on graph in Stata and copy and paste into MS Word
  \end{itemize}
\end{itemize}

\section{Exercise Solutions}\label{exercise-solutions-6}

\subsection{Ex 0: prototype}\label{ex-0-prototype-6}

\subsection{Ex 1: prototype}\label{ex-1-prototype-6}

\subsection{Ex 2: prototype}\label{ex-2-prototype-5}

\subsection{Ex 3: prototype}\label{ex-3-prototype-4}

\section{Wrap-up}\label{wrap-up-9}

\subsection{Feedback}\label{feedback-9}

These workshops are a work in progress, please provide any feedback to:
\href{mailto:help@iq.harvard.edu}{\nolinkurl{help@iq.harvard.edu}}

\subsection{Resources}\label{resources-9}

\begin{itemize}
\tightlist
\item
  IQSS

  \begin{itemize}
  \tightlist
  \item
    Workshops: \url{https://dss.iq.harvard.edu/workshop-materials}
  \item
    Data Science Services: \url{https://dss.iq.harvard.edu/}
  \item
    Research Computing Environment:
    \url{https://iqss.github.io/dss-rce/}
  \end{itemize}
\item
  HBS

  \begin{itemize}
  \tightlist
  \item
    Research Computing Services workshops:
    \url{https://training.rcs.hbs.org/workshops}
  \item
    Other HBS RCS resources:
    \url{https://training.rcs.hbs.org/workshop-materials}
  \item
    RCS consulting email: \url{mailto:research@hbs.edu}
  \end{itemize}
\item
  Stata

  \begin{itemize}
  \tightlist
  \item
    UCLA website: \url{http://www.ats.ucla.edu/stat/Stata/}
  \item
    Stata website: \url{http://www.stata.com/help.cgi?contents}
  \item
    Email list: \url{http://www.stata.com/statalist/}
  \end{itemize}
\end{itemize}


\end{document}
